Relacje mają bardzo dużo różnych własności. Większość z~nich nie jest szczególnie ciekawa, natomiast trafiają się perełki, które bardzo zapadają w~pamięć, ponieważ da się je przeczytać jak słowa. Tak, tak, mam na myśli słynne ,,rusot równa się rotusot''. Do tego też dotrzemy.
\begin{description}
	\item[Pierwsza grupa własności.] Przyjmijmy, że mamy następujące trzy relacje:
	      \begin{align*}
		      R & \subseteq A \times B \\
		      S & \subseteq B \times C \\
		      T & \subseteq C \times D
	      \end{align*}
	      Wtedy
	      \begin{itemize}
		      \item \(T \composition \pars{S \composition R} = \pars{T \composition S} \composition R\)\qquad (składanie relacji jest łączne --- własność znana także jako ,,to-sor równa się tos-or'')
		            \begin{proof}
			            ,,Typ'' się zgadza, ponieważ obie strony są podzbiorami \(A \times D\).
			            \begin{equation*}
				            \begin{split}
					            \pars{a, d} \in T \composition \pars{S \composition R}
					            \iff & a \in A \land d \in D \land \exists_{c \in C}\pars{\pars{a, c} \in S \composition R \land \pars{c, d} \in T}                     \\
					            \iff & a \in A \land d \in D                                                                                                            \\
					                 & \land \exists_{c \in C} \pars{\pars{\exists_{b \in B} \pars{\pars{a, b} \in R \land \pars{b, c} \in S}} \land \pars{c, d} \in T} \\
					            \iff & a \in A \land d \in D                                                                                                            \\
					                 & \land \exists_{c \in C}\exists_{b \in B}\pars{\pars{a, b} \in R \land \pars{b, c} \in S \land \pars{c, d} \in T}                 \\
					            \iff & a \in A \land d \in D                                                                                                            \\
					                 & \land \exists_{b \in B} \pars{\pars{a, b} \in R \land \pars{\exists_{c \in C} \pars{\pars{b, c} \in S \land \pars{c, d} \in T}}} \\
					            \iff & a \in A \land d \in D \land \exists_{b \in B} \pars{\pars{a, b} \in R \land \pars{b, d} \in T \composition S}                    \\
					            \iff & \pars{a, d} \in \pars{T \composition S} \composition R
				            \end{split}
			            \end{equation*}
		            \end{proof}
		      \item \(\pars{S \composition R}^{-1} = R^{-1} \composition S^{-1}\)
		            \begin{proof}
			            Zauważamy najpierw, że zgadza się ,,typ'', ponieważ obie strony są podzbiorami \(C \times A\).
			            \begin{equation*}
				            \begin{split}
					            \pars{c, a} \in \pars{S \composition R}^{-1}
					            \iff & \pars{a, c} \in \pars{S \composition R}                                                                  \\
					            \iff & a \in A \land c \in C \land \exists_{b \in B} \pars{\pars{a, b} \in R \land \pars{b, c} \in S}           \\
					            \iff & a \in A \land c \in C \land \exists_{b \in B} \pars{\pars{b, a} \in R^{-1} \land \pars{c, b} \in S^{-1}} \\
					            \iff & \pars{c, a} \in R^{-1} \composition S^{-1}
				            \end{split}
			            \end{equation*}
		            \end{proof}
		      \item \(R \subseteq \lproj{R} \times \rproj{R}\)
		            \begin{proof}
			            \begin{equation*}
				            \begin{split}
					            \pars{a, b} \in R
					            \implies & a \in A \land b \in B \land \pars{\exists_{y \in B} \pars{a, y} \in R} \land \pars{\exists_{x \in A} \pars{x, b} \in R} \\
					            \iff     & \pars{a \in A \land \exists_{y \in B} \pars{a, y} \in R} \land \pars{b \in B \land \exists_{x \in A} \pars{x, b} \in R} \\
					            \iff     & a \in \lproj{R} \land b \in \rproj{R}                                                                                   \\
					            \iff     & \pars{a, b} \in \lproj{R} \times \rproj{R}
				            \end{split}
			            \end{equation*}
			            Pierwsze przejście jest tylko implikacją. Oczywiście jeśli \(\pars{a, b} \in R\), to możemy powiedzieć, że zarówno \(a\), jak i~\(b\) mają parę w~\(R\). Jednak z~samego faktu, że \(a\)~i~\(b\) mają parę w~\(R\), nie wynika jeszcze, że \(a\)~i~\(b\)~\emph{są} parą w~\(R\). Dlatego w~ogólnym przypadku mamy tu zaledwie inkluzję, a~nie równość, co łatwo zobaczyć na przykładzie:
			            \begin{align*}
				            A                          & = \set{a_0, a_1}                                                           \\
				            B                          & = \set{b_0, b_0}                                                           \\
				            R                          & = \set{\pars{a_0, b_0}, \pars{a_1, b_1}}                                   \\
				            \lproj{R}                  & = \pars{a_0, a_1}                                                          \\
				            \rproj{R}                  & = \pars{b_0, b_1}                                                          \\
				            \lproj{R} \times \rproj{R} & = \set{\pars{a_0, b_0}, \pars{a_0, b_1}, \pars{a_1, b_0}, \pars{a_1, b_1}} \\
				            R                          & \subsetneq \lproj{R} \times \rproj{R}
			            \end{align*}
		            \end{proof}
		      \item \(\lproj{\pars{S \composition R}} \subseteq \lproj{R}\)
		            \begin{proof}
			            Zwróćmy uwagę, że \(S \composition R \subseteq A \times C\). Mamy
			            \begin{equation*}
				            \begin{split}
					            a \in \lproj{\pars{S \composition R}}
					            \iff     & a \in A \land \exists_{c \in C} \pars{a, c} \in S \composition R                                                                  \\
					            \iff     & a \in A \land \exists_{c \in C}\exists_{b \in B} \pars{\pars{a, b} \in R \land \pars{b, c} \in S}                                 \\
					            \implies & a \in A \land \exists_{c \in C}\pars{\pars{\exists_{b \in B} \pars{a, b} \in R} \land \pars{\exists_{b \in B} \pars{b, c} \in S}} \\
					            \implies & a \in A \land \exists_{c \in C}\pars{\exists_{b \in B} \pars{a, b} \in R}                                                         \\
					            \implies & a \in A \land \exists_{b \in B} \pars{a, b} \in R                                                                                 \\
					            \iff     & a \in \lproj{R}
				            \end{split}
			            \end{equation*}
			            Warto zauważyć, że trzy z~powyższych przejść są tylko implikacjami (bo zapominamy na przykład o~pewnych członach konjunkcji) i~nie jest to przypadek. Istotnie, zachodzi tylko inkluzja w~podaną stronę. Aby się o~tym przekonać możemy przyjąć
			            \begin{align*}
				            A & = \set{a}                                \\
				            B & = \set{b}                                \\
				            C & = \set{c}                                \\
				            R & = \set{\pars{a, b}} \subseteq A \times B \\
				            S & = \emptyset \subseteq B \times C
			            \end{align*}
			            Wtedy \(S \composition R = \emptyset\), zatem również \(\lproj{\pars{S \composition R}} = \emptyset\). Natomiast \(\lproj{R} = \set{a}\), więc istotnie zachodzi dowiedziona inkluzja (byłby przypał, gdyby było inaczej), ale nie ma równości.
		            \end{proof}
		      \item \(\rproj{\pars{S \composition R}} \subseteq \rproj{S}\)
		            \begin{proof}
			            Analogiczny jak dla lewej projekcji.
		            \end{proof}
		      \item \(\lproj{\pars{R^{-1}}} = \rproj{R}\)
		            \begin{proof}
			            \begin{equation*}
				            \begin{split}
					            b \in \lproj{\pars{R^{-1}}}
					             & \iff b \in B \land \exists_{a \in A}\pars{b, a} \in R^{-1} \\
					             & \iff b \in B \land \exists_{a \in A}\pars{a, b} \in R      \\
					             & \iff b \in \rproj{R}
				            \end{split}
			            \end{equation*}
		            \end{proof}
	      \end{itemize}
	\item[Druga grupa własności.] Przyjmijmy, że mamy następujące trzy relacje:
	      \begin{align*}
		      R & \subseteq B \times C \\
		      S & \subseteq B \times C \\
		      T & \subseteq A \times B
	      \end{align*}
	      Wtedy
	      \begin{itemize}
		      \item \(\pars{R \cup S}^{-1} = R^{-1} \cup S^{-1}\)
		            \begin{proof}
			            \begin{equation*}
				            \begin{split}
					            \pars{c, b} \in \pars{R \cup S}^{-1}
					            \iff & \pars{b, c} \in R \cup S                           \\
					            \iff & \pars{b, c} \in R \lor \pars{b, c} \in S           \\
					            \iff & \pars{c, b} \in R^{-1} \lor \pars{c, b} \in S^{-1} \\
					            \iff & \pars{c, b} \in R^{-1} \cup S^{-1}
				            \end{split}
			            \end{equation*}
		            \end{proof}
		      \item \(\pars{R \cap S}^{-1} = R^{-1} \cap S^{-1}\)
		            \begin{proof}
			            \begin{equation*}
				            \begin{split}
					            \pars{c, b} \in \pars{R \cap S}^{-1}
					            \iff & \pars{b, c} \in R \cap S                            \\
					            \iff & \pars{b, c} \in R \land \pars{b, c} \in S           \\
					            \iff & \pars{c, b} \in R^{-1} \land \pars{c, b} \in S^{-1} \\
					            \iff & \pars{c, b} \in R^{-1} \cap S^{-1}
				            \end{split}
			            \end{equation*}
		            \end{proof}
		      \item \(\pars{R^{-1}}^{-1} = R\)
		            \begin{proof}
			            Zbyt skomplikowany, zdecydowanie wykracza poza materiał przedmiotu. Dlatego pomijamy.
		            \end{proof}
		      \item \(\pars{R \cup S} \composition T = \pars{R \composition T} \cup \pars{S \composition T}\)\qquad (rozdzielność złożenia względem sumy --- ,,rusot równa się rotusot'')
		            \begin{proof}
			            \begin{equation*}
				            \begin{split}
					            \pars{a, c} \in \pars{R \cup S} \composition T
					            \iff & \exists_{b \in B} \pars{\pars{a, b} \in T \land \pars{b, c} \in R \cup S}                                                       \\
					            \iff & \exists_{b \in B} \pars{\pars{a, b} \in T \land \pars{\pars{b, c} \in R \lor \pars{b, c} \in S}}                                \\
					            \iff & \exists_{b \in B} \pars{\pars{\pars{a, b} \in T \land \pars{b, c} \in R} \lor \pars{\pars{a, b} \in T \land \pars{b, c} \in S}} \\
					            \iff & \pars{\exists_{b \in B} \pars{\pars{a, b} \in T \land \pars{b, c} \in R}}                                                       \\
					                 & \lor \pars{\exists_{b \in B}\pars{\pars{a, b} \in T \land \pars{b, c} \in S}}                                                   \\
					            \iff & \pars{a, c} \in \pars{R \composition T} \lor \pars{a, c} \in \pars{S \composition T}                                            \\
					            \iff & \pars{a, c} \in \pars{R \composition T} \cup \pars{S \composition T}
				            \end{split}
			            \end{equation*}
			            Skorzystaliśmy tutaj z~rozdzielności konjunkcji względem alternatywy oraz rozbiliśmy kwantyfikator egzystencjonalny na alternatywie.
		            \end{proof}
		      \item \(\pars{R \cap S} \composition T \subseteq \pars{R \composition T} \cap \pars{S \composition T}\)
		            \begin{proof}
			            \begin{equation*}
				            \begin{split}
					            \pars{a, c} \in \pars{R \cap S} \composition T
					            \iff     & \exists_{b \in B} \pars{\pars{a, b} \in T \land \pars{b, c} \in R \cap S}                                                        \\
					            \iff     & \exists_{b \in B} \pars{\pars{a, b} \in T \land \pars{\pars{b, c} \in R \land \pars{b, c} \in S}}                                \\
					            \iff     & \exists_{b \in B} \pars{\pars{\pars{a, b} \in T \land \pars{b, c} \in R} \land \pars{\pars{a, b} \in T \land \pars{b, c} \in S}} \\
					            \implies & \pars{\exists_{b \in B} \pars{\pars{a, b} \in T \land \pars{b, c} \in R}}                                                        \\
					                     & \land \pars{\exists_{b \in B}\pars{\pars{a, b} \in T \land \pars{b, c} \in S}}                                                   \\
					            \iff     & \pars{a, c} \in \pars{R \composition T} \land \pars{a, c} \in \pars{S \composition T}                                            \\
					            \iff     & \pars{a, c} \in \pars{R \composition T} \cap \pars{S \composition T}
				            \end{split}
			            \end{equation*}
			            Skorzystaliśmy tutaj z~rozdzielności konjunkcji względem niej samej oraz rozbiliśmy kwantyfikator egzystencjonalny --- tym razem na konjunkcji, więc przejście działa tylko w~jedną stronę.
		            \end{proof}
	      \end{itemize}
\end{description}
