Aby zdefiniować iloczyn kartezjański, będziemy musieli skorzystać z~paru aksjomatów:
\begin{itemize}
    \item aksjomat pary nieuporządkowanej
    \item aksjomat zbioru potęgowego
\end{itemize}
Potrzebujemy też następującej definicji.
\begin{definition}[Para uporządkowana]
\label{mfi:cartesian_and_relations:cartesian_definitions:def:ordered_pair}
\textbf{Parą uporządkowaną} \(\pars{a, b}\) nazywamy zbiór
\begin{equation*}
    \set{\set{a}, \set{a, b}}
\end{equation*}
Intuicyjnie, doubleton podaje elementy pary bez porządku, natomiast singleton ustala, który element jest pierwszy.
\end{definition}
Warto przy tej okazji zastanowić się, gdzie żyją pary uporządkowane, których pierwszy element pochodzi z~\(x\), a~drugi z~\(y\).
\begin{align*}
    a \in x, b \in y &\implies a, b \in x \cup y\\
    \set{a}, \set{a, b} &\in \powerset\pars{x \cup y}\\
    \set{\set{a}, \set{a, b}} &\subseteq \powerset\pars{x \cup y}\\
    \pars{a, b} = \set{\set{a}, \set{a, b}} &\in \powerset\pars{\powerset\pars{x \cup y}}
\end{align*}
To wystarcza nam, aby zdefiniować iloczyn kartezjański.
\begin{definition}[Iloczyn kartezjański]
\textbf{Iloczynem kartezjańskim} zbiorów \(x\), \(y\)~nazywamy zbiór
\begin{equation*}
    x \times y = \set{z \in \powerset\pars{\powerset\pars{x \cup y}} : \exists_{a \in x}\exists_{b \in y} \pars{a, b} = z}
\end{equation*}
Intuicyjnie, ze świata, w~którym żyją pary uporządkowane, wybieramy te elementy, które faktycznie są parami uporządkowanymi (bo mogą tam też być inne ,,śmieci'') oraz na pierwszej pozycji mają coś ze zbioru~\(x\), a~na drugiej coś ze zbioru~\(y\).
\end{definition}
