\subsection{Definiowanie przez schemat rekursji prostej}

\begin{theorem}
Niech dane będą zbiory \( A, Z \) oraz funkcje \( g: A \rightarrow Z, h : Z \times \natural \times A \rightarrow Z \).

Wtedy istnieje dokładnie jedna funkcja \( f : \natural \times A \rightarrow Z \) dla której zachodzi
\begin{enumerate}
    \item \( f(0, a) = g(a) \) -- warunek początkowy (bazowy)
    \item \( f(n', a) = h(f(n, a), n, a) \) -- rekursor
\end{enumerate}
\end{theorem}
\begin{proof}
    Rozważmy zbiór 
    \[
        P = \set{ n \in \natural \mid \exists_{f_n} : f_n : n' \times A \rightarrow Z \text{ która spełnia zadane warunki }}
    \]
    
    Pokażemy, że \( P \) jest induktywny.
    
    \begin{enumerate}
        \item \( 0 \in P \)
            
            \( f_0 : \set{0} \times A \rightarrow Z \) definiujemy jako \( f(0, a) = g(a) \) i nic innego nie możemy zrobić.
            
        \item \( n \in P \implies n' \in P \)
        
            Mając \( f_n : n' \times A \rightarrow Z \) konstruujemy \( f_{n'} : n'' \times A \rightarrow Z \) w następujący sposób:
            \[
                f_{n'}(k, a) = \begin{cases}
                    f_n(k, a) & \text{ gdy } k \leq n \\
                    h(f_n(n, a), n, a) & \text{ gdy } k = n'
                \end{cases}
            \]
            Ponieważ \( f_n \) spełniało oba warunki, a \( f_{n'} \) zdefiniowane jest jak jest to widać że \( f_{n'} \) również je spełnia.
    \end{enumerate}
    
    W takim razie \( P \) jest induktywny, czyli \( P = \natural \). 
    
    Mamy też bardzo fajną własność, a mianowicie \( f_n \subset f_{n'} \) (bo funkcje rozszerzamy przez dorzucenie do nich par (argument, wynik)).
    
    Możemy zatem wziąć \( f = \bigcup P = \bigcup_{n \in \natural} f_n \) otrzymując funkcję zdefiniowaną na całym \( \natural \times A \).
\end{proof}
