\chapter{\texorpdfstring{M\(\varphi\)}{MFI}}

% Ta kolejność jest jakaś losowa, ewidentnie nie ma tu dobrego porządku.
% Dlaczego definiowanie przez indukcję jest przed indukcją
% a domykanie relacji przed iloczynem kartezjańskim
% lol

\section*{Wprowadzenie}
Celem jest znaleźć pierwiastek funkcji na przedziale \( [a, b] \). Metody korzystają z tego: \\
\textbf{Twierdzenie Darboux} \\
Jeśli funkcja \( f: [a, b] \rightarrow \mathbb{R} \) jest ciągła i \( f(a) \leq y \leq f(b) \) lub \( f(b) \leq y \leq f(a) \), to istnieje \( x \in [a, b] \), t.że \( f(x) = y \). \\
\textbf{Twierdzenie Lagrange’a} \\
Jeśli \( f: [a, b] \rightarrow \mathbb{R} \) jest różniczkowalna, to istnieje \( x \in (a, b) \) t.że \( f'(x) = \frac{f(b) - f(a)}{b - a} \). \\
\textbf{Wzór Taylora}
\[
f(x) = f(a) + \frac{f'(a)}{1!}(x - a) + \frac{f''(a)}{2!}(x - a)^2 + \cdots + \frac{f^{(k)}(a)}{k!}(x - a)^k + R_{k+1}(x, a),
\]
gdzie \( R_{k+1} = \frac{f'(\xi)}{(k+1)!}(x - a)^{k+1} \).

\section{Definicje dodawania, mnożenia, potęgowania i~odejmowania w operaciu o~twierdzenie o~definiowaniu przez indukcję. Własności tych działań}
\section{Rozwiązywanie rekurencji liniowych}
\epigraph{I can elaborate: zrobiłam zadanka, zobaczyłam tworzące, stwierdziłam, że chce mi się spać, poszłam sobie}{\textit{Studentka TCSu o zadaniach z funkcji tworzących na kolokwium}}


\subsection{Rozkład na ułamki proste}
To nie jest formalny dowód ani formalna własność ani nic, bardziej schemat postępowania przy rozkładzie na ułamki proste. Sam dowód tego, że rozkład na ułamki proste istnieje, to \textit{sprowadź do wspólnego mianownika i zobacz co Ci wyszło}.
Jeżeli \(deg(P(x)) < deg(Q(x))\) i \(Q(x) = (x-a)^n \cdot (x-b)^k\) to:
\begin{equation*}
	\frac{P(x)}{Q(x)} = \frac{P(x)}{(x-a)^n \cdot (x-b)^k} = \frac{A_1}{x-a} + \frac{A_2}{(x-a)^2} + \dots + \frac{A_n}{(x-a)^n} + \frac{B_1}{x-b} + \frac{B_2}{(x-b)^2} + \dots + \frac{B_k}{(x-b)^k}
\end{equation*}

Oczywiście ten schemat można rozszerzać na więcej śmiesznych rzeczy w mianowniku, ale chyba widać o co chodzi.


\section{Ciąg Fibbonaciego}
\begin{theorem}[Wzór Bineta]
	\begin{equation}
		f_n = \frac{1}{\sqrt{5}} \cdot \left( \left(\frac{1 + \sqrt{5}}{2}\right)^{n} - \left(\frac{1 - \sqrt{5}}{2}\right)^{n} \right)
	\end{equation}
\end{theorem}

\begin{proof}
	Rozpisujemy sobie funkcję tworzącą ciągu \(f_n\):

	\begin{equation*}
		F(x) = f_0 + f_1 \cdot x + f_2 \cdot x^2 + f_3 \cdot x^3 \dots =
	\end{equation*}
	\begin{equation*}
		= f_0 + f_1 \cdot x + (f_0 + f_1) \cdot x^2 + (f_1 + f_2) \cdot x^3 + \dots =
	\end{equation*}
	\begin{equation*}
		= f_0 + f_1 \cdot x + f_0 \cdot x^2 + f_1 \cdot x^2 + f_1 \cdot x^3 + f_2 \cdot x^3 + \dots =
	\end{equation*}
	\begin{equation*}
		= f_0 + f_1 \cdot x + f_0 \cdot x^2 + f_1 \cdot x^3 + \dots + f_1 \cdot x^2 +  f_2 \cdot x^3 + \dots =
	\end{equation*}
	\begin{equation*}
		= f_0 + f_1 \cdot x + x^2 \cdot (f_0 + f_1 \cdot x + \dots) + x \cdot (f_1 \cdot x +  f_2 \cdot x^2 + \dots) =
	\end{equation*}
	\begin{equation*}
		= f_0 + f_1 \cdot x + x^2 \cdot F(x) + x \cdot (F(x) - f_0) =
	\end{equation*}
	\begin{equation*}
		= 0 + 1 \cdot x + x^2 \cdot F(x) + x \cdot (F(x) - 0) =
	\end{equation*}
	\begin{equation*}
		= x + x^2 \cdot F(x) + x \cdot F(x)
	\end{equation*}

	W takim razie mamy, że:
	\begin{equation*}
		F(x) = x + x^2 \cdot F(x) + x \cdot F(x)
	\end{equation*}
	\begin{equation*}
		F(x) -  x^2 \cdot F(x) - x \cdot F(x)  = x
	\end{equation*}
	\begin{equation*}
		F(x) \cdot (1 - x^2 - x) = x
	\end{equation*}
	\begin{equation*}
		F(x) = \frac{x}{-x^2 -x + 1}
	\end{equation*}

	Mianownik możemy rozbić (za pomocą liczenia jakichś delt czy coś):
	\begin{equation*}
		F(x) = \frac{x}{(-1) \cdot \left(x - \left(- \frac{1 + \sqrt{5}}{2}\right)\right) \cdot \left(x - \left(- \frac{1 - \sqrt{5}}{2}\right)\right)}
	\end{equation*}

	Nie no, serio, jeśli ktoś myśli że będę TeXować te przekształcenia to się myli. Powinno wyjść po przekształceniach że:
	\begin{equation*}
		F(x) = \frac{x}{(1-ax) \cdot (1-bx)}
	\end{equation*}
	gdzie \(a = \frac{1 + \sqrt{5}}{2}, b=\frac{1 - \sqrt{5}}{2}\)

	Dalej rozbijamy na ułamki proste:
	\begin{equation*}
		F(x) = \frac{A}{1-ax} + \frac{B}{1-bx}
	\end{equation*}
	\(A\) powinno wyjść \(\frac{1}{\sqrt{5}}\), \(B\) powinno wyjść \(- \frac{1}{\sqrt{5}}\).

	Odwijamy każdą z tych funkcji tworzących z osobna, korzystając ze wzoru podanego we wcześniejszym rozdziale i otrzymujemy wzór.
\end{proof}


\section{Ciąg Catalana}
Liczba Catalana jest to liczba ścieżek długości \(2n\) w kwadracie \(n \times n\) ,,poniżej'' przekątnej (lub na jej poziomie), idących za każdym razem jednostkę do góry lub jednostkę w prawo. Ścieżki takie nazywamy ścieżkami Dycka. Niezwykle formalna definicja. To jest jedna z tych rzeczy, które chyba po prostu trzeba narysować.

\begin{figure}[h]
	\centering
	\includegraphics[scale=0.5]{images/catalan/all_paths_1.png}
  \caption{Ścieżki Dycka długości 2; \(c_1 = 1\)}
\end{figure}

\begin{figure}[h]
	\centering
	\includegraphics[scale=0.5]{images/catalan/all_paths_2.png}
  \caption{Ścieżki Dycka długości 4; \(c_2 = 2\)}
\end{figure}

\begin{figure}[ht]
	\centering
	\includegraphics[scale=0.5]{images/catalan/all_paths_3.png}
  \caption{Ścieżki Dycka długości 6; \(c_3 = 5\)}
\end{figure}


\subsection{Wzór kombinatoryczny}
\begin{theorem}[Wzór kombinatoryczny na liczby Catalana]
	\begin{equation}
		c_n = \frac{1}{n+1} \cdot \binom{2n}{n}
	\end{equation}
\end{theorem}

Mamy sobie nasz kwadrat \(n \times n\). Przekątną możemy opisać tak jakby wzorem \(y = x\) (tak intuicyjnie, bo nie działamy w żadnym układzie współrzędnych, bla bla bla). Robimy sobie teraz prostą \(y = x+1\), idącą jakby ,,o jednostkę wyżej''. Zauważamy, że jeśli jakaś ścieżka przekracza linię naszej przekątnej, to musi ,,dotknąć'' linii \(y = x+1\). \textit{To widać}. Teraz wpadamy na świetny pomysł; jeśli jakaś ścieżka idąca po tym kwadracie ,,spotyka się'' z \(y = x+1\), to od tego momentu odbijamy ją symetrycznie względem \(y = x+1\). Zauważamy, że ścieżka ta (po odbiciu) skończy się w punkcie \((n-1, n+1)\) zamiast w \((n,n)\). Fakt ten dowodzimy stosując dowód przez rysowanie.

\begin{figure}[ht]
	\centering
	\includegraphics[scale=0.5]{images/catalan/path_with_reflection_1.png}
	\includegraphics[scale=0.5]{images/catalan/path_with_reflection_2.png}

	\caption{Przykłady odbicia niepoprawnej ścieżki}
\end{figure}

Zauważamy fascynujący fakt, mianowicie dwie różne ścieżki będą mieć 2 różne odbicia, a więc nasze przekształcenie jest iniektywne. Ponadto, jak sobie zobaczymy jakąkolwiek ścieżkę zaczynającą się w \((0,0)\), ale kończącą się w \((n-1,n+1)\), to jesteśmy w stanie zobaczyć gdzie pierwszy raz przecina się z \(y = x+1\), a następnie ją odbić, otrzymując ścieżkę idącą do \((n,n)\) i niebędącą ścieżką Dycka, której odbicie daje wyjściową ścieżkę. Zatem odbijanie jest suriektywne. A to oznacza tylko jedną rzecz: bijekcję między ścieżkami które ,,nie są catalanowe'', a ścieżkami ,,odbitymi''.

Wszystkich możliwych ścieżek od \((0,0)\) do \((n,n)\) mamy \(\binom{2n}{n}\), bo długość naszej drogi ma \(2n\) i wybieramy sobie \(n\) miejsc gdzie idziemy w prawo. Wszystkich możliwych ścieżek od \((0,0)\) do \((n-1,n+1)\) (czyli tych które są ,,złe'') mamy \(\binom{2n}{n-1}\), bo, analogicznie, ścieżka jest długości \(2n\) ale w prawo idziemy \(n-1\) razy. To prowadzi nas do wyniku:
\begin{equation*}
	\begin{split}
		c_n
		&= \binom{2n}{n} - \binom{2n}{n-1} \\
		&= \frac{(2n)!}{n! \cdot n!} - \frac{(2n)!}{(n-1)! \cdot (n+1)!} \\
		&= \frac{(n+1) \cdot (2n)!}{n! \cdot (n+1)!} - \frac{n \cdot (2n)!}{n! \cdot (n+1)!}\\
		&= \frac{(2n)!}{n! \cdot (n+1)!} \\
		&= \frac{1}{n+1} \cdot \frac {(2n)!}{n! \cdot n!} \\
		&= \frac{1}{n+1} \cdot \binom{2n}{n}
	\end{split}
\end{equation*}
\subsection{Zależność rekurencyjna}

\begin{theorem}[Wzór rekurencyjny na liczby Catalana]
	\begin{equation}
		c_n = c_{0} \cdot c_{n-1} + c_{1} \cdot c_{n - 2} + \dots + c_{n-1} \cdot c_0
	\end{equation}
\end{theorem}

\begin{proof}
	Znowuż mamy kwadrat \(n \times n\), ale tym razem dorysowujemy sobie prostą \(y = x - 1\). Każda ścieżka przetnie kiedyś tę linię i każda ścieżka dotknie kiedyś przekątnej \(y = x\) (można to udowodnić machając i pokazując na rysunek). Rzecz teraz ma się tak, że jeśli po ,,spotkaniu się'' z \(y = x - 1\) idziesz do góry, to potem musisz odbić w prawo (lub w skrajnym przypadku skończyłeś poprawną ścieżkę). Jednocześnie pierwszy wybór kierunku (tzn. ten w punkcie \((0,0)\) zawsze jest ,,w prawo'', bo jeśli ktoś pójdzie ,,do góry'' to znajdzie się w \((0,1)\), powyżej przekątnej \(y = x\)).

	Bierzemy sobie zatem pierwsze miejsce gdzie spotkałeś się z \(y = x\) i zauważamy, że jeśli dane jest ono jakimiś współrzędnymi \((i,i)\) to przecięliśmy \(y = x-1\) w \((i,i-1)\). Ponadto, ścieżka którą szliśmy od punktu \((1,0)\) do \((i,i-1)\) tak naprawdę jest ścieżką Dycka w kwadracie od punktów \((1,0)\), \((i, i-1)\) (kwadrat ten ma długość \(i-1\)). Ależ plot twist! Ścieżka którą idziemy od punktu \((i,i)\) do \((n,n)\) jest zaś już po prostu ścieżką Dycka w kwadracie o długości boku \(n-i\). Ścieżki te są od siebie niezależne i w ogóle, a długości tych ,,kwadratów catalanowych'' sumują się do \(i - 1 + n - i = n - 1\), więc teraz możemy zmajstrować wzór (w zależności od długości boków kwadratów, które z kolei są dyktowane tym kiedy się ,,spotkamy'' z \(y = x\)):
	\begin{equation*}
		c_n = \sum_{i = 0}^{n-1} c_i \cdot c_{n-1-i}
	\end{equation*}
	Co już można odwinąć do postaci która była w twierdzeniu.

	\begin{figure}[H]
		\centering
		\includegraphics[scale=0.5]{images/catalan/recursive_construction_1.png}
		\includegraphics[scale=0.5]{images/catalan/recursive_construction_2.png}

		\caption{Przykłady ,,podzielenia'' poprawnej ścieżki Dycka na podścieżki}
	\end{figure}

\end{proof}


\section{Zliczanie podziałów}

Chcemy pokazać fajny algorytm zliczania wszystkich podziałów liczby \(n\).

Oznaczmy liczbę wszystkich podziałów liczby \(n\) jako \(p(n)\). Jako ,,podziały liczby \(n\)'' mam na myśli liczbę sposobów na podzielenie liczby \(n\) na ileś składników (niezerowych), np. liczbę \(2\) mogę rozłożyć na \(1 + 1\) albo po prostu na \(2\) (i w sumie to tyle).  Funkcja tworząca ciągu \(p_n\) to: \begin{equation*}
	P(x) = (1 + x + x^2 + x^3 + \dots) \cdot (1 + x^2 + x^4 + x^6 + \dots) \cdot (1 + x^3 + x^6 + x^9 + \dots) \dots
\end{equation*}

Pierwszy nawias odpowiada wybraniu jedynki do podziału (i temu ile razy ją bierzemy), drugi dwójki, trzeci trójki, etc.

Oczywiście przy \(x^n\) będziemy mieli \(p_n\), jak to działa w funkcjach tworzących (i mam nadzieję, że widać dlaczego). Zapisujemy \(P(x)\) w fajniejszej postaci:

\begin{equation*}
	P(x) = \frac{1}{1-x} \cdot \frac{1}{1 - x^2} \cdot \frac{1}{1-x^3} \dots
\end{equation*}

Definiuję sobie \(Q(x) = (1-x) \cdot (1-x^2) \cdot (1-x^3) \dots\). Zauważam, że \(P(x) \cdot Q(x) = 1\), czyli \(Q(x)\) jest funkcją odwrotną do \(P(x)\). Okazuje się teraz, że \(Q(x)\) jest funkcją tworzącą pewnego śmiesznego ciągu, który sobie zaraz pokażemy.

Póki co musimy wprowadzić oznaczenia:
\begin{enumerate}
	\item \(e_n\) jest to liczba podziałów liczby \(n\) na parzystą liczbę składników parami różnych,
	\item \(o_n\) jest to liczba podziałów liczby \(n\) na nieparzystą liczbę składników parami różnych.
\end{enumerate}
Jak wszyscy powinniśmy już wiedzieć, funkcja tworząca ciągu \(e_n + o_n\) (czyli po prostu wszystkich podziałów \(n\) ze składnikami parami różnymi) wygląda tak:
\begin{equation*}
	(1+x) \cdot (1 + x^2) \cdot (1+x^3) \dots
\end{equation*}
Ten fakt do niczego nam się w sumie nie przyda, ale może pomóc zrozumieć co zaraz się stanie.

Możemy sobie teraz podumać, jaka jest funkcja tworząca ciągu \(e_n - o_n\). Otóż pojawia się tu plot twist, bo funkcja tworząca tego ciągu to po prostu \(Q(x)\):
\begin{equation*}
	(1-x) \cdot (1-x^2) \cdot (1-x^3) \dots
\end{equation*}

Działa to tak jak w powyższym przykładzie, z tym że jeśli wybraliśmy nieparzyście wiele składników to będzie nieparzyście wiele minusów i się ,,odejmie'' od współczynnika przy \(x^n\), a jeśli będzie parzyście wiele to się ,,doda''. Innymi słowy, do współczynnika przy \(x^n\) doda się 1 za każdy możliwy podział na parzyście wiele parami różnych składników, a odejmie się 1 za każdy możliwy podział na nieparzyście wiele parami różnych składników, czyli to co chcemy. Nie do końca mam pomysł jak to formalnie wytłumaczyć, więc proszę użyć swojej intuicji™.

Po co to wszystko? Okazuje się, że ciąg \(q_n = e_n - o_n\) ma pewne śmieszne własności (które niestety będzie trzeba udowodnić, brace yourselves).

\begin{theorem}[Eulera]
	\begin{equation}
		q_n = \begin{cases}
			0, \hspace{5pt} \mathrm{gdy} \hspace{5pt} n \not = \frac{(3 \cdot k \pm 1) \cdot k }{2} \\
			(-1)^k \hspace{5pt} \mathrm{wpp.}                                                       \\
		\end{cases}
	\end{equation}

\end{theorem}

\begin{proof}
	Zrobimy sobie przekształcenie \(f\), które przesyła prawie (dlaczego prawie to dojdziemy do tego za chwilę) każdy podział na \(n\) składników parami różnych na inny podział na \(n\) składników parami różnych (bijektywnie). Ktoś powie że sobie zrobiłem świetną bijekcję idącą z pewnego zbioru w samego siebie, but hear me out: ta bijekcja będzie mieć tę śmieszną własność, że jeśli podział był na parzyście wiele składników to będzie przesłany na nieparzyście wiele, a jeśli na nieparzyście wiele to będzie przesłany na parzyście wiele składników. To będzie fajne, bo pokażemy sobie że jest ich tyle samo (poza przypadkami gdzie definicja tej funkcji się popsuje, ale o tym za chwilę).

	Generalnie to oznaczmy sobie najmniejszy składnik w podziale \(P\) jako \(a\). Ponadto, zdefiniujmy sobie zbiór \(X\), taki że zawiera on największe składniki podziału \(P\), takie że każde dwa sąsiednie różnią się o jeden. Innymi słowy, jeśli podział \(P = (\lambda_1, \lambda_2, \lambda_3, \dots, \lambda_k)\), to \(X =\{\lambda_1, \lambda_2, \lambda_3, \dots, \lambda_d\}\), gdzie \(d\) jest największą liczbą taką, że kolejne składniki różnią się o 1  (zakładamy, że \(\lambda_1 > \lambda_2 > \dots > \lambda_k\)).

	Teraz jak mamy te zbiory zdefiniowane to możemy robić śmieszne rzeczy. Jeśli \(|X| < a - 1\), to możemy przerobić nasz podział, odejmując od każdego elementu z \(X\) 1, i dorzucając nowy element do podziału, taki że równy jest on moc \(|X|\). Otrzymaliśmy oczywiście poprawny podział (niektórym może pomóc dowód przez rysowanie).

	Dlaczego \(|X| < a - 1\), a nie po prostu \(|X| < a\)? Otóż przychodzi tutaj pewien śmieszny problem, mianowicie może być tak, że składnik podziału o wartości \(a\) ,,wpadł'' do \(X\). W takim przypadku bijekcja nam się kompletnie popsuje i wtedy jej definiujemy (ale jeszcze do tego wrócimy). Natomiast jeśli \(a\) nie należy do \(|X|\) to nasza bijekcja nadal działa. Fajnie.

	Czyli reasumując: jeśli \(|X| < a - 1\) lub (\(|X| = a - 1\) i \(a \not \in X\)) od każdego składnika z \(|X|\) odejmujemy 1 i majstrujemy nowy składnik, który wrzucamy pod składnik o wartości \(a\), który uprzednio był najmniejszy.

	\begin{figure}[H]
		\centering
		\includegraphics{images/case2.png}
		\caption{Wizualizacja przekształcenia (diagram Ferrersa). 2 ,,górne'' składniki różnią się o 1, trzeci już różni się od nich o 2; \(|X| = 2\), \(a=3\).}
	\end{figure}

	Zasadniczo to samo będziemy czynić (ale w drugą stronę), gdy okaże się że \(a < |X| \). Ordynarnie \textit{wywalam} składnik \(a\) i do odpowiedniej liczby elementów z \(X\)  ,,dodaję'' 1, tak by się wyrównało. Należy zauważyć, że być może nie wszystkie elementy z \(X\) będą mieć coś do siebie dodane, ale to mi nic nie psuje. W sumie też fajnie byłoby dodać, że dodaję te jedynki najpierw największym składnikom; inaczej mogłoby to się popsuć.

	Co dzieje się, gdy \(a = |X|\)? Jeśli \(a \in X\) to jest mi smutno, w przeciwnym razie mogę zrobić to samo co robiłem wcześniej i wszystko działa jak powinno.

	\begin{figure}[H]
		\centering
		\includegraphics{images/case_1.png}
		\caption{Wizualizacja przekształcenia (diagram Ferrersa). 3 ,,górne'' składniki różnią się o 1 więc należą do \(X\). \(|X| = 3\), \(a = 2\), więc dwóm największym elementom dodajemy 1, a składnik \(a\) usuwamy.}
	\end{figure}

	Zostają więc 2 przypadki, gdy coś może się popsuć:
	\begin{enumerate}
		\item \(|X| = a - 1\), \(a \in X\)
		      \begin{figure}[H]
			      \centering
			      \includegraphics{images/irytujacy_1.png}
			      \caption{Gdy \(|X| = a - 1\) i składnik \(a\) jest w \(X\); widać, że nic nie możemy z tym zrobić.}
		      \end{figure}

		\item \(|X| = a\), \(a \in x\)
		      \begin{figure}[H]
			      \centering
			      \includegraphics{images/irytujacy_2.png}
			      \caption{Gdy \(|X| = a\) i składnik \(a\) jest w \(X\); również widać, że nasze przekształcenie nie zadziała.}
		      \end{figure}
	\end{enumerate}

	Zauważmy, że sytuacja gdy składnik \(a\) jest w \(X\) jest bardzo dziwną sytuacją generalnie, bo jest to najmniejszy składnik; z definicji \(X\) mamy wtedy, że wszystkie kolejne składniki w \(P\) różnią się o dokładnie 1. Na podstawie tej obserwacji możemy już dokładnie powiedzieć, jakiej postaci musi być \(n\), by miało taki ,,złośliwy'' podział:

	\begin{enumerate}
		\item Gdy \(|X| = a - 1\), \(a \in X\), to \(n\) musi dla jakiegoś \(k\) być postaci \((k + 1) + (k + 2) + \dots + 2k \) (\(|X| =k, a = k+1\), wszystko się zgadza)
		\item Gdy \(|X| = a\), \(a \in x\), to \(n\) musi dla jakiegoś \(k\) być postaci \(k + (k+1) + (k+2) + \dots + (2k - 1)\) (\(|X| = k\), \(a = k\), ponownie wszystko gra)
	\end{enumerate}

	Jak zastosujemy matematykę mniej dyskretną by wysumować te nawiasy, dostaniemy że \(n\) aby miało irytujący podział to musi być postaci \(\frac{k \cdot (3k+1)}{2}\) lub \({k \cdot (3k-1)}{2}\). Jednocześnie nie ma takiego naturalnego \(k\), że wartości te są sobie równe, więc jeśli \(n\) ma irytujący podział, to ma go tylko jednego. Wtedy nie możemy przerzucić tylko jednego podziału na inny (inne są ze sobą w bijekcji) więc \(e_n - o_n = (-1)^k\) (jeśli \(k\) jest parzyste to irytujący podział ma parzyście wiele składników, a w przeciwnym razie nieparzyście wiele). Jeśli irytujący podział nie występuje, \(e_n = o_n\) z bijekcji którą pokazaliśmy. Fajnie.
\end{proof}

Dobra, ale wróćmy do tego cośmy chcieli udowodnić na samym początku. Co w ogóle wynika z tego twierdzenia Eulera? No w sumie to bardzo dużo, bo jak mamy \(q_n = e_n - o_n\) i \(Q(x)\) jest jego funkcją tworzącą:
\begin{equation*}
	Q(x) = q_0 + q_1 \cdot x + q_2 \cdot x^2 + q_3 \cdot x^3 + \dots
\end{equation*}
Ale znamy wartości współczynników \(q_i\) z twierdzenia Eulera:
\begin{equation*}
	Q(x) = 1 - x - x^2 + x^5 + x^7 - x^{12} - x^{15} + x^{22} + x^{26} + \dots
\end{equation*}
Zauważmy, że współczynników które nie są zerowe jest tylko jakoś \(O(\sqrt{n})\), czyli dosyć mało.

Pamiętajmy, że \(P(x) \cdot Q(x) = 1\), czyli że ciąg który wyjdzie po ich wymnożeniu będzie wyglądać tak: \((1,0,0,0, \dots)\) Ponieważ mnożenie w funkcjach tworzących działa jakoś tak, że w wynikowym ciągu (nazwijmy go \(r\)) element \(r_n\) można obliczyć w ten sposób:
\begin{equation*}
	r_n = \sum_{i=0}^{n} p_i \cdot q_{n-i}
\end{equation*}

I wiemy że w naszym przypadku \(r_n = 0\) dla \(n > 1\), to mamy że: \begin{equation*}
	0 = p_n - p_{n-1} - p_{n-2} + p_{n-5} + p_{n-7} - p_{n-12} - \dots
\end{equation*}
To teraz \(p_n\) przerzucamy na drugą stronę i mnożymy stronami razy \(-1\) i mamy wzór na \(p_n\), które możemy obliczyć w \(O(\sqrt{n})\). No i fajnie.



\section{Domykanie relacji ze względu na różne własności. Podaj przykłady własności na które istnieje i~nie istnieje domknięcie}

\section{Zbiory przeliczalne i~ich przykłady}

\section{Konstrukcja Cantora liczb rzeczywistych. Porządek na liczbach rzeczywistych. Twierdzenie o~rozwinięciu liczby rzeczywistej w~szereg}

\section{Iloczyn kartezjański i~jego własności. Pojęcia relacji, złożenia, relacji odwrotnej, własności tych pojęć}
\label{mfi:cartesian_and_relations}
\section{Rozwiązywanie rekurencji liniowych}
\epigraph{I can elaborate: zrobiłam zadanka, zobaczyłam tworzące, stwierdziłam, że chce mi się spać, poszłam sobie}{\textit{Studentka TCSu o zadaniach z funkcji tworzących na kolokwium}}


\subsection{Rozkład na ułamki proste}
To nie jest formalny dowód ani formalna własność ani nic, bardziej schemat postępowania przy rozkładzie na ułamki proste. Sam dowód tego, że rozkład na ułamki proste istnieje, to \textit{sprowadź do wspólnego mianownika i zobacz co Ci wyszło}.
Jeżeli \(deg(P(x)) < deg(Q(x))\) i \(Q(x) = (x-a)^n \cdot (x-b)^k\) to:
\begin{equation*}
	\frac{P(x)}{Q(x)} = \frac{P(x)}{(x-a)^n \cdot (x-b)^k} = \frac{A_1}{x-a} + \frac{A_2}{(x-a)^2} + \dots + \frac{A_n}{(x-a)^n} + \frac{B_1}{x-b} + \frac{B_2}{(x-b)^2} + \dots + \frac{B_k}{(x-b)^k}
\end{equation*}

Oczywiście ten schemat można rozszerzać na więcej śmiesznych rzeczy w mianowniku, ale chyba widać o co chodzi.


\section{Ciąg Fibbonaciego}
\begin{theorem}[Wzór Bineta]
	\begin{equation}
		f_n = \frac{1}{\sqrt{5}} \cdot \left( \left(\frac{1 + \sqrt{5}}{2}\right)^{n} - \left(\frac{1 - \sqrt{5}}{2}\right)^{n} \right)
	\end{equation}
\end{theorem}

\begin{proof}
	Rozpisujemy sobie funkcję tworzącą ciągu \(f_n\):

	\begin{equation*}
		F(x) = f_0 + f_1 \cdot x + f_2 \cdot x^2 + f_3 \cdot x^3 \dots =
	\end{equation*}
	\begin{equation*}
		= f_0 + f_1 \cdot x + (f_0 + f_1) \cdot x^2 + (f_1 + f_2) \cdot x^3 + \dots =
	\end{equation*}
	\begin{equation*}
		= f_0 + f_1 \cdot x + f_0 \cdot x^2 + f_1 \cdot x^2 + f_1 \cdot x^3 + f_2 \cdot x^3 + \dots =
	\end{equation*}
	\begin{equation*}
		= f_0 + f_1 \cdot x + f_0 \cdot x^2 + f_1 \cdot x^3 + \dots + f_1 \cdot x^2 +  f_2 \cdot x^3 + \dots =
	\end{equation*}
	\begin{equation*}
		= f_0 + f_1 \cdot x + x^2 \cdot (f_0 + f_1 \cdot x + \dots) + x \cdot (f_1 \cdot x +  f_2 \cdot x^2 + \dots) =
	\end{equation*}
	\begin{equation*}
		= f_0 + f_1 \cdot x + x^2 \cdot F(x) + x \cdot (F(x) - f_0) =
	\end{equation*}
	\begin{equation*}
		= 0 + 1 \cdot x + x^2 \cdot F(x) + x \cdot (F(x) - 0) =
	\end{equation*}
	\begin{equation*}
		= x + x^2 \cdot F(x) + x \cdot F(x)
	\end{equation*}

	W takim razie mamy, że:
	\begin{equation*}
		F(x) = x + x^2 \cdot F(x) + x \cdot F(x)
	\end{equation*}
	\begin{equation*}
		F(x) -  x^2 \cdot F(x) - x \cdot F(x)  = x
	\end{equation*}
	\begin{equation*}
		F(x) \cdot (1 - x^2 - x) = x
	\end{equation*}
	\begin{equation*}
		F(x) = \frac{x}{-x^2 -x + 1}
	\end{equation*}

	Mianownik możemy rozbić (za pomocą liczenia jakichś delt czy coś):
	\begin{equation*}
		F(x) = \frac{x}{(-1) \cdot \left(x - \left(- \frac{1 + \sqrt{5}}{2}\right)\right) \cdot \left(x - \left(- \frac{1 - \sqrt{5}}{2}\right)\right)}
	\end{equation*}

	Nie no, serio, jeśli ktoś myśli że będę TeXować te przekształcenia to się myli. Powinno wyjść po przekształceniach że:
	\begin{equation*}
		F(x) = \frac{x}{(1-ax) \cdot (1-bx)}
	\end{equation*}
	gdzie \(a = \frac{1 + \sqrt{5}}{2}, b=\frac{1 - \sqrt{5}}{2}\)

	Dalej rozbijamy na ułamki proste:
	\begin{equation*}
		F(x) = \frac{A}{1-ax} + \frac{B}{1-bx}
	\end{equation*}
	\(A\) powinno wyjść \(\frac{1}{\sqrt{5}}\), \(B\) powinno wyjść \(- \frac{1}{\sqrt{5}}\).

	Odwijamy każdą z tych funkcji tworzących z osobna, korzystając ze wzoru podanego we wcześniejszym rozdziale i otrzymujemy wzór.
\end{proof}


\section{Ciąg Catalana}
Liczba Catalana jest to liczba ścieżek długości \(2n\) w kwadracie \(n \times n\) ,,poniżej'' przekątnej (lub na jej poziomie), idących za każdym razem jednostkę do góry lub jednostkę w prawo. Ścieżki takie nazywamy ścieżkami Dycka. Niezwykle formalna definicja. To jest jedna z tych rzeczy, które chyba po prostu trzeba narysować.

\begin{figure}[h]
	\centering
	\includegraphics[scale=0.5]{images/catalan/all_paths_1.png}
  \caption{Ścieżki Dycka długości 2; \(c_1 = 1\)}
\end{figure}

\begin{figure}[h]
	\centering
	\includegraphics[scale=0.5]{images/catalan/all_paths_2.png}
  \caption{Ścieżki Dycka długości 4; \(c_2 = 2\)}
\end{figure}

\begin{figure}[ht]
	\centering
	\includegraphics[scale=0.5]{images/catalan/all_paths_3.png}
  \caption{Ścieżki Dycka długości 6; \(c_3 = 5\)}
\end{figure}


\subsection{Wzór kombinatoryczny}
\begin{theorem}[Wzór kombinatoryczny na liczby Catalana]
	\begin{equation}
		c_n = \frac{1}{n+1} \cdot \binom{2n}{n}
	\end{equation}
\end{theorem}

Mamy sobie nasz kwadrat \(n \times n\). Przekątną możemy opisać tak jakby wzorem \(y = x\) (tak intuicyjnie, bo nie działamy w żadnym układzie współrzędnych, bla bla bla). Robimy sobie teraz prostą \(y = x+1\), idącą jakby ,,o jednostkę wyżej''. Zauważamy, że jeśli jakaś ścieżka przekracza linię naszej przekątnej, to musi ,,dotknąć'' linii \(y = x+1\). \textit{To widać}. Teraz wpadamy na świetny pomysł; jeśli jakaś ścieżka idąca po tym kwadracie ,,spotyka się'' z \(y = x+1\), to od tego momentu odbijamy ją symetrycznie względem \(y = x+1\). Zauważamy, że ścieżka ta (po odbiciu) skończy się w punkcie \((n-1, n+1)\) zamiast w \((n,n)\). Fakt ten dowodzimy stosując dowód przez rysowanie.

\begin{figure}[ht]
	\centering
	\includegraphics[scale=0.5]{images/catalan/path_with_reflection_1.png}
	\includegraphics[scale=0.5]{images/catalan/path_with_reflection_2.png}

	\caption{Przykłady odbicia niepoprawnej ścieżki}
\end{figure}

Zauważamy fascynujący fakt, mianowicie dwie różne ścieżki będą mieć 2 różne odbicia, a więc nasze przekształcenie jest iniektywne. Ponadto, jak sobie zobaczymy jakąkolwiek ścieżkę zaczynającą się w \((0,0)\), ale kończącą się w \((n-1,n+1)\), to jesteśmy w stanie zobaczyć gdzie pierwszy raz przecina się z \(y = x+1\), a następnie ją odbić, otrzymując ścieżkę idącą do \((n,n)\) i niebędącą ścieżką Dycka, której odbicie daje wyjściową ścieżkę. Zatem odbijanie jest suriektywne. A to oznacza tylko jedną rzecz: bijekcję między ścieżkami które ,,nie są catalanowe'', a ścieżkami ,,odbitymi''.

Wszystkich możliwych ścieżek od \((0,0)\) do \((n,n)\) mamy \(\binom{2n}{n}\), bo długość naszej drogi ma \(2n\) i wybieramy sobie \(n\) miejsc gdzie idziemy w prawo. Wszystkich możliwych ścieżek od \((0,0)\) do \((n-1,n+1)\) (czyli tych które są ,,złe'') mamy \(\binom{2n}{n-1}\), bo, analogicznie, ścieżka jest długości \(2n\) ale w prawo idziemy \(n-1\) razy. To prowadzi nas do wyniku:
\begin{equation*}
	\begin{split}
		c_n
		&= \binom{2n}{n} - \binom{2n}{n-1} \\
		&= \frac{(2n)!}{n! \cdot n!} - \frac{(2n)!}{(n-1)! \cdot (n+1)!} \\
		&= \frac{(n+1) \cdot (2n)!}{n! \cdot (n+1)!} - \frac{n \cdot (2n)!}{n! \cdot (n+1)!}\\
		&= \frac{(2n)!}{n! \cdot (n+1)!} \\
		&= \frac{1}{n+1} \cdot \frac {(2n)!}{n! \cdot n!} \\
		&= \frac{1}{n+1} \cdot \binom{2n}{n}
	\end{split}
\end{equation*}
\subsection{Zależność rekurencyjna}

\begin{theorem}[Wzór rekurencyjny na liczby Catalana]
	\begin{equation}
		c_n = c_{0} \cdot c_{n-1} + c_{1} \cdot c_{n - 2} + \dots + c_{n-1} \cdot c_0
	\end{equation}
\end{theorem}

\begin{proof}
	Znowuż mamy kwadrat \(n \times n\), ale tym razem dorysowujemy sobie prostą \(y = x - 1\). Każda ścieżka przetnie kiedyś tę linię i każda ścieżka dotknie kiedyś przekątnej \(y = x\) (można to udowodnić machając i pokazując na rysunek). Rzecz teraz ma się tak, że jeśli po ,,spotkaniu się'' z \(y = x - 1\) idziesz do góry, to potem musisz odbić w prawo (lub w skrajnym przypadku skończyłeś poprawną ścieżkę). Jednocześnie pierwszy wybór kierunku (tzn. ten w punkcie \((0,0)\) zawsze jest ,,w prawo'', bo jeśli ktoś pójdzie ,,do góry'' to znajdzie się w \((0,1)\), powyżej przekątnej \(y = x\)).

	Bierzemy sobie zatem pierwsze miejsce gdzie spotkałeś się z \(y = x\) i zauważamy, że jeśli dane jest ono jakimiś współrzędnymi \((i,i)\) to przecięliśmy \(y = x-1\) w \((i,i-1)\). Ponadto, ścieżka którą szliśmy od punktu \((1,0)\) do \((i,i-1)\) tak naprawdę jest ścieżką Dycka w kwadracie od punktów \((1,0)\), \((i, i-1)\) (kwadrat ten ma długość \(i-1\)). Ależ plot twist! Ścieżka którą idziemy od punktu \((i,i)\) do \((n,n)\) jest zaś już po prostu ścieżką Dycka w kwadracie o długości boku \(n-i\). Ścieżki te są od siebie niezależne i w ogóle, a długości tych ,,kwadratów catalanowych'' sumują się do \(i - 1 + n - i = n - 1\), więc teraz możemy zmajstrować wzór (w zależności od długości boków kwadratów, które z kolei są dyktowane tym kiedy się ,,spotkamy'' z \(y = x\)):
	\begin{equation*}
		c_n = \sum_{i = 0}^{n-1} c_i \cdot c_{n-1-i}
	\end{equation*}
	Co już można odwinąć do postaci która była w twierdzeniu.

	\begin{figure}[H]
		\centering
		\includegraphics[scale=0.5]{images/catalan/recursive_construction_1.png}
		\includegraphics[scale=0.5]{images/catalan/recursive_construction_2.png}

		\caption{Przykłady ,,podzielenia'' poprawnej ścieżki Dycka na podścieżki}
	\end{figure}

\end{proof}


\section{Zliczanie podziałów}

Chcemy pokazać fajny algorytm zliczania wszystkich podziałów liczby \(n\).

Oznaczmy liczbę wszystkich podziałów liczby \(n\) jako \(p(n)\). Jako ,,podziały liczby \(n\)'' mam na myśli liczbę sposobów na podzielenie liczby \(n\) na ileś składników (niezerowych), np. liczbę \(2\) mogę rozłożyć na \(1 + 1\) albo po prostu na \(2\) (i w sumie to tyle).  Funkcja tworząca ciągu \(p_n\) to: \begin{equation*}
	P(x) = (1 + x + x^2 + x^3 + \dots) \cdot (1 + x^2 + x^4 + x^6 + \dots) \cdot (1 + x^3 + x^6 + x^9 + \dots) \dots
\end{equation*}

Pierwszy nawias odpowiada wybraniu jedynki do podziału (i temu ile razy ją bierzemy), drugi dwójki, trzeci trójki, etc.

Oczywiście przy \(x^n\) będziemy mieli \(p_n\), jak to działa w funkcjach tworzących (i mam nadzieję, że widać dlaczego). Zapisujemy \(P(x)\) w fajniejszej postaci:

\begin{equation*}
	P(x) = \frac{1}{1-x} \cdot \frac{1}{1 - x^2} \cdot \frac{1}{1-x^3} \dots
\end{equation*}

Definiuję sobie \(Q(x) = (1-x) \cdot (1-x^2) \cdot (1-x^3) \dots\). Zauważam, że \(P(x) \cdot Q(x) = 1\), czyli \(Q(x)\) jest funkcją odwrotną do \(P(x)\). Okazuje się teraz, że \(Q(x)\) jest funkcją tworzącą pewnego śmiesznego ciągu, który sobie zaraz pokażemy.

Póki co musimy wprowadzić oznaczenia:
\begin{enumerate}
	\item \(e_n\) jest to liczba podziałów liczby \(n\) na parzystą liczbę składników parami różnych,
	\item \(o_n\) jest to liczba podziałów liczby \(n\) na nieparzystą liczbę składników parami różnych.
\end{enumerate}
Jak wszyscy powinniśmy już wiedzieć, funkcja tworząca ciągu \(e_n + o_n\) (czyli po prostu wszystkich podziałów \(n\) ze składnikami parami różnymi) wygląda tak:
\begin{equation*}
	(1+x) \cdot (1 + x^2) \cdot (1+x^3) \dots
\end{equation*}
Ten fakt do niczego nam się w sumie nie przyda, ale może pomóc zrozumieć co zaraz się stanie.

Możemy sobie teraz podumać, jaka jest funkcja tworząca ciągu \(e_n - o_n\). Otóż pojawia się tu plot twist, bo funkcja tworząca tego ciągu to po prostu \(Q(x)\):
\begin{equation*}
	(1-x) \cdot (1-x^2) \cdot (1-x^3) \dots
\end{equation*}

Działa to tak jak w powyższym przykładzie, z tym że jeśli wybraliśmy nieparzyście wiele składników to będzie nieparzyście wiele minusów i się ,,odejmie'' od współczynnika przy \(x^n\), a jeśli będzie parzyście wiele to się ,,doda''. Innymi słowy, do współczynnika przy \(x^n\) doda się 1 za każdy możliwy podział na parzyście wiele parami różnych składników, a odejmie się 1 za każdy możliwy podział na nieparzyście wiele parami różnych składników, czyli to co chcemy. Nie do końca mam pomysł jak to formalnie wytłumaczyć, więc proszę użyć swojej intuicji™.

Po co to wszystko? Okazuje się, że ciąg \(q_n = e_n - o_n\) ma pewne śmieszne własności (które niestety będzie trzeba udowodnić, brace yourselves).

\begin{theorem}[Eulera]
	\begin{equation}
		q_n = \begin{cases}
			0, \hspace{5pt} \mathrm{gdy} \hspace{5pt} n \not = \frac{(3 \cdot k \pm 1) \cdot k }{2} \\
			(-1)^k \hspace{5pt} \mathrm{wpp.}                                                       \\
		\end{cases}
	\end{equation}

\end{theorem}

\begin{proof}
	Zrobimy sobie przekształcenie \(f\), które przesyła prawie (dlaczego prawie to dojdziemy do tego za chwilę) każdy podział na \(n\) składników parami różnych na inny podział na \(n\) składników parami różnych (bijektywnie). Ktoś powie że sobie zrobiłem świetną bijekcję idącą z pewnego zbioru w samego siebie, but hear me out: ta bijekcja będzie mieć tę śmieszną własność, że jeśli podział był na parzyście wiele składników to będzie przesłany na nieparzyście wiele, a jeśli na nieparzyście wiele to będzie przesłany na parzyście wiele składników. To będzie fajne, bo pokażemy sobie że jest ich tyle samo (poza przypadkami gdzie definicja tej funkcji się popsuje, ale o tym za chwilę).

	Generalnie to oznaczmy sobie najmniejszy składnik w podziale \(P\) jako \(a\). Ponadto, zdefiniujmy sobie zbiór \(X\), taki że zawiera on największe składniki podziału \(P\), takie że każde dwa sąsiednie różnią się o jeden. Innymi słowy, jeśli podział \(P = (\lambda_1, \lambda_2, \lambda_3, \dots, \lambda_k)\), to \(X =\{\lambda_1, \lambda_2, \lambda_3, \dots, \lambda_d\}\), gdzie \(d\) jest największą liczbą taką, że kolejne składniki różnią się o 1  (zakładamy, że \(\lambda_1 > \lambda_2 > \dots > \lambda_k\)).

	Teraz jak mamy te zbiory zdefiniowane to możemy robić śmieszne rzeczy. Jeśli \(|X| < a - 1\), to możemy przerobić nasz podział, odejmując od każdego elementu z \(X\) 1, i dorzucając nowy element do podziału, taki że równy jest on moc \(|X|\). Otrzymaliśmy oczywiście poprawny podział (niektórym może pomóc dowód przez rysowanie).

	Dlaczego \(|X| < a - 1\), a nie po prostu \(|X| < a\)? Otóż przychodzi tutaj pewien śmieszny problem, mianowicie może być tak, że składnik podziału o wartości \(a\) ,,wpadł'' do \(X\). W takim przypadku bijekcja nam się kompletnie popsuje i wtedy jej definiujemy (ale jeszcze do tego wrócimy). Natomiast jeśli \(a\) nie należy do \(|X|\) to nasza bijekcja nadal działa. Fajnie.

	Czyli reasumując: jeśli \(|X| < a - 1\) lub (\(|X| = a - 1\) i \(a \not \in X\)) od każdego składnika z \(|X|\) odejmujemy 1 i majstrujemy nowy składnik, który wrzucamy pod składnik o wartości \(a\), który uprzednio był najmniejszy.

	\begin{figure}[H]
		\centering
		\includegraphics{images/case2.png}
		\caption{Wizualizacja przekształcenia (diagram Ferrersa). 2 ,,górne'' składniki różnią się o 1, trzeci już różni się od nich o 2; \(|X| = 2\), \(a=3\).}
	\end{figure}

	Zasadniczo to samo będziemy czynić (ale w drugą stronę), gdy okaże się że \(a < |X| \). Ordynarnie \textit{wywalam} składnik \(a\) i do odpowiedniej liczby elementów z \(X\)  ,,dodaję'' 1, tak by się wyrównało. Należy zauważyć, że być może nie wszystkie elementy z \(X\) będą mieć coś do siebie dodane, ale to mi nic nie psuje. W sumie też fajnie byłoby dodać, że dodaję te jedynki najpierw największym składnikom; inaczej mogłoby to się popsuć.

	Co dzieje się, gdy \(a = |X|\)? Jeśli \(a \in X\) to jest mi smutno, w przeciwnym razie mogę zrobić to samo co robiłem wcześniej i wszystko działa jak powinno.

	\begin{figure}[H]
		\centering
		\includegraphics{images/case_1.png}
		\caption{Wizualizacja przekształcenia (diagram Ferrersa). 3 ,,górne'' składniki różnią się o 1 więc należą do \(X\). \(|X| = 3\), \(a = 2\), więc dwóm największym elementom dodajemy 1, a składnik \(a\) usuwamy.}
	\end{figure}

	Zostają więc 2 przypadki, gdy coś może się popsuć:
	\begin{enumerate}
		\item \(|X| = a - 1\), \(a \in X\)
		      \begin{figure}[H]
			      \centering
			      \includegraphics{images/irytujacy_1.png}
			      \caption{Gdy \(|X| = a - 1\) i składnik \(a\) jest w \(X\); widać, że nic nie możemy z tym zrobić.}
		      \end{figure}

		\item \(|X| = a\), \(a \in x\)
		      \begin{figure}[H]
			      \centering
			      \includegraphics{images/irytujacy_2.png}
			      \caption{Gdy \(|X| = a\) i składnik \(a\) jest w \(X\); również widać, że nasze przekształcenie nie zadziała.}
		      \end{figure}
	\end{enumerate}

	Zauważmy, że sytuacja gdy składnik \(a\) jest w \(X\) jest bardzo dziwną sytuacją generalnie, bo jest to najmniejszy składnik; z definicji \(X\) mamy wtedy, że wszystkie kolejne składniki w \(P\) różnią się o dokładnie 1. Na podstawie tej obserwacji możemy już dokładnie powiedzieć, jakiej postaci musi być \(n\), by miało taki ,,złośliwy'' podział:

	\begin{enumerate}
		\item Gdy \(|X| = a - 1\), \(a \in X\), to \(n\) musi dla jakiegoś \(k\) być postaci \((k + 1) + (k + 2) + \dots + 2k \) (\(|X| =k, a = k+1\), wszystko się zgadza)
		\item Gdy \(|X| = a\), \(a \in x\), to \(n\) musi dla jakiegoś \(k\) być postaci \(k + (k+1) + (k+2) + \dots + (2k - 1)\) (\(|X| = k\), \(a = k\), ponownie wszystko gra)
	\end{enumerate}

	Jak zastosujemy matematykę mniej dyskretną by wysumować te nawiasy, dostaniemy że \(n\) aby miało irytujący podział to musi być postaci \(\frac{k \cdot (3k+1)}{2}\) lub \({k \cdot (3k-1)}{2}\). Jednocześnie nie ma takiego naturalnego \(k\), że wartości te są sobie równe, więc jeśli \(n\) ma irytujący podział, to ma go tylko jednego. Wtedy nie możemy przerzucić tylko jednego podziału na inny (inne są ze sobą w bijekcji) więc \(e_n - o_n = (-1)^k\) (jeśli \(k\) jest parzyste to irytujący podział ma parzyście wiele składników, a w przeciwnym razie nieparzyście wiele). Jeśli irytujący podział nie występuje, \(e_n = o_n\) z bijekcji którą pokazaliśmy. Fajnie.
\end{proof}

Dobra, ale wróćmy do tego cośmy chcieli udowodnić na samym początku. Co w ogóle wynika z tego twierdzenia Eulera? No w sumie to bardzo dużo, bo jak mamy \(q_n = e_n - o_n\) i \(Q(x)\) jest jego funkcją tworzącą:
\begin{equation*}
	Q(x) = q_0 + q_1 \cdot x + q_2 \cdot x^2 + q_3 \cdot x^3 + \dots
\end{equation*}
Ale znamy wartości współczynników \(q_i\) z twierdzenia Eulera:
\begin{equation*}
	Q(x) = 1 - x - x^2 + x^5 + x^7 - x^{12} - x^{15} + x^{22} + x^{26} + \dots
\end{equation*}
Zauważmy, że współczynników które nie są zerowe jest tylko jakoś \(O(\sqrt{n})\), czyli dosyć mało.

Pamiętajmy, że \(P(x) \cdot Q(x) = 1\), czyli że ciąg który wyjdzie po ich wymnożeniu będzie wyglądać tak: \((1,0,0,0, \dots)\) Ponieważ mnożenie w funkcjach tworzących działa jakoś tak, że w wynikowym ciągu (nazwijmy go \(r\)) element \(r_n\) można obliczyć w ten sposób:
\begin{equation*}
	r_n = \sum_{i=0}^{n} p_i \cdot q_{n-i}
\end{equation*}

I wiemy że w naszym przypadku \(r_n = 0\) dla \(n > 1\), to mamy że: \begin{equation*}
	0 = p_n - p_{n-1} - p_{n-2} + p_{n-5} + p_{n-7} - p_{n-12} - \dots
\end{equation*}
To teraz \(p_n\) przerzucamy na drugą stronę i mnożymy stronami razy \(-1\) i mamy wzór na \(p_n\), które możemy obliczyć w \(O(\sqrt{n})\). No i fajnie.



\section{Konstrukcja liczb naturalnych von Neumanna, twierdzenie o~indukcji. Własności liczb naturalnych}
\label{mfi:nat_and_induction}
\section{Rozwiązywanie rekurencji liniowych}
\epigraph{I can elaborate: zrobiłam zadanka, zobaczyłam tworzące, stwierdziłam, że chce mi się spać, poszłam sobie}{\textit{Studentka TCSu o zadaniach z funkcji tworzących na kolokwium}}


\subsection{Rozkład na ułamki proste}
To nie jest formalny dowód ani formalna własność ani nic, bardziej schemat postępowania przy rozkładzie na ułamki proste. Sam dowód tego, że rozkład na ułamki proste istnieje, to \textit{sprowadź do wspólnego mianownika i zobacz co Ci wyszło}.
Jeżeli \(deg(P(x)) < deg(Q(x))\) i \(Q(x) = (x-a)^n \cdot (x-b)^k\) to:
\begin{equation*}
	\frac{P(x)}{Q(x)} = \frac{P(x)}{(x-a)^n \cdot (x-b)^k} = \frac{A_1}{x-a} + \frac{A_2}{(x-a)^2} + \dots + \frac{A_n}{(x-a)^n} + \frac{B_1}{x-b} + \frac{B_2}{(x-b)^2} + \dots + \frac{B_k}{(x-b)^k}
\end{equation*}

Oczywiście ten schemat można rozszerzać na więcej śmiesznych rzeczy w mianowniku, ale chyba widać o co chodzi.


\section{Ciąg Fibbonaciego}
\begin{theorem}[Wzór Bineta]
	\begin{equation}
		f_n = \frac{1}{\sqrt{5}} \cdot \left( \left(\frac{1 + \sqrt{5}}{2}\right)^{n} - \left(\frac{1 - \sqrt{5}}{2}\right)^{n} \right)
	\end{equation}
\end{theorem}

\begin{proof}
	Rozpisujemy sobie funkcję tworzącą ciągu \(f_n\):

	\begin{equation*}
		F(x) = f_0 + f_1 \cdot x + f_2 \cdot x^2 + f_3 \cdot x^3 \dots =
	\end{equation*}
	\begin{equation*}
		= f_0 + f_1 \cdot x + (f_0 + f_1) \cdot x^2 + (f_1 + f_2) \cdot x^3 + \dots =
	\end{equation*}
	\begin{equation*}
		= f_0 + f_1 \cdot x + f_0 \cdot x^2 + f_1 \cdot x^2 + f_1 \cdot x^3 + f_2 \cdot x^3 + \dots =
	\end{equation*}
	\begin{equation*}
		= f_0 + f_1 \cdot x + f_0 \cdot x^2 + f_1 \cdot x^3 + \dots + f_1 \cdot x^2 +  f_2 \cdot x^3 + \dots =
	\end{equation*}
	\begin{equation*}
		= f_0 + f_1 \cdot x + x^2 \cdot (f_0 + f_1 \cdot x + \dots) + x \cdot (f_1 \cdot x +  f_2 \cdot x^2 + \dots) =
	\end{equation*}
	\begin{equation*}
		= f_0 + f_1 \cdot x + x^2 \cdot F(x) + x \cdot (F(x) - f_0) =
	\end{equation*}
	\begin{equation*}
		= 0 + 1 \cdot x + x^2 \cdot F(x) + x \cdot (F(x) - 0) =
	\end{equation*}
	\begin{equation*}
		= x + x^2 \cdot F(x) + x \cdot F(x)
	\end{equation*}

	W takim razie mamy, że:
	\begin{equation*}
		F(x) = x + x^2 \cdot F(x) + x \cdot F(x)
	\end{equation*}
	\begin{equation*}
		F(x) -  x^2 \cdot F(x) - x \cdot F(x)  = x
	\end{equation*}
	\begin{equation*}
		F(x) \cdot (1 - x^2 - x) = x
	\end{equation*}
	\begin{equation*}
		F(x) = \frac{x}{-x^2 -x + 1}
	\end{equation*}

	Mianownik możemy rozbić (za pomocą liczenia jakichś delt czy coś):
	\begin{equation*}
		F(x) = \frac{x}{(-1) \cdot \left(x - \left(- \frac{1 + \sqrt{5}}{2}\right)\right) \cdot \left(x - \left(- \frac{1 - \sqrt{5}}{2}\right)\right)}
	\end{equation*}

	Nie no, serio, jeśli ktoś myśli że będę TeXować te przekształcenia to się myli. Powinno wyjść po przekształceniach że:
	\begin{equation*}
		F(x) = \frac{x}{(1-ax) \cdot (1-bx)}
	\end{equation*}
	gdzie \(a = \frac{1 + \sqrt{5}}{2}, b=\frac{1 - \sqrt{5}}{2}\)

	Dalej rozbijamy na ułamki proste:
	\begin{equation*}
		F(x) = \frac{A}{1-ax} + \frac{B}{1-bx}
	\end{equation*}
	\(A\) powinno wyjść \(\frac{1}{\sqrt{5}}\), \(B\) powinno wyjść \(- \frac{1}{\sqrt{5}}\).

	Odwijamy każdą z tych funkcji tworzących z osobna, korzystając ze wzoru podanego we wcześniejszym rozdziale i otrzymujemy wzór.
\end{proof}


\section{Ciąg Catalana}
Liczba Catalana jest to liczba ścieżek długości \(2n\) w kwadracie \(n \times n\) ,,poniżej'' przekątnej (lub na jej poziomie), idących za każdym razem jednostkę do góry lub jednostkę w prawo. Ścieżki takie nazywamy ścieżkami Dycka. Niezwykle formalna definicja. To jest jedna z tych rzeczy, które chyba po prostu trzeba narysować.

\begin{figure}[h]
	\centering
	\includegraphics[scale=0.5]{images/catalan/all_paths_1.png}
  \caption{Ścieżki Dycka długości 2; \(c_1 = 1\)}
\end{figure}

\begin{figure}[h]
	\centering
	\includegraphics[scale=0.5]{images/catalan/all_paths_2.png}
  \caption{Ścieżki Dycka długości 4; \(c_2 = 2\)}
\end{figure}

\begin{figure}[ht]
	\centering
	\includegraphics[scale=0.5]{images/catalan/all_paths_3.png}
  \caption{Ścieżki Dycka długości 6; \(c_3 = 5\)}
\end{figure}


\subsection{Wzór kombinatoryczny}
\begin{theorem}[Wzór kombinatoryczny na liczby Catalana]
	\begin{equation}
		c_n = \frac{1}{n+1} \cdot \binom{2n}{n}
	\end{equation}
\end{theorem}

Mamy sobie nasz kwadrat \(n \times n\). Przekątną możemy opisać tak jakby wzorem \(y = x\) (tak intuicyjnie, bo nie działamy w żadnym układzie współrzędnych, bla bla bla). Robimy sobie teraz prostą \(y = x+1\), idącą jakby ,,o jednostkę wyżej''. Zauważamy, że jeśli jakaś ścieżka przekracza linię naszej przekątnej, to musi ,,dotknąć'' linii \(y = x+1\). \textit{To widać}. Teraz wpadamy na świetny pomysł; jeśli jakaś ścieżka idąca po tym kwadracie ,,spotyka się'' z \(y = x+1\), to od tego momentu odbijamy ją symetrycznie względem \(y = x+1\). Zauważamy, że ścieżka ta (po odbiciu) skończy się w punkcie \((n-1, n+1)\) zamiast w \((n,n)\). Fakt ten dowodzimy stosując dowód przez rysowanie.

\begin{figure}[ht]
	\centering
	\includegraphics[scale=0.5]{images/catalan/path_with_reflection_1.png}
	\includegraphics[scale=0.5]{images/catalan/path_with_reflection_2.png}

	\caption{Przykłady odbicia niepoprawnej ścieżki}
\end{figure}

Zauważamy fascynujący fakt, mianowicie dwie różne ścieżki będą mieć 2 różne odbicia, a więc nasze przekształcenie jest iniektywne. Ponadto, jak sobie zobaczymy jakąkolwiek ścieżkę zaczynającą się w \((0,0)\), ale kończącą się w \((n-1,n+1)\), to jesteśmy w stanie zobaczyć gdzie pierwszy raz przecina się z \(y = x+1\), a następnie ją odbić, otrzymując ścieżkę idącą do \((n,n)\) i niebędącą ścieżką Dycka, której odbicie daje wyjściową ścieżkę. Zatem odbijanie jest suriektywne. A to oznacza tylko jedną rzecz: bijekcję między ścieżkami które ,,nie są catalanowe'', a ścieżkami ,,odbitymi''.

Wszystkich możliwych ścieżek od \((0,0)\) do \((n,n)\) mamy \(\binom{2n}{n}\), bo długość naszej drogi ma \(2n\) i wybieramy sobie \(n\) miejsc gdzie idziemy w prawo. Wszystkich możliwych ścieżek od \((0,0)\) do \((n-1,n+1)\) (czyli tych które są ,,złe'') mamy \(\binom{2n}{n-1}\), bo, analogicznie, ścieżka jest długości \(2n\) ale w prawo idziemy \(n-1\) razy. To prowadzi nas do wyniku:
\begin{equation*}
	\begin{split}
		c_n
		&= \binom{2n}{n} - \binom{2n}{n-1} \\
		&= \frac{(2n)!}{n! \cdot n!} - \frac{(2n)!}{(n-1)! \cdot (n+1)!} \\
		&= \frac{(n+1) \cdot (2n)!}{n! \cdot (n+1)!} - \frac{n \cdot (2n)!}{n! \cdot (n+1)!}\\
		&= \frac{(2n)!}{n! \cdot (n+1)!} \\
		&= \frac{1}{n+1} \cdot \frac {(2n)!}{n! \cdot n!} \\
		&= \frac{1}{n+1} \cdot \binom{2n}{n}
	\end{split}
\end{equation*}
\subsection{Zależność rekurencyjna}

\begin{theorem}[Wzór rekurencyjny na liczby Catalana]
	\begin{equation}
		c_n = c_{0} \cdot c_{n-1} + c_{1} \cdot c_{n - 2} + \dots + c_{n-1} \cdot c_0
	\end{equation}
\end{theorem}

\begin{proof}
	Znowuż mamy kwadrat \(n \times n\), ale tym razem dorysowujemy sobie prostą \(y = x - 1\). Każda ścieżka przetnie kiedyś tę linię i każda ścieżka dotknie kiedyś przekątnej \(y = x\) (można to udowodnić machając i pokazując na rysunek). Rzecz teraz ma się tak, że jeśli po ,,spotkaniu się'' z \(y = x - 1\) idziesz do góry, to potem musisz odbić w prawo (lub w skrajnym przypadku skończyłeś poprawną ścieżkę). Jednocześnie pierwszy wybór kierunku (tzn. ten w punkcie \((0,0)\) zawsze jest ,,w prawo'', bo jeśli ktoś pójdzie ,,do góry'' to znajdzie się w \((0,1)\), powyżej przekątnej \(y = x\)).

	Bierzemy sobie zatem pierwsze miejsce gdzie spotkałeś się z \(y = x\) i zauważamy, że jeśli dane jest ono jakimiś współrzędnymi \((i,i)\) to przecięliśmy \(y = x-1\) w \((i,i-1)\). Ponadto, ścieżka którą szliśmy od punktu \((1,0)\) do \((i,i-1)\) tak naprawdę jest ścieżką Dycka w kwadracie od punktów \((1,0)\), \((i, i-1)\) (kwadrat ten ma długość \(i-1\)). Ależ plot twist! Ścieżka którą idziemy od punktu \((i,i)\) do \((n,n)\) jest zaś już po prostu ścieżką Dycka w kwadracie o długości boku \(n-i\). Ścieżki te są od siebie niezależne i w ogóle, a długości tych ,,kwadratów catalanowych'' sumują się do \(i - 1 + n - i = n - 1\), więc teraz możemy zmajstrować wzór (w zależności od długości boków kwadratów, które z kolei są dyktowane tym kiedy się ,,spotkamy'' z \(y = x\)):
	\begin{equation*}
		c_n = \sum_{i = 0}^{n-1} c_i \cdot c_{n-1-i}
	\end{equation*}
	Co już można odwinąć do postaci która była w twierdzeniu.

	\begin{figure}[H]
		\centering
		\includegraphics[scale=0.5]{images/catalan/recursive_construction_1.png}
		\includegraphics[scale=0.5]{images/catalan/recursive_construction_2.png}

		\caption{Przykłady ,,podzielenia'' poprawnej ścieżki Dycka na podścieżki}
	\end{figure}

\end{proof}


\section{Zliczanie podziałów}

Chcemy pokazać fajny algorytm zliczania wszystkich podziałów liczby \(n\).

Oznaczmy liczbę wszystkich podziałów liczby \(n\) jako \(p(n)\). Jako ,,podziały liczby \(n\)'' mam na myśli liczbę sposobów na podzielenie liczby \(n\) na ileś składników (niezerowych), np. liczbę \(2\) mogę rozłożyć na \(1 + 1\) albo po prostu na \(2\) (i w sumie to tyle).  Funkcja tworząca ciągu \(p_n\) to: \begin{equation*}
	P(x) = (1 + x + x^2 + x^3 + \dots) \cdot (1 + x^2 + x^4 + x^6 + \dots) \cdot (1 + x^3 + x^6 + x^9 + \dots) \dots
\end{equation*}

Pierwszy nawias odpowiada wybraniu jedynki do podziału (i temu ile razy ją bierzemy), drugi dwójki, trzeci trójki, etc.

Oczywiście przy \(x^n\) będziemy mieli \(p_n\), jak to działa w funkcjach tworzących (i mam nadzieję, że widać dlaczego). Zapisujemy \(P(x)\) w fajniejszej postaci:

\begin{equation*}
	P(x) = \frac{1}{1-x} \cdot \frac{1}{1 - x^2} \cdot \frac{1}{1-x^3} \dots
\end{equation*}

Definiuję sobie \(Q(x) = (1-x) \cdot (1-x^2) \cdot (1-x^3) \dots\). Zauważam, że \(P(x) \cdot Q(x) = 1\), czyli \(Q(x)\) jest funkcją odwrotną do \(P(x)\). Okazuje się teraz, że \(Q(x)\) jest funkcją tworzącą pewnego śmiesznego ciągu, który sobie zaraz pokażemy.

Póki co musimy wprowadzić oznaczenia:
\begin{enumerate}
	\item \(e_n\) jest to liczba podziałów liczby \(n\) na parzystą liczbę składników parami różnych,
	\item \(o_n\) jest to liczba podziałów liczby \(n\) na nieparzystą liczbę składników parami różnych.
\end{enumerate}
Jak wszyscy powinniśmy już wiedzieć, funkcja tworząca ciągu \(e_n + o_n\) (czyli po prostu wszystkich podziałów \(n\) ze składnikami parami różnymi) wygląda tak:
\begin{equation*}
	(1+x) \cdot (1 + x^2) \cdot (1+x^3) \dots
\end{equation*}
Ten fakt do niczego nam się w sumie nie przyda, ale może pomóc zrozumieć co zaraz się stanie.

Możemy sobie teraz podumać, jaka jest funkcja tworząca ciągu \(e_n - o_n\). Otóż pojawia się tu plot twist, bo funkcja tworząca tego ciągu to po prostu \(Q(x)\):
\begin{equation*}
	(1-x) \cdot (1-x^2) \cdot (1-x^3) \dots
\end{equation*}

Działa to tak jak w powyższym przykładzie, z tym że jeśli wybraliśmy nieparzyście wiele składników to będzie nieparzyście wiele minusów i się ,,odejmie'' od współczynnika przy \(x^n\), a jeśli będzie parzyście wiele to się ,,doda''. Innymi słowy, do współczynnika przy \(x^n\) doda się 1 za każdy możliwy podział na parzyście wiele parami różnych składników, a odejmie się 1 za każdy możliwy podział na nieparzyście wiele parami różnych składników, czyli to co chcemy. Nie do końca mam pomysł jak to formalnie wytłumaczyć, więc proszę użyć swojej intuicji™.

Po co to wszystko? Okazuje się, że ciąg \(q_n = e_n - o_n\) ma pewne śmieszne własności (które niestety będzie trzeba udowodnić, brace yourselves).

\begin{theorem}[Eulera]
	\begin{equation}
		q_n = \begin{cases}
			0, \hspace{5pt} \mathrm{gdy} \hspace{5pt} n \not = \frac{(3 \cdot k \pm 1) \cdot k }{2} \\
			(-1)^k \hspace{5pt} \mathrm{wpp.}                                                       \\
		\end{cases}
	\end{equation}

\end{theorem}

\begin{proof}
	Zrobimy sobie przekształcenie \(f\), które przesyła prawie (dlaczego prawie to dojdziemy do tego za chwilę) każdy podział na \(n\) składników parami różnych na inny podział na \(n\) składników parami różnych (bijektywnie). Ktoś powie że sobie zrobiłem świetną bijekcję idącą z pewnego zbioru w samego siebie, but hear me out: ta bijekcja będzie mieć tę śmieszną własność, że jeśli podział był na parzyście wiele składników to będzie przesłany na nieparzyście wiele, a jeśli na nieparzyście wiele to będzie przesłany na parzyście wiele składników. To będzie fajne, bo pokażemy sobie że jest ich tyle samo (poza przypadkami gdzie definicja tej funkcji się popsuje, ale o tym za chwilę).

	Generalnie to oznaczmy sobie najmniejszy składnik w podziale \(P\) jako \(a\). Ponadto, zdefiniujmy sobie zbiór \(X\), taki że zawiera on największe składniki podziału \(P\), takie że każde dwa sąsiednie różnią się o jeden. Innymi słowy, jeśli podział \(P = (\lambda_1, \lambda_2, \lambda_3, \dots, \lambda_k)\), to \(X =\{\lambda_1, \lambda_2, \lambda_3, \dots, \lambda_d\}\), gdzie \(d\) jest największą liczbą taką, że kolejne składniki różnią się o 1  (zakładamy, że \(\lambda_1 > \lambda_2 > \dots > \lambda_k\)).

	Teraz jak mamy te zbiory zdefiniowane to możemy robić śmieszne rzeczy. Jeśli \(|X| < a - 1\), to możemy przerobić nasz podział, odejmując od każdego elementu z \(X\) 1, i dorzucając nowy element do podziału, taki że równy jest on moc \(|X|\). Otrzymaliśmy oczywiście poprawny podział (niektórym może pomóc dowód przez rysowanie).

	Dlaczego \(|X| < a - 1\), a nie po prostu \(|X| < a\)? Otóż przychodzi tutaj pewien śmieszny problem, mianowicie może być tak, że składnik podziału o wartości \(a\) ,,wpadł'' do \(X\). W takim przypadku bijekcja nam się kompletnie popsuje i wtedy jej definiujemy (ale jeszcze do tego wrócimy). Natomiast jeśli \(a\) nie należy do \(|X|\) to nasza bijekcja nadal działa. Fajnie.

	Czyli reasumując: jeśli \(|X| < a - 1\) lub (\(|X| = a - 1\) i \(a \not \in X\)) od każdego składnika z \(|X|\) odejmujemy 1 i majstrujemy nowy składnik, który wrzucamy pod składnik o wartości \(a\), który uprzednio był najmniejszy.

	\begin{figure}[H]
		\centering
		\includegraphics{images/case2.png}
		\caption{Wizualizacja przekształcenia (diagram Ferrersa). 2 ,,górne'' składniki różnią się o 1, trzeci już różni się od nich o 2; \(|X| = 2\), \(a=3\).}
	\end{figure}

	Zasadniczo to samo będziemy czynić (ale w drugą stronę), gdy okaże się że \(a < |X| \). Ordynarnie \textit{wywalam} składnik \(a\) i do odpowiedniej liczby elementów z \(X\)  ,,dodaję'' 1, tak by się wyrównało. Należy zauważyć, że być może nie wszystkie elementy z \(X\) będą mieć coś do siebie dodane, ale to mi nic nie psuje. W sumie też fajnie byłoby dodać, że dodaję te jedynki najpierw największym składnikom; inaczej mogłoby to się popsuć.

	Co dzieje się, gdy \(a = |X|\)? Jeśli \(a \in X\) to jest mi smutno, w przeciwnym razie mogę zrobić to samo co robiłem wcześniej i wszystko działa jak powinno.

	\begin{figure}[H]
		\centering
		\includegraphics{images/case_1.png}
		\caption{Wizualizacja przekształcenia (diagram Ferrersa). 3 ,,górne'' składniki różnią się o 1 więc należą do \(X\). \(|X| = 3\), \(a = 2\), więc dwóm największym elementom dodajemy 1, a składnik \(a\) usuwamy.}
	\end{figure}

	Zostają więc 2 przypadki, gdy coś może się popsuć:
	\begin{enumerate}
		\item \(|X| = a - 1\), \(a \in X\)
		      \begin{figure}[H]
			      \centering
			      \includegraphics{images/irytujacy_1.png}
			      \caption{Gdy \(|X| = a - 1\) i składnik \(a\) jest w \(X\); widać, że nic nie możemy z tym zrobić.}
		      \end{figure}

		\item \(|X| = a\), \(a \in x\)
		      \begin{figure}[H]
			      \centering
			      \includegraphics{images/irytujacy_2.png}
			      \caption{Gdy \(|X| = a\) i składnik \(a\) jest w \(X\); również widać, że nasze przekształcenie nie zadziała.}
		      \end{figure}
	\end{enumerate}

	Zauważmy, że sytuacja gdy składnik \(a\) jest w \(X\) jest bardzo dziwną sytuacją generalnie, bo jest to najmniejszy składnik; z definicji \(X\) mamy wtedy, że wszystkie kolejne składniki w \(P\) różnią się o dokładnie 1. Na podstawie tej obserwacji możemy już dokładnie powiedzieć, jakiej postaci musi być \(n\), by miało taki ,,złośliwy'' podział:

	\begin{enumerate}
		\item Gdy \(|X| = a - 1\), \(a \in X\), to \(n\) musi dla jakiegoś \(k\) być postaci \((k + 1) + (k + 2) + \dots + 2k \) (\(|X| =k, a = k+1\), wszystko się zgadza)
		\item Gdy \(|X| = a\), \(a \in x\), to \(n\) musi dla jakiegoś \(k\) być postaci \(k + (k+1) + (k+2) + \dots + (2k - 1)\) (\(|X| = k\), \(a = k\), ponownie wszystko gra)
	\end{enumerate}

	Jak zastosujemy matematykę mniej dyskretną by wysumować te nawiasy, dostaniemy że \(n\) aby miało irytujący podział to musi być postaci \(\frac{k \cdot (3k+1)}{2}\) lub \({k \cdot (3k-1)}{2}\). Jednocześnie nie ma takiego naturalnego \(k\), że wartości te są sobie równe, więc jeśli \(n\) ma irytujący podział, to ma go tylko jednego. Wtedy nie możemy przerzucić tylko jednego podziału na inny (inne są ze sobą w bijekcji) więc \(e_n - o_n = (-1)^k\) (jeśli \(k\) jest parzyste to irytujący podział ma parzyście wiele składników, a w przeciwnym razie nieparzyście wiele). Jeśli irytujący podział nie występuje, \(e_n = o_n\) z bijekcji którą pokazaliśmy. Fajnie.
\end{proof}

Dobra, ale wróćmy do tego cośmy chcieli udowodnić na samym początku. Co w ogóle wynika z tego twierdzenia Eulera? No w sumie to bardzo dużo, bo jak mamy \(q_n = e_n - o_n\) i \(Q(x)\) jest jego funkcją tworzącą:
\begin{equation*}
	Q(x) = q_0 + q_1 \cdot x + q_2 \cdot x^2 + q_3 \cdot x^3 + \dots
\end{equation*}
Ale znamy wartości współczynników \(q_i\) z twierdzenia Eulera:
\begin{equation*}
	Q(x) = 1 - x - x^2 + x^5 + x^7 - x^{12} - x^{15} + x^{22} + x^{26} + \dots
\end{equation*}
Zauważmy, że współczynników które nie są zerowe jest tylko jakoś \(O(\sqrt{n})\), czyli dosyć mało.

Pamiętajmy, że \(P(x) \cdot Q(x) = 1\), czyli że ciąg który wyjdzie po ich wymnożeniu będzie wyglądać tak: \((1,0,0,0, \dots)\) Ponieważ mnożenie w funkcjach tworzących działa jakoś tak, że w wynikowym ciągu (nazwijmy go \(r\)) element \(r_n\) można obliczyć w ten sposób:
\begin{equation*}
	r_n = \sum_{i=0}^{n} p_i \cdot q_{n-i}
\end{equation*}

I wiemy że w naszym przypadku \(r_n = 0\) dla \(n > 1\), to mamy że: \begin{equation*}
	0 = p_n - p_{n-1} - p_{n-2} + p_{n-5} + p_{n-7} - p_{n-12} - \dots
\end{equation*}
To teraz \(p_n\) przerzucamy na drugą stronę i mnożymy stronami razy \(-1\) i mamy wzór na \(p_n\), które możemy obliczyć w \(O(\sqrt{n})\). No i fajnie.



\section{Zasada minimum. Zasada maksimum. Twierdzenie o~definiowaniu przez indukcję}

\section{Relacje równoważności i~podziały zbiorów. Relacja równoważności jako środek do definiowania pojęć abstrakcyjnych}

Tymczasowy tekst żeby \LaTeX się ogarnął, bo inaczej wszystko renderuje się na jednej stronie, z jakiegoś powodu

\section{Twierdzenie Cantora-Bernsteina. Twierdzenie Cantora. Czy istnieje zbiór wszystkich zbiorów? Odpowiedź uzasadnij}
\section{Rozwiązywanie rekurencji liniowych}
\epigraph{I can elaborate: zrobiłam zadanka, zobaczyłam tworzące, stwierdziłam, że chce mi się spać, poszłam sobie}{\textit{Studentka TCSu o zadaniach z funkcji tworzących na kolokwium}}


\subsection{Rozkład na ułamki proste}
To nie jest formalny dowód ani formalna własność ani nic, bardziej schemat postępowania przy rozkładzie na ułamki proste. Sam dowód tego, że rozkład na ułamki proste istnieje, to \textit{sprowadź do wspólnego mianownika i zobacz co Ci wyszło}.
Jeżeli \(deg(P(x)) < deg(Q(x))\) i \(Q(x) = (x-a)^n \cdot (x-b)^k\) to:
\begin{equation*}
	\frac{P(x)}{Q(x)} = \frac{P(x)}{(x-a)^n \cdot (x-b)^k} = \frac{A_1}{x-a} + \frac{A_2}{(x-a)^2} + \dots + \frac{A_n}{(x-a)^n} + \frac{B_1}{x-b} + \frac{B_2}{(x-b)^2} + \dots + \frac{B_k}{(x-b)^k}
\end{equation*}

Oczywiście ten schemat można rozszerzać na więcej śmiesznych rzeczy w mianowniku, ale chyba widać o co chodzi.


\section{Ciąg Fibbonaciego}
\begin{theorem}[Wzór Bineta]
	\begin{equation}
		f_n = \frac{1}{\sqrt{5}} \cdot \left( \left(\frac{1 + \sqrt{5}}{2}\right)^{n} - \left(\frac{1 - \sqrt{5}}{2}\right)^{n} \right)
	\end{equation}
\end{theorem}

\begin{proof}
	Rozpisujemy sobie funkcję tworzącą ciągu \(f_n\):

	\begin{equation*}
		F(x) = f_0 + f_1 \cdot x + f_2 \cdot x^2 + f_3 \cdot x^3 \dots =
	\end{equation*}
	\begin{equation*}
		= f_0 + f_1 \cdot x + (f_0 + f_1) \cdot x^2 + (f_1 + f_2) \cdot x^3 + \dots =
	\end{equation*}
	\begin{equation*}
		= f_0 + f_1 \cdot x + f_0 \cdot x^2 + f_1 \cdot x^2 + f_1 \cdot x^3 + f_2 \cdot x^3 + \dots =
	\end{equation*}
	\begin{equation*}
		= f_0 + f_1 \cdot x + f_0 \cdot x^2 + f_1 \cdot x^3 + \dots + f_1 \cdot x^2 +  f_2 \cdot x^3 + \dots =
	\end{equation*}
	\begin{equation*}
		= f_0 + f_1 \cdot x + x^2 \cdot (f_0 + f_1 \cdot x + \dots) + x \cdot (f_1 \cdot x +  f_2 \cdot x^2 + \dots) =
	\end{equation*}
	\begin{equation*}
		= f_0 + f_1 \cdot x + x^2 \cdot F(x) + x \cdot (F(x) - f_0) =
	\end{equation*}
	\begin{equation*}
		= 0 + 1 \cdot x + x^2 \cdot F(x) + x \cdot (F(x) - 0) =
	\end{equation*}
	\begin{equation*}
		= x + x^2 \cdot F(x) + x \cdot F(x)
	\end{equation*}

	W takim razie mamy, że:
	\begin{equation*}
		F(x) = x + x^2 \cdot F(x) + x \cdot F(x)
	\end{equation*}
	\begin{equation*}
		F(x) -  x^2 \cdot F(x) - x \cdot F(x)  = x
	\end{equation*}
	\begin{equation*}
		F(x) \cdot (1 - x^2 - x) = x
	\end{equation*}
	\begin{equation*}
		F(x) = \frac{x}{-x^2 -x + 1}
	\end{equation*}

	Mianownik możemy rozbić (za pomocą liczenia jakichś delt czy coś):
	\begin{equation*}
		F(x) = \frac{x}{(-1) \cdot \left(x - \left(- \frac{1 + \sqrt{5}}{2}\right)\right) \cdot \left(x - \left(- \frac{1 - \sqrt{5}}{2}\right)\right)}
	\end{equation*}

	Nie no, serio, jeśli ktoś myśli że będę TeXować te przekształcenia to się myli. Powinno wyjść po przekształceniach że:
	\begin{equation*}
		F(x) = \frac{x}{(1-ax) \cdot (1-bx)}
	\end{equation*}
	gdzie \(a = \frac{1 + \sqrt{5}}{2}, b=\frac{1 - \sqrt{5}}{2}\)

	Dalej rozbijamy na ułamki proste:
	\begin{equation*}
		F(x) = \frac{A}{1-ax} + \frac{B}{1-bx}
	\end{equation*}
	\(A\) powinno wyjść \(\frac{1}{\sqrt{5}}\), \(B\) powinno wyjść \(- \frac{1}{\sqrt{5}}\).

	Odwijamy każdą z tych funkcji tworzących z osobna, korzystając ze wzoru podanego we wcześniejszym rozdziale i otrzymujemy wzór.
\end{proof}


\section{Ciąg Catalana}
Liczba Catalana jest to liczba ścieżek długości \(2n\) w kwadracie \(n \times n\) ,,poniżej'' przekątnej (lub na jej poziomie), idących za każdym razem jednostkę do góry lub jednostkę w prawo. Ścieżki takie nazywamy ścieżkami Dycka. Niezwykle formalna definicja. To jest jedna z tych rzeczy, które chyba po prostu trzeba narysować.

\begin{figure}[h]
	\centering
	\includegraphics[scale=0.5]{images/catalan/all_paths_1.png}
  \caption{Ścieżki Dycka długości 2; \(c_1 = 1\)}
\end{figure}

\begin{figure}[h]
	\centering
	\includegraphics[scale=0.5]{images/catalan/all_paths_2.png}
  \caption{Ścieżki Dycka długości 4; \(c_2 = 2\)}
\end{figure}

\begin{figure}[ht]
	\centering
	\includegraphics[scale=0.5]{images/catalan/all_paths_3.png}
  \caption{Ścieżki Dycka długości 6; \(c_3 = 5\)}
\end{figure}


\subsection{Wzór kombinatoryczny}
\begin{theorem}[Wzór kombinatoryczny na liczby Catalana]
	\begin{equation}
		c_n = \frac{1}{n+1} \cdot \binom{2n}{n}
	\end{equation}
\end{theorem}

Mamy sobie nasz kwadrat \(n \times n\). Przekątną możemy opisać tak jakby wzorem \(y = x\) (tak intuicyjnie, bo nie działamy w żadnym układzie współrzędnych, bla bla bla). Robimy sobie teraz prostą \(y = x+1\), idącą jakby ,,o jednostkę wyżej''. Zauważamy, że jeśli jakaś ścieżka przekracza linię naszej przekątnej, to musi ,,dotknąć'' linii \(y = x+1\). \textit{To widać}. Teraz wpadamy na świetny pomysł; jeśli jakaś ścieżka idąca po tym kwadracie ,,spotyka się'' z \(y = x+1\), to od tego momentu odbijamy ją symetrycznie względem \(y = x+1\). Zauważamy, że ścieżka ta (po odbiciu) skończy się w punkcie \((n-1, n+1)\) zamiast w \((n,n)\). Fakt ten dowodzimy stosując dowód przez rysowanie.

\begin{figure}[ht]
	\centering
	\includegraphics[scale=0.5]{images/catalan/path_with_reflection_1.png}
	\includegraphics[scale=0.5]{images/catalan/path_with_reflection_2.png}

	\caption{Przykłady odbicia niepoprawnej ścieżki}
\end{figure}

Zauważamy fascynujący fakt, mianowicie dwie różne ścieżki będą mieć 2 różne odbicia, a więc nasze przekształcenie jest iniektywne. Ponadto, jak sobie zobaczymy jakąkolwiek ścieżkę zaczynającą się w \((0,0)\), ale kończącą się w \((n-1,n+1)\), to jesteśmy w stanie zobaczyć gdzie pierwszy raz przecina się z \(y = x+1\), a następnie ją odbić, otrzymując ścieżkę idącą do \((n,n)\) i niebędącą ścieżką Dycka, której odbicie daje wyjściową ścieżkę. Zatem odbijanie jest suriektywne. A to oznacza tylko jedną rzecz: bijekcję między ścieżkami które ,,nie są catalanowe'', a ścieżkami ,,odbitymi''.

Wszystkich możliwych ścieżek od \((0,0)\) do \((n,n)\) mamy \(\binom{2n}{n}\), bo długość naszej drogi ma \(2n\) i wybieramy sobie \(n\) miejsc gdzie idziemy w prawo. Wszystkich możliwych ścieżek od \((0,0)\) do \((n-1,n+1)\) (czyli tych które są ,,złe'') mamy \(\binom{2n}{n-1}\), bo, analogicznie, ścieżka jest długości \(2n\) ale w prawo idziemy \(n-1\) razy. To prowadzi nas do wyniku:
\begin{equation*}
	\begin{split}
		c_n
		&= \binom{2n}{n} - \binom{2n}{n-1} \\
		&= \frac{(2n)!}{n! \cdot n!} - \frac{(2n)!}{(n-1)! \cdot (n+1)!} \\
		&= \frac{(n+1) \cdot (2n)!}{n! \cdot (n+1)!} - \frac{n \cdot (2n)!}{n! \cdot (n+1)!}\\
		&= \frac{(2n)!}{n! \cdot (n+1)!} \\
		&= \frac{1}{n+1} \cdot \frac {(2n)!}{n! \cdot n!} \\
		&= \frac{1}{n+1} \cdot \binom{2n}{n}
	\end{split}
\end{equation*}
\subsection{Zależność rekurencyjna}

\begin{theorem}[Wzór rekurencyjny na liczby Catalana]
	\begin{equation}
		c_n = c_{0} \cdot c_{n-1} + c_{1} \cdot c_{n - 2} + \dots + c_{n-1} \cdot c_0
	\end{equation}
\end{theorem}

\begin{proof}
	Znowuż mamy kwadrat \(n \times n\), ale tym razem dorysowujemy sobie prostą \(y = x - 1\). Każda ścieżka przetnie kiedyś tę linię i każda ścieżka dotknie kiedyś przekątnej \(y = x\) (można to udowodnić machając i pokazując na rysunek). Rzecz teraz ma się tak, że jeśli po ,,spotkaniu się'' z \(y = x - 1\) idziesz do góry, to potem musisz odbić w prawo (lub w skrajnym przypadku skończyłeś poprawną ścieżkę). Jednocześnie pierwszy wybór kierunku (tzn. ten w punkcie \((0,0)\) zawsze jest ,,w prawo'', bo jeśli ktoś pójdzie ,,do góry'' to znajdzie się w \((0,1)\), powyżej przekątnej \(y = x\)).

	Bierzemy sobie zatem pierwsze miejsce gdzie spotkałeś się z \(y = x\) i zauważamy, że jeśli dane jest ono jakimiś współrzędnymi \((i,i)\) to przecięliśmy \(y = x-1\) w \((i,i-1)\). Ponadto, ścieżka którą szliśmy od punktu \((1,0)\) do \((i,i-1)\) tak naprawdę jest ścieżką Dycka w kwadracie od punktów \((1,0)\), \((i, i-1)\) (kwadrat ten ma długość \(i-1\)). Ależ plot twist! Ścieżka którą idziemy od punktu \((i,i)\) do \((n,n)\) jest zaś już po prostu ścieżką Dycka w kwadracie o długości boku \(n-i\). Ścieżki te są od siebie niezależne i w ogóle, a długości tych ,,kwadratów catalanowych'' sumują się do \(i - 1 + n - i = n - 1\), więc teraz możemy zmajstrować wzór (w zależności od długości boków kwadratów, które z kolei są dyktowane tym kiedy się ,,spotkamy'' z \(y = x\)):
	\begin{equation*}
		c_n = \sum_{i = 0}^{n-1} c_i \cdot c_{n-1-i}
	\end{equation*}
	Co już można odwinąć do postaci która była w twierdzeniu.

	\begin{figure}[H]
		\centering
		\includegraphics[scale=0.5]{images/catalan/recursive_construction_1.png}
		\includegraphics[scale=0.5]{images/catalan/recursive_construction_2.png}

		\caption{Przykłady ,,podzielenia'' poprawnej ścieżki Dycka na podścieżki}
	\end{figure}

\end{proof}


\section{Zliczanie podziałów}

Chcemy pokazać fajny algorytm zliczania wszystkich podziałów liczby \(n\).

Oznaczmy liczbę wszystkich podziałów liczby \(n\) jako \(p(n)\). Jako ,,podziały liczby \(n\)'' mam na myśli liczbę sposobów na podzielenie liczby \(n\) na ileś składników (niezerowych), np. liczbę \(2\) mogę rozłożyć na \(1 + 1\) albo po prostu na \(2\) (i w sumie to tyle).  Funkcja tworząca ciągu \(p_n\) to: \begin{equation*}
	P(x) = (1 + x + x^2 + x^3 + \dots) \cdot (1 + x^2 + x^4 + x^6 + \dots) \cdot (1 + x^3 + x^6 + x^9 + \dots) \dots
\end{equation*}

Pierwszy nawias odpowiada wybraniu jedynki do podziału (i temu ile razy ją bierzemy), drugi dwójki, trzeci trójki, etc.

Oczywiście przy \(x^n\) będziemy mieli \(p_n\), jak to działa w funkcjach tworzących (i mam nadzieję, że widać dlaczego). Zapisujemy \(P(x)\) w fajniejszej postaci:

\begin{equation*}
	P(x) = \frac{1}{1-x} \cdot \frac{1}{1 - x^2} \cdot \frac{1}{1-x^3} \dots
\end{equation*}

Definiuję sobie \(Q(x) = (1-x) \cdot (1-x^2) \cdot (1-x^3) \dots\). Zauważam, że \(P(x) \cdot Q(x) = 1\), czyli \(Q(x)\) jest funkcją odwrotną do \(P(x)\). Okazuje się teraz, że \(Q(x)\) jest funkcją tworzącą pewnego śmiesznego ciągu, który sobie zaraz pokażemy.

Póki co musimy wprowadzić oznaczenia:
\begin{enumerate}
	\item \(e_n\) jest to liczba podziałów liczby \(n\) na parzystą liczbę składników parami różnych,
	\item \(o_n\) jest to liczba podziałów liczby \(n\) na nieparzystą liczbę składników parami różnych.
\end{enumerate}
Jak wszyscy powinniśmy już wiedzieć, funkcja tworząca ciągu \(e_n + o_n\) (czyli po prostu wszystkich podziałów \(n\) ze składnikami parami różnymi) wygląda tak:
\begin{equation*}
	(1+x) \cdot (1 + x^2) \cdot (1+x^3) \dots
\end{equation*}
Ten fakt do niczego nam się w sumie nie przyda, ale może pomóc zrozumieć co zaraz się stanie.

Możemy sobie teraz podumać, jaka jest funkcja tworząca ciągu \(e_n - o_n\). Otóż pojawia się tu plot twist, bo funkcja tworząca tego ciągu to po prostu \(Q(x)\):
\begin{equation*}
	(1-x) \cdot (1-x^2) \cdot (1-x^3) \dots
\end{equation*}

Działa to tak jak w powyższym przykładzie, z tym że jeśli wybraliśmy nieparzyście wiele składników to będzie nieparzyście wiele minusów i się ,,odejmie'' od współczynnika przy \(x^n\), a jeśli będzie parzyście wiele to się ,,doda''. Innymi słowy, do współczynnika przy \(x^n\) doda się 1 za każdy możliwy podział na parzyście wiele parami różnych składników, a odejmie się 1 za każdy możliwy podział na nieparzyście wiele parami różnych składników, czyli to co chcemy. Nie do końca mam pomysł jak to formalnie wytłumaczyć, więc proszę użyć swojej intuicji™.

Po co to wszystko? Okazuje się, że ciąg \(q_n = e_n - o_n\) ma pewne śmieszne własności (które niestety będzie trzeba udowodnić, brace yourselves).

\begin{theorem}[Eulera]
	\begin{equation}
		q_n = \begin{cases}
			0, \hspace{5pt} \mathrm{gdy} \hspace{5pt} n \not = \frac{(3 \cdot k \pm 1) \cdot k }{2} \\
			(-1)^k \hspace{5pt} \mathrm{wpp.}                                                       \\
		\end{cases}
	\end{equation}

\end{theorem}

\begin{proof}
	Zrobimy sobie przekształcenie \(f\), które przesyła prawie (dlaczego prawie to dojdziemy do tego za chwilę) każdy podział na \(n\) składników parami różnych na inny podział na \(n\) składników parami różnych (bijektywnie). Ktoś powie że sobie zrobiłem świetną bijekcję idącą z pewnego zbioru w samego siebie, but hear me out: ta bijekcja będzie mieć tę śmieszną własność, że jeśli podział był na parzyście wiele składników to będzie przesłany na nieparzyście wiele, a jeśli na nieparzyście wiele to będzie przesłany na parzyście wiele składników. To będzie fajne, bo pokażemy sobie że jest ich tyle samo (poza przypadkami gdzie definicja tej funkcji się popsuje, ale o tym za chwilę).

	Generalnie to oznaczmy sobie najmniejszy składnik w podziale \(P\) jako \(a\). Ponadto, zdefiniujmy sobie zbiór \(X\), taki że zawiera on największe składniki podziału \(P\), takie że każde dwa sąsiednie różnią się o jeden. Innymi słowy, jeśli podział \(P = (\lambda_1, \lambda_2, \lambda_3, \dots, \lambda_k)\), to \(X =\{\lambda_1, \lambda_2, \lambda_3, \dots, \lambda_d\}\), gdzie \(d\) jest największą liczbą taką, że kolejne składniki różnią się o 1  (zakładamy, że \(\lambda_1 > \lambda_2 > \dots > \lambda_k\)).

	Teraz jak mamy te zbiory zdefiniowane to możemy robić śmieszne rzeczy. Jeśli \(|X| < a - 1\), to możemy przerobić nasz podział, odejmując od każdego elementu z \(X\) 1, i dorzucając nowy element do podziału, taki że równy jest on moc \(|X|\). Otrzymaliśmy oczywiście poprawny podział (niektórym może pomóc dowód przez rysowanie).

	Dlaczego \(|X| < a - 1\), a nie po prostu \(|X| < a\)? Otóż przychodzi tutaj pewien śmieszny problem, mianowicie może być tak, że składnik podziału o wartości \(a\) ,,wpadł'' do \(X\). W takim przypadku bijekcja nam się kompletnie popsuje i wtedy jej definiujemy (ale jeszcze do tego wrócimy). Natomiast jeśli \(a\) nie należy do \(|X|\) to nasza bijekcja nadal działa. Fajnie.

	Czyli reasumując: jeśli \(|X| < a - 1\) lub (\(|X| = a - 1\) i \(a \not \in X\)) od każdego składnika z \(|X|\) odejmujemy 1 i majstrujemy nowy składnik, który wrzucamy pod składnik o wartości \(a\), który uprzednio był najmniejszy.

	\begin{figure}[H]
		\centering
		\includegraphics{images/case2.png}
		\caption{Wizualizacja przekształcenia (diagram Ferrersa). 2 ,,górne'' składniki różnią się o 1, trzeci już różni się od nich o 2; \(|X| = 2\), \(a=3\).}
	\end{figure}

	Zasadniczo to samo będziemy czynić (ale w drugą stronę), gdy okaże się że \(a < |X| \). Ordynarnie \textit{wywalam} składnik \(a\) i do odpowiedniej liczby elementów z \(X\)  ,,dodaję'' 1, tak by się wyrównało. Należy zauważyć, że być może nie wszystkie elementy z \(X\) będą mieć coś do siebie dodane, ale to mi nic nie psuje. W sumie też fajnie byłoby dodać, że dodaję te jedynki najpierw największym składnikom; inaczej mogłoby to się popsuć.

	Co dzieje się, gdy \(a = |X|\)? Jeśli \(a \in X\) to jest mi smutno, w przeciwnym razie mogę zrobić to samo co robiłem wcześniej i wszystko działa jak powinno.

	\begin{figure}[H]
		\centering
		\includegraphics{images/case_1.png}
		\caption{Wizualizacja przekształcenia (diagram Ferrersa). 3 ,,górne'' składniki różnią się o 1 więc należą do \(X\). \(|X| = 3\), \(a = 2\), więc dwóm największym elementom dodajemy 1, a składnik \(a\) usuwamy.}
	\end{figure}

	Zostają więc 2 przypadki, gdy coś może się popsuć:
	\begin{enumerate}
		\item \(|X| = a - 1\), \(a \in X\)
		      \begin{figure}[H]
			      \centering
			      \includegraphics{images/irytujacy_1.png}
			      \caption{Gdy \(|X| = a - 1\) i składnik \(a\) jest w \(X\); widać, że nic nie możemy z tym zrobić.}
		      \end{figure}

		\item \(|X| = a\), \(a \in x\)
		      \begin{figure}[H]
			      \centering
			      \includegraphics{images/irytujacy_2.png}
			      \caption{Gdy \(|X| = a\) i składnik \(a\) jest w \(X\); również widać, że nasze przekształcenie nie zadziała.}
		      \end{figure}
	\end{enumerate}

	Zauważmy, że sytuacja gdy składnik \(a\) jest w \(X\) jest bardzo dziwną sytuacją generalnie, bo jest to najmniejszy składnik; z definicji \(X\) mamy wtedy, że wszystkie kolejne składniki w \(P\) różnią się o dokładnie 1. Na podstawie tej obserwacji możemy już dokładnie powiedzieć, jakiej postaci musi być \(n\), by miało taki ,,złośliwy'' podział:

	\begin{enumerate}
		\item Gdy \(|X| = a - 1\), \(a \in X\), to \(n\) musi dla jakiegoś \(k\) być postaci \((k + 1) + (k + 2) + \dots + 2k \) (\(|X| =k, a = k+1\), wszystko się zgadza)
		\item Gdy \(|X| = a\), \(a \in x\), to \(n\) musi dla jakiegoś \(k\) być postaci \(k + (k+1) + (k+2) + \dots + (2k - 1)\) (\(|X| = k\), \(a = k\), ponownie wszystko gra)
	\end{enumerate}

	Jak zastosujemy matematykę mniej dyskretną by wysumować te nawiasy, dostaniemy że \(n\) aby miało irytujący podział to musi być postaci \(\frac{k \cdot (3k+1)}{2}\) lub \({k \cdot (3k-1)}{2}\). Jednocześnie nie ma takiego naturalnego \(k\), że wartości te są sobie równe, więc jeśli \(n\) ma irytujący podział, to ma go tylko jednego. Wtedy nie możemy przerzucić tylko jednego podziału na inny (inne są ze sobą w bijekcji) więc \(e_n - o_n = (-1)^k\) (jeśli \(k\) jest parzyste to irytujący podział ma parzyście wiele składników, a w przeciwnym razie nieparzyście wiele). Jeśli irytujący podział nie występuje, \(e_n = o_n\) z bijekcji którą pokazaliśmy. Fajnie.
\end{proof}

Dobra, ale wróćmy do tego cośmy chcieli udowodnić na samym początku. Co w ogóle wynika z tego twierdzenia Eulera? No w sumie to bardzo dużo, bo jak mamy \(q_n = e_n - o_n\) i \(Q(x)\) jest jego funkcją tworzącą:
\begin{equation*}
	Q(x) = q_0 + q_1 \cdot x + q_2 \cdot x^2 + q_3 \cdot x^3 + \dots
\end{equation*}
Ale znamy wartości współczynników \(q_i\) z twierdzenia Eulera:
\begin{equation*}
	Q(x) = 1 - x - x^2 + x^5 + x^7 - x^{12} - x^{15} + x^{22} + x^{26} + \dots
\end{equation*}
Zauważmy, że współczynników które nie są zerowe jest tylko jakoś \(O(\sqrt{n})\), czyli dosyć mało.

Pamiętajmy, że \(P(x) \cdot Q(x) = 1\), czyli że ciąg który wyjdzie po ich wymnożeniu będzie wyglądać tak: \((1,0,0,0, \dots)\) Ponieważ mnożenie w funkcjach tworzących działa jakoś tak, że w wynikowym ciągu (nazwijmy go \(r\)) element \(r_n\) można obliczyć w ten sposób:
\begin{equation*}
	r_n = \sum_{i=0}^{n} p_i \cdot q_{n-i}
\end{equation*}

I wiemy że w naszym przypadku \(r_n = 0\) dla \(n > 1\), to mamy że: \begin{equation*}
	0 = p_n - p_{n-1} - p_{n-2} + p_{n-5} + p_{n-7} - p_{n-12} - \dots
\end{equation*}
To teraz \(p_n\) przerzucamy na drugą stronę i mnożymy stronami razy \(-1\) i mamy wzór na \(p_n\), które możemy obliczyć w \(O(\sqrt{n})\). No i fajnie.



\section{Ciągłość i~gęstość porządku. Zbiór liczb wymiernych a~zbiór liczb rzeczywistych}

\section{Lemat Kuratowskiego-Zorna i~przykłady jego zastosowania}

\section{Konstrukcja liczb całkowitych. Działania na liczbach całkowitych. Konstrukcja liczb wymiernych i~działania na nich}

\section{Przykłady zbiorów nieprzeliczalnych}

\section{Aksjomatyczne ujęcie teorii mnogości. Aksjomat wyboru.}

\section{Twierdzenie Knastera-Tarskiego (dla zbiorów). Lemat Banacha}
\label{mfi:knaster_tarski_and_banach}
\section{Rozwiązywanie rekurencji liniowych}
\epigraph{I can elaborate: zrobiłam zadanka, zobaczyłam tworzące, stwierdziłam, że chce mi się spać, poszłam sobie}{\textit{Studentka TCSu o zadaniach z funkcji tworzących na kolokwium}}


\subsection{Rozkład na ułamki proste}
To nie jest formalny dowód ani formalna własność ani nic, bardziej schemat postępowania przy rozkładzie na ułamki proste. Sam dowód tego, że rozkład na ułamki proste istnieje, to \textit{sprowadź do wspólnego mianownika i zobacz co Ci wyszło}.
Jeżeli \(deg(P(x)) < deg(Q(x))\) i \(Q(x) = (x-a)^n \cdot (x-b)^k\) to:
\begin{equation*}
	\frac{P(x)}{Q(x)} = \frac{P(x)}{(x-a)^n \cdot (x-b)^k} = \frac{A_1}{x-a} + \frac{A_2}{(x-a)^2} + \dots + \frac{A_n}{(x-a)^n} + \frac{B_1}{x-b} + \frac{B_2}{(x-b)^2} + \dots + \frac{B_k}{(x-b)^k}
\end{equation*}

Oczywiście ten schemat można rozszerzać na więcej śmiesznych rzeczy w mianowniku, ale chyba widać o co chodzi.


\section{Ciąg Fibbonaciego}
\begin{theorem}[Wzór Bineta]
	\begin{equation}
		f_n = \frac{1}{\sqrt{5}} \cdot \left( \left(\frac{1 + \sqrt{5}}{2}\right)^{n} - \left(\frac{1 - \sqrt{5}}{2}\right)^{n} \right)
	\end{equation}
\end{theorem}

\begin{proof}
	Rozpisujemy sobie funkcję tworzącą ciągu \(f_n\):

	\begin{equation*}
		F(x) = f_0 + f_1 \cdot x + f_2 \cdot x^2 + f_3 \cdot x^3 \dots =
	\end{equation*}
	\begin{equation*}
		= f_0 + f_1 \cdot x + (f_0 + f_1) \cdot x^2 + (f_1 + f_2) \cdot x^3 + \dots =
	\end{equation*}
	\begin{equation*}
		= f_0 + f_1 \cdot x + f_0 \cdot x^2 + f_1 \cdot x^2 + f_1 \cdot x^3 + f_2 \cdot x^3 + \dots =
	\end{equation*}
	\begin{equation*}
		= f_0 + f_1 \cdot x + f_0 \cdot x^2 + f_1 \cdot x^3 + \dots + f_1 \cdot x^2 +  f_2 \cdot x^3 + \dots =
	\end{equation*}
	\begin{equation*}
		= f_0 + f_1 \cdot x + x^2 \cdot (f_0 + f_1 \cdot x + \dots) + x \cdot (f_1 \cdot x +  f_2 \cdot x^2 + \dots) =
	\end{equation*}
	\begin{equation*}
		= f_0 + f_1 \cdot x + x^2 \cdot F(x) + x \cdot (F(x) - f_0) =
	\end{equation*}
	\begin{equation*}
		= 0 + 1 \cdot x + x^2 \cdot F(x) + x \cdot (F(x) - 0) =
	\end{equation*}
	\begin{equation*}
		= x + x^2 \cdot F(x) + x \cdot F(x)
	\end{equation*}

	W takim razie mamy, że:
	\begin{equation*}
		F(x) = x + x^2 \cdot F(x) + x \cdot F(x)
	\end{equation*}
	\begin{equation*}
		F(x) -  x^2 \cdot F(x) - x \cdot F(x)  = x
	\end{equation*}
	\begin{equation*}
		F(x) \cdot (1 - x^2 - x) = x
	\end{equation*}
	\begin{equation*}
		F(x) = \frac{x}{-x^2 -x + 1}
	\end{equation*}

	Mianownik możemy rozbić (za pomocą liczenia jakichś delt czy coś):
	\begin{equation*}
		F(x) = \frac{x}{(-1) \cdot \left(x - \left(- \frac{1 + \sqrt{5}}{2}\right)\right) \cdot \left(x - \left(- \frac{1 - \sqrt{5}}{2}\right)\right)}
	\end{equation*}

	Nie no, serio, jeśli ktoś myśli że będę TeXować te przekształcenia to się myli. Powinno wyjść po przekształceniach że:
	\begin{equation*}
		F(x) = \frac{x}{(1-ax) \cdot (1-bx)}
	\end{equation*}
	gdzie \(a = \frac{1 + \sqrt{5}}{2}, b=\frac{1 - \sqrt{5}}{2}\)

	Dalej rozbijamy na ułamki proste:
	\begin{equation*}
		F(x) = \frac{A}{1-ax} + \frac{B}{1-bx}
	\end{equation*}
	\(A\) powinno wyjść \(\frac{1}{\sqrt{5}}\), \(B\) powinno wyjść \(- \frac{1}{\sqrt{5}}\).

	Odwijamy każdą z tych funkcji tworzących z osobna, korzystając ze wzoru podanego we wcześniejszym rozdziale i otrzymujemy wzór.
\end{proof}


\section{Ciąg Catalana}
Liczba Catalana jest to liczba ścieżek długości \(2n\) w kwadracie \(n \times n\) ,,poniżej'' przekątnej (lub na jej poziomie), idących za każdym razem jednostkę do góry lub jednostkę w prawo. Ścieżki takie nazywamy ścieżkami Dycka. Niezwykle formalna definicja. To jest jedna z tych rzeczy, które chyba po prostu trzeba narysować.

\begin{figure}[h]
	\centering
	\includegraphics[scale=0.5]{images/catalan/all_paths_1.png}
  \caption{Ścieżki Dycka długości 2; \(c_1 = 1\)}
\end{figure}

\begin{figure}[h]
	\centering
	\includegraphics[scale=0.5]{images/catalan/all_paths_2.png}
  \caption{Ścieżki Dycka długości 4; \(c_2 = 2\)}
\end{figure}

\begin{figure}[ht]
	\centering
	\includegraphics[scale=0.5]{images/catalan/all_paths_3.png}
  \caption{Ścieżki Dycka długości 6; \(c_3 = 5\)}
\end{figure}


\subsection{Wzór kombinatoryczny}
\begin{theorem}[Wzór kombinatoryczny na liczby Catalana]
	\begin{equation}
		c_n = \frac{1}{n+1} \cdot \binom{2n}{n}
	\end{equation}
\end{theorem}

Mamy sobie nasz kwadrat \(n \times n\). Przekątną możemy opisać tak jakby wzorem \(y = x\) (tak intuicyjnie, bo nie działamy w żadnym układzie współrzędnych, bla bla bla). Robimy sobie teraz prostą \(y = x+1\), idącą jakby ,,o jednostkę wyżej''. Zauważamy, że jeśli jakaś ścieżka przekracza linię naszej przekątnej, to musi ,,dotknąć'' linii \(y = x+1\). \textit{To widać}. Teraz wpadamy na świetny pomysł; jeśli jakaś ścieżka idąca po tym kwadracie ,,spotyka się'' z \(y = x+1\), to od tego momentu odbijamy ją symetrycznie względem \(y = x+1\). Zauważamy, że ścieżka ta (po odbiciu) skończy się w punkcie \((n-1, n+1)\) zamiast w \((n,n)\). Fakt ten dowodzimy stosując dowód przez rysowanie.

\begin{figure}[ht]
	\centering
	\includegraphics[scale=0.5]{images/catalan/path_with_reflection_1.png}
	\includegraphics[scale=0.5]{images/catalan/path_with_reflection_2.png}

	\caption{Przykłady odbicia niepoprawnej ścieżki}
\end{figure}

Zauważamy fascynujący fakt, mianowicie dwie różne ścieżki będą mieć 2 różne odbicia, a więc nasze przekształcenie jest iniektywne. Ponadto, jak sobie zobaczymy jakąkolwiek ścieżkę zaczynającą się w \((0,0)\), ale kończącą się w \((n-1,n+1)\), to jesteśmy w stanie zobaczyć gdzie pierwszy raz przecina się z \(y = x+1\), a następnie ją odbić, otrzymując ścieżkę idącą do \((n,n)\) i niebędącą ścieżką Dycka, której odbicie daje wyjściową ścieżkę. Zatem odbijanie jest suriektywne. A to oznacza tylko jedną rzecz: bijekcję między ścieżkami które ,,nie są catalanowe'', a ścieżkami ,,odbitymi''.

Wszystkich możliwych ścieżek od \((0,0)\) do \((n,n)\) mamy \(\binom{2n}{n}\), bo długość naszej drogi ma \(2n\) i wybieramy sobie \(n\) miejsc gdzie idziemy w prawo. Wszystkich możliwych ścieżek od \((0,0)\) do \((n-1,n+1)\) (czyli tych które są ,,złe'') mamy \(\binom{2n}{n-1}\), bo, analogicznie, ścieżka jest długości \(2n\) ale w prawo idziemy \(n-1\) razy. To prowadzi nas do wyniku:
\begin{equation*}
	\begin{split}
		c_n
		&= \binom{2n}{n} - \binom{2n}{n-1} \\
		&= \frac{(2n)!}{n! \cdot n!} - \frac{(2n)!}{(n-1)! \cdot (n+1)!} \\
		&= \frac{(n+1) \cdot (2n)!}{n! \cdot (n+1)!} - \frac{n \cdot (2n)!}{n! \cdot (n+1)!}\\
		&= \frac{(2n)!}{n! \cdot (n+1)!} \\
		&= \frac{1}{n+1} \cdot \frac {(2n)!}{n! \cdot n!} \\
		&= \frac{1}{n+1} \cdot \binom{2n}{n}
	\end{split}
\end{equation*}
\subsection{Zależność rekurencyjna}

\begin{theorem}[Wzór rekurencyjny na liczby Catalana]
	\begin{equation}
		c_n = c_{0} \cdot c_{n-1} + c_{1} \cdot c_{n - 2} + \dots + c_{n-1} \cdot c_0
	\end{equation}
\end{theorem}

\begin{proof}
	Znowuż mamy kwadrat \(n \times n\), ale tym razem dorysowujemy sobie prostą \(y = x - 1\). Każda ścieżka przetnie kiedyś tę linię i każda ścieżka dotknie kiedyś przekątnej \(y = x\) (można to udowodnić machając i pokazując na rysunek). Rzecz teraz ma się tak, że jeśli po ,,spotkaniu się'' z \(y = x - 1\) idziesz do góry, to potem musisz odbić w prawo (lub w skrajnym przypadku skończyłeś poprawną ścieżkę). Jednocześnie pierwszy wybór kierunku (tzn. ten w punkcie \((0,0)\) zawsze jest ,,w prawo'', bo jeśli ktoś pójdzie ,,do góry'' to znajdzie się w \((0,1)\), powyżej przekątnej \(y = x\)).

	Bierzemy sobie zatem pierwsze miejsce gdzie spotkałeś się z \(y = x\) i zauważamy, że jeśli dane jest ono jakimiś współrzędnymi \((i,i)\) to przecięliśmy \(y = x-1\) w \((i,i-1)\). Ponadto, ścieżka którą szliśmy od punktu \((1,0)\) do \((i,i-1)\) tak naprawdę jest ścieżką Dycka w kwadracie od punktów \((1,0)\), \((i, i-1)\) (kwadrat ten ma długość \(i-1\)). Ależ plot twist! Ścieżka którą idziemy od punktu \((i,i)\) do \((n,n)\) jest zaś już po prostu ścieżką Dycka w kwadracie o długości boku \(n-i\). Ścieżki te są od siebie niezależne i w ogóle, a długości tych ,,kwadratów catalanowych'' sumują się do \(i - 1 + n - i = n - 1\), więc teraz możemy zmajstrować wzór (w zależności od długości boków kwadratów, które z kolei są dyktowane tym kiedy się ,,spotkamy'' z \(y = x\)):
	\begin{equation*}
		c_n = \sum_{i = 0}^{n-1} c_i \cdot c_{n-1-i}
	\end{equation*}
	Co już można odwinąć do postaci która była w twierdzeniu.

	\begin{figure}[H]
		\centering
		\includegraphics[scale=0.5]{images/catalan/recursive_construction_1.png}
		\includegraphics[scale=0.5]{images/catalan/recursive_construction_2.png}

		\caption{Przykłady ,,podzielenia'' poprawnej ścieżki Dycka na podścieżki}
	\end{figure}

\end{proof}


\section{Zliczanie podziałów}

Chcemy pokazać fajny algorytm zliczania wszystkich podziałów liczby \(n\).

Oznaczmy liczbę wszystkich podziałów liczby \(n\) jako \(p(n)\). Jako ,,podziały liczby \(n\)'' mam na myśli liczbę sposobów na podzielenie liczby \(n\) na ileś składników (niezerowych), np. liczbę \(2\) mogę rozłożyć na \(1 + 1\) albo po prostu na \(2\) (i w sumie to tyle).  Funkcja tworząca ciągu \(p_n\) to: \begin{equation*}
	P(x) = (1 + x + x^2 + x^3 + \dots) \cdot (1 + x^2 + x^4 + x^6 + \dots) \cdot (1 + x^3 + x^6 + x^9 + \dots) \dots
\end{equation*}

Pierwszy nawias odpowiada wybraniu jedynki do podziału (i temu ile razy ją bierzemy), drugi dwójki, trzeci trójki, etc.

Oczywiście przy \(x^n\) będziemy mieli \(p_n\), jak to działa w funkcjach tworzących (i mam nadzieję, że widać dlaczego). Zapisujemy \(P(x)\) w fajniejszej postaci:

\begin{equation*}
	P(x) = \frac{1}{1-x} \cdot \frac{1}{1 - x^2} \cdot \frac{1}{1-x^3} \dots
\end{equation*}

Definiuję sobie \(Q(x) = (1-x) \cdot (1-x^2) \cdot (1-x^3) \dots\). Zauważam, że \(P(x) \cdot Q(x) = 1\), czyli \(Q(x)\) jest funkcją odwrotną do \(P(x)\). Okazuje się teraz, że \(Q(x)\) jest funkcją tworzącą pewnego śmiesznego ciągu, który sobie zaraz pokażemy.

Póki co musimy wprowadzić oznaczenia:
\begin{enumerate}
	\item \(e_n\) jest to liczba podziałów liczby \(n\) na parzystą liczbę składników parami różnych,
	\item \(o_n\) jest to liczba podziałów liczby \(n\) na nieparzystą liczbę składników parami różnych.
\end{enumerate}
Jak wszyscy powinniśmy już wiedzieć, funkcja tworząca ciągu \(e_n + o_n\) (czyli po prostu wszystkich podziałów \(n\) ze składnikami parami różnymi) wygląda tak:
\begin{equation*}
	(1+x) \cdot (1 + x^2) \cdot (1+x^3) \dots
\end{equation*}
Ten fakt do niczego nam się w sumie nie przyda, ale może pomóc zrozumieć co zaraz się stanie.

Możemy sobie teraz podumać, jaka jest funkcja tworząca ciągu \(e_n - o_n\). Otóż pojawia się tu plot twist, bo funkcja tworząca tego ciągu to po prostu \(Q(x)\):
\begin{equation*}
	(1-x) \cdot (1-x^2) \cdot (1-x^3) \dots
\end{equation*}

Działa to tak jak w powyższym przykładzie, z tym że jeśli wybraliśmy nieparzyście wiele składników to będzie nieparzyście wiele minusów i się ,,odejmie'' od współczynnika przy \(x^n\), a jeśli będzie parzyście wiele to się ,,doda''. Innymi słowy, do współczynnika przy \(x^n\) doda się 1 za każdy możliwy podział na parzyście wiele parami różnych składników, a odejmie się 1 za każdy możliwy podział na nieparzyście wiele parami różnych składników, czyli to co chcemy. Nie do końca mam pomysł jak to formalnie wytłumaczyć, więc proszę użyć swojej intuicji™.

Po co to wszystko? Okazuje się, że ciąg \(q_n = e_n - o_n\) ma pewne śmieszne własności (które niestety będzie trzeba udowodnić, brace yourselves).

\begin{theorem}[Eulera]
	\begin{equation}
		q_n = \begin{cases}
			0, \hspace{5pt} \mathrm{gdy} \hspace{5pt} n \not = \frac{(3 \cdot k \pm 1) \cdot k }{2} \\
			(-1)^k \hspace{5pt} \mathrm{wpp.}                                                       \\
		\end{cases}
	\end{equation}

\end{theorem}

\begin{proof}
	Zrobimy sobie przekształcenie \(f\), które przesyła prawie (dlaczego prawie to dojdziemy do tego za chwilę) każdy podział na \(n\) składników parami różnych na inny podział na \(n\) składników parami różnych (bijektywnie). Ktoś powie że sobie zrobiłem świetną bijekcję idącą z pewnego zbioru w samego siebie, but hear me out: ta bijekcja będzie mieć tę śmieszną własność, że jeśli podział był na parzyście wiele składników to będzie przesłany na nieparzyście wiele, a jeśli na nieparzyście wiele to będzie przesłany na parzyście wiele składników. To będzie fajne, bo pokażemy sobie że jest ich tyle samo (poza przypadkami gdzie definicja tej funkcji się popsuje, ale o tym za chwilę).

	Generalnie to oznaczmy sobie najmniejszy składnik w podziale \(P\) jako \(a\). Ponadto, zdefiniujmy sobie zbiór \(X\), taki że zawiera on największe składniki podziału \(P\), takie że każde dwa sąsiednie różnią się o jeden. Innymi słowy, jeśli podział \(P = (\lambda_1, \lambda_2, \lambda_3, \dots, \lambda_k)\), to \(X =\{\lambda_1, \lambda_2, \lambda_3, \dots, \lambda_d\}\), gdzie \(d\) jest największą liczbą taką, że kolejne składniki różnią się o 1  (zakładamy, że \(\lambda_1 > \lambda_2 > \dots > \lambda_k\)).

	Teraz jak mamy te zbiory zdefiniowane to możemy robić śmieszne rzeczy. Jeśli \(|X| < a - 1\), to możemy przerobić nasz podział, odejmując od każdego elementu z \(X\) 1, i dorzucając nowy element do podziału, taki że równy jest on moc \(|X|\). Otrzymaliśmy oczywiście poprawny podział (niektórym może pomóc dowód przez rysowanie).

	Dlaczego \(|X| < a - 1\), a nie po prostu \(|X| < a\)? Otóż przychodzi tutaj pewien śmieszny problem, mianowicie może być tak, że składnik podziału o wartości \(a\) ,,wpadł'' do \(X\). W takim przypadku bijekcja nam się kompletnie popsuje i wtedy jej definiujemy (ale jeszcze do tego wrócimy). Natomiast jeśli \(a\) nie należy do \(|X|\) to nasza bijekcja nadal działa. Fajnie.

	Czyli reasumując: jeśli \(|X| < a - 1\) lub (\(|X| = a - 1\) i \(a \not \in X\)) od każdego składnika z \(|X|\) odejmujemy 1 i majstrujemy nowy składnik, który wrzucamy pod składnik o wartości \(a\), który uprzednio był najmniejszy.

	\begin{figure}[H]
		\centering
		\includegraphics{images/case2.png}
		\caption{Wizualizacja przekształcenia (diagram Ferrersa). 2 ,,górne'' składniki różnią się o 1, trzeci już różni się od nich o 2; \(|X| = 2\), \(a=3\).}
	\end{figure}

	Zasadniczo to samo będziemy czynić (ale w drugą stronę), gdy okaże się że \(a < |X| \). Ordynarnie \textit{wywalam} składnik \(a\) i do odpowiedniej liczby elementów z \(X\)  ,,dodaję'' 1, tak by się wyrównało. Należy zauważyć, że być może nie wszystkie elementy z \(X\) będą mieć coś do siebie dodane, ale to mi nic nie psuje. W sumie też fajnie byłoby dodać, że dodaję te jedynki najpierw największym składnikom; inaczej mogłoby to się popsuć.

	Co dzieje się, gdy \(a = |X|\)? Jeśli \(a \in X\) to jest mi smutno, w przeciwnym razie mogę zrobić to samo co robiłem wcześniej i wszystko działa jak powinno.

	\begin{figure}[H]
		\centering
		\includegraphics{images/case_1.png}
		\caption{Wizualizacja przekształcenia (diagram Ferrersa). 3 ,,górne'' składniki różnią się o 1 więc należą do \(X\). \(|X| = 3\), \(a = 2\), więc dwóm największym elementom dodajemy 1, a składnik \(a\) usuwamy.}
	\end{figure}

	Zostają więc 2 przypadki, gdy coś może się popsuć:
	\begin{enumerate}
		\item \(|X| = a - 1\), \(a \in X\)
		      \begin{figure}[H]
			      \centering
			      \includegraphics{images/irytujacy_1.png}
			      \caption{Gdy \(|X| = a - 1\) i składnik \(a\) jest w \(X\); widać, że nic nie możemy z tym zrobić.}
		      \end{figure}

		\item \(|X| = a\), \(a \in x\)
		      \begin{figure}[H]
			      \centering
			      \includegraphics{images/irytujacy_2.png}
			      \caption{Gdy \(|X| = a\) i składnik \(a\) jest w \(X\); również widać, że nasze przekształcenie nie zadziała.}
		      \end{figure}
	\end{enumerate}

	Zauważmy, że sytuacja gdy składnik \(a\) jest w \(X\) jest bardzo dziwną sytuacją generalnie, bo jest to najmniejszy składnik; z definicji \(X\) mamy wtedy, że wszystkie kolejne składniki w \(P\) różnią się o dokładnie 1. Na podstawie tej obserwacji możemy już dokładnie powiedzieć, jakiej postaci musi być \(n\), by miało taki ,,złośliwy'' podział:

	\begin{enumerate}
		\item Gdy \(|X| = a - 1\), \(a \in X\), to \(n\) musi dla jakiegoś \(k\) być postaci \((k + 1) + (k + 2) + \dots + 2k \) (\(|X| =k, a = k+1\), wszystko się zgadza)
		\item Gdy \(|X| = a\), \(a \in x\), to \(n\) musi dla jakiegoś \(k\) być postaci \(k + (k+1) + (k+2) + \dots + (2k - 1)\) (\(|X| = k\), \(a = k\), ponownie wszystko gra)
	\end{enumerate}

	Jak zastosujemy matematykę mniej dyskretną by wysumować te nawiasy, dostaniemy że \(n\) aby miało irytujący podział to musi być postaci \(\frac{k \cdot (3k+1)}{2}\) lub \({k \cdot (3k-1)}{2}\). Jednocześnie nie ma takiego naturalnego \(k\), że wartości te są sobie równe, więc jeśli \(n\) ma irytujący podział, to ma go tylko jednego. Wtedy nie możemy przerzucić tylko jednego podziału na inny (inne są ze sobą w bijekcji) więc \(e_n - o_n = (-1)^k\) (jeśli \(k\) jest parzyste to irytujący podział ma parzyście wiele składników, a w przeciwnym razie nieparzyście wiele). Jeśli irytujący podział nie występuje, \(e_n = o_n\) z bijekcji którą pokazaliśmy. Fajnie.
\end{proof}

Dobra, ale wróćmy do tego cośmy chcieli udowodnić na samym początku. Co w ogóle wynika z tego twierdzenia Eulera? No w sumie to bardzo dużo, bo jak mamy \(q_n = e_n - o_n\) i \(Q(x)\) jest jego funkcją tworzącą:
\begin{equation*}
	Q(x) = q_0 + q_1 \cdot x + q_2 \cdot x^2 + q_3 \cdot x^3 + \dots
\end{equation*}
Ale znamy wartości współczynników \(q_i\) z twierdzenia Eulera:
\begin{equation*}
	Q(x) = 1 - x - x^2 + x^5 + x^7 - x^{12} - x^{15} + x^{22} + x^{26} + \dots
\end{equation*}
Zauważmy, że współczynników które nie są zerowe jest tylko jakoś \(O(\sqrt{n})\), czyli dosyć mało.

Pamiętajmy, że \(P(x) \cdot Q(x) = 1\), czyli że ciąg który wyjdzie po ich wymnożeniu będzie wyglądać tak: \((1,0,0,0, \dots)\) Ponieważ mnożenie w funkcjach tworzących działa jakoś tak, że w wynikowym ciągu (nazwijmy go \(r\)) element \(r_n\) można obliczyć w ten sposób:
\begin{equation*}
	r_n = \sum_{i=0}^{n} p_i \cdot q_{n-i}
\end{equation*}

I wiemy że w naszym przypadku \(r_n = 0\) dla \(n > 1\), to mamy że: \begin{equation*}
	0 = p_n - p_{n-1} - p_{n-2} + p_{n-5} + p_{n-7} - p_{n-12} - \dots
\end{equation*}
To teraz \(p_n\) przerzucamy na drugą stronę i mnożymy stronami razy \(-1\) i mamy wzór na \(p_n\), które możemy obliczyć w \(O(\sqrt{n})\). No i fajnie.



\section{Równoliczność zbiorów na przykładach \texorpdfstring{\(\pars{A^B}^C \eqnum A^{B \times C}\)}{(A\^B)\^C ~ A\^(B x C)} oraz \texorpdfstring{\(\pars{A \times B}^C \eqnum A^C \times B^C\)}{(A x B)\^C ~ A\^C x B\^C}}
\label{mfi:equinumerosity}
\section{Rozwiązywanie rekurencji liniowych}
\epigraph{I can elaborate: zrobiłam zadanka, zobaczyłam tworzące, stwierdziłam, że chce mi się spać, poszłam sobie}{\textit{Studentka TCSu o zadaniach z funkcji tworzących na kolokwium}}


\subsection{Rozkład na ułamki proste}
To nie jest formalny dowód ani formalna własność ani nic, bardziej schemat postępowania przy rozkładzie na ułamki proste. Sam dowód tego, że rozkład na ułamki proste istnieje, to \textit{sprowadź do wspólnego mianownika i zobacz co Ci wyszło}.
Jeżeli \(deg(P(x)) < deg(Q(x))\) i \(Q(x) = (x-a)^n \cdot (x-b)^k\) to:
\begin{equation*}
	\frac{P(x)}{Q(x)} = \frac{P(x)}{(x-a)^n \cdot (x-b)^k} = \frac{A_1}{x-a} + \frac{A_2}{(x-a)^2} + \dots + \frac{A_n}{(x-a)^n} + \frac{B_1}{x-b} + \frac{B_2}{(x-b)^2} + \dots + \frac{B_k}{(x-b)^k}
\end{equation*}

Oczywiście ten schemat można rozszerzać na więcej śmiesznych rzeczy w mianowniku, ale chyba widać o co chodzi.


\section{Ciąg Fibbonaciego}
\begin{theorem}[Wzór Bineta]
	\begin{equation}
		f_n = \frac{1}{\sqrt{5}} \cdot \left( \left(\frac{1 + \sqrt{5}}{2}\right)^{n} - \left(\frac{1 - \sqrt{5}}{2}\right)^{n} \right)
	\end{equation}
\end{theorem}

\begin{proof}
	Rozpisujemy sobie funkcję tworzącą ciągu \(f_n\):

	\begin{equation*}
		F(x) = f_0 + f_1 \cdot x + f_2 \cdot x^2 + f_3 \cdot x^3 \dots =
	\end{equation*}
	\begin{equation*}
		= f_0 + f_1 \cdot x + (f_0 + f_1) \cdot x^2 + (f_1 + f_2) \cdot x^3 + \dots =
	\end{equation*}
	\begin{equation*}
		= f_0 + f_1 \cdot x + f_0 \cdot x^2 + f_1 \cdot x^2 + f_1 \cdot x^3 + f_2 \cdot x^3 + \dots =
	\end{equation*}
	\begin{equation*}
		= f_0 + f_1 \cdot x + f_0 \cdot x^2 + f_1 \cdot x^3 + \dots + f_1 \cdot x^2 +  f_2 \cdot x^3 + \dots =
	\end{equation*}
	\begin{equation*}
		= f_0 + f_1 \cdot x + x^2 \cdot (f_0 + f_1 \cdot x + \dots) + x \cdot (f_1 \cdot x +  f_2 \cdot x^2 + \dots) =
	\end{equation*}
	\begin{equation*}
		= f_0 + f_1 \cdot x + x^2 \cdot F(x) + x \cdot (F(x) - f_0) =
	\end{equation*}
	\begin{equation*}
		= 0 + 1 \cdot x + x^2 \cdot F(x) + x \cdot (F(x) - 0) =
	\end{equation*}
	\begin{equation*}
		= x + x^2 \cdot F(x) + x \cdot F(x)
	\end{equation*}

	W takim razie mamy, że:
	\begin{equation*}
		F(x) = x + x^2 \cdot F(x) + x \cdot F(x)
	\end{equation*}
	\begin{equation*}
		F(x) -  x^2 \cdot F(x) - x \cdot F(x)  = x
	\end{equation*}
	\begin{equation*}
		F(x) \cdot (1 - x^2 - x) = x
	\end{equation*}
	\begin{equation*}
		F(x) = \frac{x}{-x^2 -x + 1}
	\end{equation*}

	Mianownik możemy rozbić (za pomocą liczenia jakichś delt czy coś):
	\begin{equation*}
		F(x) = \frac{x}{(-1) \cdot \left(x - \left(- \frac{1 + \sqrt{5}}{2}\right)\right) \cdot \left(x - \left(- \frac{1 - \sqrt{5}}{2}\right)\right)}
	\end{equation*}

	Nie no, serio, jeśli ktoś myśli że będę TeXować te przekształcenia to się myli. Powinno wyjść po przekształceniach że:
	\begin{equation*}
		F(x) = \frac{x}{(1-ax) \cdot (1-bx)}
	\end{equation*}
	gdzie \(a = \frac{1 + \sqrt{5}}{2}, b=\frac{1 - \sqrt{5}}{2}\)

	Dalej rozbijamy na ułamki proste:
	\begin{equation*}
		F(x) = \frac{A}{1-ax} + \frac{B}{1-bx}
	\end{equation*}
	\(A\) powinno wyjść \(\frac{1}{\sqrt{5}}\), \(B\) powinno wyjść \(- \frac{1}{\sqrt{5}}\).

	Odwijamy każdą z tych funkcji tworzących z osobna, korzystając ze wzoru podanego we wcześniejszym rozdziale i otrzymujemy wzór.
\end{proof}


\section{Ciąg Catalana}
Liczba Catalana jest to liczba ścieżek długości \(2n\) w kwadracie \(n \times n\) ,,poniżej'' przekątnej (lub na jej poziomie), idących za każdym razem jednostkę do góry lub jednostkę w prawo. Ścieżki takie nazywamy ścieżkami Dycka. Niezwykle formalna definicja. To jest jedna z tych rzeczy, które chyba po prostu trzeba narysować.

\begin{figure}[h]
	\centering
	\includegraphics[scale=0.5]{images/catalan/all_paths_1.png}
  \caption{Ścieżki Dycka długości 2; \(c_1 = 1\)}
\end{figure}

\begin{figure}[h]
	\centering
	\includegraphics[scale=0.5]{images/catalan/all_paths_2.png}
  \caption{Ścieżki Dycka długości 4; \(c_2 = 2\)}
\end{figure}

\begin{figure}[ht]
	\centering
	\includegraphics[scale=0.5]{images/catalan/all_paths_3.png}
  \caption{Ścieżki Dycka długości 6; \(c_3 = 5\)}
\end{figure}


\subsection{Wzór kombinatoryczny}
\begin{theorem}[Wzór kombinatoryczny na liczby Catalana]
	\begin{equation}
		c_n = \frac{1}{n+1} \cdot \binom{2n}{n}
	\end{equation}
\end{theorem}

Mamy sobie nasz kwadrat \(n \times n\). Przekątną możemy opisać tak jakby wzorem \(y = x\) (tak intuicyjnie, bo nie działamy w żadnym układzie współrzędnych, bla bla bla). Robimy sobie teraz prostą \(y = x+1\), idącą jakby ,,o jednostkę wyżej''. Zauważamy, że jeśli jakaś ścieżka przekracza linię naszej przekątnej, to musi ,,dotknąć'' linii \(y = x+1\). \textit{To widać}. Teraz wpadamy na świetny pomysł; jeśli jakaś ścieżka idąca po tym kwadracie ,,spotyka się'' z \(y = x+1\), to od tego momentu odbijamy ją symetrycznie względem \(y = x+1\). Zauważamy, że ścieżka ta (po odbiciu) skończy się w punkcie \((n-1, n+1)\) zamiast w \((n,n)\). Fakt ten dowodzimy stosując dowód przez rysowanie.

\begin{figure}[ht]
	\centering
	\includegraphics[scale=0.5]{images/catalan/path_with_reflection_1.png}
	\includegraphics[scale=0.5]{images/catalan/path_with_reflection_2.png}

	\caption{Przykłady odbicia niepoprawnej ścieżki}
\end{figure}

Zauważamy fascynujący fakt, mianowicie dwie różne ścieżki będą mieć 2 różne odbicia, a więc nasze przekształcenie jest iniektywne. Ponadto, jak sobie zobaczymy jakąkolwiek ścieżkę zaczynającą się w \((0,0)\), ale kończącą się w \((n-1,n+1)\), to jesteśmy w stanie zobaczyć gdzie pierwszy raz przecina się z \(y = x+1\), a następnie ją odbić, otrzymując ścieżkę idącą do \((n,n)\) i niebędącą ścieżką Dycka, której odbicie daje wyjściową ścieżkę. Zatem odbijanie jest suriektywne. A to oznacza tylko jedną rzecz: bijekcję między ścieżkami które ,,nie są catalanowe'', a ścieżkami ,,odbitymi''.

Wszystkich możliwych ścieżek od \((0,0)\) do \((n,n)\) mamy \(\binom{2n}{n}\), bo długość naszej drogi ma \(2n\) i wybieramy sobie \(n\) miejsc gdzie idziemy w prawo. Wszystkich możliwych ścieżek od \((0,0)\) do \((n-1,n+1)\) (czyli tych które są ,,złe'') mamy \(\binom{2n}{n-1}\), bo, analogicznie, ścieżka jest długości \(2n\) ale w prawo idziemy \(n-1\) razy. To prowadzi nas do wyniku:
\begin{equation*}
	\begin{split}
		c_n
		&= \binom{2n}{n} - \binom{2n}{n-1} \\
		&= \frac{(2n)!}{n! \cdot n!} - \frac{(2n)!}{(n-1)! \cdot (n+1)!} \\
		&= \frac{(n+1) \cdot (2n)!}{n! \cdot (n+1)!} - \frac{n \cdot (2n)!}{n! \cdot (n+1)!}\\
		&= \frac{(2n)!}{n! \cdot (n+1)!} \\
		&= \frac{1}{n+1} \cdot \frac {(2n)!}{n! \cdot n!} \\
		&= \frac{1}{n+1} \cdot \binom{2n}{n}
	\end{split}
\end{equation*}
\subsection{Zależność rekurencyjna}

\begin{theorem}[Wzór rekurencyjny na liczby Catalana]
	\begin{equation}
		c_n = c_{0} \cdot c_{n-1} + c_{1} \cdot c_{n - 2} + \dots + c_{n-1} \cdot c_0
	\end{equation}
\end{theorem}

\begin{proof}
	Znowuż mamy kwadrat \(n \times n\), ale tym razem dorysowujemy sobie prostą \(y = x - 1\). Każda ścieżka przetnie kiedyś tę linię i każda ścieżka dotknie kiedyś przekątnej \(y = x\) (można to udowodnić machając i pokazując na rysunek). Rzecz teraz ma się tak, że jeśli po ,,spotkaniu się'' z \(y = x - 1\) idziesz do góry, to potem musisz odbić w prawo (lub w skrajnym przypadku skończyłeś poprawną ścieżkę). Jednocześnie pierwszy wybór kierunku (tzn. ten w punkcie \((0,0)\) zawsze jest ,,w prawo'', bo jeśli ktoś pójdzie ,,do góry'' to znajdzie się w \((0,1)\), powyżej przekątnej \(y = x\)).

	Bierzemy sobie zatem pierwsze miejsce gdzie spotkałeś się z \(y = x\) i zauważamy, że jeśli dane jest ono jakimiś współrzędnymi \((i,i)\) to przecięliśmy \(y = x-1\) w \((i,i-1)\). Ponadto, ścieżka którą szliśmy od punktu \((1,0)\) do \((i,i-1)\) tak naprawdę jest ścieżką Dycka w kwadracie od punktów \((1,0)\), \((i, i-1)\) (kwadrat ten ma długość \(i-1\)). Ależ plot twist! Ścieżka którą idziemy od punktu \((i,i)\) do \((n,n)\) jest zaś już po prostu ścieżką Dycka w kwadracie o długości boku \(n-i\). Ścieżki te są od siebie niezależne i w ogóle, a długości tych ,,kwadratów catalanowych'' sumują się do \(i - 1 + n - i = n - 1\), więc teraz możemy zmajstrować wzór (w zależności od długości boków kwadratów, które z kolei są dyktowane tym kiedy się ,,spotkamy'' z \(y = x\)):
	\begin{equation*}
		c_n = \sum_{i = 0}^{n-1} c_i \cdot c_{n-1-i}
	\end{equation*}
	Co już można odwinąć do postaci która była w twierdzeniu.

	\begin{figure}[H]
		\centering
		\includegraphics[scale=0.5]{images/catalan/recursive_construction_1.png}
		\includegraphics[scale=0.5]{images/catalan/recursive_construction_2.png}

		\caption{Przykłady ,,podzielenia'' poprawnej ścieżki Dycka na podścieżki}
	\end{figure}

\end{proof}


\section{Zliczanie podziałów}

Chcemy pokazać fajny algorytm zliczania wszystkich podziałów liczby \(n\).

Oznaczmy liczbę wszystkich podziałów liczby \(n\) jako \(p(n)\). Jako ,,podziały liczby \(n\)'' mam na myśli liczbę sposobów na podzielenie liczby \(n\) na ileś składników (niezerowych), np. liczbę \(2\) mogę rozłożyć na \(1 + 1\) albo po prostu na \(2\) (i w sumie to tyle).  Funkcja tworząca ciągu \(p_n\) to: \begin{equation*}
	P(x) = (1 + x + x^2 + x^3 + \dots) \cdot (1 + x^2 + x^4 + x^6 + \dots) \cdot (1 + x^3 + x^6 + x^9 + \dots) \dots
\end{equation*}

Pierwszy nawias odpowiada wybraniu jedynki do podziału (i temu ile razy ją bierzemy), drugi dwójki, trzeci trójki, etc.

Oczywiście przy \(x^n\) będziemy mieli \(p_n\), jak to działa w funkcjach tworzących (i mam nadzieję, że widać dlaczego). Zapisujemy \(P(x)\) w fajniejszej postaci:

\begin{equation*}
	P(x) = \frac{1}{1-x} \cdot \frac{1}{1 - x^2} \cdot \frac{1}{1-x^3} \dots
\end{equation*}

Definiuję sobie \(Q(x) = (1-x) \cdot (1-x^2) \cdot (1-x^3) \dots\). Zauważam, że \(P(x) \cdot Q(x) = 1\), czyli \(Q(x)\) jest funkcją odwrotną do \(P(x)\). Okazuje się teraz, że \(Q(x)\) jest funkcją tworzącą pewnego śmiesznego ciągu, który sobie zaraz pokażemy.

Póki co musimy wprowadzić oznaczenia:
\begin{enumerate}
	\item \(e_n\) jest to liczba podziałów liczby \(n\) na parzystą liczbę składników parami różnych,
	\item \(o_n\) jest to liczba podziałów liczby \(n\) na nieparzystą liczbę składników parami różnych.
\end{enumerate}
Jak wszyscy powinniśmy już wiedzieć, funkcja tworząca ciągu \(e_n + o_n\) (czyli po prostu wszystkich podziałów \(n\) ze składnikami parami różnymi) wygląda tak:
\begin{equation*}
	(1+x) \cdot (1 + x^2) \cdot (1+x^3) \dots
\end{equation*}
Ten fakt do niczego nam się w sumie nie przyda, ale może pomóc zrozumieć co zaraz się stanie.

Możemy sobie teraz podumać, jaka jest funkcja tworząca ciągu \(e_n - o_n\). Otóż pojawia się tu plot twist, bo funkcja tworząca tego ciągu to po prostu \(Q(x)\):
\begin{equation*}
	(1-x) \cdot (1-x^2) \cdot (1-x^3) \dots
\end{equation*}

Działa to tak jak w powyższym przykładzie, z tym że jeśli wybraliśmy nieparzyście wiele składników to będzie nieparzyście wiele minusów i się ,,odejmie'' od współczynnika przy \(x^n\), a jeśli będzie parzyście wiele to się ,,doda''. Innymi słowy, do współczynnika przy \(x^n\) doda się 1 za każdy możliwy podział na parzyście wiele parami różnych składników, a odejmie się 1 za każdy możliwy podział na nieparzyście wiele parami różnych składników, czyli to co chcemy. Nie do końca mam pomysł jak to formalnie wytłumaczyć, więc proszę użyć swojej intuicji™.

Po co to wszystko? Okazuje się, że ciąg \(q_n = e_n - o_n\) ma pewne śmieszne własności (które niestety będzie trzeba udowodnić, brace yourselves).

\begin{theorem}[Eulera]
	\begin{equation}
		q_n = \begin{cases}
			0, \hspace{5pt} \mathrm{gdy} \hspace{5pt} n \not = \frac{(3 \cdot k \pm 1) \cdot k }{2} \\
			(-1)^k \hspace{5pt} \mathrm{wpp.}                                                       \\
		\end{cases}
	\end{equation}

\end{theorem}

\begin{proof}
	Zrobimy sobie przekształcenie \(f\), które przesyła prawie (dlaczego prawie to dojdziemy do tego za chwilę) każdy podział na \(n\) składników parami różnych na inny podział na \(n\) składników parami różnych (bijektywnie). Ktoś powie że sobie zrobiłem świetną bijekcję idącą z pewnego zbioru w samego siebie, but hear me out: ta bijekcja będzie mieć tę śmieszną własność, że jeśli podział był na parzyście wiele składników to będzie przesłany na nieparzyście wiele, a jeśli na nieparzyście wiele to będzie przesłany na parzyście wiele składników. To będzie fajne, bo pokażemy sobie że jest ich tyle samo (poza przypadkami gdzie definicja tej funkcji się popsuje, ale o tym za chwilę).

	Generalnie to oznaczmy sobie najmniejszy składnik w podziale \(P\) jako \(a\). Ponadto, zdefiniujmy sobie zbiór \(X\), taki że zawiera on największe składniki podziału \(P\), takie że każde dwa sąsiednie różnią się o jeden. Innymi słowy, jeśli podział \(P = (\lambda_1, \lambda_2, \lambda_3, \dots, \lambda_k)\), to \(X =\{\lambda_1, \lambda_2, \lambda_3, \dots, \lambda_d\}\), gdzie \(d\) jest największą liczbą taką, że kolejne składniki różnią się o 1  (zakładamy, że \(\lambda_1 > \lambda_2 > \dots > \lambda_k\)).

	Teraz jak mamy te zbiory zdefiniowane to możemy robić śmieszne rzeczy. Jeśli \(|X| < a - 1\), to możemy przerobić nasz podział, odejmując od każdego elementu z \(X\) 1, i dorzucając nowy element do podziału, taki że równy jest on moc \(|X|\). Otrzymaliśmy oczywiście poprawny podział (niektórym może pomóc dowód przez rysowanie).

	Dlaczego \(|X| < a - 1\), a nie po prostu \(|X| < a\)? Otóż przychodzi tutaj pewien śmieszny problem, mianowicie może być tak, że składnik podziału o wartości \(a\) ,,wpadł'' do \(X\). W takim przypadku bijekcja nam się kompletnie popsuje i wtedy jej definiujemy (ale jeszcze do tego wrócimy). Natomiast jeśli \(a\) nie należy do \(|X|\) to nasza bijekcja nadal działa. Fajnie.

	Czyli reasumując: jeśli \(|X| < a - 1\) lub (\(|X| = a - 1\) i \(a \not \in X\)) od każdego składnika z \(|X|\) odejmujemy 1 i majstrujemy nowy składnik, który wrzucamy pod składnik o wartości \(a\), który uprzednio był najmniejszy.

	\begin{figure}[H]
		\centering
		\includegraphics{images/case2.png}
		\caption{Wizualizacja przekształcenia (diagram Ferrersa). 2 ,,górne'' składniki różnią się o 1, trzeci już różni się od nich o 2; \(|X| = 2\), \(a=3\).}
	\end{figure}

	Zasadniczo to samo będziemy czynić (ale w drugą stronę), gdy okaże się że \(a < |X| \). Ordynarnie \textit{wywalam} składnik \(a\) i do odpowiedniej liczby elementów z \(X\)  ,,dodaję'' 1, tak by się wyrównało. Należy zauważyć, że być może nie wszystkie elementy z \(X\) będą mieć coś do siebie dodane, ale to mi nic nie psuje. W sumie też fajnie byłoby dodać, że dodaję te jedynki najpierw największym składnikom; inaczej mogłoby to się popsuć.

	Co dzieje się, gdy \(a = |X|\)? Jeśli \(a \in X\) to jest mi smutno, w przeciwnym razie mogę zrobić to samo co robiłem wcześniej i wszystko działa jak powinno.

	\begin{figure}[H]
		\centering
		\includegraphics{images/case_1.png}
		\caption{Wizualizacja przekształcenia (diagram Ferrersa). 3 ,,górne'' składniki różnią się o 1 więc należą do \(X\). \(|X| = 3\), \(a = 2\), więc dwóm największym elementom dodajemy 1, a składnik \(a\) usuwamy.}
	\end{figure}

	Zostają więc 2 przypadki, gdy coś może się popsuć:
	\begin{enumerate}
		\item \(|X| = a - 1\), \(a \in X\)
		      \begin{figure}[H]
			      \centering
			      \includegraphics{images/irytujacy_1.png}
			      \caption{Gdy \(|X| = a - 1\) i składnik \(a\) jest w \(X\); widać, że nic nie możemy z tym zrobić.}
		      \end{figure}

		\item \(|X| = a\), \(a \in x\)
		      \begin{figure}[H]
			      \centering
			      \includegraphics{images/irytujacy_2.png}
			      \caption{Gdy \(|X| = a\) i składnik \(a\) jest w \(X\); również widać, że nasze przekształcenie nie zadziała.}
		      \end{figure}
	\end{enumerate}

	Zauważmy, że sytuacja gdy składnik \(a\) jest w \(X\) jest bardzo dziwną sytuacją generalnie, bo jest to najmniejszy składnik; z definicji \(X\) mamy wtedy, że wszystkie kolejne składniki w \(P\) różnią się o dokładnie 1. Na podstawie tej obserwacji możemy już dokładnie powiedzieć, jakiej postaci musi być \(n\), by miało taki ,,złośliwy'' podział:

	\begin{enumerate}
		\item Gdy \(|X| = a - 1\), \(a \in X\), to \(n\) musi dla jakiegoś \(k\) być postaci \((k + 1) + (k + 2) + \dots + 2k \) (\(|X| =k, a = k+1\), wszystko się zgadza)
		\item Gdy \(|X| = a\), \(a \in x\), to \(n\) musi dla jakiegoś \(k\) być postaci \(k + (k+1) + (k+2) + \dots + (2k - 1)\) (\(|X| = k\), \(a = k\), ponownie wszystko gra)
	\end{enumerate}

	Jak zastosujemy matematykę mniej dyskretną by wysumować te nawiasy, dostaniemy że \(n\) aby miało irytujący podział to musi być postaci \(\frac{k \cdot (3k+1)}{2}\) lub \({k \cdot (3k-1)}{2}\). Jednocześnie nie ma takiego naturalnego \(k\), że wartości te są sobie równe, więc jeśli \(n\) ma irytujący podział, to ma go tylko jednego. Wtedy nie możemy przerzucić tylko jednego podziału na inny (inne są ze sobą w bijekcji) więc \(e_n - o_n = (-1)^k\) (jeśli \(k\) jest parzyste to irytujący podział ma parzyście wiele składników, a w przeciwnym razie nieparzyście wiele). Jeśli irytujący podział nie występuje, \(e_n = o_n\) z bijekcji którą pokazaliśmy. Fajnie.
\end{proof}

Dobra, ale wróćmy do tego cośmy chcieli udowodnić na samym początku. Co w ogóle wynika z tego twierdzenia Eulera? No w sumie to bardzo dużo, bo jak mamy \(q_n = e_n - o_n\) i \(Q(x)\) jest jego funkcją tworzącą:
\begin{equation*}
	Q(x) = q_0 + q_1 \cdot x + q_2 \cdot x^2 + q_3 \cdot x^3 + \dots
\end{equation*}
Ale znamy wartości współczynników \(q_i\) z twierdzenia Eulera:
\begin{equation*}
	Q(x) = 1 - x - x^2 + x^5 + x^7 - x^{12} - x^{15} + x^{22} + x^{26} + \dots
\end{equation*}
Zauważmy, że współczynników które nie są zerowe jest tylko jakoś \(O(\sqrt{n})\), czyli dosyć mało.

Pamiętajmy, że \(P(x) \cdot Q(x) = 1\), czyli że ciąg który wyjdzie po ich wymnożeniu będzie wyglądać tak: \((1,0,0,0, \dots)\) Ponieważ mnożenie w funkcjach tworzących działa jakoś tak, że w wynikowym ciągu (nazwijmy go \(r\)) element \(r_n\) można obliczyć w ten sposób:
\begin{equation*}
	r_n = \sum_{i=0}^{n} p_i \cdot q_{n-i}
\end{equation*}

I wiemy że w naszym przypadku \(r_n = 0\) dla \(n > 1\), to mamy że: \begin{equation*}
	0 = p_n - p_{n-1} - p_{n-2} + p_{n-5} + p_{n-7} - p_{n-12} - \dots
\end{equation*}
To teraz \(p_n\) przerzucamy na drugą stronę i mnożymy stronami razy \(-1\) i mamy wzór na \(p_n\), które możemy obliczyć w \(O(\sqrt{n})\). No i fajnie.



\section{Zasada indukcji pozaskończonej a~dobry porządek}

\section{Liczby porządkowe von Neumanna i~ich własności. Antynomia Burali-Forti}
