\begin{lemma}[Banach]
\label{mfi:banach_lemma}
Niech \(f\colon X \function Y\) oraz \(g\colon Y \implies X\).

Istnieją:
\begin{itemize}
    \item rozkład \(\set{A_1, A_2}\) zbioru \(X\) (\(A_1 \cup A_2 = X \land A_1 \cap A_2 = \emptyset\))
    \item rozkład \(\set{B_1, B_2}\) zbioru \(Y\) (\(B_1 \cup B_2 = Y \land B_1 \cap B_2 = \emptyset\))
\end{itemize}
takie, że
\begin{itemize}
    \item \(\fIm{f}\pars{A_1} = B_1\)
    \item \(\fIm{g}\pars{B_2} = A_2\)
\end{itemize}
Uwaga! Dopuszczamy tu trywialne rozkłady, to znaczy niektóre ze zbiorów \(A_1, A_2, B_1, B_2\) mogą być puste.
\end{lemma}
\begin{proof}
Zdefiniujmy następującą funkcję:
\begin{gather*}
    \varphi\colon \powerset\pars{X} \function \powerset\pars{X}\\
    \varphi\pars{x} = X \setminus \fIm{g}\pars{Y \setminus \fIm{f}\pars{x}}
\end{gather*}
Intuicyjnie, funkcja \(\varphi\)
\begin{enumerate}
    \item zaczyna w~zbiorze \(x \subseteq X\)
    \item przechodzi z~niego funkcją \(f\)~do \(Y\)
    \item wyrzuca z~\(Y\)~elementy, na które przeszła
    \item z~elementów pozostałych w~zbiorze \(Y\)~przechodzi funkcją \(g\)~do \(X\)
    \item wyrzuca z~\(X\)~elementy, na które przeszła
\end{enumerate}
Zauważmy najpierw, że \(\varphi\)~jest monotoniczna:
\begin{align*}
    A &\subseteq B\\
    \fIm{f}\pars{A} &\subseteq \fIm{f}\pars{B}\\
    Y \setminus \fIm{f}\pars{A} &\supseteq Y \setminus \fIm{f}\pars{B}\\
    \fIm{g}\pars{Y \setminus \fIm{f}\pars{A}} &\supseteq \fIm{g}\pars{Y \setminus \fIm{f}\pars{B}}\\
    X \setminus \fIm{g}\pars{Y \setminus \fIm{f}\pars{A}} &\subseteq X \setminus \fIm{g}\pars{Y \setminus \fIm{f}\pars{B}}
\end{align*}
Na mocy twierdzenia Knastera-Tarskiego, funkcja \(\varphi\) posiada punkt stały. Nazwijmy go \(A_1\). Resztę ustalmy następująco:
\begin{align*}
    B_1 &\coloneqq \fIm{f}\pars{A_1}\\
    B_2 &\coloneqq Y \setminus B_1\\
    A_2 &\coloneqq X \setminus A_1
\end{align*}
Z~definicji mamy zatem spełnione wymaganie, że \(\fIm{f}\pars{A_1} = B_1\). Zauważamy też
\begin{align*}
    \fIm{g}\pars{B_2}
        &= \fIm{g}\pars{Y \setminus B_1} && \text{z~definicji \(B_2\)}\\
        &= \fIm{g}\pars{Y \setminus \fIm{f}\pars{A_1}} && \text{z~definicji \(B_1\)}\\
        &= X \setminus \pars{X \setminus \fIm{g}\pars{Y \setminus \fIm{f}\pars{A_1}}} && \text{jesteśmy w~uniwersum \(X\)}\\
        &= X \setminus \varphi\pars{A_1} && \text{z~definicji \(\varphi\)}\\
        &= X \setminus A_1 && \text{\(A_1\)~jest punktem stałym}\\
        &= A_2 && \text{z~definicji \(A_2\)}
\end{align*}
Zatem również drugie wymaganie \(\fIm{g}\pars{B_2} = A_2\) jest spełnione.
\end{proof}
