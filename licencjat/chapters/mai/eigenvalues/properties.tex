\begin{theorem}
	Niech \(\field = (\fieldset, +, \cdot)\) będzie ciałem, a \(M \in \fieldset^{n \times n}\) będzie macierzą nad tym ciałem (gdzie \(n \in \natural\)). Ponadto, niech \(\lambda \in \fieldset\).

	Następujące stwierdzenia są równoważne:
	\begin{enumerate}
		\item \(\lambda\) jest wartością własną macierzy \(M\).
		\item \(\det{(M - \lambda I)}\) = 0, gdzie \(I\) to macierz identycznościowa (rozmiaru \(n \times n\)).
	\end{enumerate}
\end{theorem}
\begin{proof}
	Zanim przejdziemy do rozumowania, przekształćmy równoważnie warunek na to, żeby \(\lambda\) było wartością własną:
	\begin{align*}
		Mv = \lambda v                            \\
		Mv - \lambda v = 0                        \\
		(M \cdot I - \lambda \cdot I) \cdot v = 0 \\
		(M - \lambda I) \cdot v = 0               \\
	\end{align*}
	\( (1) \implies (2) \):
	Zakładamy \(\lambda\) jest wartością własną oraz (nie wprost), że \((M - \lambda I)\) ma niezerowy wyznacznik. W takim razie jest odwracalna. Lewostronnie więc mnożymy równanie przez jej odwrotność i otrzymujemy \(v = 0\) jako jedyne rozwiązanie (co przeczy założeniu że \(\lambda\) jest wartością własną). Otrzymana sprzeczność pokazuje, że wyznacznik musi być zerowy.

	\( (2) \implies (1) \):
	Załóżmy, że wyznacznik macierzy \(M - \lambda I\) jest zerowy. W takim razie wiemy, że zbiór wektorów tworzących wiersze macierzy \(M\) jest liniowo \textbf{zależny}. Z definicji liniowej zależności oznacza to, że istnieją takie \(\lambda_1, \lambda_2, \dots, \lambda_n \in \fieldset\) (przy czym oraz istnieje takie \(i\), że \(\lambda_i \not = 0\)) że:
	\[
		\sum_{i=1}^{n} \lambda_i \cdot x_i = 0
	\]
	gdzie \(x_i\) to wektor znajdujący się w \(i\)-tym wierszu macierzy.

	Oznacza to, że wektor \(v = [\lambda_1, \lambda_2, \dots, \lambda_n]\) jest taki, że:
	\[
		(M - \lambda I ) v = 0
	\]
	czyli \(v\) jest wektorem własnym, a \(\lambda\) wartością własną. Ale fajnie.

\end{proof}

\begin{definition}[Wielomian charakterystyczny]
	Niech \(\field = (\fieldset, +, \cdot)\) będzie ciałem, a \(M \in \fieldset^{n \times n}\) będzie macierzą nad tym ciałem (gdzie \(n \in \natural\)).

	Funkcję \(f: \fieldset \rightarrow \fieldset\) nazwiemy \textbf{wielomianem charakterystycznym} macierzy \(M\), jeżeli \(f(x) = \det{(M - x \cdot I)}\).
\end{definition}

\begin{fact}
	Niech \(\field = (\fieldset, +, \cdot)\) będzie ciałem, a \(M \in \fieldset^{n \times n}\) będzie macierzą nad tym ciałem, a \(f\) będzie wielomianem charakterystycznym macierzy \(M\).

	Jeśli jakieś \(\lambda \in \fieldset\) jest takie, że \(f(\lambda)=0\) (tzn. \(\lambda\) jest miejscem zerowym wielomianu \(f\)), to \(\lambda\) jest wartością własną macierzy \(M\).
\end{fact}
\begin{proof}
	\[
		f(\lambda) = 0 \implies \det{(M - \lambda I)} = 0 \iff \text{\(\lambda\) jest wartością własną}
	\]
\end{proof}

Z powyższego faktu wynika, że aby poznać wartości własne możemy szukać miejsc zerowych wielomianu charakterystycznego.