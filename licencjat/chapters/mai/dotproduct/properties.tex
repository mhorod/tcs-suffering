\begin{fact}
Niech \(v \in \vfield\). Wówczas \( \norm{\lambda v} = \lambda \norm{v}\).

\end{fact}
\begin{proof}
\[
\norm{\lambda v} = \sqrt{\dotpro{\lambda v}{\lambda v}} = \sqrt{\lambda^2 \dotpro{v}{v}} = \lambda \sqrt{\dotpro{v}{v}} = \lambda \norm{v}
\]
\end{proof}
\begin{fact}
Niech \(v \in \vfield\). Wówczas następujące warunki są równoważne:

\begin{enumerate}
    \item \(\norm{v} = 0\)
    \item \(v = 0\)
\end{enumerate}

\end{fact}
\begin{proof}
Przekształcamy pierwszy warunek równoważnie: 
\begin{align*}
    \norm{v} = 0 \\
    \sqrt{\dotpro{v}{v}} = 0 \\ 
    \dotpro{v}{v} = 0 \\
    v = 0
\end{align*}

Ostatnie przejście wynika bezpośrednio z definicji iloczynu skalarnego.
\end{proof}
\begin{fact}

\end{fact}

\begin{fact}
Niech \(v_1, v_2 \in \vfield\). Wówczas \(\norm{v_1 + v_2} \leq \norm{v_1} + \norm{v_2}\)

\end{fact}
\begin{proof}
\begin{align*}
    \norm{v_1} + \norm{v_2} &= \sqrt{\dotpro{v_1 + v_2}{v_1 + v_2}}  \\
\end{align*}
\end{proof}