Zdefiniujemy sobie tak zwany \textit{Algorytm D}. Służyć on będzie do minimalizacji deterministycznych automatów skończonych.

\begin{itemize}
	\item Wejście: DFA \( A \)
	\item Wyjście: (trójkątna) macierz Boolowska \( M \), w której \( M_{i, j} = 1 \iff a_i, a_j \) są rozróżnialne
\end{itemize}

Na początku wypełniamy \( M \) zerami.
Jeśli \( a_i \in F, a_j \notin F \) to \( M_{i, j} = 1 \).
Dopóki macierz \( M \) się zmienia, to dla każdych dwóch stanów \( q_1, q_2 \) oraz litery \( a \) jeśli \( \hat \delta(q_1, a) = q_k, \hat \delta(q_2, a) = q_l \) oraz \( M_{k, l} = 1 \) to ustawiamy \( M_{i, j} := 1 \).

Jeśli po zakończeniu algorytmu jakieś 2 stany \(q_i, q_j\) są takie, że \(M_{i,j} = 0\) to wtedy wiemy, że \(q_i\) oraz \(q_j\) nie są rozróżnialne, a więc są A-równoważne. Zauważmy, że w ten sposób znajdujemy wszystkie stany które są sobie równoważne (działa to trochę jak taki Floyd-Warshall). Teraz co robimy, to kolapsujemy wszystkie równoważne stany w jeden wierzchołek. Okazuje się, że tak powstałe DFA jest minimalne pod względem liczby stanów (w porównaniu do innych DFA rozpoznających ten sam język).

\begin{theorem}
	Automat otrzymany w wyniku pracy algorytmu \(D\) jest najmniejszy spośród DFA rozpoznających dany język.
\end{theorem}
\begin{proof}
	Dowód nie wprost: załóżmy, że w wyniku minimalizacji jakiegoś automatu otrzymaliśmy jakiś automat \(A\), a istnieje DFA \(B\) takie, że \( L(A) = L(B) \) i B ma mniej stanów. Zdefiniujmy sobie DFA \(C\)\footnote{Tworzymy je tylko i wyłącznie z przyczyn formalnych, bo A-równoważnosć jest zdefiniowana dla określonego automatu.} takie, że jego stanem startowym jest \(s_B\), a \(Q_C = Q_A \cup Q_B\), funkcje przejść również ,,sumujemy''. Zauważmy, że dla automatu C:

	\[
		\forall_{q_A \in Q_A} \exists_{q_B \in Q_B}  \hspace{5pt} \textit{\(q_a\) i \(q_b\) są \(C\)-równoważne}
	\]

	Dowód tej obserwacji jest prosty: jako, że \(A\) i \(B\) akceptują te same języki, to na pewno \(s_A\) i \(s_B\) są \(C\)-równoważne. No ale w takim razie dla każdego \(a \in \Sigma\) jest tak, że \(\delta(s_A, a)\) i \(\delta(s_B, a)\) są również \(C\)-równoważne (i tak dalej). Stosując tę konstrukcję dalej, otrzymamy że dla każego stanu z \(A\) istnieje równoważny stan w \(B\).

	Na podstawie tej informacji wykonujemy \textit{potężną} obserwację: skoro, z założenia nie wprost, \(A\) miał więcej stanów w \(B\); to w połączeniu z powyższą obserwacją oznacza, że istnieją stany \(q_1, q_2 \in A\) takie, że oba z nich są równoważne z jakimś stanem \(q_3 \in B\). Ponieważ jednak równoważnosć jest przechodnia, mamy że \(q_1\) i \(q_2\) są wzajemnie równoważne. To prowadzi do sprzeczności, bo wiemy, że algorytm \(D\) będzie na tyle miły, że skolapsuje wszystkie stany które są wzajemnie równoważne. To kończy dowód.

\end{proof}