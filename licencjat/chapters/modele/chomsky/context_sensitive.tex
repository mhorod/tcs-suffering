\begin{definition}
	\textbf{Gramatyka kontekstowa} (Context-Sensitive Grammar) to czwórka \( G = (N, \Sigma, P, S) \) gdzie
	\begin{itemize}
		\item \( N \) to skończony zbiór zmiennych (nieterminale)
		\item \( \Sigma \) - alfabet (terminale)
		\item \( P \) - produkcje \( P \subseteq
		      (N \cup \Sigma)^* \, N \,
		      (N \cup \Sigma)^* \times
		      (N \cup \Sigma)^* \)
		\item \( S \in N \) - symbol startowy
	\end{itemize}
\end{definition}
gdzie każda produkcja jest postaci:
\[
	\alpha_1 A \alpha_2 \rightarrow_G \alpha_1 \gamma \alpha_2
\]
gdzie \( \alpha_1, \alpha_2, \gamma \in (N \cup \Sigma)^* \) oraz \( A \in N \)
oraz \( |\gamma| \geq 1\) \\
Praktycznie wszystko co poznaliśmy z bezkontekstowych znajduje się też tutaj.

\begin{definition}
	\textbf{Język  generowany} przez gramatykę G to oczywiście
	\[
		L(G) = \set{w \in \Sigma^* \mid S \rightarrow_G^* w}
	\]
\end{definition}

\begin{definition}
	Język jest kontekstowy (Context-Sensitive Language) jeśli jest generowany przez jakąś gramatykę kontekstową.
\end{definition}
