\label{dmt}
\begin{definition}

	\textbf{Maszyna Turinga} to tupla
	\[
		MT = (Q, \Sigma, \Gamma, \delta, q_0, \blank, F)
	\]
	gdzie
	\begin{itemize}
		\item \( Q \) -- skończony zbiór stanów
		\item \( \Sigma \subseteq \Gamma \) -- skończony alfabet wejściowy
		\item \( \Gamma \) -- skończony alfabet taśmowy
		\item \( \delta : Q \times \Gamma \rightarrow
		      Q \times \Gamma \times \set{-1, +1}
		      \) -- stan, litera \( \rightarrow \) nowy stan, zmiana litery, ruch głowicą (lewo, prawo)
		\item \( q_0 \) -- stan startowy
		\item \( \blank \in \Gamma \setminus \Sigma \) -- pusty symbol / zero (domyślny symbol taśmy)
		\item \( F \) -- zbiór stanów akceptujących
	\end{itemize}

	Ponadto zakładamy, że \(Q \cap \Gamma = \varnothing\).
\end{definition}

\begin{definition}
	\textbf{Opis chwilowy} (konfiguracja) Maszyny Turinga to słowo nad językiem \(\Gamma^* q\Gamma \Gamma^* \).
\end{definition}

\begin{definition}
	Konfiguracja startowa Maszyny Turinga to słowo postaci \(q_0w\blank\).
\end{definition}

\begin{definition}
	Definiujemy relację \( \vdash_M \) na konfiguracjach MT.

	Przejście w lewo ,,wewnątrz taśmy'':
	\[
		\alpha A q B \beta
		\vdash_M
		\alpha q' A B' \beta
	\]

	gdzie \(\alpha, \beta \in \Gamma^*\), \(A, B \in \Gamma\), \(q, q'\in Q\), oraz \(\delta(q, B) = (q', B', -1)\).

	Analogicznie definiujemy przejście w prawo ,,wewnątrz taśmy'':

	\[
		\alpha q B \beta
		\vdash_M
		\alpha B' q' \beta
	\]

	gdzie \(\alpha, \beta \in \Gamma^*\), \(B \in \Gamma\), \(q, q'\in Q\), oraz \(   \delta(q, B) = (q', B', 1) \).

	Definiujemy również ,,skrajne'' przejścia -- przejście w prawo ,,na skraju taśmy'':

	\[
		q B
		\vdash_M
		B' q' \blank
	\]

	gdzie \(B, B' \in \Gamma\), \(q, q' \in Q\), oraz \( \delta(q, B) = (q', B', 1) \)

	Przejście w lewo ,,na skraju taśmy'':
	\[
		q B
		\vdash_M
		q' \blank B'
	\]

	gdzie \(B, B' \in \Gamma\), \(q, q' \in Q\), oraz \( \delta(q, B) = (q', B', -1)\)


\end{definition}


\begin{definition}
	\textbf{Język akceptowany przez Maszynę Turinga} definiujemy jako
	\[
		L(M) = \set{w \in \Sigma^* \mid \exists_{q_f \in F} \exists_{\alpha, \beta \in \Gamma^*} q_0w\blank \vdash_M^* \alpha q_f \beta}
	\]
\end{definition}

