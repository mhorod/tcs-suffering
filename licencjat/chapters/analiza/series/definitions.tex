\begin{definition}[Nieskończony ciąg liczbowy]
    \textbf{Nieskończonym ciągiem liczbowym} (liczb rzeczywistych) nazywamy dowolną funkcję $a: \mathbb{N} \rightarrow \mathbb{R}$.
    
    Jej wartości \(a(1), a(2), \dots\) oznaczamy przez \(a_1, a_2, \dots\), sam ciąg oznaczamy jako \(\{a_n\}\).
\end{definition}

\begin{definition}[Ciąg sum częściowych]
    Dla ciągu \(\{a_n\}\) rozważamy ciąg \(\{s_n\}\) \textbf{sum częściowych}:
    \begin{align*} 
        s_1 &=  a_1 \\ 
        s_2 &=  a_1 + a_2 \\
        &\vdots\\
        s_n &=  \sum_{i=1}^n a_i
    \end{align*}

\end{definition}

\begin{definition}[Suma szeregu liczbowego]
\textbf{Sumą szeregu liczbowego} nazwiemy wyrażenie postaci
\[
\lim_{n \rightarrow \infty} s_n
\]
\end{definition}

\begin{definition}[Zbieżność szeregu]
    Szereg $\sum_{n=1}^\infty a_n$ jest \textbf{zbieżny}, jeżeli jego ciąg sum cząstkowych $\{s_n\}$ jest zbieżny (tj. ma granicę skończoną \(s\)).
    
    Szereg, który nie jest zbieżny, nazywamy \textbf{rozbieżnym}.
\end{definition}

\begin{example}
    Szereg \(\sum_{n=0}^\infty \frac{1}{2^n}\) jest zbieżny, bo:
    \[
    s_n = \sum_{k=0}^n \frac{1}{2^k} = 1 - \frac{1}{2^n}
    \]
    Wobec tego mamy $\lim_{n\to\infty}{s_n} = 1$
\end{example}

\begin{definition}[Bezwzględna zbieżność]
    Szereg $\sum_{n=1}^\infty a_n$ jest \textbf{bezwzględnie zbieżny} jeśli szereg $\sum_{n=1}^\infty |a_n|$ jest zbieżny.
\end{definition}