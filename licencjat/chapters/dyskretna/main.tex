{
\makeatletter
\def\input@path{{../dyskretna}}
\makeatother
\graphicspath{{../dyskretna}}

% This has to be done, because we input another suffering with different sectioning scheme.
\let\realsection\section
\let\realsubsection\subsection
\let\section\subsection
\let\subsection\subsubsection
\let\subsubsection\paragraph

\chapter{Matematyka dyskretna}
\begin{definition}
    Mówimy, że maszyna M (deterministyczna lub nie) \textbf{działa w czasie} \( T : \natural \rightarrow \natural \) jeśli dla każdej konfiguracji startowej \( q_0 w \) każde obliczenie jest akceptujące lub odrzucające i ma długość co najwyżej \( T(\abs{w}) \)
\end{definition}

\begin{definition}
    Mówimy, że funkcja \( f \) jest \textbf{redukcją wielomianową} (Karpa) jeśli istnieje wielomian \( p \) oraz Maszyna Turinga obliczająca \( f \) w czasie \( p \). 
    
    Jeśli istnieje redukcja wielomianowa z języka \( L_1 \) do języka \( L_2 \) to zapisujemy to jako \( L_1 \leq_p L_2 \)
\end{definition}

\begin{definition}
    Dla klasy problemów \( C \subseteq R \) problem \( L \) jest \textbf{C-trudny} jeśli \( \forall_{L' \in C} L' \leq_p L \)
\end{definition}
\begin{definition}
    Dla klasy problemów \( C \subseteq R \) problem \( L \) jest \textbf{C-zupełny} jeśli jest \(C\)-trudny i \( L \in C \)
\end{definition}

\begin{lemma}
    Jeśli \( L_1 \) jest C-trudny i \( L_1 \leq_p L_2 \) to \( L_2 \) też jest C-trudny. 
\end{lemma}

\begin{proof}
    Skoro \( L_1 \) jest C-trudny to znaczy, że dla każdego \( L' \in C \) mamy redukcję \( f \) z \( L' \) do \( L_1 \).
    Mamy też redukcję \( g \) z \( L_1 \) do \( L_2 \).
    
    Składając \( g \circ f \) dostajemy redukcję z \( L' \)  do \( L_2 \) co dowodzi, że \( L_2 \) jest trudniejszy niż dowolny \( L' \in C \).
\end{proof}


\realsection{Zasada włączeń i wyłączeń. Przykłady zastosowań}
\section{Rozwiązywanie rekurencji liniowych}
\epigraph{I can elaborate: zrobiłam zadanka, zobaczyłam tworzące, stwierdziłam, że chce mi się spać, poszłam sobie}{\textit{Studentka TCSu o zadaniach z funkcji tworzących na kolokwium}}


\subsection{Rozkład na ułamki proste}
To nie jest formalny dowód ani formalna własność ani nic, bardziej schemat postępowania przy rozkładzie na ułamki proste. Sam dowód tego, że rozkład na ułamki proste istnieje, to \textit{sprowadź do wspólnego mianownika i zobacz co Ci wyszło}.
Jeżeli \(deg(P(x)) < deg(Q(x))\) i \(Q(x) = (x-a)^n \cdot (x-b)^k\) to:
\begin{equation*}
	\frac{P(x)}{Q(x)} = \frac{P(x)}{(x-a)^n \cdot (x-b)^k} = \frac{A_1}{x-a} + \frac{A_2}{(x-a)^2} + \dots + \frac{A_n}{(x-a)^n} + \frac{B_1}{x-b} + \frac{B_2}{(x-b)^2} + \dots + \frac{B_k}{(x-b)^k}
\end{equation*}

Oczywiście ten schemat można rozszerzać na więcej śmiesznych rzeczy w mianowniku, ale chyba widać o co chodzi.


\section{Ciąg Fibbonaciego}
\begin{theorem}[Wzór Bineta]
	\begin{equation}
		f_n = \frac{1}{\sqrt{5}} \cdot \left( \left(\frac{1 + \sqrt{5}}{2}\right)^{n} - \left(\frac{1 - \sqrt{5}}{2}\right)^{n} \right)
	\end{equation}
\end{theorem}

\begin{proof}
	Rozpisujemy sobie funkcję tworzącą ciągu \(f_n\):

	\begin{equation*}
		F(x) = f_0 + f_1 \cdot x + f_2 \cdot x^2 + f_3 \cdot x^3 \dots =
	\end{equation*}
	\begin{equation*}
		= f_0 + f_1 \cdot x + (f_0 + f_1) \cdot x^2 + (f_1 + f_2) \cdot x^3 + \dots =
	\end{equation*}
	\begin{equation*}
		= f_0 + f_1 \cdot x + f_0 \cdot x^2 + f_1 \cdot x^2 + f_1 \cdot x^3 + f_2 \cdot x^3 + \dots =
	\end{equation*}
	\begin{equation*}
		= f_0 + f_1 \cdot x + f_0 \cdot x^2 + f_1 \cdot x^3 + \dots + f_1 \cdot x^2 +  f_2 \cdot x^3 + \dots =
	\end{equation*}
	\begin{equation*}
		= f_0 + f_1 \cdot x + x^2 \cdot (f_0 + f_1 \cdot x + \dots) + x \cdot (f_1 \cdot x +  f_2 \cdot x^2 + \dots) =
	\end{equation*}
	\begin{equation*}
		= f_0 + f_1 \cdot x + x^2 \cdot F(x) + x \cdot (F(x) - f_0) =
	\end{equation*}
	\begin{equation*}
		= 0 + 1 \cdot x + x^2 \cdot F(x) + x \cdot (F(x) - 0) =
	\end{equation*}
	\begin{equation*}
		= x + x^2 \cdot F(x) + x \cdot F(x)
	\end{equation*}

	W takim razie mamy, że:
	\begin{equation*}
		F(x) = x + x^2 \cdot F(x) + x \cdot F(x)
	\end{equation*}
	\begin{equation*}
		F(x) -  x^2 \cdot F(x) - x \cdot F(x)  = x
	\end{equation*}
	\begin{equation*}
		F(x) \cdot (1 - x^2 - x) = x
	\end{equation*}
	\begin{equation*}
		F(x) = \frac{x}{-x^2 -x + 1}
	\end{equation*}

	Mianownik możemy rozbić (za pomocą liczenia jakichś delt czy coś):
	\begin{equation*}
		F(x) = \frac{x}{(-1) \cdot \left(x - \left(- \frac{1 + \sqrt{5}}{2}\right)\right) \cdot \left(x - \left(- \frac{1 - \sqrt{5}}{2}\right)\right)}
	\end{equation*}

	Nie no, serio, jeśli ktoś myśli że będę TeXować te przekształcenia to się myli. Powinno wyjść po przekształceniach że:
	\begin{equation*}
		F(x) = \frac{x}{(1-ax) \cdot (1-bx)}
	\end{equation*}
	gdzie \(a = \frac{1 + \sqrt{5}}{2}, b=\frac{1 - \sqrt{5}}{2}\)

	Dalej rozbijamy na ułamki proste:
	\begin{equation*}
		F(x) = \frac{A}{1-ax} + \frac{B}{1-bx}
	\end{equation*}
	\(A\) powinno wyjść \(\frac{1}{\sqrt{5}}\), \(B\) powinno wyjść \(- \frac{1}{\sqrt{5}}\).

	Odwijamy każdą z tych funkcji tworzących z osobna, korzystając ze wzoru podanego we wcześniejszym rozdziale i otrzymujemy wzór.
\end{proof}


\section{Ciąg Catalana}
Liczba Catalana jest to liczba ścieżek długości \(2n\) w kwadracie \(n \times n\) ,,poniżej'' przekątnej (lub na jej poziomie), idących za każdym razem jednostkę do góry lub jednostkę w prawo. Ścieżki takie nazywamy ścieżkami Dycka. Niezwykle formalna definicja. To jest jedna z tych rzeczy, które chyba po prostu trzeba narysować.

\begin{figure}[h]
	\centering
	\includegraphics[scale=0.5]{images/catalan/all_paths_1.png}
  \caption{Ścieżki Dycka długości 2; \(c_1 = 1\)}
\end{figure}

\begin{figure}[h]
	\centering
	\includegraphics[scale=0.5]{images/catalan/all_paths_2.png}
  \caption{Ścieżki Dycka długości 4; \(c_2 = 2\)}
\end{figure}

\begin{figure}[ht]
	\centering
	\includegraphics[scale=0.5]{images/catalan/all_paths_3.png}
  \caption{Ścieżki Dycka długości 6; \(c_3 = 5\)}
\end{figure}


\subsection{Wzór kombinatoryczny}
\begin{theorem}[Wzór kombinatoryczny na liczby Catalana]
	\begin{equation}
		c_n = \frac{1}{n+1} \cdot \binom{2n}{n}
	\end{equation}
\end{theorem}

Mamy sobie nasz kwadrat \(n \times n\). Przekątną możemy opisać tak jakby wzorem \(y = x\) (tak intuicyjnie, bo nie działamy w żadnym układzie współrzędnych, bla bla bla). Robimy sobie teraz prostą \(y = x+1\), idącą jakby ,,o jednostkę wyżej''. Zauważamy, że jeśli jakaś ścieżka przekracza linię naszej przekątnej, to musi ,,dotknąć'' linii \(y = x+1\). \textit{To widać}. Teraz wpadamy na świetny pomysł; jeśli jakaś ścieżka idąca po tym kwadracie ,,spotyka się'' z \(y = x+1\), to od tego momentu odbijamy ją symetrycznie względem \(y = x+1\). Zauważamy, że ścieżka ta (po odbiciu) skończy się w punkcie \((n-1, n+1)\) zamiast w \((n,n)\). Fakt ten dowodzimy stosując dowód przez rysowanie.

\begin{figure}[ht]
	\centering
	\includegraphics[scale=0.5]{images/catalan/path_with_reflection_1.png}
	\includegraphics[scale=0.5]{images/catalan/path_with_reflection_2.png}

	\caption{Przykłady odbicia niepoprawnej ścieżki}
\end{figure}

Zauważamy fascynujący fakt, mianowicie dwie różne ścieżki będą mieć 2 różne odbicia, a więc nasze przekształcenie jest iniektywne. Ponadto, jak sobie zobaczymy jakąkolwiek ścieżkę zaczynającą się w \((0,0)\), ale kończącą się w \((n-1,n+1)\), to jesteśmy w stanie zobaczyć gdzie pierwszy raz przecina się z \(y = x+1\), a następnie ją odbić, otrzymując ścieżkę idącą do \((n,n)\) i niebędącą ścieżką Dycka, której odbicie daje wyjściową ścieżkę. Zatem odbijanie jest suriektywne. A to oznacza tylko jedną rzecz: bijekcję między ścieżkami które ,,nie są catalanowe'', a ścieżkami ,,odbitymi''.

Wszystkich możliwych ścieżek od \((0,0)\) do \((n,n)\) mamy \(\binom{2n}{n}\), bo długość naszej drogi ma \(2n\) i wybieramy sobie \(n\) miejsc gdzie idziemy w prawo. Wszystkich możliwych ścieżek od \((0,0)\) do \((n-1,n+1)\) (czyli tych które są ,,złe'') mamy \(\binom{2n}{n-1}\), bo, analogicznie, ścieżka jest długości \(2n\) ale w prawo idziemy \(n-1\) razy. To prowadzi nas do wyniku:
\begin{equation*}
	\begin{split}
		c_n
		&= \binom{2n}{n} - \binom{2n}{n-1} \\
		&= \frac{(2n)!}{n! \cdot n!} - \frac{(2n)!}{(n-1)! \cdot (n+1)!} \\
		&= \frac{(n+1) \cdot (2n)!}{n! \cdot (n+1)!} - \frac{n \cdot (2n)!}{n! \cdot (n+1)!}\\
		&= \frac{(2n)!}{n! \cdot (n+1)!} \\
		&= \frac{1}{n+1} \cdot \frac {(2n)!}{n! \cdot n!} \\
		&= \frac{1}{n+1} \cdot \binom{2n}{n}
	\end{split}
\end{equation*}
\subsection{Zależność rekurencyjna}

\begin{theorem}[Wzór rekurencyjny na liczby Catalana]
	\begin{equation}
		c_n = c_{0} \cdot c_{n-1} + c_{1} \cdot c_{n - 2} + \dots + c_{n-1} \cdot c_0
	\end{equation}
\end{theorem}

\begin{proof}
	Znowuż mamy kwadrat \(n \times n\), ale tym razem dorysowujemy sobie prostą \(y = x - 1\). Każda ścieżka przetnie kiedyś tę linię i każda ścieżka dotknie kiedyś przekątnej \(y = x\) (można to udowodnić machając i pokazując na rysunek). Rzecz teraz ma się tak, że jeśli po ,,spotkaniu się'' z \(y = x - 1\) idziesz do góry, to potem musisz odbić w prawo (lub w skrajnym przypadku skończyłeś poprawną ścieżkę). Jednocześnie pierwszy wybór kierunku (tzn. ten w punkcie \((0,0)\) zawsze jest ,,w prawo'', bo jeśli ktoś pójdzie ,,do góry'' to znajdzie się w \((0,1)\), powyżej przekątnej \(y = x\)).

	Bierzemy sobie zatem pierwsze miejsce gdzie spotkałeś się z \(y = x\) i zauważamy, że jeśli dane jest ono jakimiś współrzędnymi \((i,i)\) to przecięliśmy \(y = x-1\) w \((i,i-1)\). Ponadto, ścieżka którą szliśmy od punktu \((1,0)\) do \((i,i-1)\) tak naprawdę jest ścieżką Dycka w kwadracie od punktów \((1,0)\), \((i, i-1)\) (kwadrat ten ma długość \(i-1\)). Ależ plot twist! Ścieżka którą idziemy od punktu \((i,i)\) do \((n,n)\) jest zaś już po prostu ścieżką Dycka w kwadracie o długości boku \(n-i\). Ścieżki te są od siebie niezależne i w ogóle, a długości tych ,,kwadratów catalanowych'' sumują się do \(i - 1 + n - i = n - 1\), więc teraz możemy zmajstrować wzór (w zależności od długości boków kwadratów, które z kolei są dyktowane tym kiedy się ,,spotkamy'' z \(y = x\)):
	\begin{equation*}
		c_n = \sum_{i = 0}^{n-1} c_i \cdot c_{n-1-i}
	\end{equation*}
	Co już można odwinąć do postaci która była w twierdzeniu.

	\begin{figure}[H]
		\centering
		\includegraphics[scale=0.5]{images/catalan/recursive_construction_1.png}
		\includegraphics[scale=0.5]{images/catalan/recursive_construction_2.png}

		\caption{Przykłady ,,podzielenia'' poprawnej ścieżki Dycka na podścieżki}
	\end{figure}

\end{proof}


\section{Zliczanie podziałów}

Chcemy pokazać fajny algorytm zliczania wszystkich podziałów liczby \(n\).

Oznaczmy liczbę wszystkich podziałów liczby \(n\) jako \(p(n)\). Jako ,,podziały liczby \(n\)'' mam na myśli liczbę sposobów na podzielenie liczby \(n\) na ileś składników (niezerowych), np. liczbę \(2\) mogę rozłożyć na \(1 + 1\) albo po prostu na \(2\) (i w sumie to tyle).  Funkcja tworząca ciągu \(p_n\) to: \begin{equation*}
	P(x) = (1 + x + x^2 + x^3 + \dots) \cdot (1 + x^2 + x^4 + x^6 + \dots) \cdot (1 + x^3 + x^6 + x^9 + \dots) \dots
\end{equation*}

Pierwszy nawias odpowiada wybraniu jedynki do podziału (i temu ile razy ją bierzemy), drugi dwójki, trzeci trójki, etc.

Oczywiście przy \(x^n\) będziemy mieli \(p_n\), jak to działa w funkcjach tworzących (i mam nadzieję, że widać dlaczego). Zapisujemy \(P(x)\) w fajniejszej postaci:

\begin{equation*}
	P(x) = \frac{1}{1-x} \cdot \frac{1}{1 - x^2} \cdot \frac{1}{1-x^3} \dots
\end{equation*}

Definiuję sobie \(Q(x) = (1-x) \cdot (1-x^2) \cdot (1-x^3) \dots\). Zauważam, że \(P(x) \cdot Q(x) = 1\), czyli \(Q(x)\) jest funkcją odwrotną do \(P(x)\). Okazuje się teraz, że \(Q(x)\) jest funkcją tworzącą pewnego śmiesznego ciągu, który sobie zaraz pokażemy.

Póki co musimy wprowadzić oznaczenia:
\begin{enumerate}
	\item \(e_n\) jest to liczba podziałów liczby \(n\) na parzystą liczbę składników parami różnych,
	\item \(o_n\) jest to liczba podziałów liczby \(n\) na nieparzystą liczbę składników parami różnych.
\end{enumerate}
Jak wszyscy powinniśmy już wiedzieć, funkcja tworząca ciągu \(e_n + o_n\) (czyli po prostu wszystkich podziałów \(n\) ze składnikami parami różnymi) wygląda tak:
\begin{equation*}
	(1+x) \cdot (1 + x^2) \cdot (1+x^3) \dots
\end{equation*}
Ten fakt do niczego nam się w sumie nie przyda, ale może pomóc zrozumieć co zaraz się stanie.

Możemy sobie teraz podumać, jaka jest funkcja tworząca ciągu \(e_n - o_n\). Otóż pojawia się tu plot twist, bo funkcja tworząca tego ciągu to po prostu \(Q(x)\):
\begin{equation*}
	(1-x) \cdot (1-x^2) \cdot (1-x^3) \dots
\end{equation*}

Działa to tak jak w powyższym przykładzie, z tym że jeśli wybraliśmy nieparzyście wiele składników to będzie nieparzyście wiele minusów i się ,,odejmie'' od współczynnika przy \(x^n\), a jeśli będzie parzyście wiele to się ,,doda''. Innymi słowy, do współczynnika przy \(x^n\) doda się 1 za każdy możliwy podział na parzyście wiele parami różnych składników, a odejmie się 1 za każdy możliwy podział na nieparzyście wiele parami różnych składników, czyli to co chcemy. Nie do końca mam pomysł jak to formalnie wytłumaczyć, więc proszę użyć swojej intuicji™.

Po co to wszystko? Okazuje się, że ciąg \(q_n = e_n - o_n\) ma pewne śmieszne własności (które niestety będzie trzeba udowodnić, brace yourselves).

\begin{theorem}[Eulera]
	\begin{equation}
		q_n = \begin{cases}
			0, \hspace{5pt} \mathrm{gdy} \hspace{5pt} n \not = \frac{(3 \cdot k \pm 1) \cdot k }{2} \\
			(-1)^k \hspace{5pt} \mathrm{wpp.}                                                       \\
		\end{cases}
	\end{equation}

\end{theorem}

\begin{proof}
	Zrobimy sobie przekształcenie \(f\), które przesyła prawie (dlaczego prawie to dojdziemy do tego za chwilę) każdy podział na \(n\) składników parami różnych na inny podział na \(n\) składników parami różnych (bijektywnie). Ktoś powie że sobie zrobiłem świetną bijekcję idącą z pewnego zbioru w samego siebie, but hear me out: ta bijekcja będzie mieć tę śmieszną własność, że jeśli podział był na parzyście wiele składników to będzie przesłany na nieparzyście wiele, a jeśli na nieparzyście wiele to będzie przesłany na parzyście wiele składników. To będzie fajne, bo pokażemy sobie że jest ich tyle samo (poza przypadkami gdzie definicja tej funkcji się popsuje, ale o tym za chwilę).

	Generalnie to oznaczmy sobie najmniejszy składnik w podziale \(P\) jako \(a\). Ponadto, zdefiniujmy sobie zbiór \(X\), taki że zawiera on największe składniki podziału \(P\), takie że każde dwa sąsiednie różnią się o jeden. Innymi słowy, jeśli podział \(P = (\lambda_1, \lambda_2, \lambda_3, \dots, \lambda_k)\), to \(X =\{\lambda_1, \lambda_2, \lambda_3, \dots, \lambda_d\}\), gdzie \(d\) jest największą liczbą taką, że kolejne składniki różnią się o 1  (zakładamy, że \(\lambda_1 > \lambda_2 > \dots > \lambda_k\)).

	Teraz jak mamy te zbiory zdefiniowane to możemy robić śmieszne rzeczy. Jeśli \(|X| < a - 1\), to możemy przerobić nasz podział, odejmując od każdego elementu z \(X\) 1, i dorzucając nowy element do podziału, taki że równy jest on moc \(|X|\). Otrzymaliśmy oczywiście poprawny podział (niektórym może pomóc dowód przez rysowanie).

	Dlaczego \(|X| < a - 1\), a nie po prostu \(|X| < a\)? Otóż przychodzi tutaj pewien śmieszny problem, mianowicie może być tak, że składnik podziału o wartości \(a\) ,,wpadł'' do \(X\). W takim przypadku bijekcja nam się kompletnie popsuje i wtedy jej definiujemy (ale jeszcze do tego wrócimy). Natomiast jeśli \(a\) nie należy do \(|X|\) to nasza bijekcja nadal działa. Fajnie.

	Czyli reasumując: jeśli \(|X| < a - 1\) lub (\(|X| = a - 1\) i \(a \not \in X\)) od każdego składnika z \(|X|\) odejmujemy 1 i majstrujemy nowy składnik, który wrzucamy pod składnik o wartości \(a\), który uprzednio był najmniejszy.

	\begin{figure}[H]
		\centering
		\includegraphics{images/case2.png}
		\caption{Wizualizacja przekształcenia (diagram Ferrersa). 2 ,,górne'' składniki różnią się o 1, trzeci już różni się od nich o 2; \(|X| = 2\), \(a=3\).}
	\end{figure}

	Zasadniczo to samo będziemy czynić (ale w drugą stronę), gdy okaże się że \(a < |X| \). Ordynarnie \textit{wywalam} składnik \(a\) i do odpowiedniej liczby elementów z \(X\)  ,,dodaję'' 1, tak by się wyrównało. Należy zauważyć, że być może nie wszystkie elementy z \(X\) będą mieć coś do siebie dodane, ale to mi nic nie psuje. W sumie też fajnie byłoby dodać, że dodaję te jedynki najpierw największym składnikom; inaczej mogłoby to się popsuć.

	Co dzieje się, gdy \(a = |X|\)? Jeśli \(a \in X\) to jest mi smutno, w przeciwnym razie mogę zrobić to samo co robiłem wcześniej i wszystko działa jak powinno.

	\begin{figure}[H]
		\centering
		\includegraphics{images/case_1.png}
		\caption{Wizualizacja przekształcenia (diagram Ferrersa). 3 ,,górne'' składniki różnią się o 1 więc należą do \(X\). \(|X| = 3\), \(a = 2\), więc dwóm największym elementom dodajemy 1, a składnik \(a\) usuwamy.}
	\end{figure}

	Zostają więc 2 przypadki, gdy coś może się popsuć:
	\begin{enumerate}
		\item \(|X| = a - 1\), \(a \in X\)
		      \begin{figure}[H]
			      \centering
			      \includegraphics{images/irytujacy_1.png}
			      \caption{Gdy \(|X| = a - 1\) i składnik \(a\) jest w \(X\); widać, że nic nie możemy z tym zrobić.}
		      \end{figure}

		\item \(|X| = a\), \(a \in x\)
		      \begin{figure}[H]
			      \centering
			      \includegraphics{images/irytujacy_2.png}
			      \caption{Gdy \(|X| = a\) i składnik \(a\) jest w \(X\); również widać, że nasze przekształcenie nie zadziała.}
		      \end{figure}
	\end{enumerate}

	Zauważmy, że sytuacja gdy składnik \(a\) jest w \(X\) jest bardzo dziwną sytuacją generalnie, bo jest to najmniejszy składnik; z definicji \(X\) mamy wtedy, że wszystkie kolejne składniki w \(P\) różnią się o dokładnie 1. Na podstawie tej obserwacji możemy już dokładnie powiedzieć, jakiej postaci musi być \(n\), by miało taki ,,złośliwy'' podział:

	\begin{enumerate}
		\item Gdy \(|X| = a - 1\), \(a \in X\), to \(n\) musi dla jakiegoś \(k\) być postaci \((k + 1) + (k + 2) + \dots + 2k \) (\(|X| =k, a = k+1\), wszystko się zgadza)
		\item Gdy \(|X| = a\), \(a \in x\), to \(n\) musi dla jakiegoś \(k\) być postaci \(k + (k+1) + (k+2) + \dots + (2k - 1)\) (\(|X| = k\), \(a = k\), ponownie wszystko gra)
	\end{enumerate}

	Jak zastosujemy matematykę mniej dyskretną by wysumować te nawiasy, dostaniemy że \(n\) aby miało irytujący podział to musi być postaci \(\frac{k \cdot (3k+1)}{2}\) lub \({k \cdot (3k-1)}{2}\). Jednocześnie nie ma takiego naturalnego \(k\), że wartości te są sobie równe, więc jeśli \(n\) ma irytujący podział, to ma go tylko jednego. Wtedy nie możemy przerzucić tylko jednego podziału na inny (inne są ze sobą w bijekcji) więc \(e_n - o_n = (-1)^k\) (jeśli \(k\) jest parzyste to irytujący podział ma parzyście wiele składników, a w przeciwnym razie nieparzyście wiele). Jeśli irytujący podział nie występuje, \(e_n = o_n\) z bijekcji którą pokazaliśmy. Fajnie.
\end{proof}

Dobra, ale wróćmy do tego cośmy chcieli udowodnić na samym początku. Co w ogóle wynika z tego twierdzenia Eulera? No w sumie to bardzo dużo, bo jak mamy \(q_n = e_n - o_n\) i \(Q(x)\) jest jego funkcją tworzącą:
\begin{equation*}
	Q(x) = q_0 + q_1 \cdot x + q_2 \cdot x^2 + q_3 \cdot x^3 + \dots
\end{equation*}
Ale znamy wartości współczynników \(q_i\) z twierdzenia Eulera:
\begin{equation*}
	Q(x) = 1 - x - x^2 + x^5 + x^7 - x^{12} - x^{15} + x^{22} + x^{26} + \dots
\end{equation*}
Zauważmy, że współczynników które nie są zerowe jest tylko jakoś \(O(\sqrt{n})\), czyli dosyć mało.

Pamiętajmy, że \(P(x) \cdot Q(x) = 1\), czyli że ciąg który wyjdzie po ich wymnożeniu będzie wyglądać tak: \((1,0,0,0, \dots)\) Ponieważ mnożenie w funkcjach tworzących działa jakoś tak, że w wynikowym ciągu (nazwijmy go \(r\)) element \(r_n\) można obliczyć w ten sposób:
\begin{equation*}
	r_n = \sum_{i=0}^{n} p_i \cdot q_{n-i}
\end{equation*}

I wiemy że w naszym przypadku \(r_n = 0\) dla \(n > 1\), to mamy że: \begin{equation*}
	0 = p_n - p_{n-1} - p_{n-2} + p_{n-5} + p_{n-7} - p_{n-12} - \dots
\end{equation*}
To teraz \(p_n\) przerzucamy na drugą stronę i mnożymy stronami razy \(-1\) i mamy wzór na \(p_n\), które możemy obliczyć w \(O(\sqrt{n})\). No i fajnie.



\realsection{Porządki częściowe, Twierdzenia Dilwortha i Spernera}
\section{Rozwiązywanie rekurencji liniowych}
\epigraph{I can elaborate: zrobiłam zadanka, zobaczyłam tworzące, stwierdziłam, że chce mi się spać, poszłam sobie}{\textit{Studentka TCSu o zadaniach z funkcji tworzących na kolokwium}}


\subsection{Rozkład na ułamki proste}
To nie jest formalny dowód ani formalna własność ani nic, bardziej schemat postępowania przy rozkładzie na ułamki proste. Sam dowód tego, że rozkład na ułamki proste istnieje, to \textit{sprowadź do wspólnego mianownika i zobacz co Ci wyszło}.
Jeżeli \(deg(P(x)) < deg(Q(x))\) i \(Q(x) = (x-a)^n \cdot (x-b)^k\) to:
\begin{equation*}
	\frac{P(x)}{Q(x)} = \frac{P(x)}{(x-a)^n \cdot (x-b)^k} = \frac{A_1}{x-a} + \frac{A_2}{(x-a)^2} + \dots + \frac{A_n}{(x-a)^n} + \frac{B_1}{x-b} + \frac{B_2}{(x-b)^2} + \dots + \frac{B_k}{(x-b)^k}
\end{equation*}

Oczywiście ten schemat można rozszerzać na więcej śmiesznych rzeczy w mianowniku, ale chyba widać o co chodzi.


\section{Ciąg Fibbonaciego}
\begin{theorem}[Wzór Bineta]
	\begin{equation}
		f_n = \frac{1}{\sqrt{5}} \cdot \left( \left(\frac{1 + \sqrt{5}}{2}\right)^{n} - \left(\frac{1 - \sqrt{5}}{2}\right)^{n} \right)
	\end{equation}
\end{theorem}

\begin{proof}
	Rozpisujemy sobie funkcję tworzącą ciągu \(f_n\):

	\begin{equation*}
		F(x) = f_0 + f_1 \cdot x + f_2 \cdot x^2 + f_3 \cdot x^3 \dots =
	\end{equation*}
	\begin{equation*}
		= f_0 + f_1 \cdot x + (f_0 + f_1) \cdot x^2 + (f_1 + f_2) \cdot x^3 + \dots =
	\end{equation*}
	\begin{equation*}
		= f_0 + f_1 \cdot x + f_0 \cdot x^2 + f_1 \cdot x^2 + f_1 \cdot x^3 + f_2 \cdot x^3 + \dots =
	\end{equation*}
	\begin{equation*}
		= f_0 + f_1 \cdot x + f_0 \cdot x^2 + f_1 \cdot x^3 + \dots + f_1 \cdot x^2 +  f_2 \cdot x^3 + \dots =
	\end{equation*}
	\begin{equation*}
		= f_0 + f_1 \cdot x + x^2 \cdot (f_0 + f_1 \cdot x + \dots) + x \cdot (f_1 \cdot x +  f_2 \cdot x^2 + \dots) =
	\end{equation*}
	\begin{equation*}
		= f_0 + f_1 \cdot x + x^2 \cdot F(x) + x \cdot (F(x) - f_0) =
	\end{equation*}
	\begin{equation*}
		= 0 + 1 \cdot x + x^2 \cdot F(x) + x \cdot (F(x) - 0) =
	\end{equation*}
	\begin{equation*}
		= x + x^2 \cdot F(x) + x \cdot F(x)
	\end{equation*}

	W takim razie mamy, że:
	\begin{equation*}
		F(x) = x + x^2 \cdot F(x) + x \cdot F(x)
	\end{equation*}
	\begin{equation*}
		F(x) -  x^2 \cdot F(x) - x \cdot F(x)  = x
	\end{equation*}
	\begin{equation*}
		F(x) \cdot (1 - x^2 - x) = x
	\end{equation*}
	\begin{equation*}
		F(x) = \frac{x}{-x^2 -x + 1}
	\end{equation*}

	Mianownik możemy rozbić (za pomocą liczenia jakichś delt czy coś):
	\begin{equation*}
		F(x) = \frac{x}{(-1) \cdot \left(x - \left(- \frac{1 + \sqrt{5}}{2}\right)\right) \cdot \left(x - \left(- \frac{1 - \sqrt{5}}{2}\right)\right)}
	\end{equation*}

	Nie no, serio, jeśli ktoś myśli że będę TeXować te przekształcenia to się myli. Powinno wyjść po przekształceniach że:
	\begin{equation*}
		F(x) = \frac{x}{(1-ax) \cdot (1-bx)}
	\end{equation*}
	gdzie \(a = \frac{1 + \sqrt{5}}{2}, b=\frac{1 - \sqrt{5}}{2}\)

	Dalej rozbijamy na ułamki proste:
	\begin{equation*}
		F(x) = \frac{A}{1-ax} + \frac{B}{1-bx}
	\end{equation*}
	\(A\) powinno wyjść \(\frac{1}{\sqrt{5}}\), \(B\) powinno wyjść \(- \frac{1}{\sqrt{5}}\).

	Odwijamy każdą z tych funkcji tworzących z osobna, korzystając ze wzoru podanego we wcześniejszym rozdziale i otrzymujemy wzór.
\end{proof}


\section{Ciąg Catalana}
Liczba Catalana jest to liczba ścieżek długości \(2n\) w kwadracie \(n \times n\) ,,poniżej'' przekątnej (lub na jej poziomie), idących za każdym razem jednostkę do góry lub jednostkę w prawo. Ścieżki takie nazywamy ścieżkami Dycka. Niezwykle formalna definicja. To jest jedna z tych rzeczy, które chyba po prostu trzeba narysować.

\begin{figure}[h]
	\centering
	\includegraphics[scale=0.5]{images/catalan/all_paths_1.png}
  \caption{Ścieżki Dycka długości 2; \(c_1 = 1\)}
\end{figure}

\begin{figure}[h]
	\centering
	\includegraphics[scale=0.5]{images/catalan/all_paths_2.png}
  \caption{Ścieżki Dycka długości 4; \(c_2 = 2\)}
\end{figure}

\begin{figure}[ht]
	\centering
	\includegraphics[scale=0.5]{images/catalan/all_paths_3.png}
  \caption{Ścieżki Dycka długości 6; \(c_3 = 5\)}
\end{figure}


\subsection{Wzór kombinatoryczny}
\begin{theorem}[Wzór kombinatoryczny na liczby Catalana]
	\begin{equation}
		c_n = \frac{1}{n+1} \cdot \binom{2n}{n}
	\end{equation}
\end{theorem}

Mamy sobie nasz kwadrat \(n \times n\). Przekątną możemy opisać tak jakby wzorem \(y = x\) (tak intuicyjnie, bo nie działamy w żadnym układzie współrzędnych, bla bla bla). Robimy sobie teraz prostą \(y = x+1\), idącą jakby ,,o jednostkę wyżej''. Zauważamy, że jeśli jakaś ścieżka przekracza linię naszej przekątnej, to musi ,,dotknąć'' linii \(y = x+1\). \textit{To widać}. Teraz wpadamy na świetny pomysł; jeśli jakaś ścieżka idąca po tym kwadracie ,,spotyka się'' z \(y = x+1\), to od tego momentu odbijamy ją symetrycznie względem \(y = x+1\). Zauważamy, że ścieżka ta (po odbiciu) skończy się w punkcie \((n-1, n+1)\) zamiast w \((n,n)\). Fakt ten dowodzimy stosując dowód przez rysowanie.

\begin{figure}[ht]
	\centering
	\includegraphics[scale=0.5]{images/catalan/path_with_reflection_1.png}
	\includegraphics[scale=0.5]{images/catalan/path_with_reflection_2.png}

	\caption{Przykłady odbicia niepoprawnej ścieżki}
\end{figure}

Zauważamy fascynujący fakt, mianowicie dwie różne ścieżki będą mieć 2 różne odbicia, a więc nasze przekształcenie jest iniektywne. Ponadto, jak sobie zobaczymy jakąkolwiek ścieżkę zaczynającą się w \((0,0)\), ale kończącą się w \((n-1,n+1)\), to jesteśmy w stanie zobaczyć gdzie pierwszy raz przecina się z \(y = x+1\), a następnie ją odbić, otrzymując ścieżkę idącą do \((n,n)\) i niebędącą ścieżką Dycka, której odbicie daje wyjściową ścieżkę. Zatem odbijanie jest suriektywne. A to oznacza tylko jedną rzecz: bijekcję między ścieżkami które ,,nie są catalanowe'', a ścieżkami ,,odbitymi''.

Wszystkich możliwych ścieżek od \((0,0)\) do \((n,n)\) mamy \(\binom{2n}{n}\), bo długość naszej drogi ma \(2n\) i wybieramy sobie \(n\) miejsc gdzie idziemy w prawo. Wszystkich możliwych ścieżek od \((0,0)\) do \((n-1,n+1)\) (czyli tych które są ,,złe'') mamy \(\binom{2n}{n-1}\), bo, analogicznie, ścieżka jest długości \(2n\) ale w prawo idziemy \(n-1\) razy. To prowadzi nas do wyniku:
\begin{equation*}
	\begin{split}
		c_n
		&= \binom{2n}{n} - \binom{2n}{n-1} \\
		&= \frac{(2n)!}{n! \cdot n!} - \frac{(2n)!}{(n-1)! \cdot (n+1)!} \\
		&= \frac{(n+1) \cdot (2n)!}{n! \cdot (n+1)!} - \frac{n \cdot (2n)!}{n! \cdot (n+1)!}\\
		&= \frac{(2n)!}{n! \cdot (n+1)!} \\
		&= \frac{1}{n+1} \cdot \frac {(2n)!}{n! \cdot n!} \\
		&= \frac{1}{n+1} \cdot \binom{2n}{n}
	\end{split}
\end{equation*}
\subsection{Zależność rekurencyjna}

\begin{theorem}[Wzór rekurencyjny na liczby Catalana]
	\begin{equation}
		c_n = c_{0} \cdot c_{n-1} + c_{1} \cdot c_{n - 2} + \dots + c_{n-1} \cdot c_0
	\end{equation}
\end{theorem}

\begin{proof}
	Znowuż mamy kwadrat \(n \times n\), ale tym razem dorysowujemy sobie prostą \(y = x - 1\). Każda ścieżka przetnie kiedyś tę linię i każda ścieżka dotknie kiedyś przekątnej \(y = x\) (można to udowodnić machając i pokazując na rysunek). Rzecz teraz ma się tak, że jeśli po ,,spotkaniu się'' z \(y = x - 1\) idziesz do góry, to potem musisz odbić w prawo (lub w skrajnym przypadku skończyłeś poprawną ścieżkę). Jednocześnie pierwszy wybór kierunku (tzn. ten w punkcie \((0,0)\) zawsze jest ,,w prawo'', bo jeśli ktoś pójdzie ,,do góry'' to znajdzie się w \((0,1)\), powyżej przekątnej \(y = x\)).

	Bierzemy sobie zatem pierwsze miejsce gdzie spotkałeś się z \(y = x\) i zauważamy, że jeśli dane jest ono jakimiś współrzędnymi \((i,i)\) to przecięliśmy \(y = x-1\) w \((i,i-1)\). Ponadto, ścieżka którą szliśmy od punktu \((1,0)\) do \((i,i-1)\) tak naprawdę jest ścieżką Dycka w kwadracie od punktów \((1,0)\), \((i, i-1)\) (kwadrat ten ma długość \(i-1\)). Ależ plot twist! Ścieżka którą idziemy od punktu \((i,i)\) do \((n,n)\) jest zaś już po prostu ścieżką Dycka w kwadracie o długości boku \(n-i\). Ścieżki te są od siebie niezależne i w ogóle, a długości tych ,,kwadratów catalanowych'' sumują się do \(i - 1 + n - i = n - 1\), więc teraz możemy zmajstrować wzór (w zależności od długości boków kwadratów, które z kolei są dyktowane tym kiedy się ,,spotkamy'' z \(y = x\)):
	\begin{equation*}
		c_n = \sum_{i = 0}^{n-1} c_i \cdot c_{n-1-i}
	\end{equation*}
	Co już można odwinąć do postaci która była w twierdzeniu.

	\begin{figure}[H]
		\centering
		\includegraphics[scale=0.5]{images/catalan/recursive_construction_1.png}
		\includegraphics[scale=0.5]{images/catalan/recursive_construction_2.png}

		\caption{Przykłady ,,podzielenia'' poprawnej ścieżki Dycka na podścieżki}
	\end{figure}

\end{proof}


\section{Zliczanie podziałów}

Chcemy pokazać fajny algorytm zliczania wszystkich podziałów liczby \(n\).

Oznaczmy liczbę wszystkich podziałów liczby \(n\) jako \(p(n)\). Jako ,,podziały liczby \(n\)'' mam na myśli liczbę sposobów na podzielenie liczby \(n\) na ileś składników (niezerowych), np. liczbę \(2\) mogę rozłożyć na \(1 + 1\) albo po prostu na \(2\) (i w sumie to tyle).  Funkcja tworząca ciągu \(p_n\) to: \begin{equation*}
	P(x) = (1 + x + x^2 + x^3 + \dots) \cdot (1 + x^2 + x^4 + x^6 + \dots) \cdot (1 + x^3 + x^6 + x^9 + \dots) \dots
\end{equation*}

Pierwszy nawias odpowiada wybraniu jedynki do podziału (i temu ile razy ją bierzemy), drugi dwójki, trzeci trójki, etc.

Oczywiście przy \(x^n\) będziemy mieli \(p_n\), jak to działa w funkcjach tworzących (i mam nadzieję, że widać dlaczego). Zapisujemy \(P(x)\) w fajniejszej postaci:

\begin{equation*}
	P(x) = \frac{1}{1-x} \cdot \frac{1}{1 - x^2} \cdot \frac{1}{1-x^3} \dots
\end{equation*}

Definiuję sobie \(Q(x) = (1-x) \cdot (1-x^2) \cdot (1-x^3) \dots\). Zauważam, że \(P(x) \cdot Q(x) = 1\), czyli \(Q(x)\) jest funkcją odwrotną do \(P(x)\). Okazuje się teraz, że \(Q(x)\) jest funkcją tworzącą pewnego śmiesznego ciągu, który sobie zaraz pokażemy.

Póki co musimy wprowadzić oznaczenia:
\begin{enumerate}
	\item \(e_n\) jest to liczba podziałów liczby \(n\) na parzystą liczbę składników parami różnych,
	\item \(o_n\) jest to liczba podziałów liczby \(n\) na nieparzystą liczbę składników parami różnych.
\end{enumerate}
Jak wszyscy powinniśmy już wiedzieć, funkcja tworząca ciągu \(e_n + o_n\) (czyli po prostu wszystkich podziałów \(n\) ze składnikami parami różnymi) wygląda tak:
\begin{equation*}
	(1+x) \cdot (1 + x^2) \cdot (1+x^3) \dots
\end{equation*}
Ten fakt do niczego nam się w sumie nie przyda, ale może pomóc zrozumieć co zaraz się stanie.

Możemy sobie teraz podumać, jaka jest funkcja tworząca ciągu \(e_n - o_n\). Otóż pojawia się tu plot twist, bo funkcja tworząca tego ciągu to po prostu \(Q(x)\):
\begin{equation*}
	(1-x) \cdot (1-x^2) \cdot (1-x^3) \dots
\end{equation*}

Działa to tak jak w powyższym przykładzie, z tym że jeśli wybraliśmy nieparzyście wiele składników to będzie nieparzyście wiele minusów i się ,,odejmie'' od współczynnika przy \(x^n\), a jeśli będzie parzyście wiele to się ,,doda''. Innymi słowy, do współczynnika przy \(x^n\) doda się 1 za każdy możliwy podział na parzyście wiele parami różnych składników, a odejmie się 1 za każdy możliwy podział na nieparzyście wiele parami różnych składników, czyli to co chcemy. Nie do końca mam pomysł jak to formalnie wytłumaczyć, więc proszę użyć swojej intuicji™.

Po co to wszystko? Okazuje się, że ciąg \(q_n = e_n - o_n\) ma pewne śmieszne własności (które niestety będzie trzeba udowodnić, brace yourselves).

\begin{theorem}[Eulera]
	\begin{equation}
		q_n = \begin{cases}
			0, \hspace{5pt} \mathrm{gdy} \hspace{5pt} n \not = \frac{(3 \cdot k \pm 1) \cdot k }{2} \\
			(-1)^k \hspace{5pt} \mathrm{wpp.}                                                       \\
		\end{cases}
	\end{equation}

\end{theorem}

\begin{proof}
	Zrobimy sobie przekształcenie \(f\), które przesyła prawie (dlaczego prawie to dojdziemy do tego za chwilę) każdy podział na \(n\) składników parami różnych na inny podział na \(n\) składników parami różnych (bijektywnie). Ktoś powie że sobie zrobiłem świetną bijekcję idącą z pewnego zbioru w samego siebie, but hear me out: ta bijekcja będzie mieć tę śmieszną własność, że jeśli podział był na parzyście wiele składników to będzie przesłany na nieparzyście wiele, a jeśli na nieparzyście wiele to będzie przesłany na parzyście wiele składników. To będzie fajne, bo pokażemy sobie że jest ich tyle samo (poza przypadkami gdzie definicja tej funkcji się popsuje, ale o tym za chwilę).

	Generalnie to oznaczmy sobie najmniejszy składnik w podziale \(P\) jako \(a\). Ponadto, zdefiniujmy sobie zbiór \(X\), taki że zawiera on największe składniki podziału \(P\), takie że każde dwa sąsiednie różnią się o jeden. Innymi słowy, jeśli podział \(P = (\lambda_1, \lambda_2, \lambda_3, \dots, \lambda_k)\), to \(X =\{\lambda_1, \lambda_2, \lambda_3, \dots, \lambda_d\}\), gdzie \(d\) jest największą liczbą taką, że kolejne składniki różnią się o 1  (zakładamy, że \(\lambda_1 > \lambda_2 > \dots > \lambda_k\)).

	Teraz jak mamy te zbiory zdefiniowane to możemy robić śmieszne rzeczy. Jeśli \(|X| < a - 1\), to możemy przerobić nasz podział, odejmując od każdego elementu z \(X\) 1, i dorzucając nowy element do podziału, taki że równy jest on moc \(|X|\). Otrzymaliśmy oczywiście poprawny podział (niektórym może pomóc dowód przez rysowanie).

	Dlaczego \(|X| < a - 1\), a nie po prostu \(|X| < a\)? Otóż przychodzi tutaj pewien śmieszny problem, mianowicie może być tak, że składnik podziału o wartości \(a\) ,,wpadł'' do \(X\). W takim przypadku bijekcja nam się kompletnie popsuje i wtedy jej definiujemy (ale jeszcze do tego wrócimy). Natomiast jeśli \(a\) nie należy do \(|X|\) to nasza bijekcja nadal działa. Fajnie.

	Czyli reasumując: jeśli \(|X| < a - 1\) lub (\(|X| = a - 1\) i \(a \not \in X\)) od każdego składnika z \(|X|\) odejmujemy 1 i majstrujemy nowy składnik, który wrzucamy pod składnik o wartości \(a\), który uprzednio był najmniejszy.

	\begin{figure}[H]
		\centering
		\includegraphics{images/case2.png}
		\caption{Wizualizacja przekształcenia (diagram Ferrersa). 2 ,,górne'' składniki różnią się o 1, trzeci już różni się od nich o 2; \(|X| = 2\), \(a=3\).}
	\end{figure}

	Zasadniczo to samo będziemy czynić (ale w drugą stronę), gdy okaże się że \(a < |X| \). Ordynarnie \textit{wywalam} składnik \(a\) i do odpowiedniej liczby elementów z \(X\)  ,,dodaję'' 1, tak by się wyrównało. Należy zauważyć, że być może nie wszystkie elementy z \(X\) będą mieć coś do siebie dodane, ale to mi nic nie psuje. W sumie też fajnie byłoby dodać, że dodaję te jedynki najpierw największym składnikom; inaczej mogłoby to się popsuć.

	Co dzieje się, gdy \(a = |X|\)? Jeśli \(a \in X\) to jest mi smutno, w przeciwnym razie mogę zrobić to samo co robiłem wcześniej i wszystko działa jak powinno.

	\begin{figure}[H]
		\centering
		\includegraphics{images/case_1.png}
		\caption{Wizualizacja przekształcenia (diagram Ferrersa). 3 ,,górne'' składniki różnią się o 1 więc należą do \(X\). \(|X| = 3\), \(a = 2\), więc dwóm największym elementom dodajemy 1, a składnik \(a\) usuwamy.}
	\end{figure}

	Zostają więc 2 przypadki, gdy coś może się popsuć:
	\begin{enumerate}
		\item \(|X| = a - 1\), \(a \in X\)
		      \begin{figure}[H]
			      \centering
			      \includegraphics{images/irytujacy_1.png}
			      \caption{Gdy \(|X| = a - 1\) i składnik \(a\) jest w \(X\); widać, że nic nie możemy z tym zrobić.}
		      \end{figure}

		\item \(|X| = a\), \(a \in x\)
		      \begin{figure}[H]
			      \centering
			      \includegraphics{images/irytujacy_2.png}
			      \caption{Gdy \(|X| = a\) i składnik \(a\) jest w \(X\); również widać, że nasze przekształcenie nie zadziała.}
		      \end{figure}
	\end{enumerate}

	Zauważmy, że sytuacja gdy składnik \(a\) jest w \(X\) jest bardzo dziwną sytuacją generalnie, bo jest to najmniejszy składnik; z definicji \(X\) mamy wtedy, że wszystkie kolejne składniki w \(P\) różnią się o dokładnie 1. Na podstawie tej obserwacji możemy już dokładnie powiedzieć, jakiej postaci musi być \(n\), by miało taki ,,złośliwy'' podział:

	\begin{enumerate}
		\item Gdy \(|X| = a - 1\), \(a \in X\), to \(n\) musi dla jakiegoś \(k\) być postaci \((k + 1) + (k + 2) + \dots + 2k \) (\(|X| =k, a = k+1\), wszystko się zgadza)
		\item Gdy \(|X| = a\), \(a \in x\), to \(n\) musi dla jakiegoś \(k\) być postaci \(k + (k+1) + (k+2) + \dots + (2k - 1)\) (\(|X| = k\), \(a = k\), ponownie wszystko gra)
	\end{enumerate}

	Jak zastosujemy matematykę mniej dyskretną by wysumować te nawiasy, dostaniemy że \(n\) aby miało irytujący podział to musi być postaci \(\frac{k \cdot (3k+1)}{2}\) lub \({k \cdot (3k-1)}{2}\). Jednocześnie nie ma takiego naturalnego \(k\), że wartości te są sobie równe, więc jeśli \(n\) ma irytujący podział, to ma go tylko jednego. Wtedy nie możemy przerzucić tylko jednego podziału na inny (inne są ze sobą w bijekcji) więc \(e_n - o_n = (-1)^k\) (jeśli \(k\) jest parzyste to irytujący podział ma parzyście wiele składników, a w przeciwnym razie nieparzyście wiele). Jeśli irytujący podział nie występuje, \(e_n = o_n\) z bijekcji którą pokazaliśmy. Fajnie.
\end{proof}

Dobra, ale wróćmy do tego cośmy chcieli udowodnić na samym początku. Co w ogóle wynika z tego twierdzenia Eulera? No w sumie to bardzo dużo, bo jak mamy \(q_n = e_n - o_n\) i \(Q(x)\) jest jego funkcją tworzącą:
\begin{equation*}
	Q(x) = q_0 + q_1 \cdot x + q_2 \cdot x^2 + q_3 \cdot x^3 + \dots
\end{equation*}
Ale znamy wartości współczynników \(q_i\) z twierdzenia Eulera:
\begin{equation*}
	Q(x) = 1 - x - x^2 + x^5 + x^7 - x^{12} - x^{15} + x^{22} + x^{26} + \dots
\end{equation*}
Zauważmy, że współczynników które nie są zerowe jest tylko jakoś \(O(\sqrt{n})\), czyli dosyć mało.

Pamiętajmy, że \(P(x) \cdot Q(x) = 1\), czyli że ciąg który wyjdzie po ich wymnożeniu będzie wyglądać tak: \((1,0,0,0, \dots)\) Ponieważ mnożenie w funkcjach tworzących działa jakoś tak, że w wynikowym ciągu (nazwijmy go \(r\)) element \(r_n\) można obliczyć w ten sposób:
\begin{equation*}
	r_n = \sum_{i=0}^{n} p_i \cdot q_{n-i}
\end{equation*}

I wiemy że w naszym przypadku \(r_n = 0\) dla \(n > 1\), to mamy że: \begin{equation*}
	0 = p_n - p_{n-1} - p_{n-2} + p_{n-5} + p_{n-7} - p_{n-12} - \dots
\end{equation*}
To teraz \(p_n\) przerzucamy na drugą stronę i mnożymy stronami razy \(-1\) i mamy wzór na \(p_n\), które możemy obliczyć w \(O(\sqrt{n})\). No i fajnie.



\realsection{Twierdzenie Ramseya. Przykłady zastosowań}
\section{Rozwiązywanie rekurencji liniowych}
\epigraph{I can elaborate: zrobiłam zadanka, zobaczyłam tworzące, stwierdziłam, że chce mi się spać, poszłam sobie}{\textit{Studentka TCSu o zadaniach z funkcji tworzących na kolokwium}}


\subsection{Rozkład na ułamki proste}
To nie jest formalny dowód ani formalna własność ani nic, bardziej schemat postępowania przy rozkładzie na ułamki proste. Sam dowód tego, że rozkład na ułamki proste istnieje, to \textit{sprowadź do wspólnego mianownika i zobacz co Ci wyszło}.
Jeżeli \(deg(P(x)) < deg(Q(x))\) i \(Q(x) = (x-a)^n \cdot (x-b)^k\) to:
\begin{equation*}
	\frac{P(x)}{Q(x)} = \frac{P(x)}{(x-a)^n \cdot (x-b)^k} = \frac{A_1}{x-a} + \frac{A_2}{(x-a)^2} + \dots + \frac{A_n}{(x-a)^n} + \frac{B_1}{x-b} + \frac{B_2}{(x-b)^2} + \dots + \frac{B_k}{(x-b)^k}
\end{equation*}

Oczywiście ten schemat można rozszerzać na więcej śmiesznych rzeczy w mianowniku, ale chyba widać o co chodzi.


\section{Ciąg Fibbonaciego}
\begin{theorem}[Wzór Bineta]
	\begin{equation}
		f_n = \frac{1}{\sqrt{5}} \cdot \left( \left(\frac{1 + \sqrt{5}}{2}\right)^{n} - \left(\frac{1 - \sqrt{5}}{2}\right)^{n} \right)
	\end{equation}
\end{theorem}

\begin{proof}
	Rozpisujemy sobie funkcję tworzącą ciągu \(f_n\):

	\begin{equation*}
		F(x) = f_0 + f_1 \cdot x + f_2 \cdot x^2 + f_3 \cdot x^3 \dots =
	\end{equation*}
	\begin{equation*}
		= f_0 + f_1 \cdot x + (f_0 + f_1) \cdot x^2 + (f_1 + f_2) \cdot x^3 + \dots =
	\end{equation*}
	\begin{equation*}
		= f_0 + f_1 \cdot x + f_0 \cdot x^2 + f_1 \cdot x^2 + f_1 \cdot x^3 + f_2 \cdot x^3 + \dots =
	\end{equation*}
	\begin{equation*}
		= f_0 + f_1 \cdot x + f_0 \cdot x^2 + f_1 \cdot x^3 + \dots + f_1 \cdot x^2 +  f_2 \cdot x^3 + \dots =
	\end{equation*}
	\begin{equation*}
		= f_0 + f_1 \cdot x + x^2 \cdot (f_0 + f_1 \cdot x + \dots) + x \cdot (f_1 \cdot x +  f_2 \cdot x^2 + \dots) =
	\end{equation*}
	\begin{equation*}
		= f_0 + f_1 \cdot x + x^2 \cdot F(x) + x \cdot (F(x) - f_0) =
	\end{equation*}
	\begin{equation*}
		= 0 + 1 \cdot x + x^2 \cdot F(x) + x \cdot (F(x) - 0) =
	\end{equation*}
	\begin{equation*}
		= x + x^2 \cdot F(x) + x \cdot F(x)
	\end{equation*}

	W takim razie mamy, że:
	\begin{equation*}
		F(x) = x + x^2 \cdot F(x) + x \cdot F(x)
	\end{equation*}
	\begin{equation*}
		F(x) -  x^2 \cdot F(x) - x \cdot F(x)  = x
	\end{equation*}
	\begin{equation*}
		F(x) \cdot (1 - x^2 - x) = x
	\end{equation*}
	\begin{equation*}
		F(x) = \frac{x}{-x^2 -x + 1}
	\end{equation*}

	Mianownik możemy rozbić (za pomocą liczenia jakichś delt czy coś):
	\begin{equation*}
		F(x) = \frac{x}{(-1) \cdot \left(x - \left(- \frac{1 + \sqrt{5}}{2}\right)\right) \cdot \left(x - \left(- \frac{1 - \sqrt{5}}{2}\right)\right)}
	\end{equation*}

	Nie no, serio, jeśli ktoś myśli że będę TeXować te przekształcenia to się myli. Powinno wyjść po przekształceniach że:
	\begin{equation*}
		F(x) = \frac{x}{(1-ax) \cdot (1-bx)}
	\end{equation*}
	gdzie \(a = \frac{1 + \sqrt{5}}{2}, b=\frac{1 - \sqrt{5}}{2}\)

	Dalej rozbijamy na ułamki proste:
	\begin{equation*}
		F(x) = \frac{A}{1-ax} + \frac{B}{1-bx}
	\end{equation*}
	\(A\) powinno wyjść \(\frac{1}{\sqrt{5}}\), \(B\) powinno wyjść \(- \frac{1}{\sqrt{5}}\).

	Odwijamy każdą z tych funkcji tworzących z osobna, korzystając ze wzoru podanego we wcześniejszym rozdziale i otrzymujemy wzór.
\end{proof}


\section{Ciąg Catalana}
Liczba Catalana jest to liczba ścieżek długości \(2n\) w kwadracie \(n \times n\) ,,poniżej'' przekątnej (lub na jej poziomie), idących za każdym razem jednostkę do góry lub jednostkę w prawo. Ścieżki takie nazywamy ścieżkami Dycka. Niezwykle formalna definicja. To jest jedna z tych rzeczy, które chyba po prostu trzeba narysować.

\begin{figure}[h]
	\centering
	\includegraphics[scale=0.5]{images/catalan/all_paths_1.png}
  \caption{Ścieżki Dycka długości 2; \(c_1 = 1\)}
\end{figure}

\begin{figure}[h]
	\centering
	\includegraphics[scale=0.5]{images/catalan/all_paths_2.png}
  \caption{Ścieżki Dycka długości 4; \(c_2 = 2\)}
\end{figure}

\begin{figure}[ht]
	\centering
	\includegraphics[scale=0.5]{images/catalan/all_paths_3.png}
  \caption{Ścieżki Dycka długości 6; \(c_3 = 5\)}
\end{figure}


\subsection{Wzór kombinatoryczny}
\begin{theorem}[Wzór kombinatoryczny na liczby Catalana]
	\begin{equation}
		c_n = \frac{1}{n+1} \cdot \binom{2n}{n}
	\end{equation}
\end{theorem}

Mamy sobie nasz kwadrat \(n \times n\). Przekątną możemy opisać tak jakby wzorem \(y = x\) (tak intuicyjnie, bo nie działamy w żadnym układzie współrzędnych, bla bla bla). Robimy sobie teraz prostą \(y = x+1\), idącą jakby ,,o jednostkę wyżej''. Zauważamy, że jeśli jakaś ścieżka przekracza linię naszej przekątnej, to musi ,,dotknąć'' linii \(y = x+1\). \textit{To widać}. Teraz wpadamy na świetny pomysł; jeśli jakaś ścieżka idąca po tym kwadracie ,,spotyka się'' z \(y = x+1\), to od tego momentu odbijamy ją symetrycznie względem \(y = x+1\). Zauważamy, że ścieżka ta (po odbiciu) skończy się w punkcie \((n-1, n+1)\) zamiast w \((n,n)\). Fakt ten dowodzimy stosując dowód przez rysowanie.

\begin{figure}[ht]
	\centering
	\includegraphics[scale=0.5]{images/catalan/path_with_reflection_1.png}
	\includegraphics[scale=0.5]{images/catalan/path_with_reflection_2.png}

	\caption{Przykłady odbicia niepoprawnej ścieżki}
\end{figure}

Zauważamy fascynujący fakt, mianowicie dwie różne ścieżki będą mieć 2 różne odbicia, a więc nasze przekształcenie jest iniektywne. Ponadto, jak sobie zobaczymy jakąkolwiek ścieżkę zaczynającą się w \((0,0)\), ale kończącą się w \((n-1,n+1)\), to jesteśmy w stanie zobaczyć gdzie pierwszy raz przecina się z \(y = x+1\), a następnie ją odbić, otrzymując ścieżkę idącą do \((n,n)\) i niebędącą ścieżką Dycka, której odbicie daje wyjściową ścieżkę. Zatem odbijanie jest suriektywne. A to oznacza tylko jedną rzecz: bijekcję między ścieżkami które ,,nie są catalanowe'', a ścieżkami ,,odbitymi''.

Wszystkich możliwych ścieżek od \((0,0)\) do \((n,n)\) mamy \(\binom{2n}{n}\), bo długość naszej drogi ma \(2n\) i wybieramy sobie \(n\) miejsc gdzie idziemy w prawo. Wszystkich możliwych ścieżek od \((0,0)\) do \((n-1,n+1)\) (czyli tych które są ,,złe'') mamy \(\binom{2n}{n-1}\), bo, analogicznie, ścieżka jest długości \(2n\) ale w prawo idziemy \(n-1\) razy. To prowadzi nas do wyniku:
\begin{equation*}
	\begin{split}
		c_n
		&= \binom{2n}{n} - \binom{2n}{n-1} \\
		&= \frac{(2n)!}{n! \cdot n!} - \frac{(2n)!}{(n-1)! \cdot (n+1)!} \\
		&= \frac{(n+1) \cdot (2n)!}{n! \cdot (n+1)!} - \frac{n \cdot (2n)!}{n! \cdot (n+1)!}\\
		&= \frac{(2n)!}{n! \cdot (n+1)!} \\
		&= \frac{1}{n+1} \cdot \frac {(2n)!}{n! \cdot n!} \\
		&= \frac{1}{n+1} \cdot \binom{2n}{n}
	\end{split}
\end{equation*}
\subsection{Zależność rekurencyjna}

\begin{theorem}[Wzór rekurencyjny na liczby Catalana]
	\begin{equation}
		c_n = c_{0} \cdot c_{n-1} + c_{1} \cdot c_{n - 2} + \dots + c_{n-1} \cdot c_0
	\end{equation}
\end{theorem}

\begin{proof}
	Znowuż mamy kwadrat \(n \times n\), ale tym razem dorysowujemy sobie prostą \(y = x - 1\). Każda ścieżka przetnie kiedyś tę linię i każda ścieżka dotknie kiedyś przekątnej \(y = x\) (można to udowodnić machając i pokazując na rysunek). Rzecz teraz ma się tak, że jeśli po ,,spotkaniu się'' z \(y = x - 1\) idziesz do góry, to potem musisz odbić w prawo (lub w skrajnym przypadku skończyłeś poprawną ścieżkę). Jednocześnie pierwszy wybór kierunku (tzn. ten w punkcie \((0,0)\) zawsze jest ,,w prawo'', bo jeśli ktoś pójdzie ,,do góry'' to znajdzie się w \((0,1)\), powyżej przekątnej \(y = x\)).

	Bierzemy sobie zatem pierwsze miejsce gdzie spotkałeś się z \(y = x\) i zauważamy, że jeśli dane jest ono jakimiś współrzędnymi \((i,i)\) to przecięliśmy \(y = x-1\) w \((i,i-1)\). Ponadto, ścieżka którą szliśmy od punktu \((1,0)\) do \((i,i-1)\) tak naprawdę jest ścieżką Dycka w kwadracie od punktów \((1,0)\), \((i, i-1)\) (kwadrat ten ma długość \(i-1\)). Ależ plot twist! Ścieżka którą idziemy od punktu \((i,i)\) do \((n,n)\) jest zaś już po prostu ścieżką Dycka w kwadracie o długości boku \(n-i\). Ścieżki te są od siebie niezależne i w ogóle, a długości tych ,,kwadratów catalanowych'' sumują się do \(i - 1 + n - i = n - 1\), więc teraz możemy zmajstrować wzór (w zależności od długości boków kwadratów, które z kolei są dyktowane tym kiedy się ,,spotkamy'' z \(y = x\)):
	\begin{equation*}
		c_n = \sum_{i = 0}^{n-1} c_i \cdot c_{n-1-i}
	\end{equation*}
	Co już można odwinąć do postaci która była w twierdzeniu.

	\begin{figure}[H]
		\centering
		\includegraphics[scale=0.5]{images/catalan/recursive_construction_1.png}
		\includegraphics[scale=0.5]{images/catalan/recursive_construction_2.png}

		\caption{Przykłady ,,podzielenia'' poprawnej ścieżki Dycka na podścieżki}
	\end{figure}

\end{proof}


\section{Zliczanie podziałów}

Chcemy pokazać fajny algorytm zliczania wszystkich podziałów liczby \(n\).

Oznaczmy liczbę wszystkich podziałów liczby \(n\) jako \(p(n)\). Jako ,,podziały liczby \(n\)'' mam na myśli liczbę sposobów na podzielenie liczby \(n\) na ileś składników (niezerowych), np. liczbę \(2\) mogę rozłożyć na \(1 + 1\) albo po prostu na \(2\) (i w sumie to tyle).  Funkcja tworząca ciągu \(p_n\) to: \begin{equation*}
	P(x) = (1 + x + x^2 + x^3 + \dots) \cdot (1 + x^2 + x^4 + x^6 + \dots) \cdot (1 + x^3 + x^6 + x^9 + \dots) \dots
\end{equation*}

Pierwszy nawias odpowiada wybraniu jedynki do podziału (i temu ile razy ją bierzemy), drugi dwójki, trzeci trójki, etc.

Oczywiście przy \(x^n\) będziemy mieli \(p_n\), jak to działa w funkcjach tworzących (i mam nadzieję, że widać dlaczego). Zapisujemy \(P(x)\) w fajniejszej postaci:

\begin{equation*}
	P(x) = \frac{1}{1-x} \cdot \frac{1}{1 - x^2} \cdot \frac{1}{1-x^3} \dots
\end{equation*}

Definiuję sobie \(Q(x) = (1-x) \cdot (1-x^2) \cdot (1-x^3) \dots\). Zauważam, że \(P(x) \cdot Q(x) = 1\), czyli \(Q(x)\) jest funkcją odwrotną do \(P(x)\). Okazuje się teraz, że \(Q(x)\) jest funkcją tworzącą pewnego śmiesznego ciągu, który sobie zaraz pokażemy.

Póki co musimy wprowadzić oznaczenia:
\begin{enumerate}
	\item \(e_n\) jest to liczba podziałów liczby \(n\) na parzystą liczbę składników parami różnych,
	\item \(o_n\) jest to liczba podziałów liczby \(n\) na nieparzystą liczbę składników parami różnych.
\end{enumerate}
Jak wszyscy powinniśmy już wiedzieć, funkcja tworząca ciągu \(e_n + o_n\) (czyli po prostu wszystkich podziałów \(n\) ze składnikami parami różnymi) wygląda tak:
\begin{equation*}
	(1+x) \cdot (1 + x^2) \cdot (1+x^3) \dots
\end{equation*}
Ten fakt do niczego nam się w sumie nie przyda, ale może pomóc zrozumieć co zaraz się stanie.

Możemy sobie teraz podumać, jaka jest funkcja tworząca ciągu \(e_n - o_n\). Otóż pojawia się tu plot twist, bo funkcja tworząca tego ciągu to po prostu \(Q(x)\):
\begin{equation*}
	(1-x) \cdot (1-x^2) \cdot (1-x^3) \dots
\end{equation*}

Działa to tak jak w powyższym przykładzie, z tym że jeśli wybraliśmy nieparzyście wiele składników to będzie nieparzyście wiele minusów i się ,,odejmie'' od współczynnika przy \(x^n\), a jeśli będzie parzyście wiele to się ,,doda''. Innymi słowy, do współczynnika przy \(x^n\) doda się 1 za każdy możliwy podział na parzyście wiele parami różnych składników, a odejmie się 1 za każdy możliwy podział na nieparzyście wiele parami różnych składników, czyli to co chcemy. Nie do końca mam pomysł jak to formalnie wytłumaczyć, więc proszę użyć swojej intuicji™.

Po co to wszystko? Okazuje się, że ciąg \(q_n = e_n - o_n\) ma pewne śmieszne własności (które niestety będzie trzeba udowodnić, brace yourselves).

\begin{theorem}[Eulera]
	\begin{equation}
		q_n = \begin{cases}
			0, \hspace{5pt} \mathrm{gdy} \hspace{5pt} n \not = \frac{(3 \cdot k \pm 1) \cdot k }{2} \\
			(-1)^k \hspace{5pt} \mathrm{wpp.}                                                       \\
		\end{cases}
	\end{equation}

\end{theorem}

\begin{proof}
	Zrobimy sobie przekształcenie \(f\), które przesyła prawie (dlaczego prawie to dojdziemy do tego za chwilę) każdy podział na \(n\) składników parami różnych na inny podział na \(n\) składników parami różnych (bijektywnie). Ktoś powie że sobie zrobiłem świetną bijekcję idącą z pewnego zbioru w samego siebie, but hear me out: ta bijekcja będzie mieć tę śmieszną własność, że jeśli podział był na parzyście wiele składników to będzie przesłany na nieparzyście wiele, a jeśli na nieparzyście wiele to będzie przesłany na parzyście wiele składników. To będzie fajne, bo pokażemy sobie że jest ich tyle samo (poza przypadkami gdzie definicja tej funkcji się popsuje, ale o tym za chwilę).

	Generalnie to oznaczmy sobie najmniejszy składnik w podziale \(P\) jako \(a\). Ponadto, zdefiniujmy sobie zbiór \(X\), taki że zawiera on największe składniki podziału \(P\), takie że każde dwa sąsiednie różnią się o jeden. Innymi słowy, jeśli podział \(P = (\lambda_1, \lambda_2, \lambda_3, \dots, \lambda_k)\), to \(X =\{\lambda_1, \lambda_2, \lambda_3, \dots, \lambda_d\}\), gdzie \(d\) jest największą liczbą taką, że kolejne składniki różnią się o 1  (zakładamy, że \(\lambda_1 > \lambda_2 > \dots > \lambda_k\)).

	Teraz jak mamy te zbiory zdefiniowane to możemy robić śmieszne rzeczy. Jeśli \(|X| < a - 1\), to możemy przerobić nasz podział, odejmując od każdego elementu z \(X\) 1, i dorzucając nowy element do podziału, taki że równy jest on moc \(|X|\). Otrzymaliśmy oczywiście poprawny podział (niektórym może pomóc dowód przez rysowanie).

	Dlaczego \(|X| < a - 1\), a nie po prostu \(|X| < a\)? Otóż przychodzi tutaj pewien śmieszny problem, mianowicie może być tak, że składnik podziału o wartości \(a\) ,,wpadł'' do \(X\). W takim przypadku bijekcja nam się kompletnie popsuje i wtedy jej definiujemy (ale jeszcze do tego wrócimy). Natomiast jeśli \(a\) nie należy do \(|X|\) to nasza bijekcja nadal działa. Fajnie.

	Czyli reasumując: jeśli \(|X| < a - 1\) lub (\(|X| = a - 1\) i \(a \not \in X\)) od każdego składnika z \(|X|\) odejmujemy 1 i majstrujemy nowy składnik, który wrzucamy pod składnik o wartości \(a\), który uprzednio był najmniejszy.

	\begin{figure}[H]
		\centering
		\includegraphics{images/case2.png}
		\caption{Wizualizacja przekształcenia (diagram Ferrersa). 2 ,,górne'' składniki różnią się o 1, trzeci już różni się od nich o 2; \(|X| = 2\), \(a=3\).}
	\end{figure}

	Zasadniczo to samo będziemy czynić (ale w drugą stronę), gdy okaże się że \(a < |X| \). Ordynarnie \textit{wywalam} składnik \(a\) i do odpowiedniej liczby elementów z \(X\)  ,,dodaję'' 1, tak by się wyrównało. Należy zauważyć, że być może nie wszystkie elementy z \(X\) będą mieć coś do siebie dodane, ale to mi nic nie psuje. W sumie też fajnie byłoby dodać, że dodaję te jedynki najpierw największym składnikom; inaczej mogłoby to się popsuć.

	Co dzieje się, gdy \(a = |X|\)? Jeśli \(a \in X\) to jest mi smutno, w przeciwnym razie mogę zrobić to samo co robiłem wcześniej i wszystko działa jak powinno.

	\begin{figure}[H]
		\centering
		\includegraphics{images/case_1.png}
		\caption{Wizualizacja przekształcenia (diagram Ferrersa). 3 ,,górne'' składniki różnią się o 1 więc należą do \(X\). \(|X| = 3\), \(a = 2\), więc dwóm największym elementom dodajemy 1, a składnik \(a\) usuwamy.}
	\end{figure}

	Zostają więc 2 przypadki, gdy coś może się popsuć:
	\begin{enumerate}
		\item \(|X| = a - 1\), \(a \in X\)
		      \begin{figure}[H]
			      \centering
			      \includegraphics{images/irytujacy_1.png}
			      \caption{Gdy \(|X| = a - 1\) i składnik \(a\) jest w \(X\); widać, że nic nie możemy z tym zrobić.}
		      \end{figure}

		\item \(|X| = a\), \(a \in x\)
		      \begin{figure}[H]
			      \centering
			      \includegraphics{images/irytujacy_2.png}
			      \caption{Gdy \(|X| = a\) i składnik \(a\) jest w \(X\); również widać, że nasze przekształcenie nie zadziała.}
		      \end{figure}
	\end{enumerate}

	Zauważmy, że sytuacja gdy składnik \(a\) jest w \(X\) jest bardzo dziwną sytuacją generalnie, bo jest to najmniejszy składnik; z definicji \(X\) mamy wtedy, że wszystkie kolejne składniki w \(P\) różnią się o dokładnie 1. Na podstawie tej obserwacji możemy już dokładnie powiedzieć, jakiej postaci musi być \(n\), by miało taki ,,złośliwy'' podział:

	\begin{enumerate}
		\item Gdy \(|X| = a - 1\), \(a \in X\), to \(n\) musi dla jakiegoś \(k\) być postaci \((k + 1) + (k + 2) + \dots + 2k \) (\(|X| =k, a = k+1\), wszystko się zgadza)
		\item Gdy \(|X| = a\), \(a \in x\), to \(n\) musi dla jakiegoś \(k\) być postaci \(k + (k+1) + (k+2) + \dots + (2k - 1)\) (\(|X| = k\), \(a = k\), ponownie wszystko gra)
	\end{enumerate}

	Jak zastosujemy matematykę mniej dyskretną by wysumować te nawiasy, dostaniemy że \(n\) aby miało irytujący podział to musi być postaci \(\frac{k \cdot (3k+1)}{2}\) lub \({k \cdot (3k-1)}{2}\). Jednocześnie nie ma takiego naturalnego \(k\), że wartości te są sobie równe, więc jeśli \(n\) ma irytujący podział, to ma go tylko jednego. Wtedy nie możemy przerzucić tylko jednego podziału na inny (inne są ze sobą w bijekcji) więc \(e_n - o_n = (-1)^k\) (jeśli \(k\) jest parzyste to irytujący podział ma parzyście wiele składników, a w przeciwnym razie nieparzyście wiele). Jeśli irytujący podział nie występuje, \(e_n = o_n\) z bijekcji którą pokazaliśmy. Fajnie.
\end{proof}

Dobra, ale wróćmy do tego cośmy chcieli udowodnić na samym początku. Co w ogóle wynika z tego twierdzenia Eulera? No w sumie to bardzo dużo, bo jak mamy \(q_n = e_n - o_n\) i \(Q(x)\) jest jego funkcją tworzącą:
\begin{equation*}
	Q(x) = q_0 + q_1 \cdot x + q_2 \cdot x^2 + q_3 \cdot x^3 + \dots
\end{equation*}
Ale znamy wartości współczynników \(q_i\) z twierdzenia Eulera:
\begin{equation*}
	Q(x) = 1 - x - x^2 + x^5 + x^7 - x^{12} - x^{15} + x^{22} + x^{26} + \dots
\end{equation*}
Zauważmy, że współczynników które nie są zerowe jest tylko jakoś \(O(\sqrt{n})\), czyli dosyć mało.

Pamiętajmy, że \(P(x) \cdot Q(x) = 1\), czyli że ciąg który wyjdzie po ich wymnożeniu będzie wyglądać tak: \((1,0,0,0, \dots)\) Ponieważ mnożenie w funkcjach tworzących działa jakoś tak, że w wynikowym ciągu (nazwijmy go \(r\)) element \(r_n\) można obliczyć w ten sposób:
\begin{equation*}
	r_n = \sum_{i=0}^{n} p_i \cdot q_{n-i}
\end{equation*}

I wiemy że w naszym przypadku \(r_n = 0\) dla \(n > 1\), to mamy że: \begin{equation*}
	0 = p_n - p_{n-1} - p_{n-2} + p_{n-5} + p_{n-7} - p_{n-12} - \dots
\end{equation*}
To teraz \(p_n\) przerzucamy na drugą stronę i mnożymy stronami razy \(-1\) i mamy wzór na \(p_n\), które możemy obliczyć w \(O(\sqrt{n})\). No i fajnie.



\realsection{Funkcje tworzące. Wyznaczanie liczb Fibonacciego za pomocą funkcji tworzących}

\epigraph{I can elaborate: zrobiłam zadanka, zobaczyłam tworzące, stwierdziłam, że chce mi się spać, poszłam sobie}{\textit{Studentka TCSu o zadaniach z funkcji tworzących na kolokwium}}
\subsection{Rozkład na ułamki proste}
To nie jest formalny dowód ani formalna własność ani nic, bardziej schemat postępowania przy rozkładzie na ułamki proste. Sam dowód tego, że rozkład na ułamki proste istnieje, to \textit{sprowadź do wspólnego mianownika i zobacz co Ci wyszło}.
Jeżeli \(deg(P(x)) < deg(Q(x))\) i \(Q(x) = (x-a)^n \cdot (x-b)^k\) to:
\begin{equation*}
	\frac{P(x)}{Q(x)} = \frac{P(x)}{(x-a)^n \cdot (x-b)^k} = \frac{A_1}{x-a} + \frac{A_2}{(x-a)^2} + \dots + \frac{A_n}{(x-a)^n} + \frac{B_1}{x-b} + \frac{B_2}{(x-b)^2} + \dots + \frac{B_k}{(x-b)^k}
\end{equation*}

Oczywiście ten schemat można rozszerzać na więcej śmiesznych rzeczy w mianowniku, ale chyba widać o co chodzi.


\realsubsection{Wzór Bineta}
\begin{theorem}[Wzór Bineta]
	\begin{equation}
		f_n = \frac{1}{\sqrt{5}} \cdot \left( \left(\frac{1 + \sqrt{5}}{2}\right)^{n} - \left(\frac{1 - \sqrt{5}}{2}\right)^{n} \right)
	\end{equation}
\end{theorem}

\begin{proof}
	Rozpisujemy sobie funkcję tworzącą ciągu \(f_n\):

	\begin{equation*}
		F(x) = f_0 + f_1 \cdot x + f_2 \cdot x^2 + f_3 \cdot x^3 \dots =
	\end{equation*}
	\begin{equation*}
		= f_0 + f_1 \cdot x + (f_0 + f_1) \cdot x^2 + (f_1 + f_2) \cdot x^3 + \dots =
	\end{equation*}
	\begin{equation*}
		= f_0 + f_1 \cdot x + f_0 \cdot x^2 + f_1 \cdot x^2 + f_1 \cdot x^3 + f_2 \cdot x^3 + \dots =
	\end{equation*}
	\begin{equation*}
		= f_0 + f_1 \cdot x + f_0 \cdot x^2 + f_1 \cdot x^3 + \dots + f_1 \cdot x^2 +  f_2 \cdot x^3 + \dots =
	\end{equation*}
	\begin{equation*}
		= f_0 + f_1 \cdot x + x^2 \cdot (f_0 + f_1 \cdot x + \dots) + x \cdot (f_1 \cdot x +  f_2 \cdot x^2 + \dots) =
	\end{equation*}
	\begin{equation*}
		= f_0 + f_1 \cdot x + x^2 \cdot F(x) + x \cdot (F(x) - f_0) =
	\end{equation*}
	\begin{equation*}
		= 0 + 1 \cdot x + x^2 \cdot F(x) + x \cdot (F(x) - 0) =
	\end{equation*}
	\begin{equation*}
		= x + x^2 \cdot F(x) + x \cdot F(x)
	\end{equation*}

	W takim razie mamy, że:
	\begin{equation*}
		F(x) = x + x^2 \cdot F(x) + x \cdot F(x)
	\end{equation*}
	\begin{equation*}
		F(x) -  x^2 \cdot F(x) - x \cdot F(x)  = x
	\end{equation*}
	\begin{equation*}
		F(x) \cdot (1 - x^2 - x) = x
	\end{equation*}
	\begin{equation*}
		F(x) = \frac{x}{-x^2 -x + 1}
	\end{equation*}

	Mianownik możemy rozbić (za pomocą liczenia jakichś delt czy coś):
	\begin{equation*}
		F(x) = \frac{x}{(-1) \cdot \left(x - \left(- \frac{1 + \sqrt{5}}{2}\right)\right) \cdot \left(x - \left(- \frac{1 - \sqrt{5}}{2}\right)\right)}
	\end{equation*}

	Nie no, serio, jeśli ktoś myśli że będę TeXować te przekształcenia to się myli. Powinno wyjść po przekształceniach że:
	\begin{equation*}
		F(x) = \frac{x}{(1-ax) \cdot (1-bx)}
	\end{equation*}
	gdzie \(a = \frac{1 + \sqrt{5}}{2}, b=\frac{1 - \sqrt{5}}{2}\)

	Dalej rozbijamy na ułamki proste:
	\begin{equation*}
		F(x) = \frac{A}{1-ax} + \frac{B}{1-bx}
	\end{equation*}
	\(A\) powinno wyjść \(\frac{1}{\sqrt{5}}\), \(B\) powinno wyjść \(- \frac{1}{\sqrt{5}}\).

	Odwijamy każdą z tych funkcji tworzących z osobna, korzystając ze wzoru podanego we wcześniejszym rozdziale i otrzymujemy wzór.
\end{proof}


\realsection{Skojarzenia w grafach dwudzielnych. Twierdzenie Halla}
\epigraph{Dlaczego wysoki odsetek pracowników służby drogowej ma rodziny? Bo dużo Hall'ują.}{\textit{Niezwykle suchy żart pewnego studenta}}
\begin{theorem}[Halla]
	Graf dwudzielny \(G = (X,Y,E)\), gdzie \(|X|\) = \(|Y|\) ma dopasowanie doskonałe wtedy i tylko wtedy, gdy dla dowolnego \(A \subset X\) zachodzi: \begin{equation}
		|A| \leq |N(A)|
	\end{equation}
\end{theorem}

\begin{proof}
	Ponieważ twierdzenie mówi \textit{wtedy i tylko wtedy}, musimy udowadniać w dwie strony. Zacznijmy od tej prostszej strony, czyli pokażmy że gdy graf dwudzielny w którym \(|X| = |Y|\) ma dopasowanie doskonałe to \(|A| \leq |N(A)|\). Zasadniczo od razu to widać, bo skoro dopasowanie doskonałe istnieje to wystarczy sobie je wziąć. Każdy wierzchołek z \(|X|\) ma wówczas jakąś krawędź do wierzchołka z \(|Y|\). Zauważamy, że siłą rzeczy w samym dopasowaniu (czyli jakimś podgrafie oryginalnego grafu dwudzielnego, z którego być może ,,wywalono'' jakieś krawędzie) jest tak, że \(|A| = |N(A)|\), z czego w szczególności wynika teza. To chyba widać.

	W drugą stronę jest ciekawiej, bo po pierwsze musimy sobie wprowadzić instytucję \textit{ścieżki powiększającej}. Nie należy tego mylić ze ścieżką powiększającą z przepływów, bo one mówią o innych rzeczach (ale idea jest ta sama). Generalnie to załóżmy sobie, że mam jakieś dopasowanie \(M\) które nie jest doskonałe. Oznacza to, że w \(X\) jest jakiś wierzchołek \(x_0\) poza dopasowaniem. Jeśli \(x_0\) łączy się z jakimś wierzchołkiem \(y_0 \in Y\) i \(y_0 \in M\). \(y\) łączy się z jakimś \(x_1 \in M\) (bo są razem w dopasowaniu). Teraz jeśli \(x_1\) łączy się z jakimś \(y_1 \in Y\) takim, że \(y_1 \not \in M\) to ja to dopasowanie mogę przerobić: ,,połączyć'' \(x\) z \(y\) i \(x_1\) z \(y_1\), dorzucając dodatkowy wierzchołek do dopasowania. To jest przykład bardzo krótkiej ścieżki powiększającej, ale generalna idea to jest taki ,,zygzak'' którego można przerobić, żeby dopasowanie powiększyło się o jeden wierzchołek. At this point wszyscy już chyba wiedzą, że zamiłowania do formalizmu to ja nie mam.

	\begin{figure}[H]
		\centering
		\includegraphics[scale=0.3]{images/hall/augmenting_path_before.png}
		\includegraphics[scale=0.3]{images/hall/augmenting_path_after.png}
		\caption{Ścieżka powiększająca przed i po zamianie krawędzi}
	\end{figure}
	\pagebreak

	No dobra, ale co ma ścieżka powiększająca do twierdzenia Halla? Okazuje się że jest ona bardzo wygodnym narzędziem.

	Załóżmy sobie nie wprost, że mamy jakiś graf dwudzielny \(G = (X,Y,E)\), w którym zachodzi warunek Halla ale nie ma dopasowania doskonałego. W takim razie weźmy dopasowanie maksymalne \(M\). Istnieje jakiś \(x\), który nie należy do tego dopasowania (bo nie jest doskonałe). Z warunku Halla (\(|A| \leq |N(A)|\) dla dowolnego \(A \subset X\)) mam, że \(x\) musi mieć jakiegoś sąsiada w \(Y\). Zbiór wszystkich wierzchołków, z którymi połączony jest \(x\) (być może jest ich więcej, być może tylko jeden) oznaczam jako \(B_0\). Każdy wierzchołek z \(B_0\) musi należeć do dopasowania \(M\) (bo inaczej mógłbym je rozszerzyć, biorąc krawędź między tym wierzchołkiem a \(x\)). Wszystkie wierzchołki z \(X\) które są razem w dopasowaniu z wierzchołkami z \(B_0\) oznaczam jako \(A_1\). Oczywiście \(|A_1| = |B_0|\). Zauważmy, że \(|A_1 \cup \{x\}| \geq |B_0|\), a zatem musi istnieć jakiś zbiór wierzchołków \(B_1\) który ma krawędzie do wierzchołków zbioru \(A_1\). Ponownie, wszystkie krawędzie z \(B_1\) muszą być w dopasowaniu, bo inaczej mielibyśmy ścieżkę powiększającą (aha!) od \(x\) do jakiegoś wierzchołka z \(B_1\). W takim razie mamy jakiś zbiór \(A_2\) wierzchołków które są w dopasowaniu z wierzchołkami z \(B_1\), ponownie \(|A_2| = |B_1|\). \(|A_2| + |A_1| + 1 \geq |B_0| + |B_1|\), skąd wierzchołki z \(A_2\) łączą się jeszcze z jakimiś innymi wierzchołkami z \(Y\), ich zbiór nazwiemy \(B_2\). I ponownie, wierzchołki z \(B_2\) muszą być w dopasowaniu, bo inaczej mielibyśmy ścieżkę powiększającą. Korzystamy teraz z faktu który zawsze zakładamy, tj. faktu że grafy są skończone.

	\epigraph{I tak dalej, aż do wyczerpania zasobów}{\textit{Stefan ,,Siara'' Siarzewski do senatora Ferdynanda Lipskiego, ,,Kilerów-ów 2-óch''}}

	Nietrudno bowiem zauważyć, że w końcu wierzchołki się skończą i albo dostaniemy sprzeczność z założeniem że warunek Halla zachodzi, albo wyjdzie nam ścieżka powiększająca (a dopasowanie miało być maksymalne). Tym samym kończymy dowód.
\end{proof}


\realsection{Kolorowanie grafów, twierdzenia Brooksa}
\epigraph{
	Zgodnie z twierdzeniem Vizinga każdy graf można pokolorować przy użyciu co najwyżej \(\Delta(G)+1\) kolorów. Twierdzenie Brooksa określa dla jakich grafów to ograniczenie jest osiągane.
}{\textit{
		Użytkownik ,,Esculapa'' na polskojęzycznej Wikipedii w artykule ,,Twierdzenie Brooksa''}}

Oczywiście przytoczona wypowiedź jest nonsensem gdyż, jak dobrze wiemy, twierdzenie Vizinga mówi o kolorowaniu \textbf{krawędziowym}.
Wypowiedzmy zatem poprawną formę twierdzenia Brooksa.

\begin{theorem}[Brooks]
	Jeśli spójny graf \(G\) jest kliką lub cyklem nieparzystym to \(\chi(G) = \Delta(G) + 1\).
	W przeciwnym razie \(\chi(G) \leq \Delta(G)\)
\end{theorem}

\begin{proof}
	Widzimy, że cykl nieparzysty wymaga użycia \(3 = \Delta(G) + 1\) kolorów,
	a w przypadku kliki sąsiedzi każdego wierzchołka używają \(\Delta(G)\) kolorów, a jeszcze jeden potrzebujemy na ten wierzchołek. Przyjmijmy zatem, że nasz graf \(G\) nie jest
	ani cyklem nieparzystym ani kliką.

	Jeśli \(\Delta(G) \leq 2\) to \(G\) jest ścieżką lub cyklem parzystym i widzimy, że \(G\) jest dwudzielny.

	Niech \(\Delta(G) \geq 3\).

	Idea dowodu jest taka, że będziemy chcieli jakoś skonstruować kolorowanie używające co najwyżej \(\Delta(G)\) kolorów. W związku z tym będziemy inkrementalnie odfiltrowywać grafy, dla których takie kolorowanie stworzymy.
	Przedstawiam zatem kolejne własności grafu, dla którego będzie się trzeba trochę bardziej namęczyć.

	\begin{enumerate}
		\item \(G\) jest \(\Delta\)-regularny
		      Pokażemy, że w przeciwnym razie \(col(G) \leq \Delta\)
		      Jeśli \(G\) nie jest \(\Delta\)-regularny to w \(G\) istnieje wierzchołek \(v\) o stopniu mniejszym niż \(\Delta\).
		      Postawmy ten wierzchołek na końcu permutacji i spójrzmy na graf \(G - v\).
		      \(v\) miał jakichś sąsiadów, których stopień wynosił co najwyżej \(\Delta\).
		      W takim razie po usunięciu \(v\) jego byli sąsiedzi na pewno mają teraz stopień mniejszy niż \(\Delta\)
		      i możemy powtórzyć całe to rozumowanie aż skończą nam się wierzchołki i wygenerujemy całą permutację.
		      Zauważamy, że dzięki konstrukcji każdy wierzchołek ma na lewo mniej niż \(\Delta\) sąsiadów i dostajemy \(col(G) \leq \Delta\)

		      Wybierzmy dowolny wierzchołek \(v\) i oznaczmy \(H = G - v\).
		      Sąsiadom \(v\) zmniejszyliśmy stopień, zatem z powyższego wywodu wynika, że \(H\) jest \(\Delta\)-kolorowalny.
		      Pokolorujmy zatem \(H\) i przejdźmy do kolejnej własności.

		\item Sąsiedzi \(v\) dostają parami różne kolory w kolorowaniu grafu \(H\)
		      W przeciwnym razie któryś z \(\Delta\) kolorów jest wolny w \(v\) i możemy go użyć kończąc kolorowanie.

		      Nazwijmy sąsiadów \(v\) przez \(v_1, ..., v_\Delta\) i niech będą pokolorowani kolorami \(1, ..., \Delta\).

		      Oznaczmy \(C_{ij} = H[\{v \in V \mid c(v) \in \{i, j\}]\) - podgraf indukowany
		      grafu \(H\), który zawiera wszystkie wierzchołki w kolorach \(i, j\)

		\item Sąsiedzi \(v_i, v_j\) leżą w tym samym komponencie grafu \(C_{ij}\)
		      W przeciwnym razie możemy wziąć komponent do którego należy \(v_i\)
		      i przekolorować go tak, że wierzchołki o kolorze \(i\) dostają kolor \(j\),
		      a te o kolorze \(j\) dostają kolor \(i\). Tym samym wierzchołki \(v_i, v_j\)
		      otrzymują oba kolor \(j\) sprowadzając problem do poprzedniego podpunktu.

		\item Każdy \(C_{ij}\) jest ścieżką.
		      Założmy że tak nie jest i weźmy pierwszy licząc od \(v_i\) wierzchołek, który ma co najmniej trzech sąsiadów w \(C_{ij}\) i nazwijmy go \(x\).
		      Bez straty ogólności powiedzmy, że \(x\) ma kolor \(i\).
		      Rozważmy kolory jakie mają sąsiedzi \(x\).
		      Co najmniej trzech sąsiadów ma kolor \(j\), a pozostałych jest \(\Delta - 3\)
		      W takim razie sąsiedzi \(x\) używają co najwyżej \(\Delta - 2\) różnych kolorów. Oczywiście jeden z wolnych kolorów to \(i\), ale możemy teraz przekolorować \(x\) na ten drugi.
		      Ponieważ \(x\) był pierwszym rozgałęzieniem między \(v_i\) a \(v_j\)
		      to po przekolorowaniu \(v_i\) oraz \(v_j\) muszą leżeć w różnych komponentach nowego \(C_{ij}\)

		      \begin{figure}[ht]
			      \centering
			      \includegraphics[scale=0.5]{images/brooks/branching_path.png}
			      \caption{Wierzchołek \(x\) ma trzech sąsiadów w kolorze \(j\)}
		      \end{figure}

		\item Każde dwa \(C_{ij}, C_{ik}, k \neq j\) przecinają się tylko w \(v_i\)
		      W przeciwnym razie istnieje wierzchołek \(x\) w kolorze \(i\),
		      który ma dwóch sąsiadów w kolorze \(j\) i dwóch sąsiadów w kolorze \(k\).
		      Jak się dobrze policzy to tak jak poprzednio wyjdzie nam, że jakiś kolor jest wolny i możemy zrobić ten sam myk z przekolorowaniem.

		      \begin{figure}[ht]
			      \centering
			      \includegraphics[scale=0.5]{images/brooks/branching_path_three_colors.png}
			      \caption{Wierzchołek \(x\) ma dwóch sąsiadów w kolorze \(j\) i dwóch w kolorze \(k\)}
		      \end{figure}

		\item Istnieje para \(v_i, v_j\), która nie jest połączoną krawędzią.
		      W przeciwnym razie wierzchołki \(v, v_1, ..., v_\Delta\) tworzą klikę, co jest sprzeczne z założeniem.

		      Bez straty ogólności powiedzmy, że \(v_1\) i \(v_2\) nie są połączone krawędzią. Jednak z własności (5) musi istnieć ścieżka między nimi. Niech więc \(u\) to będzie pierwszy wierzchołek w kolorze \(1\) na ścieżce od \(v_2\) do \(v_1\).

		      Teraz dzieje się magia.
		      W \(C_{23}\) zamieniamy kolory \(2\) i \(3\) i takie kolorowanie przepuszczamy przez warunki \((2) - (5)\).
		      Jeśli w którymś miejscu udało nam się stworzyć dobre kolorowanie to super, a jeśli nie to ups.
		      Na szczęście zauważamy teraz fajną rzecz. Otóż wierzchołek \(u\) nadal jest połączony ścieżką w kolorach \(1\) i \(2\) z wierzchołkiem \(v_1\), zatem należy do komponentu \(C_{12}\), ale z drugiej strony jest połączony krawędzią z wierzchołkiem \(v_2\), który ma teraz kolor \(3\) zatem należy też do komponentu \(C_{13}\).
		      W takim razie nowe kolorowanie narusza warunek \((5)\) co już umiemy rozwiązać.

		      \begin{figure}[H]
			      \centering
			      \includegraphics[scale=0.4]{images/brooks/disconnected_before_swap.png}
			      \includegraphics[scale=0.4]{images/brooks/disconnected_after_swap.png}
			      \caption{Wierzchołki \(v_1\) i \(v_2\) przed i po przekolorowaniu komponentu \(C_{23}\)}
		      \end{figure}


	\end{enumerate}

	To tyle, nie ma więcej warunków, które musimy rozważać. Fajnie.

\end{proof}


\realsection{Liczba chromatyczna a liczba kolorująca grafów}
\subsection{Co to w ogóle jest}
Liczba kolorująca jest źródłem konfuzji dla niezliczonych studentów. Co ona w ogóle robi, po co jest i dlaczego nie ma jej w żadnej literaturze? Na to ostatnie pytanie nie odpowiem, ale liczba kolorująca, oznaczana również jako \(col(G)\) jest użyteczna, bo, jak za niedługo pokażemy, ogranicza od góry \(\chi(G)\) oraz da się ją obliczyć w czasie wielomianowym (na wypadek gdyby ktoś się jeszcze nie zorientował, kolorowanie grafów jest problemem klasy NP-przykrej). Poza tym jestem zdania, że nazwanie tego \textit{liczba kolorująca} to fatalna decyzja, bo dosyć ładnie kamufluje co to w ogóle jest; w tym momencie chciałbym napisać coś śmiesznego o matematykach i dziwnych decyzjach, ale nic śmiesznego nie przychodzi mi do głowy.
\subsubsection{Definicja}
Rozpatrzmy wszystkie permutacje wierzchołków grafu \(G\). Zdefiniujmy funkcję \(f\), która dla każdego wierzchołka w obrębie danej permutacji przypisuje liczbę jego sąsiadów, którzy wystąpili ,,przed nim'' w permutacji. Zdefiniujmy funkcję \(g\), która dla danej permutacji wynosi maksymalnej wartości funkcji \(f\) policzonej dla wszystkich wierzchołków z tej permutacji i dodaje do tego jeden. Skąd wytrzasnęła się ta jedynka zaraz się dowiemy, obiecuję że to nie jest kolejny arbitralny wymysł matematyków którzy wypili za dużo kawy (tak jak nazwa tego bytu).

Fajnie teraz zauważyć, że \(g\) dla danej permutacji oszacowuje z góry jak bardzo może ,,skopać'' kolorowanie algorytm First-Fit, jeśli pokoloruje wierzchołki zgodnie z tą permutacją; w najgorszym przypadku, gdy istnieje jakiś wierzchołek mający \(d\) sąsiadów ,,na lewo'' i wszyscy mają różne kolory, to FF da mu kolor \(d+1\). Stąd maksymalna liczba sąsiadów na lewo wśród wierzchołków (\(+1\)) daje nam górne oszacowanie na to, jak First-Fit może popsuć kolorowanie \textit{jeśli będzie kolorować według tej permutacji}.

Nikogo teraz chyba nie zdziwi fakt, że \(col(G)\) to jest po prostu minimum po funkcjach \(g\) dla wszystkich permutacji wierzchołków grafu \(G\).
\subsection{Relacja z liczbą chromatyczną}
\subsubsection{Ograniczenie liczby chromatycznej przez liczbę kolorującą}
\begin{theorem}[Relacja liczby kolorującej z liczbą chromatyczną]
	\begin{equation}
		\chi(G) \leq col(G)
	\end{equation}
\end{theorem}
\begin{proof}
	Gdy tak nie było, to istniałaby taka permutacja wierzchołków grafu że First-Fit pokolorowałby graf lepiej niż \(\chi(G)\) mówi, że da się pokolorować. Koniec dowodu. Serio.
\end{proof}
\subsubsection{Ograniczenie liczby kolorującej przez jakąś funkcję liczby chromatycznej}
Nie da się. Znaczy da się, ale tylko w pewnych klasach grafów. W ogólnej klasie grafów istnieją takie grafy, że liczba kolorująca leci do nieskończoności, a \(\chi(G) = 2\). Grafem takim jest na przykład klika dwudzielna \(K_{n,n}\). Swoją drogą to jest protip: jeśli potrzebujesz udowodnić że \(col(G)\) nie da się ograniczyć w jakiejś klasie grafów przez funkcję \(\chi(G)\), spróbuj pokazać że da się tam skonstruować klikę dwudzielną. Nie ma za co.
\subsection{Algorytm obliczania liczby kolorującej}
\begin{theorem}[Magiczny wzór na liczbę kolorującą]
	\begin{equation}
		col(G) = \mathrm{max}\{ \delta(H) : H \subset_{ind.} G \} + 1
	\end{equation}
\end{theorem}

\begin{proof}
	Ten wzór wygląda na początku dziwnie, ale w sumie ma sens. Dowodzić to będziemy trochę dziwnie, bo zamiast pokazywać od razu równość, pokażemy ograniczenie z góry i z dołu (skąd będziemy mieć, że istotnie zachodzi równość).

	Pokażmy zatem nierówność w pierwszą stronę:
	\begin{equation*}
		col(G) \geq \mathrm{max}\{ \delta(H) : H \subset_{ind.} G \} + 1
	\end{equation*}

	Dowód jest dosyć prosty; jeśli \(col(G) = k\) dla jakiegoś \(k\), to znaczy to że istnieje jakaś permutacja wierzchołków \(G\) taka, że każdy wierzchołek ,,na lewo'' ma  co najwyżej \(k-1\) sąsiadów. Teraz dla każdego podgrafu indukowanego \(H\) jest tak, że gdzieś w tej permutacji jest ,,ostatni'' wierzchołek należący do tego podgrafu indukowanego. Nie ma on w ogóle krawędzi ,,na prawo'' i ma same krawędzie ,,na lewo''. Minimalnie może ich mieć \(\delta(H)\), czyli w takim razie \(\delta(H) \leq k - 1\), skąd mamy że \(k \geq \delta(H) + 1\) dla dowolnego podgrafu indukowanego (skąd mamy już tezę).

	W drugą stronę dowód przy okazji pokazuje nam wielomianowy algorytm obliczania \(col(G)\). Nie ukrywam że jest on bardzo fajny.

	\begin{equation*}
		col(G) \leq \mathrm{max}\{ \delta(H) : H \subset_{ind.} G \} + 1
	\end{equation*}

	Konstruujemy sobie permutację wierzchołków grafu \(G\). W jaki sposób? Bierzemy wierzchołek o najmniejszym stopniu i wrzucamy go \textit{na koniec} permutacji. Następnie rozpatrujemy podgraf indukowany \(H\), bez tamtego wierzchołka o najmniejszym stopniu. \(H\) znowu ma jakiś wierzchołek o minimalnym stopniu, więc dorzucam go na koniec permutacji (przed wcześniejszym wierzchołkiem o minimalnym stopniu) i kontynuuję ,,obgryzanie''. Nietrudno zauważyć, że każdy wierzchołek ma ,,na lewo'' od siebie jakieś \(\delta(H)\) sąsiadów (gdzie \(H\) jest jakimś podgrafem indukowanym). Tym samym col jest z pewnością mniejszy lub równy niż maksymalny minimalny stopień wierzchołka w jakimś podgrafie indukowanym (\(+1\)).

	Pokazaliśmy więc, że nasz ,,algorytm'' generuje optymalną permutację, bo pokazaliśmy wcześniej że lepiej się nie da (pokazując ograniczenie dolne na \(col(G)\)). Jednocześnie dowodzi to równości.
\end{proof}



\realsection{Kolorowania krawędziowe grafów. Twierdzenie Vizinga}
Jest to dowód, który najłatwiej zrozumieć samemu rozrysowując sobie proces na kartce. Niemniej, z pomocą rysunków spróbuję go Wam przybliżyć.

\begin{theorem}[Vizing]
	\[\Delta(G) \leq \chi'(G) \leq \Delta(G) + 1\]
\end{theorem}

\begin{proof}
	Ograniczenie dolne widzimy od razu -- \(\Delta\) krawędzi spotykających się w jednym wierzchołku musi dostać różne kolory.

	Ograniczenie górne pokazujemy indukcją po liczbie pokolorowanych krawędzi. Jedną krawędź umiemy pokolorować bez problemu.
	Weźmy zatem częściowe kolorowanie i powiedzmy, że chcemy pokolorować krawędź \((x, y)\).

	Skoro mamy do dyspozycji \(\Delta + 1\) kolorów to znaczy, że każdy wierzchołek ma jakiś kolor wolny (nie wychodzi z niego krawędź w tym kolorze).

	Obserwacja, z której będziemy dużo korzystać: jeśli dowolne wierzchołki \(x\), \(y\) połączone krawędzią mają wolny ten sam kolor \(\beta\)
	to krawędź między nimi możemy pokolorować na tenże kolor. Kolory wolne dla danego wierzchołka będziemy oznaczać linią przerywaną. dotykającą tego wierzchołka.

	\begin{figure}[ht]
		\centering
		\includegraphics[scale=0.6]{images/vizing/trivial_case.png}
		\caption{prosty przypadek; \(x\) i \(y\) mają wolny ten sam kolor \(\beta\)}
	\end{figure}

	Załóżmy więc, że mamy pecha i wierzchołki \(x, y\) nie mają wspólnych wolnych kolorów tj. jeśli \(x\) ma wolny kolor \(\beta\)
	to \(y\) ma ten kolor zajęty i vice versa. Dodatkowo niech \(\alpha_0\) będzie wolnym kolorem wierzchołka \(y\).

	Od tego momentu będziemy tak kombinować, żeby \(x\) zwolnić \(\alpha_0\), być może zajmując przy tym \(\beta\).

	\begin{figure}[H]
		\centering
		\includegraphics[scale=0.6]{images/vizing/pre_step_one.png}
		\caption{\(x\) i \(y\) nie mają wspólnych wolnych kolorów}
	\end{figure}


	Niech \(x_0\) będzie taki, że krawędź \((x, x_0)\) ma kolor \(\alpha_0\). Jeśli \(x_0\) ma wolny kolor \(\beta\)
	to krawędź \((x, x_0)\) możemy przekolorować na kolor \(\beta\),
	tym samym sprawiając, że \(x\) ma wolne \(\alpha_0\).
	Ale w takiej sytuacji możemy pokolorować \((x, y)\) na kolor \(\alpha_0\).

	Zatem sytuacja ma się teraz tak:

	\begin{figure}[ht]
		\centering
		\includegraphics[scale=0.6]{images/vizing/step_one.png}
		\caption{\(x\) i \(y\) mają wolne różne kolory, \(x_0\) ma zajętą \(\beta\) i wolne \(\alpha_0\)}
	\end{figure}

	Jeżeli teraz \(x\) ma wolny kolor \(\alpha_1\) to krawędź \((x, x_0)\)
	możemy przekolorować na \(\alpha_1\), a krawędź \((x, y)\) na \(\alpha_0\). Niech więc \((x, x_1)\) będzie w kolorze \(\alpha_1\).

	Jeśli \(x_1\) miałby wolny kolor \(\beta\) to możemy przekolorować \((x, x_1)\) na \(\beta\), wtedy \(x\) zwalnia się \(\alpha_1\) a z tym wiemy co robić.
	No to niech w \(x_1\) \(\beta\) będzie zajęta, a wolny będzie kolor \(\alpha_2\). Poniżej ilustracja:

	\begin{figure}[ht]
		\centering
		\includegraphics[scale=0.6]{images/vizing/step_two.png}
		\caption{\(x_1\) ma zajętą \(\beta\) a wolne \(\alpha_2\)}
	\end{figure}

	Podobnie jak wcześniej stwierdzamy, że z \(x\) wychodzi krawędź w kolorze \(\alpha_2\), bo inaczej przekolorujemy \((x, x_1)\) na \(\alpha_2\). Kontynuujemy to rozumowanie aż napotkamy wierzchołek \(x_k\) o wolnym kolorze \(\alpha_j\), który już znajduje się wśród kolorów \(\alpha_0, ..., \alpha_{k-1}\)


	\begin{figure}[ht]
		\centering
		\includegraphics[scale=0.6]{images/vizing/step_three.png}
		\caption{\(x_k\) ma wolny kolor \(\alpha_j\), który już widzieliśmy.}
	\end{figure}

	Niestety nie możemy wykonać tej samej sztuczki z przekolorowaniem co wcześniej, ale to nic nie szkodzi bo zrobimy co innego.
	Otóż wyjdźmy z wierzchołka \(x_k\) i pójdźmy ścieżką w kolorach na przemian \(\beta\) i \(\alpha_j\).
	Oczywiście kiedyś skończą nam się krawędzie i wylądujemy w jakimś wierzchołku \(v\).

	Rozważmy sobie teraz przypadki czym ten wierzchołek \(v\) jest.

	\begin{enumerate}
		\item \(v \notin \{x, x_0, x_1, ..., x_k\}\)
		      Najfajniejszy przypadek - ścieżka kończy się w niezbyt istotnym miejscu. Zamieniamy kolory na ścieżce miejscami. Możemy tak zrobić, bo wewnętrznym wierzchołkom się nic nie zmienia, a na końcach odpowiedni kolor jest wolny.

		      Teraz, począwszy od krawędzi \((x, x_k)\) przekolorujemy wachlarz.
		      Dzięki przekolorowaniu, \(x_k\) ma teraz wolną \(\beta\) tak jak \(x\), zatem \((x, x_k)\) możemy dać kolor \(\beta\)
		      W takim razie \(x\) ma teraz wolny kolor \(\alpha_k\) tak jak \(x_{k - 1}\),
		      zatem \((x, x_{k-1})\) dostanie kolor \(\alpha_k\).
		      Podobnie \((x, x_{k-2})\) dostanie kolor \(\alpha_{k-1}\).

		      W końcu dojdziemy do \((x, x_0)\), które dostanie kolor \(\alpha_1\). Sprawiliśmy, że \(x\) ma wolne \(\alpha_0\),
		      więc z czystym sumieniem kolorujemy \((x, y)\) na \(\alpha_0\).

		      \begin{figure}[H]
			      \centering
			      \includegraphics[scale=0.45]{images/vizing/fan_case_one_before.png}
			      \includegraphics[scale=0.45]{images/vizing/fan_case_one_after.png}
			      \caption{Przekolorowanie wachlarza gdy ścieżka z \(x_k\) kończy się w poza wierzchołkami \(x, x_0, ..., x_k\)}
		      \end{figure}

		\item \(v = x_{j - 1}\)
		      Taka sytuacja niestety może zajść, bo \(x_{j-1}\) ma wolny kolor \(x_j\) i zajętą \(\beta\).
		      Zauważmy, że nie możemy zrobić tego co w przypadku pierwszym, bo przekolorowanie ścieżki sprawia, że \(x_{j-1}\) ma kolor \(\alpha_j\) zajęty, a taki kolor by otrzymał przy poprawianiu wachlarza. Zrobimy zatem co innego.

		      Tak jak wcześniej przekolorujemy ścieżkę,
		      ale zamiast przekolorowywać krawędź \((x, x_k)\) na \(\beta\)
		      przekolorujemy \((x, x_{j-1})\) na \(\beta\).
		      Dalej możemy kontynuować tak jak poprzednio: \((x, x_{j-1})\) dostanie kolor \(\alpha_{j-2}\) itd. (Musieliśmy przyjść do \(v\) krawędzią o kolorze \(\beta\) bo inaczej nie byłby to koniec ścieżki)


		      \begin{figure}[H]
			      \centering
			      \includegraphics[scale=0.45]{images/vizing/fan_case_three_before.png}
			      \includegraphics[scale=0.45]{images/vizing/fan_case_three_after.png}
			      \caption{Przekolorowanie gdy ścieżka z \(x_k\) kończy się w \(x_{j-1}\)}
		      \end{figure}

	\end{enumerate}



\end{proof}


\realsection{Przepływy w sieciach. Twierdzenie o maksymalnym przepływie i minimalnym
	przekroju}
\section{Rozwiązywanie rekurencji liniowych}
\epigraph{I can elaborate: zrobiłam zadanka, zobaczyłam tworzące, stwierdziłam, że chce mi się spać, poszłam sobie}{\textit{Studentka TCSu o zadaniach z funkcji tworzących na kolokwium}}


\subsection{Rozkład na ułamki proste}
To nie jest formalny dowód ani formalna własność ani nic, bardziej schemat postępowania przy rozkładzie na ułamki proste. Sam dowód tego, że rozkład na ułamki proste istnieje, to \textit{sprowadź do wspólnego mianownika i zobacz co Ci wyszło}.
Jeżeli \(deg(P(x)) < deg(Q(x))\) i \(Q(x) = (x-a)^n \cdot (x-b)^k\) to:
\begin{equation*}
	\frac{P(x)}{Q(x)} = \frac{P(x)}{(x-a)^n \cdot (x-b)^k} = \frac{A_1}{x-a} + \frac{A_2}{(x-a)^2} + \dots + \frac{A_n}{(x-a)^n} + \frac{B_1}{x-b} + \frac{B_2}{(x-b)^2} + \dots + \frac{B_k}{(x-b)^k}
\end{equation*}

Oczywiście ten schemat można rozszerzać na więcej śmiesznych rzeczy w mianowniku, ale chyba widać o co chodzi.


\section{Ciąg Fibbonaciego}
\begin{theorem}[Wzór Bineta]
	\begin{equation}
		f_n = \frac{1}{\sqrt{5}} \cdot \left( \left(\frac{1 + \sqrt{5}}{2}\right)^{n} - \left(\frac{1 - \sqrt{5}}{2}\right)^{n} \right)
	\end{equation}
\end{theorem}

\begin{proof}
	Rozpisujemy sobie funkcję tworzącą ciągu \(f_n\):

	\begin{equation*}
		F(x) = f_0 + f_1 \cdot x + f_2 \cdot x^2 + f_3 \cdot x^3 \dots =
	\end{equation*}
	\begin{equation*}
		= f_0 + f_1 \cdot x + (f_0 + f_1) \cdot x^2 + (f_1 + f_2) \cdot x^3 + \dots =
	\end{equation*}
	\begin{equation*}
		= f_0 + f_1 \cdot x + f_0 \cdot x^2 + f_1 \cdot x^2 + f_1 \cdot x^3 + f_2 \cdot x^3 + \dots =
	\end{equation*}
	\begin{equation*}
		= f_0 + f_1 \cdot x + f_0 \cdot x^2 + f_1 \cdot x^3 + \dots + f_1 \cdot x^2 +  f_2 \cdot x^3 + \dots =
	\end{equation*}
	\begin{equation*}
		= f_0 + f_1 \cdot x + x^2 \cdot (f_0 + f_1 \cdot x + \dots) + x \cdot (f_1 \cdot x +  f_2 \cdot x^2 + \dots) =
	\end{equation*}
	\begin{equation*}
		= f_0 + f_1 \cdot x + x^2 \cdot F(x) + x \cdot (F(x) - f_0) =
	\end{equation*}
	\begin{equation*}
		= 0 + 1 \cdot x + x^2 \cdot F(x) + x \cdot (F(x) - 0) =
	\end{equation*}
	\begin{equation*}
		= x + x^2 \cdot F(x) + x \cdot F(x)
	\end{equation*}

	W takim razie mamy, że:
	\begin{equation*}
		F(x) = x + x^2 \cdot F(x) + x \cdot F(x)
	\end{equation*}
	\begin{equation*}
		F(x) -  x^2 \cdot F(x) - x \cdot F(x)  = x
	\end{equation*}
	\begin{equation*}
		F(x) \cdot (1 - x^2 - x) = x
	\end{equation*}
	\begin{equation*}
		F(x) = \frac{x}{-x^2 -x + 1}
	\end{equation*}

	Mianownik możemy rozbić (za pomocą liczenia jakichś delt czy coś):
	\begin{equation*}
		F(x) = \frac{x}{(-1) \cdot \left(x - \left(- \frac{1 + \sqrt{5}}{2}\right)\right) \cdot \left(x - \left(- \frac{1 - \sqrt{5}}{2}\right)\right)}
	\end{equation*}

	Nie no, serio, jeśli ktoś myśli że będę TeXować te przekształcenia to się myli. Powinno wyjść po przekształceniach że:
	\begin{equation*}
		F(x) = \frac{x}{(1-ax) \cdot (1-bx)}
	\end{equation*}
	gdzie \(a = \frac{1 + \sqrt{5}}{2}, b=\frac{1 - \sqrt{5}}{2}\)

	Dalej rozbijamy na ułamki proste:
	\begin{equation*}
		F(x) = \frac{A}{1-ax} + \frac{B}{1-bx}
	\end{equation*}
	\(A\) powinno wyjść \(\frac{1}{\sqrt{5}}\), \(B\) powinno wyjść \(- \frac{1}{\sqrt{5}}\).

	Odwijamy każdą z tych funkcji tworzących z osobna, korzystając ze wzoru podanego we wcześniejszym rozdziale i otrzymujemy wzór.
\end{proof}


\section{Ciąg Catalana}
Liczba Catalana jest to liczba ścieżek długości \(2n\) w kwadracie \(n \times n\) ,,poniżej'' przekątnej (lub na jej poziomie), idących za każdym razem jednostkę do góry lub jednostkę w prawo. Ścieżki takie nazywamy ścieżkami Dycka. Niezwykle formalna definicja. To jest jedna z tych rzeczy, które chyba po prostu trzeba narysować.

\begin{figure}[h]
	\centering
	\includegraphics[scale=0.5]{images/catalan/all_paths_1.png}
  \caption{Ścieżki Dycka długości 2; \(c_1 = 1\)}
\end{figure}

\begin{figure}[h]
	\centering
	\includegraphics[scale=0.5]{images/catalan/all_paths_2.png}
  \caption{Ścieżki Dycka długości 4; \(c_2 = 2\)}
\end{figure}

\begin{figure}[ht]
	\centering
	\includegraphics[scale=0.5]{images/catalan/all_paths_3.png}
  \caption{Ścieżki Dycka długości 6; \(c_3 = 5\)}
\end{figure}


\subsection{Wzór kombinatoryczny}
\begin{theorem}[Wzór kombinatoryczny na liczby Catalana]
	\begin{equation}
		c_n = \frac{1}{n+1} \cdot \binom{2n}{n}
	\end{equation}
\end{theorem}

Mamy sobie nasz kwadrat \(n \times n\). Przekątną możemy opisać tak jakby wzorem \(y = x\) (tak intuicyjnie, bo nie działamy w żadnym układzie współrzędnych, bla bla bla). Robimy sobie teraz prostą \(y = x+1\), idącą jakby ,,o jednostkę wyżej''. Zauważamy, że jeśli jakaś ścieżka przekracza linię naszej przekątnej, to musi ,,dotknąć'' linii \(y = x+1\). \textit{To widać}. Teraz wpadamy na świetny pomysł; jeśli jakaś ścieżka idąca po tym kwadracie ,,spotyka się'' z \(y = x+1\), to od tego momentu odbijamy ją symetrycznie względem \(y = x+1\). Zauważamy, że ścieżka ta (po odbiciu) skończy się w punkcie \((n-1, n+1)\) zamiast w \((n,n)\). Fakt ten dowodzimy stosując dowód przez rysowanie.

\begin{figure}[ht]
	\centering
	\includegraphics[scale=0.5]{images/catalan/path_with_reflection_1.png}
	\includegraphics[scale=0.5]{images/catalan/path_with_reflection_2.png}

	\caption{Przykłady odbicia niepoprawnej ścieżki}
\end{figure}

Zauważamy fascynujący fakt, mianowicie dwie różne ścieżki będą mieć 2 różne odbicia, a więc nasze przekształcenie jest iniektywne. Ponadto, jak sobie zobaczymy jakąkolwiek ścieżkę zaczynającą się w \((0,0)\), ale kończącą się w \((n-1,n+1)\), to jesteśmy w stanie zobaczyć gdzie pierwszy raz przecina się z \(y = x+1\), a następnie ją odbić, otrzymując ścieżkę idącą do \((n,n)\) i niebędącą ścieżką Dycka, której odbicie daje wyjściową ścieżkę. Zatem odbijanie jest suriektywne. A to oznacza tylko jedną rzecz: bijekcję między ścieżkami które ,,nie są catalanowe'', a ścieżkami ,,odbitymi''.

Wszystkich możliwych ścieżek od \((0,0)\) do \((n,n)\) mamy \(\binom{2n}{n}\), bo długość naszej drogi ma \(2n\) i wybieramy sobie \(n\) miejsc gdzie idziemy w prawo. Wszystkich możliwych ścieżek od \((0,0)\) do \((n-1,n+1)\) (czyli tych które są ,,złe'') mamy \(\binom{2n}{n-1}\), bo, analogicznie, ścieżka jest długości \(2n\) ale w prawo idziemy \(n-1\) razy. To prowadzi nas do wyniku:
\begin{equation*}
	\begin{split}
		c_n
		&= \binom{2n}{n} - \binom{2n}{n-1} \\
		&= \frac{(2n)!}{n! \cdot n!} - \frac{(2n)!}{(n-1)! \cdot (n+1)!} \\
		&= \frac{(n+1) \cdot (2n)!}{n! \cdot (n+1)!} - \frac{n \cdot (2n)!}{n! \cdot (n+1)!}\\
		&= \frac{(2n)!}{n! \cdot (n+1)!} \\
		&= \frac{1}{n+1} \cdot \frac {(2n)!}{n! \cdot n!} \\
		&= \frac{1}{n+1} \cdot \binom{2n}{n}
	\end{split}
\end{equation*}
\subsection{Zależność rekurencyjna}

\begin{theorem}[Wzór rekurencyjny na liczby Catalana]
	\begin{equation}
		c_n = c_{0} \cdot c_{n-1} + c_{1} \cdot c_{n - 2} + \dots + c_{n-1} \cdot c_0
	\end{equation}
\end{theorem}

\begin{proof}
	Znowuż mamy kwadrat \(n \times n\), ale tym razem dorysowujemy sobie prostą \(y = x - 1\). Każda ścieżka przetnie kiedyś tę linię i każda ścieżka dotknie kiedyś przekątnej \(y = x\) (można to udowodnić machając i pokazując na rysunek). Rzecz teraz ma się tak, że jeśli po ,,spotkaniu się'' z \(y = x - 1\) idziesz do góry, to potem musisz odbić w prawo (lub w skrajnym przypadku skończyłeś poprawną ścieżkę). Jednocześnie pierwszy wybór kierunku (tzn. ten w punkcie \((0,0)\) zawsze jest ,,w prawo'', bo jeśli ktoś pójdzie ,,do góry'' to znajdzie się w \((0,1)\), powyżej przekątnej \(y = x\)).

	Bierzemy sobie zatem pierwsze miejsce gdzie spotkałeś się z \(y = x\) i zauważamy, że jeśli dane jest ono jakimiś współrzędnymi \((i,i)\) to przecięliśmy \(y = x-1\) w \((i,i-1)\). Ponadto, ścieżka którą szliśmy od punktu \((1,0)\) do \((i,i-1)\) tak naprawdę jest ścieżką Dycka w kwadracie od punktów \((1,0)\), \((i, i-1)\) (kwadrat ten ma długość \(i-1\)). Ależ plot twist! Ścieżka którą idziemy od punktu \((i,i)\) do \((n,n)\) jest zaś już po prostu ścieżką Dycka w kwadracie o długości boku \(n-i\). Ścieżki te są od siebie niezależne i w ogóle, a długości tych ,,kwadratów catalanowych'' sumują się do \(i - 1 + n - i = n - 1\), więc teraz możemy zmajstrować wzór (w zależności od długości boków kwadratów, które z kolei są dyktowane tym kiedy się ,,spotkamy'' z \(y = x\)):
	\begin{equation*}
		c_n = \sum_{i = 0}^{n-1} c_i \cdot c_{n-1-i}
	\end{equation*}
	Co już można odwinąć do postaci która była w twierdzeniu.

	\begin{figure}[H]
		\centering
		\includegraphics[scale=0.5]{images/catalan/recursive_construction_1.png}
		\includegraphics[scale=0.5]{images/catalan/recursive_construction_2.png}

		\caption{Przykłady ,,podzielenia'' poprawnej ścieżki Dycka na podścieżki}
	\end{figure}

\end{proof}


\section{Zliczanie podziałów}

Chcemy pokazać fajny algorytm zliczania wszystkich podziałów liczby \(n\).

Oznaczmy liczbę wszystkich podziałów liczby \(n\) jako \(p(n)\). Jako ,,podziały liczby \(n\)'' mam na myśli liczbę sposobów na podzielenie liczby \(n\) na ileś składników (niezerowych), np. liczbę \(2\) mogę rozłożyć na \(1 + 1\) albo po prostu na \(2\) (i w sumie to tyle).  Funkcja tworząca ciągu \(p_n\) to: \begin{equation*}
	P(x) = (1 + x + x^2 + x^3 + \dots) \cdot (1 + x^2 + x^4 + x^6 + \dots) \cdot (1 + x^3 + x^6 + x^9 + \dots) \dots
\end{equation*}

Pierwszy nawias odpowiada wybraniu jedynki do podziału (i temu ile razy ją bierzemy), drugi dwójki, trzeci trójki, etc.

Oczywiście przy \(x^n\) będziemy mieli \(p_n\), jak to działa w funkcjach tworzących (i mam nadzieję, że widać dlaczego). Zapisujemy \(P(x)\) w fajniejszej postaci:

\begin{equation*}
	P(x) = \frac{1}{1-x} \cdot \frac{1}{1 - x^2} \cdot \frac{1}{1-x^3} \dots
\end{equation*}

Definiuję sobie \(Q(x) = (1-x) \cdot (1-x^2) \cdot (1-x^3) \dots\). Zauważam, że \(P(x) \cdot Q(x) = 1\), czyli \(Q(x)\) jest funkcją odwrotną do \(P(x)\). Okazuje się teraz, że \(Q(x)\) jest funkcją tworzącą pewnego śmiesznego ciągu, który sobie zaraz pokażemy.

Póki co musimy wprowadzić oznaczenia:
\begin{enumerate}
	\item \(e_n\) jest to liczba podziałów liczby \(n\) na parzystą liczbę składników parami różnych,
	\item \(o_n\) jest to liczba podziałów liczby \(n\) na nieparzystą liczbę składników parami różnych.
\end{enumerate}
Jak wszyscy powinniśmy już wiedzieć, funkcja tworząca ciągu \(e_n + o_n\) (czyli po prostu wszystkich podziałów \(n\) ze składnikami parami różnymi) wygląda tak:
\begin{equation*}
	(1+x) \cdot (1 + x^2) \cdot (1+x^3) \dots
\end{equation*}
Ten fakt do niczego nam się w sumie nie przyda, ale może pomóc zrozumieć co zaraz się stanie.

Możemy sobie teraz podumać, jaka jest funkcja tworząca ciągu \(e_n - o_n\). Otóż pojawia się tu plot twist, bo funkcja tworząca tego ciągu to po prostu \(Q(x)\):
\begin{equation*}
	(1-x) \cdot (1-x^2) \cdot (1-x^3) \dots
\end{equation*}

Działa to tak jak w powyższym przykładzie, z tym że jeśli wybraliśmy nieparzyście wiele składników to będzie nieparzyście wiele minusów i się ,,odejmie'' od współczynnika przy \(x^n\), a jeśli będzie parzyście wiele to się ,,doda''. Innymi słowy, do współczynnika przy \(x^n\) doda się 1 za każdy możliwy podział na parzyście wiele parami różnych składników, a odejmie się 1 za każdy możliwy podział na nieparzyście wiele parami różnych składników, czyli to co chcemy. Nie do końca mam pomysł jak to formalnie wytłumaczyć, więc proszę użyć swojej intuicji™.

Po co to wszystko? Okazuje się, że ciąg \(q_n = e_n - o_n\) ma pewne śmieszne własności (które niestety będzie trzeba udowodnić, brace yourselves).

\begin{theorem}[Eulera]
	\begin{equation}
		q_n = \begin{cases}
			0, \hspace{5pt} \mathrm{gdy} \hspace{5pt} n \not = \frac{(3 \cdot k \pm 1) \cdot k }{2} \\
			(-1)^k \hspace{5pt} \mathrm{wpp.}                                                       \\
		\end{cases}
	\end{equation}

\end{theorem}

\begin{proof}
	Zrobimy sobie przekształcenie \(f\), które przesyła prawie (dlaczego prawie to dojdziemy do tego za chwilę) każdy podział na \(n\) składników parami różnych na inny podział na \(n\) składników parami różnych (bijektywnie). Ktoś powie że sobie zrobiłem świetną bijekcję idącą z pewnego zbioru w samego siebie, but hear me out: ta bijekcja będzie mieć tę śmieszną własność, że jeśli podział był na parzyście wiele składników to będzie przesłany na nieparzyście wiele, a jeśli na nieparzyście wiele to będzie przesłany na parzyście wiele składników. To będzie fajne, bo pokażemy sobie że jest ich tyle samo (poza przypadkami gdzie definicja tej funkcji się popsuje, ale o tym za chwilę).

	Generalnie to oznaczmy sobie najmniejszy składnik w podziale \(P\) jako \(a\). Ponadto, zdefiniujmy sobie zbiór \(X\), taki że zawiera on największe składniki podziału \(P\), takie że każde dwa sąsiednie różnią się o jeden. Innymi słowy, jeśli podział \(P = (\lambda_1, \lambda_2, \lambda_3, \dots, \lambda_k)\), to \(X =\{\lambda_1, \lambda_2, \lambda_3, \dots, \lambda_d\}\), gdzie \(d\) jest największą liczbą taką, że kolejne składniki różnią się o 1  (zakładamy, że \(\lambda_1 > \lambda_2 > \dots > \lambda_k\)).

	Teraz jak mamy te zbiory zdefiniowane to możemy robić śmieszne rzeczy. Jeśli \(|X| < a - 1\), to możemy przerobić nasz podział, odejmując od każdego elementu z \(X\) 1, i dorzucając nowy element do podziału, taki że równy jest on moc \(|X|\). Otrzymaliśmy oczywiście poprawny podział (niektórym może pomóc dowód przez rysowanie).

	Dlaczego \(|X| < a - 1\), a nie po prostu \(|X| < a\)? Otóż przychodzi tutaj pewien śmieszny problem, mianowicie może być tak, że składnik podziału o wartości \(a\) ,,wpadł'' do \(X\). W takim przypadku bijekcja nam się kompletnie popsuje i wtedy jej definiujemy (ale jeszcze do tego wrócimy). Natomiast jeśli \(a\) nie należy do \(|X|\) to nasza bijekcja nadal działa. Fajnie.

	Czyli reasumując: jeśli \(|X| < a - 1\) lub (\(|X| = a - 1\) i \(a \not \in X\)) od każdego składnika z \(|X|\) odejmujemy 1 i majstrujemy nowy składnik, który wrzucamy pod składnik o wartości \(a\), który uprzednio był najmniejszy.

	\begin{figure}[H]
		\centering
		\includegraphics{images/case2.png}
		\caption{Wizualizacja przekształcenia (diagram Ferrersa). 2 ,,górne'' składniki różnią się o 1, trzeci już różni się od nich o 2; \(|X| = 2\), \(a=3\).}
	\end{figure}

	Zasadniczo to samo będziemy czynić (ale w drugą stronę), gdy okaże się że \(a < |X| \). Ordynarnie \textit{wywalam} składnik \(a\) i do odpowiedniej liczby elementów z \(X\)  ,,dodaję'' 1, tak by się wyrównało. Należy zauważyć, że być może nie wszystkie elementy z \(X\) będą mieć coś do siebie dodane, ale to mi nic nie psuje. W sumie też fajnie byłoby dodać, że dodaję te jedynki najpierw największym składnikom; inaczej mogłoby to się popsuć.

	Co dzieje się, gdy \(a = |X|\)? Jeśli \(a \in X\) to jest mi smutno, w przeciwnym razie mogę zrobić to samo co robiłem wcześniej i wszystko działa jak powinno.

	\begin{figure}[H]
		\centering
		\includegraphics{images/case_1.png}
		\caption{Wizualizacja przekształcenia (diagram Ferrersa). 3 ,,górne'' składniki różnią się o 1 więc należą do \(X\). \(|X| = 3\), \(a = 2\), więc dwóm największym elementom dodajemy 1, a składnik \(a\) usuwamy.}
	\end{figure}

	Zostają więc 2 przypadki, gdy coś może się popsuć:
	\begin{enumerate}
		\item \(|X| = a - 1\), \(a \in X\)
		      \begin{figure}[H]
			      \centering
			      \includegraphics{images/irytujacy_1.png}
			      \caption{Gdy \(|X| = a - 1\) i składnik \(a\) jest w \(X\); widać, że nic nie możemy z tym zrobić.}
		      \end{figure}

		\item \(|X| = a\), \(a \in x\)
		      \begin{figure}[H]
			      \centering
			      \includegraphics{images/irytujacy_2.png}
			      \caption{Gdy \(|X| = a\) i składnik \(a\) jest w \(X\); również widać, że nasze przekształcenie nie zadziała.}
		      \end{figure}
	\end{enumerate}

	Zauważmy, że sytuacja gdy składnik \(a\) jest w \(X\) jest bardzo dziwną sytuacją generalnie, bo jest to najmniejszy składnik; z definicji \(X\) mamy wtedy, że wszystkie kolejne składniki w \(P\) różnią się o dokładnie 1. Na podstawie tej obserwacji możemy już dokładnie powiedzieć, jakiej postaci musi być \(n\), by miało taki ,,złośliwy'' podział:

	\begin{enumerate}
		\item Gdy \(|X| = a - 1\), \(a \in X\), to \(n\) musi dla jakiegoś \(k\) być postaci \((k + 1) + (k + 2) + \dots + 2k \) (\(|X| =k, a = k+1\), wszystko się zgadza)
		\item Gdy \(|X| = a\), \(a \in x\), to \(n\) musi dla jakiegoś \(k\) być postaci \(k + (k+1) + (k+2) + \dots + (2k - 1)\) (\(|X| = k\), \(a = k\), ponownie wszystko gra)
	\end{enumerate}

	Jak zastosujemy matematykę mniej dyskretną by wysumować te nawiasy, dostaniemy że \(n\) aby miało irytujący podział to musi być postaci \(\frac{k \cdot (3k+1)}{2}\) lub \({k \cdot (3k-1)}{2}\). Jednocześnie nie ma takiego naturalnego \(k\), że wartości te są sobie równe, więc jeśli \(n\) ma irytujący podział, to ma go tylko jednego. Wtedy nie możemy przerzucić tylko jednego podziału na inny (inne są ze sobą w bijekcji) więc \(e_n - o_n = (-1)^k\) (jeśli \(k\) jest parzyste to irytujący podział ma parzyście wiele składników, a w przeciwnym razie nieparzyście wiele). Jeśli irytujący podział nie występuje, \(e_n = o_n\) z bijekcji którą pokazaliśmy. Fajnie.
\end{proof}

Dobra, ale wróćmy do tego cośmy chcieli udowodnić na samym początku. Co w ogóle wynika z tego twierdzenia Eulera? No w sumie to bardzo dużo, bo jak mamy \(q_n = e_n - o_n\) i \(Q(x)\) jest jego funkcją tworzącą:
\begin{equation*}
	Q(x) = q_0 + q_1 \cdot x + q_2 \cdot x^2 + q_3 \cdot x^3 + \dots
\end{equation*}
Ale znamy wartości współczynników \(q_i\) z twierdzenia Eulera:
\begin{equation*}
	Q(x) = 1 - x - x^2 + x^5 + x^7 - x^{12} - x^{15} + x^{22} + x^{26} + \dots
\end{equation*}
Zauważmy, że współczynników które nie są zerowe jest tylko jakoś \(O(\sqrt{n})\), czyli dosyć mało.

Pamiętajmy, że \(P(x) \cdot Q(x) = 1\), czyli że ciąg który wyjdzie po ich wymnożeniu będzie wyglądać tak: \((1,0,0,0, \dots)\) Ponieważ mnożenie w funkcjach tworzących działa jakoś tak, że w wynikowym ciągu (nazwijmy go \(r\)) element \(r_n\) można obliczyć w ten sposób:
\begin{equation*}
	r_n = \sum_{i=0}^{n} p_i \cdot q_{n-i}
\end{equation*}

I wiemy że w naszym przypadku \(r_n = 0\) dla \(n > 1\), to mamy że: \begin{equation*}
	0 = p_n - p_{n-1} - p_{n-2} + p_{n-5} + p_{n-7} - p_{n-12} - \dots
\end{equation*}
To teraz \(p_n\) przerzucamy na drugą stronę i mnożymy stronami razy \(-1\) i mamy wzór na \(p_n\), które możemy obliczyć w \(O(\sqrt{n})\). No i fajnie.


}