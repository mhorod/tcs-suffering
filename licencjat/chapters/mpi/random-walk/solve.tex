Zamodelujemy nasz spacer nieco inaczej. Zamiast losować kąt z \([0, 2\pi]\), będziemy losować go z przedziału \( [0, \frac{\pi}{2}]\) a następnie będziemy z prawdopodobieństwem \(\frac{1}{2}\) ,,odbijać'' nasz wektor jednostkowy w osi OY (i niezależnie od tego z prawdopodobieństwem \(\frac{1}{2}\) ,,odbijać'' tenże wektor jednostkowy w osi OX).

Wyliczmy na początek o ile przesuwa się pionek w osi Y jak i osi X, jeśli wylosowaliśmy kąt \( \alpha_i \in [0, \frac{\pi}{2}] \). Sytuacja ma się wtedy następująco:

\begin{center}

	\begin{tikzpicture}[scale=1]

		\draw[->] (0,0) -- (5,0);
		\draw[->] (0,0) -- (0,5);
		\draw[] (0, 0) -- (2, 3);

		\draw[red, dashed] (2, 0) -- (2, 3);
		\draw[blue, dashed] (0, 3) -- (2, 3);

		\tkzDefPoints{0/0/O,3/0/A,2/3/B};
		\tkzMarkAngle[size = 1.25, arc=l,mkcolor = red,mkpos=.33](A,O,B);
		\tkzLabelAngle[dist=0.75](A,O,B){\(\alpha_i\)};

		\tkzDefPoint(1,3){X};
		\tkzLabelPoint[above, blue](X){\(x\)};

		\tkzDefPoint(2,1.5){Y};
		\tkzLabelPoint[right, red](Y){\(y\)};

		\tkzDefPoint(1,1.5){Z};
		\tkzLabelPoint[left](Z){\(1\)};

	\end{tikzpicture}
\end{center}

Wobec tego mamy, że:

\[
	\sin{\alpha_i} = \frac{y}{1} = y
\]
\[
	\cos{\alpha_i} = \frac{x}{1} = x
\]

Aby wyrazić formalnie dystans który przebywamy przez nas w \(i\)-tym kroku w osi OX (i OY) zdefiniujemy sobie 2 niezależne od siebie zmienne losowe:

\begin{enumerate}
	\item \(D_{X_i}\), która przyjmie \(1\) jeśli w \(i\)-tym kroku idziemy w prawo i \(-1\) jeśli idziemy w lewo. Oczywiście, \( \mathrm{P}(D_X = 1) = \frac{1}{2}\) i \(\mathrm{P}(D_X = -1) = \frac{1}{2}\).
	\item \(D_{Y_i}\), którą definiujemy analogicznie jak \(D_{X_i}\), ale dla \(i\)-tego kroku osi OY.
\end{enumerate}

Możemy teraz policzyć dystans przebywany przez nas w \(i\)-tym kroku w osiach OX i OY:

\[
	X_i = D_{X_i} \cdot \cos{\alpha_i}
\]

\[
	Y_i = D_{Y_i} \cdot \sin{\alpha_i}
\]

Przypomnijmy sobie, że w pytaniu nas pytają o kwadrat odległości od punktu \((0, 0)\), czyli o wartość wyrażenia:

\[
	\pars{\sum_{i=1}^{n} X_i}^2 + \pars{\sum_{i=1}^{n} Y_i}^2
\]

Rozpisujemy:

\begin{align*}
	\pars{\sum_{i=1}^{n} X_i}^2 + \pars{\sum_{i=1}^{n} Y_i}^2                                                                & = \sum_{i=1}^{n} X_{i}^2 + \sum_{i < j} X_{i}X_{j} + \sum_{i=1}^{n} Y_{i}^2 + \sum_{i < j} Y_{i} Y_{j}    \\
	                                                                                                                         & = \sum_{i=1}^{n} D_{X_i}^2 \cos^2{\alpha_i} + \sum_{i < j} D_{X_i}D_{X_j} \cos{\alpha_i} \cos{\alpha_j} + \\ + \sum_{i=1}^{n} D_{Y_i}^2 \sin^2{\alpha_i} + \sum_{i < j} D_{Y_i} D_{Y_j} \sin{\alpha_i} \sin{\alpha_j}
	                                                                                                                         & = \sum_{i=1}^{n} D_{X_i}^2 \cos^2{\alpha_i} + D_{Y_i}^2 \sin^2{\alpha_i} +                                \\
	+ \sum_{i < j} D_{X_i}D_{X_j} \cos{\alpha_i} \cos{\alpha_j} + \sum_{i < j} D_{Y_i} D_{Y_j} \sin{\alpha_i} \sin{\alpha_j} & = \sum_{i=1}^{n} \cos^2{\alpha_i} + \sin^2{\alpha_i} +                                                    \\
	+ \sum_{i < j} D_{X_i}D_{X_j} \cos{\alpha_i} \cos{\alpha_j} + \sum_{i < j} D_{Y_i} D_{Y_j} \sin{\alpha_i} \sin{\alpha_j}
\end{align*}

Jako, że znaną tożsamością trygonometryczną jest to, że \(\sin^2{\alpha} + \cos^2{\alpha} = 1\), to pierwszy składnik musi wynosić \(n\).

Pytamy się zatem o:

\[
	\expected{n + \sum_{i < j} D_{X_i}D_{X_j} \cos{\alpha_i} \cos{\alpha_j} + \sum_{i < j} D_{Y_i} D_{Y_j} \sin{\alpha_i} \sin{\alpha_j}}
\]

co możemy zapisać jako:

\[
	\expected{n} + \expected{\sum_{i < j} D_{X_i}D_{X_j} \cos{\alpha_i} \cos{\alpha_j} } + \expected{\sum_{i < j} D_{Y_i} D_{Y_j} \sin{\alpha_i} \sin{\alpha_j}}
\]

Korzystając z tego, że \(\alpha_i\) są losowane niezależnie od siebie, podobnie jak \(D_{X_i}\) możemy to zapisać jako\footnote{Jeśli \(X\) i \(Y\) są niezależne, to \(\expected{XY} = \expected{X}\expected{Y}\)}:

\[
	n + \sum_{i < j} \expected{D_{X_i}}\expected{D_{X_j}} \expected{ \cos{\alpha_i}} \expected{\cos{\alpha_j}} +
	{\sum_{i < j} \expected{D_{Y_i}} \expected{D_{Y_j}} \expected{\sin{\alpha_i}} \expected{\sin{\alpha_j}}}
\]

Zauważamy, że \( \expected{D_{X_i}} = \expected{D_{Y_i}} = \frac{1}{2} \cdot (-1) + \frac{1}{2} \cdot 1 = 0\)

Co upraszcza nasze wyrażenie do:

\[
	n + \sum_{i < j} 0 \cdot 0 \cdot \expected{ \cos{\alpha_i}} \expected{\cos{\alpha_j}} +
	{\sum_{i < j} 0 \cdot 0 \cdot  \expected{\sin{\alpha_i}} \expected{\sin{\alpha_j}}}
\]

czyli w oczekiwaniu jesteśmy oddaleni od punktu \((0, 0)\) o \(n\). Ale super.