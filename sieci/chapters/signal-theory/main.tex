\section{Twierdzenie Fouriera}
\begin{theorem}[Fourier]
	Rozsądne funkcje okresowe wyrażają się szeregiem funkcji trygonometrycznych.
	\[
		f(x) = c_0 + \sum_{i=1}^\infty a_i \sin(i \cdot x) + \sum_{i=1}^\infty b_i \cos(i \cdot x)
	\]

	\[
		c_0 = \frac{1}{2\pi} \int_{-\pi}^{\pi} f(x)dx
	\]
	\[
		a_i = \frac{1}{\pi} \int_{-\pi}^{\pi} f(x)\sin(i \cdot x) dx, \ b_i = \frac{1}{\pi} \int_{-\pi}^{\pi} f(x) \cos(i \cdot x) dx
	\]
\end{theorem}
\begin{proof}
	Trygonometria jest \textit{magiczna}.
\end{proof}

\section{Twierdzenie Nyquista}
\begin{theorem}[Nyquist]
	Jeżeli funkcja \(f\) nie ma składowych o częstotliwościach większych niż \(B\) Hz i próbkujemy ją z częstotliwością \(2B\) Hz, to możemy jednoznacznie odtworzyć \(f\).
\end{theorem}
W szczególności więcej próbek nie ma sensu. Dlatego większa częstotliwość (np. światłowód) daje lepszą przepustowość niż mniejsza (kabel miedziany).
\begin{corollary}
	Maksymalna przepustowość to \(2B\log\Sigma\), gdzie \(\Sigma\) to liczba bitów w każdej próbce.
\end{corollary}

\section{Twierdzenie Shannona}
\begin{theorem}[Shannon]
	Jeżeli \(\frac{S}{N}\) to stosunek mocy sygnału do mocy szumu, to maksymalna przepustowość to \(B \log(1+ \frac{S}{N})\), gdzie \(B\) jest najwyższą częstotliwością sygnału.
\end{theorem}

