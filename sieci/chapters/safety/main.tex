\section{Bezpieczeństwo i poufność}

\subsection{Rodzaje zagrożeń}
\begin{itemize}
	\item Podglądanie.
	\item Modyfikacja, usuwanie komunikatów.
	\item Blokowanie komunikacji.
\end{itemize}

\subsection{Hasze kryptograficzne}
\begin{itemize}
	\item \textbf{MD4} -- została złamana i można wygenerować kolizję w czasie rzędu sekund, przez to wyparta przez MD5, która jest jej następnikiem.
	\item \textbf{MD5} -- z ciągu danych o dowolnej długości generuje 128 bitowy hasz, znaleziono sposób na generowanie kolizji, jednak i tak jest użyteczna w niektórych zastosowaniach.
	\item \textbf{SHA1} -- tworzy 160 bitowy hasz z wiadomości o rozmiarze maksymalnym \(2^{64}\) bitów, ciężki do złamania, jednak powoli się z tym pracuje więc w nowych aplikacjach lepiej używać SHA2.
	\item \textbf{SHA2} -- następnik SHA1, składa się z zestawu czterech funkcji generujących odpowiednio 224, 256, 384 lub nawet 512 bitowe hasze, ma podobną implementacje co SHA1.
	\item \textbf{SHA3} -- wyłoniony w 2012 w konkursie następnik SHA2, działa na bazie algorytmu Keccak. Ma zupełnie inną budowę niż SHA2 dzięki czemu jest znacznie wydajniejszy zachowując zbliżone parametry bezpieczeństwa.
\end{itemize}

\subsection{Szyfrowanie symetryczne}
Algorytmy symetryczne do szyfrowania i deszyfrowania informacji używają tego samego klucza, lub takich dwu kluczy, z których mając jeden można jednoznacznie wyznaczyć drugi. Szyfrując wiadomość wynikowy szyfr jest równy na długość wiadomości. Dzielą się na \textbf{szyfry strumieniowe}, gdzie przetwarzamy informacje bit po bicie i \textbf{blokowe}, gdzie szyfrujemy na raz bloki danych i potem je sklejamy.

\begin{itemize}
	\item \textbf{XOR} -- xorujemy z kluczem. Często używany w połączeniu z innych szyfrem. Jest ekstra bo jest idealnie zbalansowany, każdy bit ma statystycznie równe szanse stać się 0 jak i 1. Jak ktoś zgadnie wiadomość, to odzyska z tego klucz. Ale jest to dobry pomysł, gdy wiadomości są nieprzewidywalne.
	\item \textbf{AES} -- Szyfr blokowy, o rozmiarze bloku 128 bitów, bardzo bezpieczny, zoptymalizowany pod szybkość działania i niskie zużycie pamięci, standard do szyfrowania tajnych informacji przez agencje wywiadowcze.
	\item \textbf{CBC} -- (z ang. Cipher Block Chaining) tryb pracy szyfrów blokowych, gdzie każdy blok jest przed zaszyfrowaniem jest xorowany z szyfrem poprzedniego bloku.
	\item \textbf{CTR} --  mając losowe bity \(v\) szyfrujemy je i xorujemy przez wiadomość, potem inkrementujemy \(v\). Jest to o tyle dobre, że da się to zrównoleglić.
	\item \textbf{Diffie-Hellman} -- jedna strona chce przesłać drugiej \(a\), druga pierwszej \(b\). Mamy ustalone liczby pierwsze \(g, p \). Teraz \(g^{a}, g^{b} \pmod p\) są nieodwracalne, obie strony mogą policzyć \(g^{ab}\), ale nikt nie zna obu \(a\) i \(b\) -- tak ustalamy klucz. To działa, o ile nikt nie zmienia przesyłanych komunikatów. Wtedy ktoś inny może zrobić to samo z liczbą \(c\) i przesłać \(g^{c}\) obu stronom -- wtedy one stworzą klucz korzystający z \(c\) i pośrednik będzie mógł zmieniać komunikację. Takiej sytuacji (Man in the Middle Attack) ciężko uniknąć. Dlatego ważne są certyfikaty SSL -- wiemy dokładnie z kim rozmawiamy.
	\item \textbf{Needham-Schroeder} -- istnieje centralne \(S\), z którym \(A\) i \(B\) mają już wspólny klucz. Wtedy \(A\) prosi \(S\) o nawiązanie komunikacji. \(S\) tworzy klucz \(K_{AB}\), wysyła go \(A\) i zakodowane bity \(\mathrm{enc}_{K_{BS}}\left( K_{AB}, A \right) \). Wtedy \(A\) może przesłać je \(B\) i \(B\) będzie wiedzieć, że ta wiadomość to poprawny klucz komunikacji z \(A\). Na koniec \(B\) wysyła \(A\) jakąś liczbę, a \(A\) odsyła ją zdekrementowaną -- dzięki temu potwierdzamy, że klucz jest taki sam po obu stronach.
	\item \textbf{Needham-Schroeder (Kerberos style)} -- poprzedni protokół ma taką wadę, że jeśli ktoś podsłucha \(\mathrm{enc}_{K_{BS}}\left( K_{AB},A \right) \), to może przekonać \(B\), żeby znów korzystać z tego klucza, co jest problemem, jeśli \(K_{AB}\) wycieknie. Dlatego dodatkowo przesyła się czas, klucz działa tylko przez chwilę. Wtedy trzeba pamiętać o synchronizacji czasu. Takie coś działa np. w Kerberosie.
\end{itemize}

\subsection{Szyfrowanie kluczem publicznym -- RSA}
RSA to najpopularniejszy asymetryczny algorytm kryptograficzny. Może być stosowany zarówno do szyfrowania jak i do podpisów cyfrowych. Jego bezpieczeństwo opiera się na trudności faktoryzacji dużych liczb. Na terenie USA opatentowany, więc można go tam używać tylko do celów niekomercyjnych.

\subsubsection{Generowanie kluczy}
\begin{enumerate}
	\item Wybieramy losowo dwie duże liczby \(p\) i \(q\).
	\item obliczamy \(n=p\cdot q\).
	\item obliczamy \(\lambda = \mathrm{NWW}(p-1, q-1)\).
	\item wybieramy liczbę \(e\) względnie pierwszą z \(\lambda\), z przedziału \((1,\lambda)\).
	\item znajdujemy liczbę \(d\), dla której \(d\cdot e \equiv 1 \pmod \lambda\).
\end{enumerate}

\subsubsection{Szyfrowanie i deszyfrowanie}
Dzielimy wiadomość na bloki a następnie szyfrujemy i deszyfrujemy każdy blok używając wzorów:
\begin{multicols}{2}
	\begin{center}
		\[c \equiv m^e \pmod{n}\]
		(szyfrowanie)
	\end{center}
	\begin{center}
		\[m \equiv c^d \pmod{n}\]
		(deszyfrowanie)
	\end{center}
\end{multicols}

\subsubsection{Bezpieczeństwo}
Im większe liczby wybierzemy tym trudniejszy jest do złamania. W 2020r. największy złamany klucz miał 829 bitów. Potencjalnym zagrożeniem dla RSA jest skonstruowanie stabilnego komputera kwantowego, gdyż w teorii mogą one z łatwością poradzić sobie z problemem faktoryzacji, jednak na razie ze względu na niestabilność największa zfaktoryzowana przez nie liczba ma zaledwie 72 bity.

\subsection{Podpis cyfrowy}
Cyfrowy podpis służy do stwierdzenia czy wiadomość pochodzi od właściwego nadawcy i nie została zmieniona podczas transmisji. Sprawdzamy czy szyfr wiadomości jest równy oczekiwanemu.

PGP (z ang. Pretty Good Privacy) jest jednym z najczęściej używanych programów do elektronicznego podpisywania i szyfrowania plików. Używa RSA oraz DSA.

\subsection{SSL}
Protokół, który umożliwia bezpieczną komunikację w Internecie w ramach HTTPS, chroni dane w warstwie transportowej poprzez ich zaszyfrowanie, pozwala na weryfikacje tożsamości i zapewnia integralność danych. Zapobiega atakom Man-in-the-Middle. Używa do tego podpisów cyfrowych i haszowania SHA2.

\subsubsection{Wystawianie certyfikatów}
Certyfikaty SSL strony uzyskują od urzędów certyfikacji na pewien okres, po tym jak sprawdzą one w jakiś sposób (np. przez to, że uruchomimy jakiś skrypt je pingujący z serwera, do którego jest podpięta domena) naszą tożsamość. Wystawiają także specjalne certyfikaty prywatnym instytucjom, które umożliwiają im podpisywanie kolejnych domen, tworzy się w ten sposób łańcuch certyfikatów, gdyż każdy certyfikat, aby móc potwierdzić jego autentyczność, musi także podać certyfikat wystawiającego.

\subsubsection{Inicjacja połączenia}
\begin{enumerate}
	\item W trakcie handshake'u serwer wysyła swój certyfikat SSL (albo ich łańcuch) i użytkownik decyduje czy im ufać czy nie. Wysyła także informacje o metodzie szyfrowania, czasem też prosi klienta o jego certyfikat.
	\item Klient generuje klucz sesji, wysyła go serwerowi szyfrując go kluczem publicznym serwera, dzięki temu tylko serwer z jego unikalnym kluczem prywatnym może odszyfrować klucz sesji.
	\item Wysyłane są komunikaty potwierdzające pomyślne wymianę kluczami od teraz całość interakcji jest szyfrowana tajnym kluczem sesji.
\end{enumerate}

\section{Bezpieczne Sieci}
\subsection{IPsec}
Zbiór protokołów służących do implementacji bezpiecznych połączeń oraz wymiany kluczy szyfrowania. Polega na szyfrowaniu całego ruchu IP. Może być wykorzystany do tworzenia VPNów.

Potrzebujemy systemu podobnego do certyfikatów, tylko dla adresów IP. Można ręcznie wgrywać certyfikaty w system. Można też stosować do tego DNS, ale on nie jest wiarygodny -- nie jest szyfrowany. Dlatego wprowadzamy pomysł podpisywania rekordów DNS, mamy root servery, więc naturalną hierarchię dziedziczenia zaufania.

\textbf{IKE} (z ang. Internet Key exchange) polega na:
\begin{itemize}
	\item Uwierzytelnieniu obu stron przez hasło, RSA lub certyfikaty
	\item nawiązaniu bezpiecznego kanału IKE
	\item uzgodnienie bezpiecznych kluczy kryptograficznych oraz kanału do komunikacji.
\end{itemize}
Główną zaletą jest fakt, że nie trzeba ręcznie ustawiać kluczy tylko ustalić wspólne hasło i samo się zrobi.

\subsection{VPN}
Możemy używać serwerów VPN -- proxy, do którego idzie nasz ruch. On jest szyfrowany, potem idzie dalej. Jesteśmy bezpieczniejsi, bo komunikujemy się bezpośrednio z jednym serwerem po zabezpieczonym połączeniu. VPN może działać na różnych poziomach -- zazwyczaj na TCP, ale możemy stworzyć połączenie na niższych poziomach (np. w Ethernecie, tworząc sztuczną kartę sieciową).
