Po uważnym przeczytaniu tego kodu zauważamy o co tak naprawdę jest pytanie.
Otóż chodzi o to, w jaki sposób programy wykonane przez \texttt{fork} oraz \texttt{exec} zachowują się względem obsługi sygnałów, którą ustawił rodzic.

Mamy tutaj trzy sposoby na obsługę sygnału:
\begin{enumerate}
	\item ustawienie handlera
	\item zablokowanie sygnału za pomocą \texttt{sigprocmask}
	\item ustawienie sygnału \texttt{SIGINT} jako ignorowanego
\end{enumerate}

Śmieszna rzecz polega na tym, że \texttt{SIGUSR1} jest tutaj ,,blokowane'' poprzez ustawienie jakiegoś handlera (czyli tak naprawdę nie jest to blokada sygnału); to z kolei powoduje, że jak poleci exec to ten handler jest brutalnie i ordynarnie \textbf{wywalany}.

Natomiast w przypadku sygnałów \texttt{SIGUSR2} i \texttt{SIGINT} ustawiamy nie handler, a maskę sygnałów dla systemu. Maska sygnałów nie jest \textbf{wywalana} po execu, więc posłanie dziecku takich sygnałów generalnie go nie zakończy.

Ponieważ jednak w kodzie najpierw dowalamy dziecku \texttt{SIGUSR1}, to spada ono z rowerka po wypisaniu jednej kropki.

Dla pełności podajemy jeszcze inne możliwe scenariusze (bo ten jest nudny):
\begin{enumerate}
	\item Najpierw strzelamy \texttt{SIGUSR2}, potem \texttt{SIGUSR1}, na końcu \texttt{SIGINT}.

	      W takiej sytuacji wypisane są dwie kropki, bo tak jak napisaliśmy, maska jest zachowana przy \texttt{exec}u.

	\item Najpierw strzelamy \texttt{SIGINT}, potem \texttt{SIGUSR2}, na końcu \texttt{SIGUSR1}

	      W tej sytuacji wypisane będą trzy kropki, bo \texttt{SIGINT} zostanie olany, \texttt{SIGUSR2} zablokowany, więc dopiero \texttt{SIGUSR1} ubije nam kota.
\end{enumerate}

TL;DR: Przy execu maski się dziedziczą, handlery nie. Ale heca.