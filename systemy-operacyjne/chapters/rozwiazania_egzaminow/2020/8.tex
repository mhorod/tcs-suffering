\epigraph{Nie jestem dumny z tego zadania}{dr hab. Jakub Kozik}

Generalnie to zadanie jest lekko zepsute.

\texttt{open} powoduje powiązania deskryptora z i-nodem, ale najpierw trzeba go znaleźć w systemie plików.
Potrzebujemy więc z grubsza tyle odczytów ile wynosi głębokość (liczona w katalogach) ścieżki pliku, który otwieramy.

\texttt{lseek} z tego co rozumiem to zdaje się nic nie rusza tylko ustawia jakieś dane w tablicy deskryptorów

\texttt{read} czyta sobie reprezentację pliku do znalezienia odpowiedniego bajtu w pliku (no i czyta te kilka sektorów z dysku)