Generalnie to dup po prostu da nam inny sposób ,,odnoszenia'' się do pliku? Kinda? Na przykładzie to działa w ten sposób, że jak sobie otworzę jakiś plik openem i będę z niego czytać, a potem zduplikuje deskryptor na niego za pomocą \textit{dup}, to po próbie odczytu dalej czytamy z miejsca w którym wcześniej byliśmy; z kolei, jak otworzymy ten sam plik openem (jeszcze raz) na innym deskryptorze to będziemy czytać go na tamtym deskryptorze od początku (a na wcześniejszym od tego miejsca gdzie byliśmy).

A formalniej rzecz biorąc, \textit{open} robi nam \textit{open file description} i ustawia nam wskaźnik do niego na jakimś \textit{file descriptor}; \textit{dup} z kolei po prostu ordynarnie sklonuje nam \textit{file descriptor} (ale \textit{ofd} na które wskazujemy pozostanie to samo). Patrz: pytanie piąte z 2022 roku