\section{2016/2017}
    \subsection{Zadanie 5}
    Jak niby pod spodem działa read?

    \subsection{Zadanie 6}
    Czy możemy robić selecta na pipe'ach do write'owania?

    \subsection{Zadanie 7}
    Czy da się powiedzieć coś więcej niż ,,leci przerwanie, budzi się scheduler i szereguje driver''?
    
    \subsection{Zadanie 8}
    Czy tyle co napisaliśmy w rozwiązaniu wystarcza? 

\section{2017/18}
    \subsection{Zadanie 5}
    jakie \texttt{CANCEL}, o co chodzi?
    
    \subsection{Zadanie 6}
    Jedyne co umiemy powiedzieć to \texttt{read}, \texttt{write}, \texttt{ioctl}, ale pewnie chodzi o jakieś ciekawsze mechanizmy ?
    
    \subsection{Zadanie 8}
    Chodzi o \texttt{argc} i \texttt{argv} czy o jakiś mechanizm tego jak to działa. Jeśli tak to jaki ?
    
    \subsection{Zadanie 10}
    Nie widzimy innego powodu niż ,,Pamięć jak pamięć, czemu miałaby być wyzerowana?''
    
\section{2018/19}
    \subsection{Zadanie 9}
        Obstawiamy, że (pesymistycznie) chodzi o różnicę o logarytmicznie wiele dostępów do dysku.
        
\section{2019/2020}
    \subsection{Zadanie 8}
        JAK DZIAŁAJĄ SYSTEMY PLIKÓW AAAAAAAAAAAAAA

\section{2020/2021}
    \subsection{Zadanie 2}
    Mogą być 2 kropki, 3 kropki, 4 kropki (ale te najbardziej prawdopodobne) lub 5 kropek, ale pytanie jest o output którego należy należy się spodziewać? Pewnie 4, ale fajnie byłoby się upewnić.
    
    \subsection{Zadanie 7}
    Czy aby tak na pewno działa \texttt{CLOCK\_TASK} i czy jest to wystarczający opis? 
    
