\section{Programowanie liniowe}
Zadanie programowania liniowego (ZPL) polega na maksymalizacji \( f(x) = c^Tx \), przy założeniach:
\begin{itemize}
\onehalfspacing
    \item \( Ax \leq b \)
    \item \( x \geq 0 \)
\end{itemize}
Obszar dopuszczalny (spełniający ograniczenia) jest hiperwielościanem wypukłym. Jeśli ZPL ma rozwiązanie, to musi być któryś z jego wierzchołków. \\
Zadanie podstawowe przekształcamy na dualne tak:
\begin{center}
    \( \max c^Tx \quad\quad\quad\quad\quad \min b^Ty \) \\
    \( Ax \leq b \quad\quad\rightarrow\quad\quad A^Ty \geq c \) \\
    \( x \geq 0 \quad\quad\quad\quad\quad\quad\quad y \geq 0 \)
\end{center}

W zadaniu dualnym:
\begin{itemize}
\onehalfspacing
    \item max przechodzi w min i odwrotnie
    \item równania zamieniają się na zmienne
    \item ograniczenia \( \leq \) w warunkach przechodzą na warunki \( y_i \geq 0 \) \\
    (nierówności \( \leq \) tylko z dodatnim współczynnikiem)
    \item ograniczenia \( \geq \) w warunkach przechodzą na warunki \( y_i \leq 0 \) \\
    (nierówności \( \geq \) z ujemnym współczynnikiem)
    \item ograniczenia równościowe przechodzą na zmienne nieograniczone \( y_i \in \mathbb{R} \) \\
    (równości z dowolnym współczynnikiem)
\end{itemize}
Jeśli jedno zadanie jest nieograniczone, to drugie musi być niespełnialne.
Jeśli zadanie podstawowe (prymalne) jest niespełnialne, to dualne jest nieograniczone lub niespełnialne, i odwrotnie.

\noindent
\textbf{Silne twierdzenie o dualności} \\
Rozwiązanie zadania dualnego jest równe rozwiązaniu zadania prymalnego.