\section{Rozwiązywanie układu równań}
Problem równoważny: zminimalizować \( \norm{x}^2 \) dla \( x \) spełniających \( A^TAx = A^Tb \). \\
Wstawiamy \( A = U\Sigma V^T \). Wówczas \( \Sigma V^Tx = U^Tb \). Oznaczając \( y = V^Tx \), trzeba zminimalizować \( \norm{y}^2 \). Rozwiązaniem jest \( \Sigma^+U^Tb \), gdzie \( \Sigma^+ \) ma na przekątnej odwrotności wartości z \( \Sigma \) lub zera. To prowadzi do rozwiązania \( x = V\Sigma^+U^Tb \). Macierz \( A^+ = V\Sigma^+U^Tb \) to pseudoodwrotność \( A \).
\begin{itemize}
	\onehalfspacing
	\item Jeśli \( A \) jest kwadratowa i nieosobliwa, to \( A^+ = A^{-1} \).
	\item Jeśli \( Ax = b \) jest nadokreślony, to \( A^+b \) jest przybliżonym rozwiązaniem w sensie najmniejszych kwadratów.
	\item Jeśli \( Ax = b \) jest niedookreślony, to \( A^+b \) jest najmniejszym w sensie normy rozwiązaniem
\end{itemize}