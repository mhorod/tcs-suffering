\section{Metoda punktu stałego}
Celem jest znaleźć punkt stały funkcji \( f \).

\begin{theorem} [Banacha o punkcie stałym]
    Niech \( f: V \rightarrow V \) będzie odwzorowaniem w przestrzeni Banacha. Jeśli \( \norm{f(x) - f(y)} \leq \lambda \cdot \norm{x - y} \) dla pewnego \( 0 \leq \lambda < 1 \) (odzworowanie jest zwężające), to ma ono dokładnie jeden punkt stały i poniższa metoda zawsze do niego zbiega: \\
    \[
        x_n = f(x_{n-1})
    \]
\end{theorem}

Czyli w \( \mathbb{R} \) działa dla funkcji \( f \), t.że \( f' < 1 \) na pewnym przedziale domkniętym - zbiega kwadratowo. \\