\section{Rozkład QR}
Jeśli zbiór \( \{v_1, \dots, v_n\} \) stanowi bazę przestrzeni liniowej, to każdy wektor \( x \) wyraża się w tej bazie łatwo:
\[
    x = \sum_{i} \langle x, v_i \rangle v_i
\]
Mając ortonormalną macierz \( Q \), rozwiązaniem \( Qx = b \) jest \( x = Q^T b \). Pozostaje tylko rozłożyć \( A = QR \), gdzie \( Q \) jest ortogonalna, a R górnotrójkątna.