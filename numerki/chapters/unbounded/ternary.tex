\section{Ekstrema funkcji bitonicznych}

\subsection{Wyszukiwanie ternarne}
Algorytm do szukania ekstremów funkcji bitonicznych, czyli ciągłych z jednym ekstremum:

\noindent
(wersja dla maksimum)
\begin{lstlisting}[language=Cpp, morekeywords={ternary}]
function ternary(p, q)
    r = p + (q-p)/3
    s = q - (q-p)/3
    if f(r) < f(s)
        return ternary(r, q)
    else
        return ternary(p, s)
\end{lstlisting}
Przedział zmniejsza się do \( \frac{2}{3} \) za pomocą dwóch wywołań f, czyli \( \sqrt{\frac{2}{3}} \) razy na jedno obliczenie f.

\subsection{Wyszukiwanie ze złotym podziałem}
Czyli ulepszona wersja poprzedniego.
\begin{lstlisting}[language=Cpp, morekeywords={golden}]
function golden(p, q)
    r = q - (q-p)*phi
    s = p + (q-p)*phi
    if f(r) < f(s)
        return golden(r, q)
    else
        return golden(p, s)
\end{lstlisting}
Wykorzystujemy obliczone wcześniej wartości f, więc przedział zmniejsza się o \( \Phi \approx 0.618 \) \linebreak w jednym wywołaniu.