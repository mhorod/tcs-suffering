\section{Kompresja fal}
\subsection{Kompresja dźwięku}
Pomysł: Rozbić funkcję (sygnał) na małe przedziały i zamienić za pomocą DFT na częstotliwości. Zapamiętuje się tylko najmocniejsze, po czym kompresuje wybranym standardowym, bezstratnym algorytmem (np. Huffman).
\begin{warning}
	Pojawiają się problemy na granicach przedziałów.
\end{warning}
Aby ich uniknąć, zamiast transformaty Fouriera można użyć podobnej transformaty cosinusowej, na częściowo nakładających się przedziałach. Jest obliczalna w takim samym czasie. Najbardziej znany wariant to Zmodyfikowana Transformata Cosinusowa (MDCT), używana m. in. w kompresji MP3.
\[
	(x_0, \dots, x_{2n-1}) \rightarrow (X_0, \dots, X_{n-1})
\]
\[
	X_k = \sum x_j \cos\left(\frac{\pi}{n}\left(j + \frac{n + 1}{2}\right)\cdot\left(\frac{k+1}{2}\right)\right)
\]
\subsection{Kompresja obrazu}
Kompresja JPEG działa na podobnej zasadzie. Dzielimy obrazek na bloki i zamieniamy transformatą na częstotliwości. Potem zmniejszamy dokładność zapisu częstotliwości i na koniec kompresujemy bezstratnie (np. Huffman).