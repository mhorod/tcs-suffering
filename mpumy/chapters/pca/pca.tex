\section{Analiza składowych głównych (PCA)}

Mamy dane w jakiejś przestrzeni \( \real^d \). Szukamy hiperpłaszczyzny takiej, że jak zrzutujemy na nią punkty to otrzymamy wysoką wariancję.
Równoważnie chcemy aby suma kwadratów odległości do tejże hiperpłaszczyzny była jak najmniejsza.

Szukamy zatem rzutowania \( P : R^d \rightarrow R^l \), takiego że \( P^2 = P \)

Ponieważ będziemy szukać odwzorowań liniowych to zakładamy, że nasze dane są wyśrodkowane tj. \( \sum_i x^{(i)} = 0 \)

\subsection{Przestrzeń jednowymiarowa}

W przypadku jednowymiarowym (\( l = 1 \)) nasze rzutowanie jest postaci \( \pi(x) = u^T x \) gdzie \( u \in \real^d \)

Chcemy zmaksymalizować
\[
	\Var\pars{u^T X}
\]
przy czym zakładamy, że \( \norm{u} = 1 \).

Mamy ponadto
\[
	\Var\pars{u^T X} = u^T \Cov(X) u = \frac{1}{m} u^TX^TXu
\]
Dla skrócenia zapisu przyjmujemy \( C = X^TX\) i mówimy, że szukamy
\[
	\argmax_{u \in \real^d, \norm{u} = 1} u^T C u
\]

Aby pozbyć się warunku \( \norm{u} = 1 \) przerabiamy ten problem na znane i lubiane mnożniki Lagrange'a

\[
	L(u, \lambda) = u^T C u - \lambda(u^Tu - 1)
\]

dostajemy
\[
	\frac{\partial L}{\partial u} = 2Cu - 2 \lambda u
\]
Z czego mamy
\[
	Cu = \lambda u
\]
W takim razie \( u \) jest wektorem własnym, a ponadto:
\[
	u^T C u = u^T \lambda u = \lambda
\]
Szukamy zatem największej wartości własnej

\subsection{Przestrzeń wielowymiarowa}

Jeśli \( l > 1 \) to po prostu szukamy przestrzeni rozpiętej na \( l \) wektorach i analogicznie jak w przypadku \( l = 1 \) znajdujemy \( l \) największych wartości własnych macierzy \( X^TX \)


\subsection{Zastosowanie SVD}
Dzięki SVD mamy następujący algorytm na rzutowanie:
\begin{enumerate}
	\item Rozkładamy \( X = U\Sigma V^T \)
	\item Wybieramy \( l \) największych wartości i wektorów własnych dostając macierz
	      \[
		      V_r = \begin{bmatrix}
			      v_1 \dots v_l
		      \end{bmatrix}
	      \]
	\item Definiujemy rzutowanie
	      \[
		      \pi(x) = V_r^T x
	      \]
\end{enumerate}

\subsection{Wybór liczby składowych}

Pozostaje pytanie -- jakie \( l \) wybieramy ?
Ano takie, dla którego suma wybranych składowych jest wystarczająco duża np.
\[
	\frac{\sum_{i=1}^l \lambda_i}{\sum_{i=1}^d \lambda_i} \geq 0.95
\]