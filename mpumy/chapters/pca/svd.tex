\section{Rozkład według wartości osobliwych (SVD)}

Niech \( A \) będzie dowolną macierzą o wymiarach \( m \times n \) i niech \( m \geq n \).

Szukamy:
\begin{itemize}
	\item wektorów ortonormalnych
	      \[
		      v_1, \dots, v_n \in \real^n
	      \]

	\item wektorów ortonormalnych
	      \[
		      u_1, \dots, u_n \in \real^m
	      \]

	\item wartości osobliwych
	      \[
		      \sigma_1, \dots, \sigma_n \in \real_{\geq 0}
	      \]
\end{itemize}
dla których
\[
	Av_j = \sigma_j u_j
\]

W postaci macierzowej mamy
\[
	A \begin{bmatrix}
		v_1 \dots v_n
	\end{bmatrix}
	=\begin{bmatrix}
		u_1 \dots u_n
	\end{bmatrix}
	\cdot
	\begin{bmatrix}
		\sigma_1 & \hdots & 0        \\
		\vdots   & \ddots & \vdots   \\
		0        & \hdots & \sigma_n
	\end{bmatrix}
\]
co zapisujemy dla uproszczenia:
\[
	AV = \widehat U \widehat \Sigma
\]

Ponieważ \( V \) składa się z wektorów ortonormalnych to mamy
\[ VV^T = V^TV = I \]
zatem
\[ A = \widehat U \widehat \Sigma V^T\]

Możemy rozszerzyć \( \widehat U \), które ma wymiar \( m \times n \) do macierzy unitarnej \( U \), która już jest kwadratowa i ma wymiary \( m \times m \).

Możemy też uzupełnić \( \widehat \Sigma \) zerami do macierzy o wymiarach \( m \times n \)

Rozkład \( A = U\Sigma V^T \) nazywamy \textbf{rozkładem na wartości osobliwe}

Zachodzi twierdzenie
\begin{theorem}
	Każda macierz ma rozkład na wartości osobliwe, przyczym \( \sigma_1, \dots, \sigma_n \) są jednoznaczne, a ponadto jeśli są parami różne to \( u_1, \dots, u_n \) oraz \(v_1, \dots, v_n\) są jednoznaczne z dokładnością do znaku.
\end{theorem}

\subsection{Wyznaczanie rozkładu}
\[
	A^TA = (U\Sigma V^T)^T(U\Sigma V^T) = V\Sigma^T\Sigma V^T
\]
mamy zatem
\[
	A^TAV = V\Sigma^T\Sigma
\]
W szczególności mamy
\[
	A^TAv_i = \sigma^2_i v_i
\]
czyli kolumny macierzy \( V \) są wektorami własnymi \( A^TA\).

W podobny sposób kolumny \( U \) są wektorami własnymi \( AA^T \)
