\section{Algorytm maksymalizacji wartości oczekiwanej}

\subsection{Dywergencja Kullbacka-Leiblera}
Chcielibyśmy mieć jakiś sposób na obliczanie podobieństwa dwóch rozkładów, będzie nam to za chwilę przydatne w konstrukcji algorytmu.

\begin{definition}
	\textbf{Dywergencja Kullbacka-Leiblera} dla rozkładów \( q, p \) zadana jest wzorem dla rozkładów ciągłych
	\[
		\KL{q}{p} = \int q(x) \ln \frac{q(x)}{p(x)} \diff x
	\]
	oraz
	\[
		\KL{q}{p} = \sum_x q(x) \ln \frac{q(x)}{p(x)}
	\]
\end{definition}

Podstawowe jej własności:
\begin{itemize}
	\item \( \KL{p}{p} = 0 \)
	\item \( \KL{p}{q} \geq 0 \)
	\item \( \KL{p}{q} \) nie musi być równe \( \KL{q}{p} \)
\end{itemize}

\subsection{Sformułowanie problemu}

Podobnie jak w GMM zakładamy, że dane pochodzą z mieszanego modelu o \( K \) rozkładach o parametrach \( \theta \).
Innymi słowy
\[
	p(x^{(i)} \given \theta) = \sum_{c=1}^K p(z^{(i)} = c \given \theta) \cdot p(x^{(i)} \given z^{(i)} = c, \theta)
\]

Będziemy starali się znaleźć taki parametr modelu \( \theta \) który maksymalizuje log wiarygodność.
\[
	\theta^* = \argmax_\theta \sum_{i=1}^m \ln \sum_{c=1}^K p(x^{(i)}, z^{(i)} = c \given \theta)
\]

Mamy tutaj ten sam problem co wcześniej -- nie wejdziemy z logarytmem pod sumę, więc musimy kombinować inaczej.

\subsection{Ograniczenie dolne}

Niech \( q \) będzie dowolnym rozkładem na \( K \) wartościach. Możemy szacować log-wiarygodność od dołu za pomocą funkcji \( \mathcal{L}(\theta, q) \)

\begin{align*}
	\ln \ell(\theta)
	 & = \sum_{i=1}^m \ln \sum_{c=1}^K p(x^{(i)}, z^{(i)} = c \given \theta)                                        \\
	 & = \sum_{i=1}^m \ln \sum_{c=1}^K q(z^{(i)} = c) \frac{p(x^{(i)}, z^{(i)} = c \given \theta)}{q(z^{(i)}=c)}    \\
	 & \geq \sum_{i=1}^m \sum_{c=1}^K q(z^{(i)} = c) \ln \frac{p(x^{(i)}, z^{(i)} = c \given \theta)}{q(z^{(i)}=c)} \\
	 & = \mathcal{L}(\theta, q)
\end{align*}

Nierówność mamy dzięki nierówności Jensena, wklęsłości logarytmu oraz faktu, że \( q \) jest rozkładem prawdopodobieństwa tj. \( \sum_{c=1}^K q(z = c) = 1 \)

Skoro \( \mathcal{L}(\theta, q) \) jest ograniczeniem dolnym to będziemy chcieli znaleźć największe takie, a więc szukamy \( q, \theta \) maksymalizujące tę wartość.

Bardziej formalnie -- jeśli po \( k \) krokach mamy parametr \( \theta_k \) to obliczamy (krok \textbf{E})
\[
	q_{k + 1} = \argmax_q \mathcal{L}(\theta_k, q)
\]
a następnie (krok \textbf{M})
\[
	\theta_{k + 1} = \argmax_\theta \mathcal{L}(\theta, q_{k + 1})
\]

\subsection{Krok E}

Szukamy największego ograniczenia dolnego czyli w praktyce minimalizujemy różnicę
\begin{align*}
	\ln \ell(\theta) - \mathcal{L}(\theta, q)
	 & = \sum_{i=1}^m \ln p(x^{(i)}, \theta) - \sum_{i=1}^m \sum_{c=1}^K q(z^{(i)} = c) \ln \frac{p(x^{(i)}, z^{(i)} = c \given \theta)}{q(z^{(i)}=c)}                             \\
	 & = \sum_{i=1}^m \sum_{c=1}^K q(z^{(i)} = c) \ln p(x^{(i)}, \theta) - \sum_{i=1}^m \sum_{c=1}^K q(z^{(i)} = c) \ln \frac{p(x^{(i)}, z^{(i)} = c \given \theta)}{q(z^{(i)}=c)} \\
	 & = \sum_{i=1}^m \sum_{c=1}^K  q(z^{(i)} = c) \pars{\ln p(x^{(i)}, \theta) - \ln \frac{p(x^{(i)}, z^{(i)} = c \given \theta)}{q(z^{(i)}=c)}}                                  \\
	 & = \sum_{i=1}^m \sum_{c=1}^K q(z^{(i)} = c) \ln \frac{p(x^{(i)}, \theta) \cdot q(z^{(i)}=c)}{p(x^{(i)}, z^{(i)} = c \given \theta)}                                          \\
	 & = \sum_{i=1}^m \sum_{c=1}^K q(z^{(i)} = c) \ln \frac{\cdot q(z^{(i)}=c)}{p(x^{(i)}, z^{(i)} = c \given x^{(i)}, \theta)}                                                    \\
	 & = \KL{q(z^{(i)})}{p(z^{(i)} \given x^{(i)}, \theta)}
\end{align*}

A wiemy że jest ona równa zero tylko gdy \( q(z^{(i)}) = p(z^{(i)} \given x^{(i)}, \theta) \).

\subsection{Krok M}

Teraz chcemy znaleźć \( \theta \) maksymalizujące \( \mathcal{L}(\theta, q) \).

Mamy
\[
	\mathcal{L}(\theta, q) = \sum_{i=1}^m \sum_{c=1}^K q(z^{(i)} = c) \ln \frac{p(x^{(i)}, z^{(i)} = c \given \theta)}{q(z^{(i)}=c)}
\]
Zauważamy jednak, że \( q(z^{(i)}=c) \) w mianowniku jest stałe, zatem maksymalizujemy
\[
	\sum_{i=1}^m \sum_{c=1}^K q(z^{(i)} = c) \ln p(x^{(i)}, z^{(i)} = c \given \theta) = \expected{\ln p(X, Z \given \theta)}
\]
gdzie wartość oczekiwana przechodzi po rozkładzie wyznaczonym przez \( q \) tak jak miało to miejsce w przypadku GMM.

\subsection{Zbieżność}

W każdym kroku tylko maksymalizujemy, więc log-wiarygodność nie maleje
\[
	\ln p(X \given \theta_k) = \mathcal{L}(\theta_k, q_{k + 1}) \leq \mathcal{L}(\theta_{k+1}, q_{k+1}) \leq \ln p(X \given \theta_{k + 1})
\]