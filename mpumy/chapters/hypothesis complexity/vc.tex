\section{Wymiar Wapnika-Czerwonienkisa}

\begin{definition}
	Dla przestrzeni hipotez \( H \) definiujemy \textbf{funkcję wzrostu} \( \Pi_H : \natural \rightarrow \natural \) jako:
	\[
		\Pi_H(m) = \max_{x^{(1)}, \dots, x^{(m)} \in X} \card{\set{(h(x^{(1)}), \dots, h(x^{(m)}) ) : h \in H}}
	\]
\end{definition}
Innymi słowy -- dla każdego \( m \)-elementowego podzbioru \( X \) zliczamy liczbę etykietowań, które umiemy uzyskać i wybieramy największą wartość.

\begin{definition}
	Mówimy, że \( H \) rozbija \(m\)-elementowy \( S \subseteq X \) jeśli \( \Pi_H(m) = 2^m \) tj. realizuje wszystkie możliwe jego etykietowania.
\end{definition}
\begin{definition}
	\textbf{Wymiar Wapnika-Czerwonienkisa} przestrzeni hipotez \( H \) to moc największego zbioru, który jest rozbity przez \( H \)
	\[
		VC(H) = \max \set{m : \Pi_{H}(m) = 2^m}
	\]
\end{definition}

\begin{lemma}[Sauer]
	Niech \( VC(H) = d \). Wtedy dla dowolnego \( m \in \natural \) zachodzi
	\[
		\Pi_H(m) \leq \sum_{i=0}^d \binom{m}{i}
	\]
\end{lemma}
\begin{proof}

\end{proof}