\section{Twierdzenie o reprezentacji}
\subsection{Pomocnicze lematy}
\begin{lemma}
    Norma \( \norm{\cdot} : \real^d \rightarrow \real \) pochodzi od iloczynu skalarnego tj. \( \norm{x}^2 = \dotproduct{x, x} \) wtedy i tylko wtedy gdy 
    \[
        \forall_{x, y \in \real^d} : 2\norm{x}^2  + 2\norm{y}^2 = \norm{x + y}^2 + \norm{x - y}^2
    \]
\end{lemma}

\begin{proof} \( \)
    \begin{description}
        \item \( \implies \)
        
        \[
            \norm{x + y}^2 + \norm{x - y}^2 
                = \dotproduct{x + y, x + y}
                    + \dotproduct{x - y, x - y}
        \]
        
        \item \( \impliedby \)
        
        Definiujemy \[ 
            \dotproduct{x, y} = \frac{1}{4} \pars{\norm{x + y}^2 - \norm{x - y}^2}
        \]
        
        \begin{enumerate}
            \item \( \dotproduct{x} = \norm{x}^2 \)
            \item \( \dotproduct{x, y} = \dotproduct{y, x} \)
            \item \( \dotproduct{x + y, z} = \dotproduct{x, z} + \dotproduct{y, z} \)
            \item \( \dotproduct{ax, y} = a\dotproduct{x, y} \)
        \end{enumerate}
        
    \end{description}
\end{proof}

\begin{lemma}
    Niech \( M \) będzie domkniętą podprzestrzenią \( \real^d \).
    Wtedy dla dowolnego \( x \in \real^d \) istnieje dokładnie jedno \( P_M(x) = m_0 \in M \) takie że
    \[
        \forall_{m \in M} \norm{x - m_0} \leq \norm{x - m}
    \]
    gdzie \( P_M(x) \) jest rzutem ortogonalnym tj.
    \begin{itemize}
        \item \( P_M(ax + by) = aP_M(x) + bP_M(y) \)
        \item \( P_M = P_M \circ P_M \)
        \item \( \dotproduct{P_M(x), y} = \dotproduct{x, P_M(y)}\)
        \item \( \norm{P_M(x)} \leq \norm{x} \)
    \end{itemize}
\end{lemma}
\begin{proof} \( \)

    Bierzemy 
    \[
        d = \inf_{m \in M} \norm{x - m}
    \]
    
    Istnieje ciąg
    \[
        \pars{z_n}_{n \in \natural} \subseteq M : \norm{x - z_n}^2 \leq d^2 + \frac{1}{n}
    \]
    
    Dla \( n_1, n_2 \in \natural \), \( \frac{z_{n_1} + z_{n_2}}{2} \in M \)
    \begin{align*}
        \norm{z_{n_1} - z_{n_2}}^2
            &= \norm{z_{n_1} - x + x - z_{n_2}}^2 \\
            &= 2\norm{z_{n_1} - x}^2 + 2\norm{z_{n_2} - x}^2 
                - \norm{z_{n_1} - x - x + z_{n_2}}^2
            \\
            &\leq 4d^2 + \frac{2}{n_1} + \frac{2}{n_2} - 4\norm{\dots}
    \end{align*}
    
    W takim razie \( \lim_{n \rightarrow \infty} z_n = m_0 \)
    Czyli \( d \) jest realizowane przez \( m_0 \).
    
    Jeszcze trzeba pokazać jedyność:
    Niech istnieje \( m_1 \) takie że
    \[
        \forall_{m \in M} \norm{x - m_0} = \norm{x - m_1} \leq \norm{x - m}
    \]
    Policzmy
    \begin{align*}
        \norm{m_0 - m_1}
            &= \norm{m_0 - x + x - m_1}^2 \\
            &= 2\norm{m_0 - x}^2 + 2\norm{m_1 - x} 
                - \norm{m_0 - x + m_1 - x} \\
            &\leq 2d^2 + 2d^2 - 4d^2 = 0
    \end{align*}
    Czyli \( m_0 = m_1 \)

\end{proof}

\subsection{Właściwe twierdzenie}
\begin{theorem}[O reprezentacji]
Niech \( x^{(1)}, \dots, x^{(m)} \in X \), \( \psi : X \rightarrow \real^k \)
\( L : \real^k \rightarrow \real \) oraz niech \( R : \real_{\geq 0} \rightarrow 0 \) będzie niemalejąca.
Definiujemy funkcję straty
\[
    J(w) = R(\norm{w}) + L(\dotproduct{w, \psi(x^{(1)})}, \dots \dotproduct{w, \psi(x^{(1)})})
\]

Jeśli \( J(w) \) osiąga minimum to istnieje 
\[
    \argmin J(w) = w^* = \sum_{i=1}^m \alpha_i \psi(x^{(i)})
\]

A ponadto, jeśli \( R \) jest silnie rosnąca to wszystkie argumenty są tej postaci.

\end{theorem}
\begin{proof}
    Niech \( M = \text{span} \set{\psi\pars{x^{(1)}}, \dots \psi\pars{x^{(m)}}} \)
    
    Niech \( w \) minimalizuje \( J(w) \).
    
    \( w^* = P_M(w) \)
    
    \( w^\perp = w - w^* \)
    
    Dla dowolnego \( z \in M \) mamy
    \begin{align*}
        \dotproduct{w^\perp, z}
        &= \dotproduct{w - w^*, z} \\
        &= \dotproduct{w - w^*, P_M(z)} \\
        &= \dotproduct{w, P_M(z)} - \dotproduct{w^*, P_M(z)} \\
        &= \dotproduct{P_M(w), z} - \dotproduct{P_M(w^*), z} \\
        &= 0
    \end{align*}
    W takim razie 
    \[
        \dotproduct{w^*, \psi\pars{x^{(i)}}}
        = 
        \dotproduct{w, \psi\pars{x^{i}}}
    \]
    
    Zatem wartość funkcji \( L \) się nie zmienia.
    Mamy ponadto \( \norm{w^*} \leq \norm{w} \) a \( R \) jest niemalejąca, zatem \( J \) nie może być większe.
\end{proof}