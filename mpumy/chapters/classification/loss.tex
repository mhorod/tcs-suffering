\section{Funkcje straty w klasyfikacji}
Przyjmijmy że mamy dwie klasy tj. \( Y = \set{-1, 1} \)

Moglibyśmy od razu jako odpowiedź dawać \( \widehat y \in \set{-1, 1}\)
wtedy błąd zadany jest takim wzorem:
\[
    \ell(f(x), y) = \indicator{f(x) \neq q}
\]

Możemy jednak być nieco sprytniejsi i przewidywać \( \widehat y \in \real \) (im większe na moduł tym silniejsze przekonanie) i dopiero na podstawie znaku odpowiadać 1 lub -1.

Wtedy szczególności ryzyko empiryczne wyraża się:
\[
    \widehat R_m(h) = \frac{1}{m} \sum_{i=1}^m \indicator{y^{(i)} \neq h(x^{(i)} \leq 0}
\]
Co jest trochę słabe, bo nie jest to ani ciągłe ani różniczkowalne -- słabo się z nich uczy.

\begin{definition}
    \textbf{Marginesem} dla przewidywania \( \widehat y \in \real \) i prawdziwej odpowiedzi \( y \) nazywamy \( \widehat y \cdot y \)
\end{definition}

Na podstawie marginesu \( m \) możemy zdefiniować różne funkcje straty:
\begin{itemize}
    \item strata zawiasowa
    \[
        \max(0, 1 - m)
    \]
    \item strata Savage'a
    \[
        \frac{1}{(1 + e^m)^2}
    \]
    \item strata logistyczna
    \[
        \ln(1 + e^{-m})
    \]
    \item strata wykładnicza
    \[
        e^{-m}
    \]
    \item strata kwadratowa
    \[
        (1 - m)^2
    \]
    
\end{itemize}