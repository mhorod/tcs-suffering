\section{Rodzaje klasyfikacji}

\subsection{Klasyfikacja generatywna}

W klasyfikacji generatywnej staramy się dla każdej klasy określić jej rozkład. Kiedy dostajemy nowe wejście sprawdzamy z jakim prawdopodobieństwem należy do której z klas i na tej podstawie przydzielamy mu etykietę.

Co ciekawe takie podejście da się stosować w uczeniu bez nadzoru.

Przykładowe klasyfikatory:
\begin{itemize}
	\item Naiwny Bayes
	\item Sieci Bayesowskie
	\item Pola Markowa
	\item GANy
	\item gaussowska analiza dyskryminacyjna
\end{itemize}

\subsection{Klasyfikacja dyskryminatywna}

W klasyfikacji dyskryminatywnej z kolei staramy się rozgraniczyć dane od siebie -- nie interesuje nas dokładnie w jaki sposób powstają dane.
Decyzję dla nowego wejścia podejmujemy sprawdzając po której stronie granicy się znajduje.

Na ogół to podejście sprawdza się w uczeniu nadzorowanym -- skądś musimy mieć dane treningowe na podstawie których tworzymy granice, a danych dobrze jak jest sensownie dużo.

Przykładowe klasyfikatory:
\begin{itemize}
	\item Regresja logistyczna
	\item SVM
	\item Sieci neuronowe
	\item KNN
	\item drzewa decyzyjne
\end{itemize}