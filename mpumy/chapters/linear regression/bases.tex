\section{Funkcje bazowe}

Możemy obłożyć wejścia jakąś funkcją \( \phi \) -- w ten sposób przeniesiemy je do innej przestrzeni w której być może prościej się robi regresję.

Dla modelu
\[
	h_\theta(x) = \theta^T\phi(x) + \varepsilon
\]

gdzie \( \varepsilon \) jest szumem gaussowskim

tworzymy macierz planowania:
\[
	\Phi = \begin{bmatrix}
		\phi_1(x^{(1)}_1) & \hdots & \phi_k(x^{(1)}_k) \\
		\vdots            & \ddots & \vdots            \\
		\phi_1(x^{(m)}_1) & \hdots & \phi_k(x^{(m)}_k) \\
	\end{bmatrix}
\]

i dalej liczymy tak jak normalnie tj. dostajemy
\[
	\widehat \theta = (\Phi^T\Phi)^{-1}\Phi^T y
\]