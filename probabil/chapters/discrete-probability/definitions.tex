\subsection{Wartość oczekiwana}
\begin{definition}
	\textbf{Wartość oczekiwaną} zmiennej losowej \( X \) definiujemy jako
	\[
		\expected{X} = \sum_{x \in \im X} x \cdot P(X = x)
	\]
\end{definition}
\begin{theorem}[Lemat 2.9 P\&C]
	\label{expected-value-of-natural-random-variable}
	Niech \( X \) będzie zmienną losową przyjmującą jedynie wartości w liczbach naturalnych. Wtedy
	\[
		\expected{X} = \sum_{n=1}^\infty P(X \geq n)
	\]
\end{theorem}
\begin{proof}
	\begin{align*}
		\sum_{n=1}^\infty P(X \geq n)
		 & = \sum_{n=1}^\infty \sum_{k=n}^\infty P(X = k) \\
		 & = \sum_{k=1}^\infty \sum_{n=1}^k P(X = k)      \\
		 & = \sum_{k=1}^\infty k \cdot P(X = k)           \\
		 & = \expected{X}
	\end{align*}
\end{proof}

\begin{definition}
	\textbf{Warunkową wartość oczekiwaną} \( \expected{X \mid Y = y} \) definiujemy jako
	\[
		\expected{X \mid Y = y} = \sum_{x \in \im X} P\pars{X = x \mid Y = y}
	\]

	Ponadto \( \expected{X \mid Y} \) definiujemy jako zmienną losową taką, że
	\[
		\expected{X \mid Y}(y) = \expected{X \mid Y = y}
	\]
\end{definition}

\begin{lemma}
	\[
		\expected{X} = \sum_{y \in \im Y} \expected{X \mid Y = y} \cdot P(Y = y)
	\]
\end{lemma}
\begin{proof}
	\begin{align*}
		\sum_{y \in \im Y} \expected{X \mid Y = y} \cdot P(Y = y)
		 & = \sum_{y \in \im Y} \pars{P(Y = y) \sum_{x \in \im X} x \cdot P(X = x \mid Y = y)}       \\
		 & = \sum_{x \in \im X} \pars{x \cdot \sum_{y \in \im Y}P(Y = y)  \cdot P(X = x \mid Y = y)} \\
		 & = \sum_{x \in \im X} \pars{x \cdot \sum_{y \in \im Y}P(Y = y)  \cdot
		\frac{P(X = x \land Y = y)}{P(Y = y)}}                                                       \\
		 & = \sum_{x \in \im X} x \cdot P(X = x)                                                     \\
		 & = \expected{X}
	\end{align*}
\end{proof}

\begin{lemma}[Lemat Syntaktyczny]
	\[
		\expected{\expected{X \mid Y}} = \expected{X}
	\]
\end{lemma}
\begin{proof}
	\begin{equation*}
		\expected{\expected{X \mid Y}}
		= \sum_{y \in \im Y} \expected{X \mid Y = y} \cdot P(Y = y) = \expected{X}
	\end{equation*}
\end{proof}

\subsection{Wariancja}
\begin{definition}
	\textbf{Wariancję} zmiennej losowej \( X \) definiujemy jako
	\[
		\variance{X} = \expected{\pars{X - \expected{X}^2}} = \expected{X^2} - \expected{X}^2
	\]
\end{definition}
\begin{definition}
	\textbf{Kowariancję} zmiennych losowych \( X \) oraz \( Y \) definiujemy jako
	\[
		\covariance{X}{Y} = \expected{\pars{X - \expected{X}} \cdot \pars{Y - \expected{Y}}}
	\]
\end{definition}

\begin{theorem}
	\[
		\forall_{a, b \in \real} \variance{bX + a} = b^2\variance{X}
	\]
\end{theorem}
\begin{proof}
	\begin{align*}
		\variance{bX + a}
		 & = \expected{(bX + a)^2} - \expected{bX + a}^2                                           \\
		 & = \expected{b^2X^2 + 2abX + a^2} - \pars{b\expected{X} + a}^2                           \\
		 & = b^2\expected{X^2} + 2ab\expected{X} + a^2 - b^2\expected{X}^2 - 2ab\expected{X} - a^2 \\
		 & = b^2\pars{\expected{X^2} - \expected{X}^2}                                             \\
		 & = b^2\variance{X}
	\end{align*}
\end{proof}

\begin{theorem}
	Dla dowolnych zmiennych losowych \( X, Y \) zachodzi
	\[
		\variance{X + Y} = \variance{X} + \variance{Y} + 2\covariance{X}{Y}
	\]
\end{theorem}
\begin{proof}
	Rozpisujemy \( \variance{X + Y} \) z definicji.
	\begin{align*}
		\variance{X + Y}
		 & = \expected{\pars{X + Y - \expected{X + Y}}^2}                                \\
		 & = \expected{\pars{\pars{X - \expected{X}} + \pars{Y - \expected{Y}}}^2}       \\
		 & = \expected{\pars{X - \expected{X}}^2} + \expected{\pars{Y - \expected{Y}}^2}
		+ 2\expected{\pars{X - \expected{X}}\cdot\pars{Y - \expected{Y}}}                \\
		 & = \variance{X} + \variance{Y} + 2\covariance{X}{Y}
	\end{align*}
\end{proof}

\begin{theorem} Dla niezależnych zmiennych losowych \( X, Y \)
	\[
		\covariance{X}{Y} = 0
	\]
	a co za tym idzie
	\[
		\variance{X + Y} = \variance{X} + \variance{Y}
	\]
\end{theorem}
\begin{proof}
	\begin{align*}
		\covariance{X}{Y}
		 & = \expected{\pars{X - \expected{X}} \cdot \pars{Y - \expected{Y}}}            \\
		 & = \expected{X - \expected{X}} \cdot \expected{Y - \expected{Y}}               \\
		 & = \pars{\expected{X} - \expected{X}} \cdot \pars{\expected{Y} - \expected{Y}} \\
		 & = 0
	\end{align*}
\end{proof}

\begin{theorem}
	\label{variance-of-sum-of-independent-variables}
	Niech \( X_1, \dots, X_n \) będą parami niezależne. Wtedy
	\[
		\variance{\sum_{i=1}^n X_i} = \sum_{i=1}^n \variance{X_i}
	\]
\end{theorem}
\begin{proof}
	Skoro nasze zmienne są parami niezależne, to dla dowolnych \( X_i \neq X_j \) mamy \( \covariance{X_i}{X_j} = 0 \). W takim razie
	\begin{align*}
		\variance{\sum_{i=1}^n X_i}
		 & = \expected{\pars{\sum_{i=1}^n \pars{X_i - \expected{X_i}}}^2}                                  \\
		 & = \sum_{i=1}^n \expected{\pars{X_i - \expected{X_i}}^2}
		+ \sum_{i=1}^n \sum_{j=1}^n \expected{\pars{X_i - \expected{X_i}}\cdot\pars{X_j - \expected{X_j}}} \\
		 & = \sum_{i=1}^n \variance{X_i} + \sum_{i=1}^n \sum_{j=1}^n \covariance{X_i}{X_j}                 \\
		 & = \sum_{i=1}^n \variance{X_i}
	\end{align*}
\end{proof}


\subsection{Wyższe momenty}
\begin{definition}
	\(n\)-tym momentem zmiennej losowej \( X \) nazywamy \( \expected{X^n} \)
\end{definition}
