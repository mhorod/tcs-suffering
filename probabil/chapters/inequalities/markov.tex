\subsection{Definicja}
\begin{theorem}
    Jeśli \( X \) jest zmienną losową, która przyjmuje nieujemne wartości to
    \[
        P(X \geq a) \leq \frac{\expected{X}}{a}
    \]
\end{theorem}
\begin{proof}
    Niech \( I \) będzie indykatorem
    \[
        I = \begin{cases}
            1 & \text{ gdy } X \geq a \\
            0 & \text{ wpp. }
        \end{cases}
    \]
    Skoro \( X \geq 0 \) to \( I \leq \frac{X}{a} \).
    Zatem
    \[
        P(X \geq a) = P(I = 1) = \expected{I} \leq \frac{\expected{X}}{a}
    \]
\end{proof}

\subsection{Kolekcjoner kuponów}
Spróbujmy użyć nierówności Markowa do oszacowania jakoś czasu zebrania wszystkich kuponów.
Niech \( X \) będzie czasem zebrania wszystkich kuponów.
W sekcji \ref{coupon-collectors-problem} pokazaliśmy, że \( \expected{X} = nH_n = \Theta(n \ln n) \)

Możemy zatem skorzystać z nierówności Markowa aby otrzymać
\[
    P(X \geq 2n H_n) \leq \frac{\expected{X}}{2n H_n} = \frac{1}{2}
\]

Nie jest to jakieś szczególnie satysfakcjonujące oszacowanie -- prawdopodobieństwo, że musimy czekać dwa
razy dłużej niż tego oczekujemy może wynosić aż \( \frac{1}{2} \) :(
