2-SATa nikomu nie trzeba przedstawiać -- mamy \( n \) zmiennych i \( k \) klauzul postaci \( a \lor b \), gdzie \( a = x_i \) albo \( a = \bar x_i \).

Przykładem instancji problemu 2-SAT jest na przykład taka formuła:

\[ 
    (x_1 \lor x_3) \land (\neg x_1 \lor x_2) \land (x_3 \lor \neg x_1)
\]

Znamy algorytm rozwiązujący ten problem w czasie \( O(n + k) \) za pomocą silnie spójnych składowych, ale tutaj pokażemy \textbf{wolniejszy} algorytm.

\subsection{Algorytm}
\label{2-sat-algorithm}
Niech \( \lambda \in \natural \) będzie parametrem algorytmu (stałą czasu działania).
\begin{enumerate}
    \item Wylosuj dowolne wartościowanie zmniennych \( x_1, \dots, x_n \)
    \item Powtarzaj maksymalnie \( 2\lambda n^2 \) razy lub do znalezienia rozwiązania
    \begin{enumerate}
        \item Wylosuj niespełnioną klazulę
        \item Wylosuj literał z tej klauzuli i odwróć wartość zmiennej tego literału
    \end{enumerate}
    \item Jeśli mamy wartościowanie to je zwracamy
    \item W przeciwnym razie orzekamy, że formuła jest niespełnialna
\end{enumerate}

\subsection{Własności algorytmu}

\begin{theorem}[Lemat 7.1 P\&C]
    Jeśli dana jest formuła spełnialna, oraz pozwalamy działać algorytmowi dowolnie długo 
    to oczekiwana liczba kroków wynosi co najwyżej \( n^2 \)
\end{theorem}
\begin{proof}
    Będziemy modelować zachowanie algorytmu jako łańcuch Markowa (a jakże).
    Wybierzmy sobie dowolne wartościowanie \( S \), które spełnia formułę.
    Nazwijmy wartościowanie stworzone przez algorytm w \(i\)-tym kroku przez \( A_i \)
    
    Niech \( X_i \) oznacza liczbę zmiennych, które mają to samo wartościowanie w \( S \) oraz w \( A_i \)
    
    Mamy zatem taki proces \( X_0, X_1, \dots \), który niestety nie jest łańcuchem Markowa, bo tracimy informacje o tym, które zmienne mają jakie wartościowanie, a prawdopodobieństwo przejścia z \( X_i \) do \( X_{i+1} \) jest zadane wartościowaniem \( A_i \), które nie jest częścią łańcucha.
    
    Zrobimy zatem sztuczkę i rozważymy proces \( Y_0, Y_1, \dots \), który będzie łańcuchem Markowa i jednocześnie będzie pesymistyczną sytuacją naszego procesu.
    
    Zauważmy, że jeśli wybieramy klauzulę, która nie jest spełniona, to wartościowania \( A_i \) oraz \( S \) wartościują którąś ze zmiennych (być może obie) inaczej.
    W takim razie
    \[
        P(X_{i+1} = j + 1 \mid X_i = j) \geq \frac{1}{2}
    \]
    \[
        P(X_{i+1} = j - 1 \mid X_i = j) \leq \frac{1}{2}
    \]
    
    Zatem w pesymistycznej sytuacji, którą modeluje nasz \( Y_i \) mamy:
    \[
        Y_0 = X_0
    \]
    \[
        P(Y_{i+1} = j + 1 \mid Y_i = j) = \frac{1}{2}
    \]
    \[
        P(Y_{i+1} = j - 1 \mid Y_i = j) = \frac{1}{2}
    \]
    Przy czym (z uwagi na to, że mamy tylko jedną opcję)
    \[
        P(Y_{i+1} = 1 \mid Y_i = 0) = 1
    \]
    
    Niech \( Z_i \) oznacza liczbę kroków potrzebną do pierwszego dotarcia do stanu \( n \) zaczynając w stanie \( i \).
    Mamy
    \[
        h_i = \expected{Z_i} = \expected{1 + \frac{Z_{i-1}}{2} + \frac{Z_{i+1}}{2}}
    \]
    Dostajemy zatem układ \( n + 1 \) równań
    \[
        \begin{cases}
            h_0 = h_1 + 1 \\
            h_i = 1 + \frac{1}{2}\pars{h_{i-1} + h_{i+1}} \\
            h_n = 0 
        \end{cases}
    \]
    
    Przekształcamy środkową zależność do postaci
    \[
        h_{i + 1} = 2h_i - h_{i - 1} - 2
    \]
    Rozwiązując indukcyjnie dostajemy
    \begin{align*}
        h_1 &= h_0 - 1 \\
        h_2 &= 2h_1 - h_0 - 2 = h_1 - 3 = h_0 - 1 - 3 \\
        h_3 &= 2h_2 - h_1 - 2 = h_2 - 5 = h_0 - 1 - 3 - 5 \\
        h_i &= h_0 - i^2 \\
    \end{align*}
    W takim razie
    \[
        h_n = 0 = h_0 - n^2
    \]
    czyli \[ h_0 = n^2 \]
    
    Start w stanie 0 jest najbardziej pesymistyczny a pokazaliśmy, że w oczekiwaniu po \( n^2 \) krokach znajdziemy wartościowanie, co należało pokazać.
\end{proof}

\begin{theorem}[Lemat 7.2 P\&C]
    Jeśli formuła jest spełnialna, to algorytm \ref{2-sat-algorithm} myli się z prawdopodobieństwem \( 2^{-\lambda} \)
\end{theorem}
\begin{proof}
    Załóżmy na chwilę, że pozwalamy algorytmowi działać w nieskończoność.
    
    Dzielimy wykonanie algorytmu na bloki długości \( 2n^2 \).
    Niech \( Z_i \) oznacza liczbę kroków wykonaną od początku \( i \)-tego bloku (zakładając, że nie znaleźliśmy wcześniej wartościowania).
    
    Pokazaliśmy przed chwilą, że \(\expected{Z_i} \leq n^2\) zatem z nierówności Markowa
    \[
        P(Z_i > 2n^2) \leq \frac{n^2}{2n^2} = \frac{1}{2}
    \]
    
    W takim razie prawdopodobieństwo, że po wykonaniu \( \lambda \) bloków nie znaleźliśmy wartościowania wynosi
    \[
        \pars{\frac{1}{2}}^\lambda
    \]
    
\end{proof}
