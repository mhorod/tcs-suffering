\begin{definition}
    \textbf{Procesem stochastycznym} nazywamy dowolny zbiór zmiennych losowych \(\set{X(t) : t \in T}\).
    Zwykle \(t\) oznacza moment w czasie, a \(X(t)\) jest \textbf{stanem} tego procesu w czasie \(t\) i zapisujemy \(X_t\) \\
    Mówimy, że proces jest \textbf{skończony} jeśli zmienne \(X_t\) przyjmują skończenie wiele wartości.
\end{definition}

\begin{definition}
    \textbf{Procesem Markowa} (czasu homogenicznego) nazywamy taki proces stochastyczny \(X_0, X_1, X_2, \dots\) w którym dla dowolnego \(t\) zachodzi
    \[ 
        P(X_t = a_t \mid X_{t-1} = a_{t-1}, X_{t-2} = a_{t-2}, \dots X_0 = a_0) =
        P(X_t = a_t \mid X_{t-1} = a_{t-1}) 
    \]
\end{definition}

Innymi słowy aby dostać rozkład zmiennej \(X_t\) wystarczy, że znamy rozkład zmiennej \(X_{t-1}\) tzn. łańcuch Markowa jest bez pamięci.

Warto zauważyć, że \textbf{nie oznacza to}, że \(X_t\) jest niezależne od \(X_{t-2}, X_{t-3}, \dots\) --
jest, ale cała ta zależność jest zawarta w zależności od stanu \(X_{t-1}\).

\begin{definition}
    Łańcuch Markowa jest \textbf{czasu homogenicznego} jeśli \(
    P(X_t = a_t \mid X_{t-1} = a_{t-1} \land t = t_0) = P(X_t = a_t \mid X_{t-1} = a_{t-1})
    \)
\end{definition}
Mniej formalnie - na rozwój wydarzeń ma jedynie wpływ stan łańcucha, a nie czas w którym ten stan ma miejsce.

\begin{definition}
    \textbf{Macierzą przejścia} nazywamy macierz \(\mathbf{P}\) taką, że 
    \[
        P_{i, j} = P(X_t = j \mid X_{t-1} = i)
    \]
    oraz
    \[
        \forall_i : \sum_j P_{i, j} = 1
    \]
\end{definition}

\begin{definition}
    \textbf{Rozkładem stacjonarnym} nazywamy wektor \( \bar \pi \) taki, że \( \bar \pi = \bar \pi \mathbf{P} \) oraz \( \sum_i \bar \pi_i = 1 \)
\end{definition}
Intuicyjnie rozkład stacjonarny opisuje jak często asymptotycznie odwiedzamy każdy ze stanów niezależnie od tego skąd zaczęliśmy. Rozkład stacjonarny nie zawsze istnieje - np. łańcuch na liczbach naturalnych, taki, że \( p(n, n + 1) = 1\) w oczywisty sposób nie ma rozkładu stacjonarnego.

\begin{exercise}
    Niech dana będzie macierz \( Q \) wymiaru \( n \times n \), taka, że suma wartości w każdym wierszu wynosi 1.
    Rozważmy łańcuch Markowa na \( n \) stanach zadany tą macierzą.
    
    \begin{enumerate}
        \item Czy ma on rozkład stacjonarny?
        \item Co jeśli wszystkie wartości macierzy są dodatnie?
        \item Co jeśli dodatkowo suma wartości w każdej kolumnie wynosi 1?
    \end{enumerate}
\end{exercise}
\begin{proof}
    \begin{enumerate}
        \item Nie.
        Łańcuch zadany macierzą
        \[
            \begin{bmatrix}
                1 & 0 \\
                0 & 1 
            \end{bmatrix}
        \]
        nie ma jednoznacznego rozkładu stacjonarnego.
        
        \item Tak
        
        \item Tak, a ponadto \( \bar \pi = \brackets{\frac{1}{n}, \dots, \frac{1}{n}} \)
    \end{enumerate}
\end{proof}