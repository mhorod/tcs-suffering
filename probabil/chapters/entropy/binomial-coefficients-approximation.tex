\begin{theorem}[Lemat 10.2 P\&C]
    \label{entropy-binomial-coefficients-base}
    Niech \( n \in \natural, q \in \brackets{0, 1} \) oraz \( nq \in \natural \). Wtedy
    \[
        \frac{2^{nH(q)}}{n + 1} \leq \binom{n}{nq} \leq 2^{nH(q)}
    \]
\end{theorem}
\begin{proof}
    Dla \( q = 0 \) i \( q = 1 \) dowód jest oczywisty. Niech zatem \( q \in (0, 1) \)
    \begin{enumerate}
        \item Pokażmy najpierw ograniczenie górne.
        
        Ponieważ \( nq \in \natural \) wyrażenie \( \binom{n}{nq} \) występuje we wzorze
        \[
            1 = (q + (1-q))^n = \sum_{k=0}^n \binom{n}{k} q^k(1-q)^{n-k} = \binom{n}{nq}
            q^{qn}(1-q)^{(1-q)n} + \text{coś}
        \]
        Zatem 
        \[
            \binom{n}{nq} \leq q^{-qn}(1-q)^{-(1-q)n} = 2^{-nq\lg q - n(q-1)\lg(1-q)} = 2^{nH(q)}
        \]
        
        \item Teraz pokażemy ograniczenie dolne. 
        
        Będziemy chcieli pokazać, że \( \binom{n}{q}q^{qn}(1-q)^{(1-q)n} \) 
        jest największym składnikiem sumy -- w ten sposób dostaniemy
        \[
            \binom{n}{nq}q^{qn}(1-q)^{(1-q)n} \geq \frac{1}{n+1}
        \]
        co przekształcamy do postulowanej nierówności tak jak zrobiliśmy to przed chwilą.
        
        Różnica dwóch kolejnych składników sumy wynosi
        \[
            \binom{n}{k}q^k(1-q)^{n-k} - \binom{n}{k+1}q^{q+1}(1-q)^{n-k-1}
        \]
        \[
            = \binom{n}{k}q^k(1-q)^{n-k} \cdot \pars{1 - \frac{n - k}{k + 1} \cdot \frac{q}{1-q}}
        \]
        
        Jak wyliczymy nawias do sensownej postaci to nam wyjdzie, że jest on dodatni tylko gdy 
        \[
            k \geq qn - 1 + q
        \]
        
        W takim razie pierwszym takim \( k \) jest \( k = qn \), czyli \( qn\)-ty składnik jest tym największym.
    \end{enumerate}
\end{proof}
\begin{theorem}[Wniosek 10.3 P\&C] \( \) \\
   Dla \( 0 \leq q \leq \frac{1}{2} \)
    \[
        \frac{2^{nH(q)}}{n + 1} \leq \binom{n}{\floor{nq}} \leq 2^{nH(q)} 
    \]
    Dla \( \frac{1}{2} \leq q \leq 1 \)
    \[
        \frac{2^{nH(q)}}{n + 1} \leq \binom{n}{\ceil{nq}} \leq 2^{nH(q)} \\
    \]
\end{theorem}
\begin{proof}
    \begin{enumerate}
        \item Pokażmy najpierw ograniczenia górne.

    
        Skorzystamy tutaj z prostej obserwacji:
        \[
            \floor{nq} \leq nq \leq \ceil{nq}
        \]
        
        Zobaczmy jak to wygląda dla pierwszego przypadku.
        Skoro \( \floor{nq} \leq nq \) oraz \( q \leq \frac{1}{2} \) to 
        \( q^{qn} \leq q^{\floor{qn}} \) 
        oraz 
        \( (1-q)^{n - qn} \leq (1-q)^{n - \floor{qn}} \)
        , a zatem
        
        \[
            \binom{n}{\floor{nq}} q^{qn}(1-q)^{(1-q)n} 
            \leq \binom{n}{\floor{nq}} q^{\floor{qn}}(1-q)^{n - \floor{qn}} \leq 1
        \]
        
        Stąd stosujemy rozumowanie pokazane w dowodzie twierdzenia \ref{entropy-binomial-coefficients-base} aby dostać ograniczenie górne.
        
        Analogicznie pokazujemy przypadek drugi.
        
        
    \item Przejdźmy teraz do ograniczeń dolnych i ponownie ograniczmy się do pierwszego przypadku.
    
        Najlepiej byłoby pokazać
        \[
            \binom{n}{\floor{nq}}q^{qn}(1-q)^{(1-q)n} \geq \frac{1}{n+1}
        \]
 
        ale niestety w książce tego nie ma (jest za to blef) a na wykładzie tego nie omawialiśmy.
    \end{enumerate}
\end{proof}