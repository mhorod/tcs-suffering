\begin{definition}
	Długa igła jest to jeszcze piękniejsze zdarzenie, różni się tym że długa igła może przeciąć dwie krawędzie deski \( l \geq d \)
\end{definition}

\begin{theorem}
	Prawdopodobieństwo na przecięcie długiej igły z krawędzią wynosi
	\[
		\frac{2l}{d\pi} - \frac{2}{d\pi}\left(\sqrt{l^2 - d^2} + d\arcsin\frac{d}{l}\right) + 1
	\]
\end{theorem}

\begin{proof}
	Zobaczmy, że jeśli \( l\sin{\theta} > d\) to wówczas prawdopodobieństwo, że dana igła przetnie krawędź wynosi 1. Natomiast w w przeciwnym przypadku zachowuje się tak samo jak krótka igła. Gdy \( 0 \leq \theta \leq \arcsin\left({\frac{d}{l}}\right) \) to mamy krótką igłę. A zatem:

	\[
		\int_{\theta=0}^{\arcsin{\frac{d}{l}}} \int_{x=0}^{\frac{l}{2}\sin{\theta}} \frac{4}{d\pi} \, dx\, d\theta + \int_{\arcsin{\frac{d}{l}}}^{\frac{\pi}{2}} \frac{2}{\pi} d\theta =
		\frac{2l}{d\pi} - \frac{2}{d\pi}\left(\sqrt{l^2 - d^2} + d\arcsin\frac{d}{l}\right) + 1
	\]
	Po trywialnych przeliczeniach.

\end{proof}