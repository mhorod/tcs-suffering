W problemie Igły Buffona rzucamy ,,igłą''  na podłogę złożoną z ,,desek''. Pytamy się o prawdopodobieństwo przecięcia igły z krawędzią deski.\\ \\
Bardziej formalnie, mamy płaszczyznę na której są zaznaczone proste, które są parami równoległe. Odstęp między kolejnymi prostymi jest ustalony i wynosi on \(d\) - szerokość naszej deski.\\ \\
Następnie losujemy odcinek o długości \(l\) - długość igły, gdzieś na tej płaszczyźnie i pytamy się o prawdopodobieństwo, że ten odcinek przecina jedną z naszych prostych.
