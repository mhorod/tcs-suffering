\begin{definition}[Bramka Hadamarda]
	Bramka zdefiniowana jako
	\[
		H\pars{\begin{bmatrix} \alpha \\ \beta \end{bmatrix}} = \begin{bmatrix} \frac{1}{\sqrt{2}}(\alpha + \beta) \\ \frac{1}{\sqrt{2}}(\alpha - \beta) \end{bmatrix}
	\]
	o własności \( H(H(x)) = x \).
\end{definition}

\begin{definition}[Bramka Toffolego]
	Trójargumentowa bramka
	\[ T(x, y, z) = (x, y, (x \land y) \oplus z) \]
\end{definition}
Każda bramka musi być określona macierzą odwracalną i unitarną.

Algorytm Grovera to kwantowy algorytm wyszukiwania. Dla zadanej funkcji:
\[
	f : \set{0, 1, \ldots, N - 1} \rightarrow \set{0, 1}
\]
znajduje z prawdopodobieństwem \( \frac{2}{3} \), ciąg \( x^{*} \) wartościowany przez \( f \) inaczej niż pozostałe.
Na potrzeby algorytmu zakładamy, że \( N = 2^n \) oraz \( f \) jest określona na ciągach z \( \set{0, 1}^n \), z których tylko na \( x^{*} \) przyjmuje wartość -1, a na pozostałych 1.

Tworzymy \( n \)-wejściową bramkę \( O_f \), określoną regułą \( O_f (|x\rangle) = f(x) \cdot |x \rangle \).

Wykorzystując bramkę \( O_f \), konstruujemy obwód kwantowy, który:
\begin{itemize}
	\item składa się z bramek \( H \), \( T \) i \( O_f \),
	\item ma \( n \) wejść i \( n \) wyjść,
	\item otrzymuje na wejściu \( |00\ldots0 \rangle \),
	\item zwraca na wyjściu ciąg \( x^{*} \), dla którego \( f(x^{*}) = -1 \) z prawdopodobieństwem \( \frac{1}{10} \),
	\item zawiera \( \bigO(\sqrt{N} \log N) \) bramek, w tym \( \bigO(\sqrt{N}) = \bigO(\sqrt{2^n}) \) bramek \( O_f \).
\end{itemize}

Dysponując opisanym wyżej obwodem, możemy przejść do właściwego algorytmu.
\begin{greyframe}
	Algorytm Grovera:
	\begin{enumerate}
		\item Przepuść każdy z \( n \) kubitów przez bramkę H:
		      \[
			      |0\rangle \rightarrow \frac{1}{\sqrt{2}}|0\rangle + \frac{1}{\sqrt{2}}|1\rangle
		      \]
		      Układ \( n \) kubitów ma stan \( \frac{1}{\sqrt{2^n}} \sum_{x \in \set{0,1}^n} |x\rangle \), czyli każdy \( n \)-bitowy ciąg jest równie prawdopodobny.
		\item Kilkukrotnie przepuść kubity przez warstwę bramek, przekształcającą:
		      \[
			      \sum_{x \in \set{0,1}^n} \delta_x |x\rangle \rightarrow \sum_{x \in \set{0,1}^n} \delta'_x |x\rangle
		      \]
		      Każda warstwa zwiększa współczynnik przy \( x^{*} \), a zmniejsza przy \( x \neq x^{*} \), zachowując:
		      \begin{itemize}
			      \item \( \delta_x \in \real^{+} \),
			      \item \( \delta'_{x^{*}} > \delta_{x^{*}} + \frac{1}{\alpha\sqrt{N}} \) dla stałej \( \alpha \).
		      \end{itemize}
		      Po \( \bigO(\sqrt{N}) \) kroków \( \abs{\delta_{x^{*}}}^2 > \frac{1}{10} \), czyli z prawdopodobieństwem co najmniej \( \frac{1}{10} \) ciągiem wyjściowym jest \( x^{*} \).
		      Do osiągnięcia tego służą operacje liniowe:
		      \begin{itemize}
			      \item bramka \( O_f \):
			            \[
				            \delta_{x^{*}} \rightarrow -\delta_{x^{*}}, \ \delta_{x} \text{ bez zmian dla } x \neq x^{*}
			            \]
			      \item bramka dyfuzyjna Grovera (\( D \)):
			            \[
				            (\delta_0, \delta_1, \ldots, \delta_{N-1}) \rightarrow (2s - \delta_0, 2s - \delta_1, \ldots, 2s - \delta_{N-1}),
			            \]
			            gdzie \( s = \frac{1}{N}(\delta_0 + \delta_1 + \ldots + \delta_{N-1}) \), czyli jest średnią współczynników.
			            Operacja \( D \) odbija symetrycznie każde \( \delta_x \) względem \( s \).
		      \end{itemize}
	\end{enumerate}
\end{greyframe}

\begin{lemma}
	Operacje \( D \) i \( O_f \) wystarczają, żeby zwiększyć współczynnik przy \( x^{*} \), a zmniejszyć przy pozostałych \( x \neq x^{*} \).
\end{lemma}
\begin{proof}
	Stan układu kubitów to \( \sum_{x} \delta'_x |x\rangle \). Początkowo \( \delta_x = \frac{1}{\sqrt{N}} \) dla każdego \( x \),
	a w każdym kroku jeden ze współczynników \( (\delta_{x^{*}}) \) jest zamieniany na przeciwny. Następnie odejmujemy wszystkie współczynniki od \( 2s \).
	Wynika z tego, że po \( k \) krokach wszystkie \( (\delta_x) \) dla \( x \neq x^{*} \) są sobie równe z wyjątkiem \( (\delta_{x^{*}}) = b_k \).
	Bramka \( O_f \) zamienia \( b_k \) na \( -b_k \) , po czym bramka \( D \) oblicza \( b_{k+1} = 2s_k + b_k \), gdzie \( s_k \) jest średnią współczynników w \( k \)-tym kroku.
	Można udowodnić indukcyjnie, że \( b_{k+1} - b_k > \frac{1}{\sqrt{N}} \) tak długo, jak \( b_k < \frac{1}{2} \).
	Stąd po co najwyżej \( \frac{1}{10} \sqrt{N} \) krokach dostaniemy \( b_k > \frac{1}{10} \), co było do pokazania.
\end{proof}
\begin{lemma}
	Operację \( D \) można wyrazić jako kombinację \( \bigO(\log N) \) bramek \( H \) i \( T \).
\end{lemma}
\begin{proof}
	Bramka \( D \) jest opisana macierzą:
	\[
		D = \frac{2}{N}
		\begin{bmatrix}
			1 & 1 & \ldots & 1 \\
			1 & 1 & \ldots & 1 \\
			\ldots             \\
			1 & 1 & \ldots & 1
		\end{bmatrix}
		- I,
	\]
	czyli \( D = 2 \cdot v \cdot v^T - I \), gdzie \( v = \frac{1}{\sqrt{N}}
	\pars{\begin{smallmatrix}
			1 \\
			\vdots \\
			1
		\end{smallmatrix}} \).
	Macierz \( D \) jest postaci Householdera, więc jest unitarna. Faktem jest, że \( D = H_n \cdot Z_0 \cdot H_n \), gdzie \( Z_0 \) jest macierzą jednostkową z -1 zamiast 1 w komórce \( (0,0) \),
	a \( H_n \) jest macierzą opisującą bramkę \( H \) na każdym z kubitów. Składa się ona z \( n \) bramek \( H \). Do konstrukcji bramki \( Z_0 \) również wystarcza \( \bigO(n) \) dwuwejściowych bramek logicznych.
	Zatem całościowo potrzebujemy \( \bigO(n) = \bigO(\log N) \) bramek, co było do pokazania.
\end{proof}
