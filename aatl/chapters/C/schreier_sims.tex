\begin{definition}
	\textbf{Grupa permutacji} to zbiór \( S_n \) z działaniem składania permutacji.
\end{definition}

\begin{definition}
	\textbf{Stabilizatorem} \( i \) w grupie permutacji \( A \) jest zbiór \( A_i = \set{\pi \in A : \pi(i) = i} \).
\end{definition}

\begin{definition}
	\textbf{Orbitą} \( i \) w grupie permutacji \( A \) jest zbiór \(O_i = \set{\pi(i) : \pi \in A}\).
\end{definition}

Dla danego zbioru permutacji \(\Sigma = \set{\sigma_1, \sigma_2, \ldots, \sigma_k} \subset S_n\) chcemy określić, ile elementów ma grupa \(A \subseteq S_n\) generowana przez \(\Sigma\).

Korzystamy z następującego twierdzenia:
\begin{theorem}
	\( \abs{A} = \abs{A_1} \cdot \abs{O_1} \)
\end{theorem}
\begin{proof}
	Dzielimy grupę \( A \) na podzbiory \( B_1, \ldots, B_n \) zdefiniowane jako:
	\[
		B_x = \set{\pi \in A : \pi(1) = x}
	\]
	Zbiór \( B_x \) jest niepusty, jeśli \( x \in O_1 \), czyli istnieje permutacja \( \pi \in A \) taka, że \(\pi_x(1) = x\).
	Zatem odwzorowanie z \( A_1 \) do \( B_x \):
	\[
		\zeta \mapsto \pi_x \circ \zeta
	\]
	jest bijekcją i każdy niepusty zbiór \( B_x \) jest równoliczny z \( A_1 \). Stąd otrzymujemy:
	\[
		\abs{A} = \abs{B_1} + \ldots + \abs{B_n} = \abs{A_1} \cdot \abs{O_1},
	\]
	co było do pokazania.
\end{proof}

Z twierdzenia wynika schemat algorytmu:
\begin{greyframe}
	\begin{enumerate}
		\item Znajdź zbiór \( O_1 \).
		\item Utwórz zbiór \( \Sigma' \subseteq \Sigma \), który generuje \( A_1 \).
		\item Rekurencyjnie wyznacz \(\abs{A_1}\).
		\item Oblicz \(\abs{A} = \abs{O_1} \cdot \abs{A_1}\).
	\end{enumerate}
\end{greyframe}

\newpage
\subsubsection*{Jak wyznaczyć orbitę \(O_1\)?}
\begin{greyframe}
	\begin{enumerate}
		\item Utwórz graf \( G=(\set{1, \ldots, n}, E) \), gdzie \( E = \set{(x, \sigma_i(x)) : x \in V, \ \sigma_i \in \Sigma } \).
		\item Wyznacz \( O_1 \), czyli elementy osiągalne z \( 1 \), na przykład BFSem.
		\item Dla każdego \( x \in O_1 \) zapamiętaj \( \pi_x \in A \) taką, że \( \pi_x(1) = x \). \\
		      Permutacja \(\pi_x\) jest złożeniem generatorów na ścieżce od \( 1 \) do \( x \).
	\end{enumerate}
\end{greyframe}

\subsubsection*{Jak wyznaczyć zbiór \(\Sigma'\) generujący \(A_1\)?}
Dla \(\sigma_i \in \Sigma\) oraz \(x \in \{1, 2, \ldots, n\}\) konstruujemy:
\[
	\tau_{i,x} = \pi_{\sigma_i(x)}^{-1} \circ \sigma_i \circ \pi_x
\]
Permutacja \(\tau_{i,x}\) działa następująco:
\[
	1 \xrightarrow{\pi_x} x \xrightarrow{\sigma_i} \sigma_i(x) \xrightarrow{\pi_{\sigma_i(x)}^{-1}} 1,
\]
czyli \(\tau_{i,x} \in A_1\).

\begin{lemma}[Schreiera]
	Zbiór \( \Sigma' = \set{\tau_{i,x} : 1 \leq i \leq k, \ 1 \leq x \leq n} \) generuje \( A_1 \).
\end{lemma}
\begin{proof}
	Dowolna permutacja \(\zeta \in A_1\) musi być iloczynem generatorów:
	\[
		\zeta = \sigma_s \circ \sigma_{s-1} \circ \ldots \circ \sigma_2 \circ \sigma_1
	\]
	Ustalamy \(y_j = (\sigma_j \circ \ldots \circ \sigma_1)(1)\), dla \(j = 1, 2, \ldots, s\) i dodajemy \(\pi_{y_j} \circ \pi_{y_j}^{-1}\) między każde \(\sigma_j\) i \(\sigma_{j+1}\):
	\[
		\zeta = \sigma_s \circ \pi_{y_{s-1}} \circ \pi_{y_{s-1}}^{-1} \circ \sigma_{s-1} \circ \pi_{y_{s-2}} \circ \pi_{y_{s-2}}^{-1} \circ \ldots \circ \sigma_2 \circ \pi_{y_1} \circ \pi_{y_1}^{-1} \circ \sigma_1
	\]
	Ponieważ \(\pi_1 = \text{id}\), otrzymujemy:
	\[
		\zeta = (\pi_1^{-1} \circ \sigma_s \circ \pi_{y_{s-1}}) \circ (\pi_{y_{s-1}}^{-1} \circ \sigma_{s-1} \circ \pi_{y_{s-2}}) \circ \ldots \circ (\pi_{y_1}^{-1} \circ \sigma_1 \circ \pi_1)
	\]
	Udało się wygenerować permutację \(\zeta\) jako złożenie generatorów z \(\Sigma'\).
\end{proof}
Konstrukcja zbioru \( \Sigma' \) jest poprawna, ale w rekurencji jego rozmiar może dojść do \( n! \), ponieważ \( \abs{\Sigma'} \approx n \cdot \abs{\Sigma} \).
Do kontroli nad tym służy filtr Simsa.

\newpage
\subsection{Filtr Simsa}
\begin{theorem}
	Dla dowolnego zbioru \( P \subseteq S_n \) istnieje zbiór \( Q \subseteq S_n \) taki, że podgrupy generowane przez \(P\) i \(Q\) są równe oraz \( \abs{Q} \leq \frac{n(n-1)}{2} \).
\end{theorem}
Dowód twierdzenia jest algorytmizowalny: \\
Dzielimy permutacje na klasy -- permutacja \(\zeta \neq \text{id}\) jest klasy \((i, j)\), jeśli:
\[
	\zeta(1) = 1, \zeta(2) = 2, \ldots, \zeta(i - 1) = i - 1,
\]
\[
	\zeta(i) = j > i.
\]
Chcemy, żeby w \( Q \) był co najwyżej jeden element z każdej klasy, ponieważ to gwarantuje, że \( \abs{Q} \leq \frac{n(n-1)}{2} \).

Zaczynając od \(Q = \emptyset\), przekładamy permutacje z \(P\) do \(Q\), utrzymując niezmienniki, że grupa generowana przez \( P \cup Q \) nie zmienia się i w \( Q \) jest co najwyżej jeden element każdej klasy.

Dla każdego elementu \(\zeta \in P\) klasy \((i, j)\):
\begin{enumerate}
	\item Jeśli w \(Q\) nie ma nic klasy \((i, j)\), przekładamy \(\zeta\) do \(Q\).
	\item Jeśli w \(Q\) jest element \(\varphi\) klasy \((i, j)\), zastępujemy \(\zeta\) przez \(\zeta' = \varphi^{-1} \circ \zeta\) i powtarzamy.
\end{enumerate}
Procedura zachowuje niezmienniki, ponieważ \(\zeta = \varphi \circ \zeta'\), więc zbiór generowany przez \(P \cup Q\) pozostaje taki sam.
Zmienia się natomiast klasa elementu, ponieważ \(\zeta'(i) = i\), więc \(\zeta'\) jest klasy \((i', j')\), gdzie \(i' > i\).

Algorytm się nie zapętli, ponieważ \( P \) ma co najwyżej \(k\) elementów, które można przerzucić do \(Q\) i można co najwyżej \( n \) razy zmienić klasę \( \zeta \), bo każda zmiana zwiększa \( i \).
Filtr Simsa działa więc w czasie \(\bigO(k \cdot n^2)\), bo operacje na permutacjach są liniowe.

Poszczególne etapy algorytmu Schreiera-Simsa mają złożoność:
\begin{enumerate}
	\item Obliczenie orbity \(O_1\) \( \longrightarrow \bigO(\abs{\Sigma} \cdot n) \)
	\item Wyznaczenie \(\Sigma'\) \( \longrightarrow \abs{\Sigma'} = \bigO(n \cdot \abs{\Sigma}) \)
	\item Filtrowanie \(\Sigma'\) \( \longrightarrow n^2 \cdot \abs{\Sigma'} = \bigO(n^3 \cdot \abs{\Sigma}) \)
\end{enumerate}
Algorytm wykonuje \( n \) wywołań rekurencyjnych, a \( \abs{\Sigma} \) w każdym z nich mieści się w \( \bigO(n^2) \), więc łączna złożoność to \( \bigO(n^6) \).