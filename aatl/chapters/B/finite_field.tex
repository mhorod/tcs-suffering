% \begin{theorem}(O charakterystyce)
% Charakterystyka ciała skończonego jest zawsze dodatnią liczbą pierwszą.
% \end{theorem}
% \begin{proof}
%     \textit{Charakterystyka jest liczbą dodatnią.} \\
%     Załóżmy nie wprost, że po dodaniu jedynki \( n = \abs{\mathbb{F}} \) razy, nie udało się otrzymać zera. Jakiś element musiał się więc powtórzyć, czyli dla pewnych \( j < k \) zachodzi \( j \cdot 1 = k \cdot 1 \) w \( \mathbb{F} \).
%     Wynika z tego, że \( (k - j) \cdot 1 = 0 \), więc mamy skończoną dodatnią charakterystykę. Inaczej każda suma jedynek musiałaby być równa innej wartości, czyli \( \mathbb{F} \) miałoby nieskończenie wiele elementów, co daje sprzeczność.

%     \textit{Charakterystyka jest liczbą pierwszą.} \\
%     Załóżmy nie wprost, że charakterystyka jest liczbą złożoną \( n = p \cdot q \). Wynika z tego, że
%     \[
%         \underbrace{(1 + \dots + 1)}_n = \underbrace{(1 + \dots + 1)}_p \cdot \underbrace{(1 + \dots + 1)}_q = 0,
%     \]
%     czyli \( p \) lub \( q \) jest równe 0. Zatem \( n \) nie jest najmniejszą liczbą jedynek, które po dodaniu dają 0, co daje sprzeczność z tym, że \( n \) jest charakterystyką.
% \end{proof}

% \newpage
\begin{theorem}
	Ciało skończone \( \mathbb{F} \)  o charakterystyce \( p \) ma \( p^k \) elementów dla pewnego \( k \in \integer \).
\end{theorem}
\begin{proof}
	Wiedząc, że \( \integer_p \) jest ciałem, możemy wprowadzić mnożenie przez element z \( \integer_p \) elementów z ciała \( \mathbb{F} \):
	\[
		a \cdot x = \underbrace{x + x + ... + x}_a, \ a \in \integer_p,\; x \in \mathbb{F}
	\]
	Definiujemy w ten sposób przestrzeń liniową \( \mathbb{F} \) nad \( \integer_p \), jako że mamy dodawanie i mnożenie przez skalar. Nasza przestrzeń liniowa ma skończony wymiar \( k \) i bazę \( x_1, \dots, x_k \). \\
	Każdy element \( x \in \mathbb{F} \) możemy zapisać jako:
	\[
		a_1x_1 + a_2x_2 + \dots + a_kx_k
	\]
	Różne ciągi \( (a_1, \dots, a_k) \) dają różne elementy \( \mathbb{F} \) z definicji bazy. Stąd liczba elementów \( \mathbb{F} \) jest równa liczbie ciągów \( (a_1, \dots, a_k) \), czyli \( p^k \).
\end{proof}