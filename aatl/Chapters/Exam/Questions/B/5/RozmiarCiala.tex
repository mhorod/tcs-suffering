\subsubsection{Twierdzenie 1. (O charakterystyce):}
Charakterystyka ciała skończonego zawsze jest dodatnia i jest liczbą pierwszą.
\subsubsection{Dowód Tw. 1:}
 1. Nie zerowość. \newline \newline
 Nie wprost zakładamy, że nie otrzymujemy 0 (poprzez dodawanie jedynki) a wykonaliśmy więcej niz ilość elementów w ciele dodań 1. Co oznacza, że jakiś element musiał się powtórzyć. Jeżeli jakiś element się powtórzył to z definicji suma między powtórzeniami wynosi 0, gdyż jest ono elementem neutralnym dodawania. Czyli 0 musiało wystąpić sprzeczność.
\newline \newline
 2. Charakterystaka jest liczbą pierwszą.\\ \\
 Załóżmy nie wprost, że charakterystyka jest liczbą złożoną równą $n$. Możemy z tego faktu rozbić $n = pq$ gdzie $p,q > 1$. W takim razie $(1+1+...+1) (n$ razy$) = (1+1+...+1) (q$ razy$) \cdot (1+1+...+1) (p$ razy$)$ z czego wynika, że $p$ lub $q$ jest 0. Co oznacza że $n$ nie jest najmniejsza taką liczbą która po dodaniu $n$ razy 1 otrzymamy 0. Czyli mamy sprzeczność z definicji charakterystyki.
 \subsubsection{Twierdzenie 2.}
 Niech $\mathbb{F}$ będzie ciałem skończonym charakterystyki $p$. Wtedy istnieje takie $k$, że $|\mathbb{F}| = p^k$.
 \subsubsection{Dowód Tw. 2}
Wiemy, że $\mathbb{Z}_p$ jest ciałem. Możemy więc wprowadzić mnożenie przez element z $\mathbb{Z}_p$ elementów z ciała $\mathbb{F}$ zdefiniowane jako $a\cdot x = x + x + ... + x (a razy)$. Wynika z tego, że mamy teraz przestrzeń linową na $Z_p$ gdyż mamy ciało oraz mnożenie przez skalar. Skoro $\mathbb{F}$ jest przestrzenią liniową nad $Z_p$ to ma ona skończony wymiar, nazwijmy go $k$, oraz jakąś baze $x_1,...,x_k$. A więc każdy element $x \in \mathbb{F}$ możemy zapisać jako $a_1x_1 + a_2x_2 + ... + a_kx_k$. Różne ciągi $(a_1,...,a_k)$ dają różne elementy z $\mathbb{F}$ z czego wynika, że liczba elementów $\mathbb{F}$ jest taka sama jak liczba ciągów $(a_1,...,a_k)$ czyli $p^k$.
