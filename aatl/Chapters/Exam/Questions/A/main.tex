\section{Grupa A}

\subsection{Opisać rozszerzony algorytm Euklidesa znajdowania NWD}
Dla danych $a,b > 0$, algorytm zwraca $d = \gcd(a,b)$ oraz takie, że $s,t \in \mathbb{Z}$, że $d = s\cdot a + t\cdot b$. 
\newline

Algorytm Euklidesa
\begin{itemize}
    \item 1. Jeśli $a < b$, zamień $a$ i $b$.

    \item 2. Jeśli $b = 0$, zwróć $d = a$, oraz parę $(1,0)$.
    \item 3. Podziel z resztą $a$ przez $b$, otrzymując $a = q \cdot b + r$.
    \item 4. Wywołaj $\gcd(b,r)$, otrzymując $d$ oraz parę $(s,t)$, taką, że $s\cdot b + t\cdot r = d$
    \item 5. Zwróć $d$ oraz parę $(t,s - t \cdot q)$
\end{itemize}

Dlaczego ten algorytm działa ? \newline
To, że zwraca on poprawne $d$ to chyba już każdy widział wiele razy (pokazuje się, że skoro $d \mid a$ oraz $d \mid b$ to $d \mid (a - b)$, jeżeli $a \geq b$, a to idzie odrazu z przystawania modulo $d$). Jedyne co musimy pokazać, to fakt, że zwraca on poprawne $s,t$, czyli takie że zachodzi $s\cdot a + t \cdot b = d$. A więc założmy sobie niezmiennik, że w każdym kroku algorytmu nasze $s,t$ dla obecnych $a,b$ jest poprawne.\newline \newline

Baza ($b = 0$):\newline
Zwracamy, że $d = a$, oraz zwracamy $(1,0)$, z czego wynika, że $s = 1, t = 0$ czyli $1\cdot a + 0 \cdot b = a = d$. Czyli jest OK
\newline
\newline

Krok: \newline
Mamy, $a,b$, oraz $a = q\cdot b + r$ oraz zwrócone z rekursji ( na argumentach $(b,r)$) $d,s,t$ takie, że $s\cdot b + t\cdot r = d$. Chcemy teraz zmienić nasze $s,t$ tak aby warunek zachodził dla $a,b$. A więc chcemu powiedzieć że dla nas będzie to $x = t$, $y = s - t \cdot q$ (konstruujemy nasze nowe $s,t$). Sprawdźmy teraz czy dla naszego $x,y$ zachodzi nasz warunek. $$x\cdot a + y \cdot b = t \cdot a + (s - t\cdot q) \cdot b = t \cdot a + s \cdot b - t \cdot q \cdot b$$
Tutaj skorzystamy z faktu, że $a = q\cdot b + r$, z czego wynika, że $q\cdot b = a - r$.

$$t \cdot a + s \cdot b - t \cdot q \cdot b = t\cdot a + s\cdot b - t\cdot(a - r) =  t\cdot a + s \cdot b - t \cdot a + t\cdot r = s \cdot b + t \cdot r = d = x\cdot a + y \cdot b$$

Czyli jak widać nasz niezmiennik jest zachowany czyli algorytm działa. \newline

Złożność:
Zauważmy, że każdym przynajmniej jedna z liczb $a,b$ spadnie nam o 2 (casologia):
\begin{itemize}
    \item 1. $b \leq a/2$ z czego wynika, że $a\mod b < b < a/2$ czyli do rekurencji przekazaliśmy argument conajmniej dwukrotnie mniejszy
    
    \item 2. $a > b > a/2$ wynika z tego, że $a - b < a/2$ ( a odejmowanie w tym przypadku zachowuje się tak samo jak modulo, gdyż $a = 1 \cdot b + r$, z czego wynika, że $a - b = 1 \cdot b + r - b = r$ )
\end{itemize}
Czyli mamy $\bigO(\log a)$ (założenie, że $a \geq b$) kroków rekursji co daje nam złożoność $\bigO(\log a \cdot M(a))$, gdzie $M(a)$ to złożonośc operacji w jednym kroku rekurencji (Według wykładu złożoność całkowita to $\bigO(\log^2 a)$ ale nie za bardzo wiem czemu jak w każdym kroku wykonujemy mnożenie,dzielenie i modulo co nie jest tanie chyba że da się go jakość zoopcić).


\subsection{Opisać binarny algorytm Euklidesa znajdowania NWD}
Pseudokod: \\
Binarny Algorytm Euklidesa:
\begin{itemize}
    \item 1. Jeśli $a < b$, zamień $a$ i $b$,
    
    \item 2. Jeśli $b = 0$, zwróć $a$,
    
    \item 3. Jeśli $2 \mid a$ i $2 \nmid b$, zwróć $gcd(a/2, b)$,
    
    \item 4. Jeśli $2 \nmid a$ i $2 \mid b$, zwróć $\gcd(a, b/2)$,
    
    \item 5. Jeśli $2 \mid a$ i $2 \mid b$, zwróć $2 \cdot \gcd(a/2, b/2)$, 
    
    \item 6. Jeśli $2 \nmid a$ i $2 \nmid b$, zwróć $\gcd(b, a - b)$.
\end{itemize}

Jak możemy zauważyć algorytm ten jest bardzo podobny do normalnego algorytmu euklidesa lecz rozważa on wszystkie możliwe parzystości $a$ oraz $b$ w danym kroku algorytmu.
\newline

\textbf{Dlaczego owy algorytm działa?} \newline 

Tutaj musimy się mocno wycaseować. Zauważmy na początek, że przypadek 1 oraz 2 są dokładnie takie same jak w zwykłym euklidesie.\newline 

Następnie weźmy sobie przypadek 3 i 4 na cel. W tych przypadkach jedna z liczb jest podzielna przez 2 a druga nie jest. Wynika z tego, że w ich nwd na pewno nie jest podzielne przez 2, czyli możemy podzielić tą liczbę, która jest podzielna przez 2 i uruchomić się rekurencyjnie i otrzymamy poprawne $\gcd$.


Następny przypadek jest przypadek 5 i wynika z niego, że $2 \mid a$ oraz $2 \mid b$, co oznacza, że w ich nwd możemy uzględnić czynnik $2$ oraz uruchomić się na $a/2$ oraz $b/2$, gdyż wyciągamy ten czynnik przed funkcję $\gcd$ i w ten sposób otrzymujemy poprawne $\gcd$.



Ostanim przypadkiem jest, $2 \nmid a$ oraz $2 \nmid b$. Ale w nim wykonujemy normalny krok z algorytmu euklidesa czyli $\gcd(a,b) = \gcd(b,b-a)$, który też oczywiście jest poprawny.


Złożność Algorytmu:
\newline
Zauważymy, że w każdym z przypadków od 3 do 5, $a$ lub $b$ spada conajmniej dwukrotnie. Jedynm problemem wydaje się przypadek 6. Ale okazuje się, że nie jest to duży problem, gdyż jeżeli $2 \nmid a$ oraz $2 \nmid b$ to $2 \mid b-a$, czyli w wywołaniu rekurencyjnym zajdzie już przypadek od 3 do 5. A więc w conajwyżej dwóch krokach jedna z liczb spadnie dwukrotnie. A więc otrzymujemy złożoność $\bigO(\log (a + b) \cdot M(a,b))$, gdzie $M(a,b)$ to koszt wykonania operacji podziel przez 2, wymnóż razy 2, sprawdź podzielność przez 2, oraz odejmij liczby $a$ oraz $b$ a każdą z tych operacji jesteśmy w stanie wykonać liniowo względem zapisu liczby czyli w czasie $\bigO(\log n)$, czyli całkowita złożność z podliczonymi operacjami wynosi $\bigO(\log^2 (a+b)) = \bigO(\log^2 (a))$, przy założeniu, że $a \geq b$.


\subsection{Zapisać algorytmy szyfrowania RSA oraz El-Gamal i opisać, na czym polega trudność ich łamania}
Szybkie przypomnienie jak szyfrujemy ( przynajmniej w tym przypadku ) mamy sobie klucz publiczny i klucz prywatny, klucz publiczny udostępniamy i mówimy jak chcesz do mnie coś wysłać to użyj mojego klucza publicznego (no i oczywiście tej samej metody co ja) a ja sobie to odszyfruje moim kluczem prywatnym i przeczytam twoją wiadomość (funkcję szyfrującą oznaczamy zazwyczaj $E(X)$, a deszyfrująca $D(x)$. \newline \newline

RSA Idea:
\begin{itemize}
    \item Wybiersz sobie liczby pierwsze $p,q$, oraz oblicz $N = p\cdot q$, oraz zauważ, że $\phi(N) = (p-1)(q-1)$ (zamiast $\phi(n)$ można wszędzie użyć lcm($p-1,q-1$))
    \item Wybierz $e$ względnie pierwsze z $\phi(N)$ oraz oblicz takie $d$, że $e\cdot d = 1 \mod \phi(N)$ (można tu użyć rozszerzonego algorytmu euklidesa)
    \item Niech $E(x) = x^e \mod N$ oraz $D(x) = x^d \mod N$, czyli nasz klucz publiczny to $(N,e)$
\end{itemize}

Dlaczego deszyfracja działa? \newline
Rząd grupy multiplikatywnej modulo $N$ wynosi $\phi(N)$, a więc dla każdego $x$ względnie pierwszego z $N$ zachodzi $(x^{e})^d = x^{e\cdot d} = x^1$, gdyż $e\cdot d = 1 \mod \phi(N)$ czyli $e\cdot d$ dzieli rząd naszej grupy multiplikatywnej modulo $N$.
\newline \newline

RSA możemy złamać, gdy rozłożymy $N$ na czynniki. Czyli otrzymamy $p,q$. Obliczymy wtedy $\phi(N)$ oraz znajdziemy $d$ (rozszerzonym euklidesem), dla którego zachodzi $e\cdot d = 1 \mod \phi(N)$. Obecnie nie umiemy dobrze rozkładać na czynniki dlatego nie umiemy też łatwo łamać RSA o ile ktoś dobrze dobrał liczby.
\newline \newline

Idea El-Gamal:
\begin{itemize}
    \item Wybierzmy sobie jakąś grupę $G$ (Na przykład grupę multiplikatywną ciała skończonego $\mathbb{F}_q$) oraz element $g \in G$ (najlepiej generator).
    \item Wylosuj liczbę $x \in G$.
    \item Klucz publiczny to $(g,g^x)$, a prywatny to $(g,x)$
    \item Szyfrowanie wygląda w następujący sposób, wylosuj liczbę $y$ oraz oblicz $g^y$ oraz $g^{xy}$ oraz wyślij wiadomość $P$ w postaci $(g^y,P\cdot g^{xy})$. 
\end{itemize}
Znając $x$ oraz $g^y$ możemy wykonać operację $(g^y)^x = g^{xy}$, następnie znaleść odwrotność $g^{xy}$ i odzyskać $P$, gdyż $P\cdot g^{xy} \cdot g^{-xy} = P$
\newline \newline
Problem w odszyfrowaniu polega na tym, że znająć $g^x,g^y$ nie potrafimy obliczyć $g^{xy}$, czyli sprowadza się do to szyfrowanie Diffiego-Hellmana, co rozwiązuje się logarytmem dyskretnym, którego nie potrafimy obecnie szybko liczyć.



\subsection{Opisać metodę klucza jednorazowego Vernama, trudność jej łamania i praktyczne zastosowanie}
Idea klucza jednorazowego Vernama polega na tym, że mamy sobie osobę A oraz osobę B. Osoba A chce przesłać osobie B wiadomość $x$ zapisaną bitowo oraz mają one między sobą bezpieczny kanał i normalny (być może niebezpieczny) kanał do przesyłania wiadomości. Teraz osoba A losuje sobie $k$ o tej samej długości co wiadomość $x$ oraz wykonuje operacje $w = x \oplus k$. Następnie osoba A przesyła bezpiecznym kanałem wartość $k$ oraz normalnym kanałem wartość $w$. Aby osoba B mogła odczytać $x$ wystarczy, że weźmie i wykona operację $w \oplus k$, gdyż $x = x \oplus k \oplus k$ i xor jest łączny i przemienny a $k \oplus k = 0$. \newline \newline
Dlaczego to było by super bezpieczne? \newline
Okazuje się że losując nasze $k$ to wygląda tak samo jakbyśmy osobno losowali jej każdy bit, a co za tym idzie to to że jeżeli wylosujemy 1 na jakimś miejscu to że mienimy ten sam bit na przeciwny w zapisie $x$. Czyli w pełni (no być może pseudolosowo) losowo pozmieniemy wszystkie bity $x$ co za tym idzie, że miedzy kolenymi wysłaniami wiadomości nie ma żadnych relacji i jest on nie do złamania.
\newline \newline
Jakie są jego przypadki użycia? \newline 
Żadne !!! Po co robić cokolwiek jeżeli mamy bezpieczny kanał to poprostu wyślimy $x$ bezpiecznym kanałem i tyle, nie jest wtedy potrzebna żadna kryptografia.

\subsection{Zdefiniować problemy PRIMES oraz FACTORING i podać ich umiejscowienie w klasach złożoności}
Definicja problemu PRIMES:\newline
Mając na wejściu liczbę $p$ stwierdź czy $p$ jest liczbą pierwszą (Odpowiedź TAK/NIE).
\newline \newline
PRIMES $\in coNP$: \newline
Zgadnij $d \in \{2,...,p-1\}$, jeżeli $d \mid p$ odpowiedz NIE.
\newline \newline
PRIMES $\in NP$: \newline
Tw. $\mathbb{Z}_p^*$ ma rząd $p-1$ wtedy i tylko wtedy, gdy $p$ jest pierwsze.\newline
Zgadujemy $g$(generator $\mathbb{Z}_p^*$), oraz rozkład na czynniki pierwsze $p-1$, czyli $p_1^{\alpha_1}\cdot ... \cdot p_s^{\alpha_s} = p-1$. Następnie sprawdzamy:
\begin{itemize}
    \item Rekurencyjnie dla każdego zgadniętego $p_i$ czy jest pierwsze.
    \item Czy $g^{p-1} = 1 \mod p$.
    \item Czy dla każdego $p_i$ zachodzi, że $g^{\frac{p-1}{p_i}} \neq 1 \mod p$ (Zauważ że nie uwzględniamy potęg liczb pierwszych gdyż są to najwieksze dzielniki, w których brakuje dokładnie czynnika więc jeżeli dla jakiegoś mniejszego dzielnika by to zachodziło to dla jednego z tych też zajdzie, gdyż dzieli on jeden z naszych dzielników).
\end{itemize}
Jeżeli wszystkie te warunki są prawdziwe to odpowaiadamy TAK.
\newline \newline
PRIMES $\in BPP$: \newline
Dowód jest przez pokazanie algorytmu Millera-Rabina.

%\newline \newline

PRIMES $\in P$:\newline
Dowód to pokazanie algorytmu AKS którego tutaj raczej nie trzeba będzie pokazać.

%\newline \newline

Definicja problemu FACTORING:\newline
Na wejściu dana jest liczba $n$ oraz liczba $k$. Stwierdź czy istnieje dzielnik $n$ mniejszy lub równy $k$. Czyli formalnie $\exists_d: 2 \leq d \leq k \land d \mid n$.
\newline \newline

FACTORING $\in NP$: \newline
Zgadnij $d \in \{2,...,k\}$ jeżeli $d | n$ odpowiedz TAK.
\newline \newline

FACTORING $\in coNP$: \newline
Zgadujemy rozkład na czynniki pierwsze liczby $n$, czyli mamy $p_1^{\alpha_1}\cdot ... \cdot p_s^{\alpha_s}$. Następnie sprawdzamy czy $p_1^{\alpha_1}\cdot ... \cdot p_s^{\alpha_s} = n$ oraz czy każdego $p_i$ jest pierwsze (albo poprzez AKS albo odpalamy się na algorytmie w $NP$). Jeżeli dla każego $p_i$ nasze sprawdzenie odpowiedziało tak to znajdujemy najmniejsze $p_i$ w naszym rozkładzie i zwracamy TAK, jeżeli najmniejsze $p_i$ jest mniejsze lub równe $k$ w przeciwnym odpowiedz NIE.
\newline
Więcej o tym problemie nie potrafimy narazie powiedzieć.


\subsection{Podać efektywną metodę znalezienia liczby pierwszej o zadanej liczbie bitów}
Mamy podana liczbę $k$ oraz chcemy znaleść liczbę pierwszą $p$, która ma $k$ bitów, z czego wynika, że $p \in [2^k,2^{k+1}-1]$.
\newline \newline
Pierwszym faktem jaki zauważymy jest gęstość liczb pierwszych:\newline
W przedziale od 1 do n jest asymptotycznie $\bigO(\frac{n}{\log n})$ liczb pierwszych. 
\newline
Wynika z tego, że liczby pierwsze sa upakowane dosyć gęsto. Wiemy, że w przedziale od 1 do $2^{k+1} - 1$ jest rzędu $c\cdot \frac{2^{k+1}-1}{k+1}$ liczb pierwszych oraz w przedziale od 1 do $2^{k} - 1$ jest rzędu $c\cdot \frac{2^{k}-1}{k}$ wynika z tego, że w przedziale $[2^k,2^{k+1}-1]$ jest $c\cdot \frac{2^{k+1}-1}{k} - c\cdot \frac{2^{k}-1}{k}$ około tyle liczb pierwszych. Z czego wynika że mamy tam dalej $\bigO(\frac{n}{\log n})$ liczb pierwszych (troche machane).
\newline
\newline
Skoro są one upakowane dosyć gęsto to wykonajmy następującą procedurę.
\begin{itemize}
    \item Wylosuj $p$ z przedziału $[2^k,2^{k+1}-1]$.
    \item Za pomocą algorytmu Millera-Rabina sprawdź czy $p$ jest liczbą pierwszą
    \item Jeżeli jest pierwsza to ją zwróć, w przeciwnym wypadku powtórz procedurę.
\end{itemize}

Skoro wiemy, że w tym przedziale jest $\bigO(\frac{2^k}{k})$ liczb pierwszych to oznacza że w oczekiwaniu po $\bigO(k)$ losowaniach trafimy na liczbę pierwszą. Sprawdzanie z wolnym monożeniem czy liczba jest pierwsza z pomocą algorytmu Millera-Rabina wykonuje się w czasie $\bigO(k^3)$ czyli w oczekiwaniu otrzymujemy algorytm w złożoności $\bigO(k^4)$.


\subsection{Opisać efektywną implementację działań arytmetycznych w ciele skończonym $Z_p/(W)$}
\begin{itemize}
    \item Mamy dodawanie w $\bigO(n)$ - trywialne.
    \item Mnożenie standardowe w $\bigO(n^2)$, można użyć Karatsubę, Tooma-Cooka lub Schonhagego-Strassena by zejść niżej do jakiegoś $\bigO(n\log n)$
    \item dzielenie mamy standardowo Hornerem $\bigO(n^2)$. Można też dzielić szybciej. Jeśli dzielimy $A(X)$ przez $B(X)$, to wtedy dajemy sobie funkcję pomocniczą $rev_k(P(x))=x^kP(\frac{1}{x})$, a następnie szukamy $rev_m(B(x))^{-1}$ w pierścieniu Taylora (a istnieje ona, bo wyraz wolny u nas to $1$, szukamy to Newtonem). I mamy coś takiego 
    $$rev_n(A)\equiv rev_m(B)\cdot rev_{n-m}(Q) \mod y^{n-m+1}$$
    $$rev_n(A)\cdot rev_m(B)^{-1}\equiv rev_{n-m}(Q) \mod y^{n-m+1}$$
    Z tego liczymy $Q=rev_{n-m}(rev_{n-m}(Q))$, a następnie $R=A-BQ$. Lub poprostu robimy rozszerzony algorytm euklidesa bo sensownie i szybko działa i zwraca nam piękinie odwrotność w tym ciele.
\end{itemize}



\subsection{Opisać ideę algorytmu AKS (schemat, bez dowodu)}
Twierdzenie 1. \newline
Niech $n,a \in \mathbb{Z}$ oraz $\gcd(a,n) = 1$.
$$(X + a)^n = X^n + a \mod n$$
Zachodzi wtedy i tylko wtedy, gdy $n$ jest liczbą pierwsza.
\newline
Szkic dowodu na wszelki wypadek: \newline \newline
Jeśli $n$ jest liczbą pierwszą, to wszystkie współczynniki $n \choose k$ dla $0 < k < n$ są podzielne przez $n$, a zatem $(X + a)^n = X^n + a^n \mod n$, a dodatkowo z małego twierdzenia Fermata $a^n = a \mod n$. Jeśli $n$ nie jest liczbą pierwszą, to przynajmniej jedno $n \choose k$ nie jest podzielne przez $n$, a więc wielomian $(X + a)^n$ zawiera wyraz
$n \choose k$ $X^ka^{n-k}$ (wystarczy wziąć za $k$ najmniejszy dzielnik $n$), nie może więc być równy $X^n + a \mod n$
\newline \newline
Od teraz zacznie się ciekawiej. Niestety obliczenie wielomianu $(X + a)^n$ jest za drogie dlatego obliczenia będziemy prowadzić w pierścieniu ilorazowym $\mathbb{Z}_n[X]/(X^r - 1)$ ($r$ sobie za chwile wyczarujemy). Po tej redukcji okaże się, że jednak jedno $a$ nie wystarczy ale będziemy musieli sprawdzić ich stosunkowo mało.
\newline \newline
Idea Algorytmu:
\begin{itemize}
    \item 1. Sprawdź czy n jest potęgą liczby pierwszej tzn. $n = p^k$ dla pewngo $k \geq 2$ lub czy $2 \mid n$ jeżeli tak to zwróć złożona.
    \item 2. Znajdź najmniejsze $r$ takie, że rząd $n \mod r$ jest większy niż $\log^2 n$.
    \item 3. Jeżeli dla jakiegoś $a \leq min(r,n-1)$ $\gcd(a,n) \neq 1$,  to zwróć złożona.
    \item 4. Jeżeli $n \leq r$ zwróć pierwsza.
    \item 5. Niech $l = \sqrt{r} \log n$, Dla każdego $a$ takiego, że $1 \leq a \leq l$ sprawdź równość $(X + a)^n = X^n + a \mod (n,X^r - 1)$, jeżeli równość nie zajdzie zwróć złożona.
    \item 6. Jak nic się wcześniej nie wywaliło to zwróć pierwsza

\end{itemize}

Szybki argument złożoności:\newline
Najwięcej sprawdzamy w punkcie 5, gdyż $r$ jest rzedu $\bigO(\log^5 n)$. Z tego wynika, że $l$ jest rzędu $\bigO(\log^{3.5} n)$, A na obliczenie każdego równania potrzebujesz czasu $\bigO(r \log^2 n) = \bigO(log^7 n)$ co wymnażając otrzymujemy $\bigO(\log^{10.5} n)$.


\subsection{Pokazać, że wielomianowy algorytm na problem pierwiastka dyskretnego da się zamienić na wielomianowy algorytm na faktoryzację}
Twierdzenie 1.\newline
Mamy liczbę $n$ złożoną oraz $n$ nie jest potęgą liczby pierwszej (możemy też założyć że $n$ jest nieparzyste). Wtedy dla dowolnych dla dowolnego $u$ rówanie:
$$x = u^2 \mod n$$
Jeżeli to równanie ma jakiegkolwiek rozwiązanie to ma ich conajmniej 4.\newline \newline
Dowód: \newline
Przedstawmy liczbę $n$ jak $n = pq$, takie że $p,q$ są względnie pierwsze (formalnie $\gcd(p,q) = 1$). Skorzystajmy teraz z Twierdzenia o nieresztach kwatradowych czyli z faktu, że każde rozwiązanie ma $x = z^2 \mod w$ ma 0 rozwiazań lub conajmniej 2 ( jeżeli ma jedno jakiez $l$ to $-l$ też jest rozwiązaniem ) (jeżeli dla $p$ lub $q$ ma 0 to nasz pierwiastek dla $n$ nie istniałby) czyli musi mieć conajmniej 2 rozwiązania. A więc zapiszmy teraz chińskie twierdzenie o resztach:
$$x = u^2 \mod p$$
$$x = u^2 \mod q$$
Z chinśkiego twierdzenia o resztach wynika, że rozwiązań równania modulo n jest conajmniej tyle ile rozwiązań modulo $q$ razy ilość rozwiązań modulo $p$. Czyli wynika z tego, że mamy conajmniej 4 rozwiązania.
\newline \newline
Idea algorytmu(funkcją $Root(x,n)$ oznaczamy nasza maszynkę do liczenia pierwiastka dystretnego z $x \mod n$): 
\begin{itemize}
    \item 1. Wylosuj losowo $x$ z przedziału $\{1,...,n-1\}$.
    \item 2. Jeżeli $\gcd(x,n) \neq 1$ znaleźliśmy jakiś dzielnik.
    \item 3. Obliczmy $y = x^2 \mod n$ oraz $s = Root(y,n)$.
    \item 4. Jeżeli $s = x$ lub $s = -x$ to powróć do punktu 1.
    \item 5. (bez straty ogólniości założmy, że $s > x$ ( jeżeli nie to $swap(x,s)$)) Mamy teraz $s^2 = x^2 \mod n$ z czego wynika, że $(s+x)(s-x) = 0 \mod n$, co oznacza, że $\gcd(s+x,n)$ lub $\gcd(s-x,n)$ jest nietrywialnym dzielnikiem. 
\end{itemize}

Ogólnie to idee widać dosyć dobrze. Jedyne teraz zobaczmy czemu mamy sensowne prawdobodobieństwo przejścia punktu 4.\newline Zauważmy, że $x$ losujemy, a algorytmowi podajemy już $x^2$. Więc algorytm nie wie który z pierwiastków my wylosowaliśmy więc sam odpowiada którymś. A jest są conajmniej 4 różne pierwiastki, a my nie możemy dostać dwóch z nich, co oznacza, że mamy szansę conajmniej $\frac{1}{2}$, że nie pokryjemy się z odpowiedzią naszej czarnej skrzynki. 


\subsection{Zdefiniować problemy Discrete-Log i Diffie-Helman, ich miejsce w klasach złożoności, opisać protokół Diffiego-Helmana}
Definicja problemu Discrete-Log:\newline
Mamy dowolną grupę cykliczną $G$ oraz element $g \in G$ będący generatorem wtedy:\newline \newline

Mając na wejściu $a \in G$ znajdź taki $x$, że $g^x = a$.
\newline \newline
Discrete-Log $\in NP$:\newline
Zgadnij $x$ oraz sprawdź czy $g^x = a$, jeżeli tak to zwróć $x$.
\newline \newline
Niestety więcej o tym problemie nie wiemy. W kryptografi zakładamy, że Discrete-Log $\notin P$ oraz Discrete-Log $\notin BPP$ ale tego nie wiemy! Nie wiemy też czy jest to problem trudny w klasie $NP$.
\newline \newline
Definicja problemu Diffie-Helman:
Mamy dowolną grupę cykliczną $G$ oraz element $g \in G$ będący generatorem wtedy:\newline \newline

Mając na wejsciu $g^x,g^y$ ($x,y$ nie jest podane) znajdź $g^{xy}$.
\newline \newline
Diffie-Hellman $\in NP$:\newline
Zgadnij $x$, sprawdź czy $g^x$ równa się temu z wejścia równa się temu z wejścia. Wykonaj teraz $(g^{y})^x = g^{xy}$ bo znamy $x$ i zwróc $g^{xy}$.
\newline \newline
Niestety w tym przypadku też nie jesteśmy w stanie powiedzieć wiecej na temat należenia tego problemu do innych klas, które nas interesują. Natomias możemy stwierdzić, że za pomocą Discrete-Log możemy rozwiązać Diffie-Helmana (poprostu obliczmy $x$ i postępujemy tak jak w dowodzie dla $NP$).

%\newline \newline
Protokół Diffiego-Helmana:\newline
Jest to protokół symetryczny czyli wyślemy sobie nawzajem klucze publiczne i stworzymy na podstawie go nasz klucz symetryczny. Klucz prywatny to $a$ oraz dla drugiej osoby $b$. Do publicznej wiadomości dajemy $g^a$ oraz druga osoba $g^b$. Naszym kluczem symetrycznym będzie $g^{ab}$. Jak widać jest on dosyć podobny do El-Gammal.

\subsection{Podać definicję krzywej eliptycznej i grupy z nią związanej}
\subsubsection{Definicja Krzywa eliptyczna:}
To zbiór rozwiązań w pewnym ciele $\mathbb{F}$ (punktów $(x,y)$) równania:
$$y^2 = x^3 + ax + b$$
(Jest to krzywa w postaci Weierstrassa i każdą krzywą można do takiej postaci sprowadzić rówanie ogólne wygląda dziko czyli tak: $y^2 + a_1xy + a_3y = x^3 + a_2x^2 + a_4x + a_6$).
\newline \newline
My chcemy aby nasz krzywa była "gładka" i nie miała "ostrza" czyli formanie wyznacznik krzywej $\delta = 4a^3 + 27b^2$ musi być różny od 0.

\subsubsection{Definicja grupy:}
Jeśli $P = (x, y)$ leży na krzywej eliptycznej, to $(x, -y)$ też i oznaczmy go przez $-P$. Jeśli $P$ i $Q$ są punktami na krzywej eliptycznej, to prosta $PQ$ musi
(prawie zawsze!) przeciąć krzywą w jeszcze jednym punkcie $R$.
Definiujemy sumę punktów $P + Q$ jako $-R$ (uwaga, tam jest minus przed $R$!). Aby obsłużyć przypadek $Q = -P$, dodajemy zatem do krzywej jeszcze sztuczny punkt $O$, leżący "w nieskończoności" i
definiujemy $P + (-P) = O$ (a także $P + O = P$). Aby z kolei obliczyć sumę $P + P$, rysujemy styczną do krzywej w
punkcie $P$, znajdujemy jej punkt przecięcia $Q$ z krzywą, i bierzemy $P + P = -Q$, chyba że nie ma takiego $Q$ to zwracamy $O$. (Ogólnie jak nie ma jakiegoś punktu to mówmy w skrócie $O$, za wyjątkiem $P+O$)

\subsubsection{Dzikie wzory:}
Mając $P = (x_P, y_P)$ oraz $Q = (x_Q, y_Q)$, to możemy obliczć $P + Q = S = (x_S , y_S )$ ze wzorów:
$$x_S = \lambda^2 - x_P - x_Q$$
$$y_S = -y_P - \lambda(x_S - x_P)$$

gdzie $\lambda = \frac{3x_{P}^2 +a }{2y_p}$ jeżeli $P = Q$, w przeciwnym wypadku $\lambda = \frac{y_Q - y_P}{x_Q - x_P}$
