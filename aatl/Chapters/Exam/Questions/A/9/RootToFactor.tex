Twierdzenie 1.\newline
Mamy liczbę $n$ złożoną oraz $n$ nie jest potęgą liczby pierwszej (możemy też założyć że $n$ jest nieparzyste). Wtedy dla dowolnych dla dowolnego $u$ rówanie:
$$x = u^2 \mod n$$
Jeżeli to równanie ma jakiegkolwiek rozwiązanie to ma ich conajmniej 4.\newline \newline
Dowód: \newline
Przedstawmy liczbę $n$ jak $n = pq$, takie że $p,q$ są względnie pierwsze (formalnie $\gcd(p,q) = 1$). Skorzystajmy teraz z Twierdzenia o nieresztach kwatradowych czyli z faktu, że każde rozwiązanie ma $x = z^2 \mod w$ ma 0 rozwiazań lub conajmniej 2 ( jeżeli ma jedno jakiez $l$ to $-l$ też jest rozwiązaniem ) (jeżeli dla $p$ lub $q$ ma 0 to nasz pierwiastek dla $n$ nie istniałby) czyli musi mieć conajmniej 2 rozwiązania. A więc zapiszmy teraz chińskie twierdzenie o resztach:
$$x = u^2 \mod p$$
$$x = u^2 \mod q$$
Z chinśkiego twierdzenia o resztach wynika, że rozwiązań równania modulo n jest conajmniej tyle ile rozwiązań modulo $q$ razy ilość rozwiązań modulo $p$. Czyli wynika z tego, że mamy conajmniej 4 rozwiązania.
\newline \newline
Idea algorytmu(funkcją $Root(x,n)$ oznaczamy nasza maszynkę do liczenia pierwiastka dystretnego z $x \mod n$): 
\begin{itemize}
    \item 1. Wylosuj losowo $x$ z przedziału $\{1,...,n-1\}$.
    \item 2. Jeżeli $\gcd(x,n) \neq 1$ znaleźliśmy jakiś dzielnik.
    \item 3. Obliczmy $y = x^2 \mod n$ oraz $s = Root(y,n)$.
    \item 4. Jeżeli $s = x$ lub $s = -x$ to powróć do punktu 1.
    \item 5. (bez straty ogólniości założmy, że $s > x$ ( jeżeli nie to $swap(x,s)$)) Mamy teraz $s^2 = x^2 \mod n$ z czego wynika, że $(s+x)(s-x) = 0 \mod n$, co oznacza, że $\gcd(s+x,n)$ lub $\gcd(s-x,n)$ jest nietrywialnym dzielnikiem. 
\end{itemize}

Ogólnie to idee widać dosyć dobrze. Jedyne teraz zobaczmy czemu mamy sensowne prawdobodobieństwo przejścia punktu 4.\newline Zauważmy, że $x$ losujemy, a algorytmowi podajemy już $x^2$. Więc algorytm nie wie który z pierwiastków my wylosowaliśmy więc sam odpowiada którymś. A jest są conajmniej 4 różne pierwiastki, a my nie możemy dostać dwóch z nich, co oznacza, że mamy szansę conajmniej $\frac{1}{2}$, że nie pokryjemy się z odpowiedzią naszej czarnej skrzynki. 