\section{Algorytm Karatsuby}

\section{Algorytm Tooma-Cooka}

Mamy do pomnożenia liczby A i B.

Zapisujemy \( A = A_0 + A_1 \cdot K + A_2 \cdot K^2 \) oraz \( B = B_0 + B_1 \cdot K + B_2 \cdot K^2 \) (gdzie \( K = \frac{n}{3} \) co podejrzanie wygląda jak wielomiany drugiego stopnia.

Możemy zatem potraktować te liczby jako wielomiany i wymnożyć \( C(X) = A(X) \cdot B(X) \), a następnie wyliczyć \( C(K) \).

\( C \) jest wielomianem czwartego stopnia, więc jak wyliczymy wartości w pięciu punktach to będziemy w stanie wyliczyć z układu równań jego współczynniki.

Ewaluujemy zatem:
\begin{enumerate}
    \item \( C(-2) = A(-2) \cdot B(-2) \)
    \item \( C(-1) \)
    \item \( C(0) = A(0) \cdot B(0) = C_0 \)
    \item \( C(1) = A(1) \cdot B(1) = C_0 + C_1 + \dots + C_4 \)
    \item \( C(2) = C_0 + 2 C_1 + 4 C_2 + 8 C_3 + 16 C_4 \)
\end{enumerate}

Wartości \( C(-2), \dots, C(2) \) wyliczamy rekurencyjnie, a potem to już układ równań (nie przejmując się tym, że na ANach nam nie było wolno).

\section{Algorytm Euklidesa}