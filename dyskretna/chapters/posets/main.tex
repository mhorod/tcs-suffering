Dla przypomnienia z MFI, posetem nazywamy parę \((X, \preceq)\), gdzie \(\preceq\  \subset X \times X\)
jest relacją zwrotną, przechodnią i antysymetryczną.
Przez \(\mathbb B_n\) oznaczamy poset \((\mathcal P([n]), \subseteq)\).

\section{Twierdzenie Dilwortha}
 \begin{theorem}[Twierdzenie Dilwortha]
      Jeśli długość maksymalnego antyłańcucha w posecie wynosi $k$, poset ten da się pokryć w całości z użyciem $k$ łańcuchów.
    \end{theorem}

    \begin{proof}
        Robimy indukcję po liczbie elementów posetu; gdy poset \(P\) składa się z jednego elementu, twierdzenie zachodzi w trywialny sposób. W dalszej części dowodu, pisząc \(P\) będziemy mieli na myśli zbiór, na którym zdefiniowany jest nasz poset (bo formalne poset to tupla).
        
        Załóżmy teraz, że mamy poset zdefiniowany na \(n\) elementach. Wiemy, że jego antyłańcuch maksymalny ma długość \(k\). Antyłańcuchów maksymalny spełniający ten warunek nie musi być jeden; oznaczymy zbiór wszystkich antyłańcuchów maksymalnych w tym posecie jako \(A\).
        
        Zdefiniujmy teraz (dla danego antyłańcucha maksymalnego \( \alpha \in A\) zbiory \(U_{\alpha}\) i \(D_{\alpha}\), które będziemy określać odpowiednio jako \textit{upset} i \textit{downset} antyłańcucha \(\alpha\). Do zbioru \(U_{\alpha}\) należą wszystkie elementy \(P\), takie że są (ostro) większe od jakiegokolwiek elementu z \(\alpha\). Do zbioru \(D_{\alpha}\) należą zaś wszystkie elementy \(P\), takie że są mniejsze (ostro) od jakiegokolwiek elementu z \(\alpha\). Bardziej formalnie:
        \begin{equation*}
            U_{\alpha} = \{\,x \in P \mid \exists_{y \in \alpha} \;  y \leq x \} \setminus \alpha
        \end{equation*}
        \begin{equation*}
            D_{\alpha} = \{\,x \in P \mid \exists_{y \in \alpha} \; y \geq x \} \setminus \alpha
        \end{equation*}

        Pierwsza obserwacja którą należy wykonać, to taka że dla dowolnego \(\alpha \in A\) jest tak, że  \(U_{\alpha} \cup D_{\alpha} \cup \alpha = P\). Jest to oczywiste; jeśli istniałby jakiś element z \(P\) który nie należałby ani do downsetu \(\alpha\), ani do upsetu \(\alpha\), ani do antyłańcucha maksymalnego \(\alpha\), to z faktu że nie należy ani do downsetu ani do upsetu wynikałoby, że musiałby należeć do antyłańcucha maksymalnego \(\alpha\) (bo nie jest porównywalny z żadnym jego elementem).
        
        Druga obserwacja: dla dowolnego \(\alpha \in A\) nie istnieje element, który należy jednocześnie do \(U_{\alpha}\) i \(D_{\alpha}\). Aby dowieść tę obserwację, załóżmy nie wprost, że istnieją jakieś \(x, y, z \in P\) takie, że:
        \begin{enumerate}
            \item \(y, z \in \alpha\)
            \item \( x \geq y\)
            \item \( x \leq z\) 
        \end{enumerate}
        
        Wówczas otrzymujemy, że \(y \leq x \leq z\), a więc z tranzytywności w posetach dostalibyśmy, że \(y \leq z\). To prowadziłoby do sprzeczności, bo założyliśmy że \(y, z \in \alpha\), a więc znajdują się w jednym antyłańcuchu (i nie mogą być porównywalne).

        Udowodnimy teraz szybki lemacik.
        
        \begin{lemma}
            \label{dilworth-lemma-1}
            Następujące warunki są równoważne:
            
            \begin{enumerate}
                \item Antyłańcuch maksymalny \(\alpha\) jest taki, że \(D_{\alpha} = \emptyset\);
                \item Antyłańcuch maksymalny \(\alpha\) składa się \textbf{jedynie} ze wszystkich elementów minimalnych posetu \(P\). 
            \end{enumerate}
        \end{lemma}
        \begin{proof}
            \begin{enumerate}
                \item \( (1) \implies (2) \); stosujemy dowód nie wprost. Załóżmy, że istnieje taki antyłańcuch maksymalny \(\alpha\), że \(D_{\alpha} = \emptyset\), ale do \(\alpha\) nie należy jakiś element minimalny z \(P\)\footnote{Należy również uważać na to, że nie możemy w tym dowodzie założyć \textit{czegokolwiek} o elementach w \(\alpha\) -- w szczególności \textit{a priori} możliwe jest, że \(\alpha\) zawiera jakieś elementy które nie są minimalne w \(P\).}. Nazwijmy go \(x\). Rozważmy zbiór \( \alpha' = \alpha \cup \set{x}\). Jeśli \(\alpha'\) jest antyłańcuchem, to znaczy że \(\alpha\) nie był antyłańcuchem maksymalnym i otrzymujemy sprzeczność z założeniami. Jeśli \(\alpha'\) nie jest antyłańcuchem, to oznacza że istnieje jakieś \(y \in \alpha\) takie, że \( y \leq x\) lub \(y \geq x\). 
                
                Nie może być tak, że \(y \leq x\), bo \(x\) jest elementem minimalnym w \(P\). Jeśli \(y \geq x\), to z kolei mamy, że \(x \in D_{\alpha}\), skąd otrzymujemy sprzeczność.
                
                Należy tutaj dodać, że ten dowód nie wprost pokazał jedynie, że \(\alpha\) w takim przypadku zawiera wszystkie elementy minimalne z \(P\), ale nie pokazaliśmy że \textit{nie należą do niego} inne elementy z \(P\). Na szczęście, wiedząc że wszystkie elementy minimalne z \(P\) znajdują się w \(\alpha\), wiemy że jakikolwiek inny element nie może się tam znaleźć (bo skoro nie jest minimalny to jest porównywalny z jakimś minimum, a więc nie należy do antyłańcucha). To już kończy dowód.
                
                \item \( (2) \implies (1) \); element minimalny to taki, że nie istnieje element który byłby od niego mniejszy. Z definicji \(D_{\alpha}\) musi zatem być tak, że \(D_{\alpha} = \emptyset\). 
            \end{enumerate}
        \end{proof}
        
        Niemal identycznym dowodem można posłużyć się, by dowieść następujący lemat:
        
        \begin{lemma}
            \label{dilworth-lemma-2}
            Następujące warunki są równoważne:
            
            \begin{enumerate}
                \item Antyłańcuch maksymalny \(\alpha\) jest taki, że \(U_{\alpha} = \emptyset\);
                \item Antyłańcuch maksymalny \(\alpha\) składa się \textbf{jedynie} ze wszystkich elementów maksymalnych posetu \(P\). 
            \end{enumerate}
        \end{lemma}

        Teraz musimy rozpatrzyć trzy przypadki:
        \begin{enumerate}
            \item \( \exists_{\alpha \in A} \; U_{\alpha} = D_{\alpha} = \emptyset \) \\ 
            W tym przypadku istnieje antyłańcuch maksymalny, którego upset i downset są puste. Nietrudno pokazać, że jest to jedyny antyłańcuch maksymalny (ale to nie ma znaczenia dla dowodu). Co ma znaczenie dla dowodu to to, że wystarczy z każdego elementu tego antyłańcucha utworzyć jednoelementowy łańcuch zawierający tylko siebie samego. Jako że \(\alpha\) ma \(k\) elementów, dostajemy podział \(P\) na \(k\) antyłańcuchów.
            
            \item \( \exists_{\alpha \in A} \; U_{\alpha} \not = \emptyset \wedge D_{\alpha} \not = \emptyset\) \\
            Rozpatruję sobie posety na zbiorach \(B = A \cup U_{\alpha}\) i \(C = A \cup D_{\alpha}\). Jako, że \(U_{\alpha} \not = \emptyset\) i \(D_{\alpha} \not = \emptyset\), to z pewnością \( |B| < |P|\) i \( |C| < |P|\). W takim razie, \(B\) i \(C\) z założenia indukcyjnego da się podzielić na \(k\) łańcuchów.
            
            Ponadto, każdy element \(\alpha\) (zarówno w pokryciu łańcuchowym zbioru \(B\), jak i \(C\)) należy do łańcucha innego niż jakikolwiek inny element \(\alpha\), jako że 2 elementy z jednego antyłańcucha nie mogą znaleźć się w jednym łańcuchu. W dodatku, z definicji zbiorów \(U_{\alpha}\) i \(D_{\alpha}\) bezpośrednio wynika, że każdy element \(\alpha\) jest elementem najmniejszym w odpowiednim łańcuchu z pokrycia łańcuchowego zbioru \(B\), i elementem największym w odpowiednim łańcuchu ze zbioru \(C\). W takim razie po prostu ,,sklejam'' łańcuchy z \(B\) i \(C\) w danym elemencie z \(\alpha\) i mam poprawne pokrycie łańcuchowe całego posetu \(P\).
            
            \item Przypadek przeciwny do dwóch wcześniejszych; \(\forall_{\alpha \in A } \; U_{\alpha} = \emptyset \hspace{5pt} \mathbf{ALBO} \hspace{5pt} D_{\alpha} = \emptyset\) \\

            Korzystając z lematów \ref{dilworth-lemma-1} i \ref{dilworth-lemma-2}, wiemy, że dla każdego \(\alpha \in A\) jest tak, że składa się (jedynie) ze wszystkich elementów maksymalnych lub ze wszystkich elementów minimalnych w \(P\). Nawiasem mówiąc, to bezpośrednio implikuje, że w tym przypadku \( |A| \leq 2 \), ale nie jest to specjalnie ważne. 
            
            Weźmy z \(P\) takie \(x, y\), że \(x\) jest elementem maksymalnym, a \(y\) jest elementem minimalnym w \(P\), przy czym chcemy, by te dwa elementy były porównywalne (tzn. \( x \geq y\)). Taka para dwóch elementów szczęśliwie zawsze istnieje -- jeśli w \(A\) istnieje antyłańcuch bez downsetu, to parę tę stanowi dowolne maksimum i jego świadek bycia w upsecie; analogicznie w dualnym przypadku.  Rozważmy więc poset \(P' = P \setminus \set{x,y} \). Zauważmy, że:
            \begin{itemize}
                \item \( |P'| = |P| - 2\), co pozwala nam zastosować założenie indukcyjne;
                \item jako, że każde \( \alpha \in A\) zawierało zbiór wszystkich elementów maksymalnych lub minimalnych \(P\), wiemy że długość \textbf{wszystkich} antyłańcuchów maksymalnych zmniejszyła się o 1, a więc długość najdłuższego antyłańcucha wynosi \(k-1\).
            \end{itemize}
            To oznacza, że z założenia indukcyjnego \( P' \) możemy podzielić na \(k-1\) łańcuchów. Tak więc dodając do \(P'\) łańcuch \( \{x, y\} \) otrzymujemy podział \(P\) na \(k\) łańcuchów, a to kończy dowód.
            
    \end{enumerate}
        
     \end{proof}

\section{Twierdzenie dualne do Dilwortha}
  \begin{theorem}[Twierdzenie dualne do twierdzenia Dilwortha]
      Jeśli długość maksymalnego łańcucha w posecie wynosi k, poset ten da się pokryć w całości z użyciem k antyłańcuchów.
    \end{theorem}

    \begin{proof}
        Zdefiniujmy sobie poset $P$ i funkcję $\varphi$ idącą z elementów $P$ w liczby naturalne, taką że $\varphi(x)$ jest to moc najdłuższego łańcucha w $P$, którego maksimum wynosi $x$. Zauważmy, że $\varphi$ może przyjmować jedynie wartości w zakresie $1 \dots k$, bo $k$ to długość najdłuższego łanćucha w ogóle. Zauważamy, że wszystkie elementy $P$ które przechodzą na jakąś liczbę $m$ muszą być ze sobą nieporównywalne, a więc formować antyłańcuch. Gdyby tak nie było i istniałyby jakieś elementy $x, y$, takie że, bez straty ogólności, $x \leq y$ i $\varphi(x) = \varphi(y) = z$, to łańcuch w którym $x$ jest elementem maksymalnym i który ma długość $z$ mógłbym ,,rozszerzyć'' dodając do niego $y$, które stałoby się nowym elementem maksymalnym; tym samym maksymalna długość łańcucha w którym $y$ byłoby elementem maksymalnym wynosiłaby nie $z$, a $z+1$, co prowadziłoby do sprzeczności. W takim razie dla każdej liczby naturalnej w zakresie $1 \dots k$ mam jakiś antyłańcuch i wiem, że te antyłańcuchy w sumie muszą pokrywać cały poset $P$, co kończy dowód.
     \end{proof}

\section[Lemat Erdősa-Szekeresa o podciągach monotonicznych]{Lemat Erdősa-Szekeresa o podciągach\\monotonicznych}
\begin{theorem}[Lemat Erdősa-Szekeresa o podciągach monotonicznych]
	W ciągu składającym się z \(n \cdot m + 1\) liczb naturalnych (\(n,m \leq 1\)) znajduje
	się podciąg niemalejący długości co najmniej \(n + 1\) lub nierosnący długości co najmniej
	\(m + 1\).
\end{theorem}

\begin{proof}
	Zdefiniujmy sobie porządek częściowy na elementach ciągu. Mówimy, że \(a \preceq b\),
	gdy \(b\) występuje później niż \(a\) w ciągu oraz \(a \geq b\). Zauważmy, że łańcuch w
	tak zdefiniowanym posecie jest podciągiem nierosnącym naszego ciągu, zaś antyłańcuch
	musi być podciągiem niemalejącym.
	Z twierdzenia dualnego do twierdzenia Dilwortha wnioskujemy, że w dowolnym posecie
	zachodzi \(\posetwidth(P) \cdot \posetheight(P) \geq \card{P}\).
	Prowadzi to nas już zasadniczo do tezy, którą możemy teraz dowieść nie wprost:
	załóżmy, że istnieje taki ciąg długości \(n \cdot m + 1\), w którym każdy podciąg
	niemalejący ma długość maksymalnie \(n\), a nierosnący ma długość maksymalnie \(m\).
	Oznacza to, że najdłuższy łańcuch w naszym wcześniej zdefiniowanym
	posecie ma długość \(m\), a antyłańcuch długość \(n\), skąd otrzymujemy że
	\(n \cdot m \geq n \cdot m + 1\), co prowadzi nas do sprzeczności.
\end{proof}



\section{Nierówność LYM}
    \begin{theorem}[Nierówność LYM (Lubella, Yamamoto, Meshalkina)]
       Przez $f_k$ oznaczmy liczbę elementów $k$-elementowych w danym antyłańcuchu w kracie $B_n$. Mamy wówczas:
       \begin{equation}
           \sum_{k=0}^{n} \frac{f_k}{\binom{n}{k}} \leq 1 
       \end{equation}
    \end{theorem}

    \begin{proof}
       Zdefiniujemy teraz dziwną strukturę, ale obiecuję że ona ma sens. Mamy jakiś antyłańcuch $F$ i definiujemy zbiór $C$, który dla każdego elementu z antyłańcucha $C$ trzyma wszystkie możliwe pary tego elementu i łańcucha maksymalnego w kracie zbiorów, do którego należy dany element antyłańcucha $F$. Jeden element z antyłańcucha w zbiorze $C$ może występować w różnych parach (np. jeśli przechodzi przez niego $k$ łańcuchów maksymalnych w kracie zbiorów to wystąpi on $k$ razy). 
       
       Ciekawszą obserwacją natomiast jest, że jeden łańcuch maksymalny w zbiorze $C$ może wystąpić maksymalnie raz. Wynika to z faktu, że gdyby wystąpił dwukrotnie, to znaczyłoby że dwa różne elementy z $F$ należą do jednego łańcucha, a więc są ze sobą porównywalne; prowadziłoby to do sprzeczności, bo z założenia te dwa elementy mają ze sobą nie być porównywalne, jako że należą do jednego antyłańcucha. 

       Warto zauważyć, że liczba łańcuchów maksymalnych w kracie zbiorów \(B_n\) wynosi $n!$. Jest to dosyć prosta do udowodnienia obserwacja -- bierzemy sobie najwyższy element; elementów o mocy od niego mniejszej o 1 (tzn. takich że są ,,poziom niżej'') zawierających się w nim jest dokładnie $n$ (bo na $n$ sposobów mogę ,,wyrzucić'' z niego jakiś element tak, by powstał jego podzbiór mający moc mniejszą od niego o 1). Gdy mam zbiór mający $n-1$ elementów, mogę uzyskać podzbiór na $n-2$ elementowy na \(n-1\) sposobów, i tak dalej. ,,Wyrzucanie'' elementów w ten sposób aż nie otrzymamy zbioru pustego tworzy zawsze jakiś łańcuch maksymalny; oczywistym jest, że łańcuchy te są różne jak i to, że w ten sposób otrzymujemy wszystkie możliwe łańcuchy maksymalne. 

       Oznacza to, że elementów w zbiorze $C$ jest maksymalnie $n!$, bo każdy łańcuch maksymalny występuje maksymalnie raz. Jednocześnie okazuje się, że możemy zliczyć ile dokładnie razy w zbiorze $C$ występuje dany element z antyłańcucha $F$. Weźmy sobie element $x \in F$; zauważmy, że przechodzi przez niego dokładnie $|x|! \cdot (n - |x|)!$ łańcuchów maksymalnych. Żeby to pokazać, stosujemy tę samą obserwację co powyżej, ale jakby ,,zawężamy'' kratę zbiorów do tego elementu, tzn. bierzemy taki jej podzbiór że mamy inną kratę zbiorów, gdzie $x$ jest ,,na górze'', a więc łańcuchów które idą ,,od dołu'' do $x$ jest $x!$. Podobne rozumowanie stosujemy by pokazać, że łańcuchów idących od $x$ ,,na górę'' jest $(n - |x|)!$, a więc wszystkich łańcuchów maksymalnych idących przez $x$ jest dokładnie $|x|! \cdot (n - |x|)!$. 
       Mamy więc, że
       \begin{equation*}
            \sum_{x \in F} |x|! \cdot (n - |x|)! = |C| \leq |n!|
       \end{equation*}
       A zatem 
       \begin{equation*}
            \sum_{x \in F} |x|! \cdot (n - |x|)! \leq |n!|
       \end{equation*}
        \begin{equation*}
            \sum_{x \in F} \frac{|x|! \cdot (n - |x|)!}{n!} \leq 1
       \end{equation*}
       \begin{equation*}
            \sum_{x \in F} \frac{1}{\binom{n}{|x|}} \leq 1
       \end{equation*}
       Teraz żeby było ładniej mówimy, że $f_k$ to jest liczba elementów $F$ takich, że mają moc równą dokładnie $k$, skąd otrzymujemy już 
        \begin{equation*}
           \sum_{k=0}^{n} \frac{f_k}{\binom{n}{k}} \leq 1 
       \end{equation*}
     \end{proof}

\section{Twierdzenie Spernera}
\begin{theorem}[Twierdzenie Spernera]
	Najdłuższy antyłańcuch \(\mathcal D\) w kracie zbiorów \(\mathbb B_n\) ma moc \(\binom{n}{\lfloor\frac{n}{2}\rfloor} = \binom{n}{\lceil\frac{n}{2}\rceil}\).
\end{theorem}

\begin{proof}[Dowód przez nierówność LYM]
	Przedstawimy dowody dla \(\binom{n}{\lfloor\frac n2\rfloor}\) -- dla sufitu są one analogiczne.
	Jesteśmy w stanie wskazać antyłańcuch takiej długości -- \(\binom{[n]}{\lfloor\frac{n}{2}\rfloor}\).
	Wystaczy pokazać więc, że nie istnieje dłuższy antyłańcuch.
	Niech \(\mathcal{D}\) będzie antyłańcuchem, wtedy:
	\begin{align*}
		\forall_{0 \leq k \leq n}: \binom{n}{k}                       & \leq \binom{n}{\lfloor\frac{n}{2}\rfloor}                            & \implies \text{(tr. obserwacja)} \\
		1 \geq \sum_{X \in \mathcal{D}} \frac{1}{\binom{n}{\card{X}}} & \geq \frac{\card{\mathcal{D}}}{\binom{n}{\lfloor\frac{n}{2}\rfloor}} & \implies \text{(nier. LYM)}      \\
		\binom{n}{\lfloor\frac{n}{2}\rfloor}                          & \geq \card{\mathcal{D}}                                              & \text{(mnożenie stronami)}
	\end{align*}
\end{proof}

\begin{definition}[Łańcuchy symetryczne]
	Łańcuch \(C\) w \(\mathbb{B}_n\) nazywamy symetrycznym, jeśli \(C=\{X_k, X_{k+1},
	\ldots, X_{n-k}\}\), gdzie \(X_k \subset X_{k+1} \subset \ldots \subset
	X_{n-k}\) oraz \(|X_i|=i\) dla pewnego \(k\). Taki łańcuch narysowany na kracie
	jest symetryczny względem środkowego poziomu.
\end{definition}

\begin{proof}[Dowód tw. Spernera przez łańcuchy symetryczne]
	Rozważmy podział \(\mathbb B_n\) na łańcuchy symetryczne. Każdy taki łańcuch
	zawiera dokładnie jeden element ze środkowego poziomu (rozmiaru  \(\lfloor \frac
	n2 \rfloor\)), a więc podział ma \(\binom{n}{\lfloor \frac n2 \rfloor}\)
	elementów. Z twierdzenia Dilwortha antyłańcuch nie może mieć więcej niż tyle
	elementów.
\end{proof}
Niestety powyższy dowód nie jest jeszcze kompletny, bo nie wiemy jeszcze czy taki
podział na łańcuchy symetryczne wogóle istnieje. Na szczęście właśnie to pokażemy

\begin{theorem}[Podział na łańcuchy symetryczne, rekurencyjnie]
	Dla każdej kraty boolowskiej \(\mathbb B_n\) istnieje jej podział na łańcuchy symetryczne.
\end{theorem}
\begin{proof}
	\(\mathbb B_0\) ma jeden element, on sam jest symetrycznym łańcuchem. Mając
	podział \(\mathbb B_n\) na symetryczne łańcuchy \(\mathcal C\), konstruujemy
	podział \(\mathbb B_{n+1}\): dla \(C = \{X_k,\ldots, X_{n-k}\}\in\mathcal C\)
	łańcuchami w \(\mathbb B_{n+1}\) są \(C' = \{X_k,X_{k+1},\ldots,
	X_{n-k}\cup\{n+1\}\}\) oraz \(C'' = \{X_k\cup\{n+1\},
	x_{k+1}\cup\{n+1\},\ldots, X_{n-k-1}\cup\{n+1\}\}\). Te łańcuchy są symetryczne
	w \(\mathbb B_{n+1}\) (pierwszy to zbiory ze środkowych poziomów mające od \(k\)
	do \(n+1-k\) elementów, a drugi od \(k+1\) do \(n+1-k-1\)), a stworzenie takich
	łańcuchów dla wszystkich \(C\in\mathcal C\) daje nam podział.
\end{proof}



\section{Nawiasowania i liczby Dedekinda}
Powrócimy na chwilę do wprowadzonych w poprzednim dziale łańcuchów symetrycznych:
\begin{theorem}[Podział na łańcuchy symetryczne, przez nawiasowania]
	Dla każdej kraty boolowskiej \(\mathbb B_n\) istnieje jej podział na łańcuchy symetryczne.
\end{theorem}
\begin{proof}[Dowód]
	Niech \(A \subseteq [n]\) -- będziemy utożsamiać \(A\) z ciągiem \(n\) nawiasów, gdzie
	\(i\)-ty nawias jest zamykający wtedy i tylko wtedy, gdy gdy \(i \in A\). Innymi słowy
	bierzemy funkcję charakterystyczną \(\chi_A: [n] \to \set{0, 1}\) i traktujemy
	ją jako ciąg gdzie \(0\) to \texttt{(}, a \(1\) to \texttt{)}\footnote{Jeżeli nie jesteście pewni,
		skąd wynika taka interpretacja, to warto sobie przypomnieć, że w XX wieku, gdy
		powstawała kombinatoryka o wiele łatwiej było o środki psychoaktywne}. Weźmy teraz owy ciąg i ,,sparujmy''
	wszystkie możliwe nawiasy -- tj. wybieramy sobie wszystkie spójne podciągi,
	gdzie każdy nawias otwierający ma odpowiadający mu nawias zamykający i wice-wersa.
	Niech \(M_A\) będzie zbiorem wszystkich sparowanych nawiasów, \(I_A = M_A \cap A\)
	(tj. zbiór sparowanych nawiasów zamykających), a \(F_A = [n] \setminus M_A\) (tj. zbiór niesparowanych elementów).
	Niech \(F_A = \set{x_1, x_2, \ldots, x_k}\), gdzie \(x_1 < x_2 < \ldots < x_k\).
	Definiujemy teraz łańcuch \(\mathcal C\): jego kolejne elementy to \(C_0 =
	I_A, C_1 = I_A\cup\set{x_1}, \ldots, C_k = I_A\cup\set{x_1,\ldots,x_k} = C_{k-1} \cup \set{x_k}\).
	Mamy \(\card{C_0} = \frac{\card{M_A}}2\) oraz \(\card{C_k} = \frac{\card{M_A}}2 +
	\card{F_A}\), więc \(\card{C_0}+\card{C_k} = \card{M_A}+\card{F_A} = n\) i \(\mathcal C\) jest symetryczny.

	Zauważmy również, że dla dowolnego zbioru z \(\mathcal C\), patrząc na elementy \((x_1,\ldots,x_k)\)
	w nawiasowaniu, dostaniemy ciąg najpierw zamkniętych nawiasów, a potem otwartych -- wynika to z prostej analizy definicji.
	Zauważmy, że zbiór \(A\) jednoznacznie wyznacza \(M_A\), a co za tym idzie funkcja \(M = \lambda A \to M_A\) definiuje
	rozkład \(\mathbb B_n\) na klasy abstrakcji\footnote{Własności relacji równoważności przenoszą się z równości zbiorów}.
	Ciekawszym jest fakt, że klasa, do której należy \(A\) to właśnie skonstruowany przez nas łańcuch \(\mathcal C\)
	-- mając ustalone \(M_A\) mamy dowolność tylko na elementach \(F_A\),
	w \(C_0\) one wszystkie są otwarte, po kolei domykamy kolejne, dodając kolejne elementy \(F_A\)
	do zbiorów tworzących łańcuch -- łańcuch nie zepsuje się z racji wcześniejszej obserwacji
	o postaci \(F_A\) jako ciąg postaci \texttt{))...))((...((}.
	Udowodniliśmy więc, że podział na klasy wyznaczony przez operator \(M\) dzieli \(\mathbb B_n\) na symetryczne łańcuchy.
\end{proof}
Powyższy dowód jest alternatywną konstrukcją podziału na łańcuchy symetryczne, który możemy wykorzystać
w dowodzie twierdzenia Spernera.

\begin{definition}
	Liczba Dedekinda \(D_n\) to liczba antyłańcuchów w \(\mathbb B_n\).
\end{definition}

\begin{theorem}[Ograniczenie na liczby Dedekinda]
	Liczba Dedekinda \(D_n\) ograniczona jest nierównościami
	\[2^{\binom{n}{\floor{\frac n2}}} \le D_n \le 3^{\binom{n}{\floor{\frac n2}}}.\]
\end{theorem}
\begin{proof}
	Dolne ograniczenie wynika z tego, że największy antyłańcuch ma
	\(\binom{n}{\floor{\frac n2}}\) elementów, a dowolny jego podzbiór jest antyłańcuchem.

	Pozostało wykazać ograniczenie górne.
	Wykorzystamy obserwację, że antyłańcuchy można utożsamiać z ich stożkami dolnymi
	(zbiorami elementów, które są mniejsze lub równe elementom antyłańcucha) --
	antyłańcuch zadaje swój stożek dolny i można go odzyskać biorąc
	elementy maksymalne. Natomiast stożki dolne można utożsamiać z monotonicznymi
	funkcjami \(\mathbb B_n \to \set{0,1}\), które są dopełnieniami funkcji
	charakterystycznych tych zbiorów. Będziemy zliczać funkcje monotoniczne.

	Rozważmy podział na łańcuchy symetryczne \(\mathcal C\) zadany przez konstrukcję
	z nawiasowaniem. Każdy łańcuch \(C\in\mathcal C\) ma ustalone elementy
	sparowane, a zmieniają się elementy niesparowane. Dla \(\set{A_0,\ldots,
		A_k}\in\mathcal C\), gdzie \(A_0\subset\ldots\subset A_k\) i \(k\ge 2\), w zbiorze
	\(A_i\) \(i\)-ty niesparowany nawias jest ostatnim domkniętym. Dla \(0<i<k\)
	istnieje za nim nawias otwarty. Możemy obrócić te nawiasy i sparować je,
	tworząc zbiór \(B_i\), który należy do pewnego krótszego łańcucha w \(\mathcal
	C\) (są w nim dwa nowe sparowane nawiasy). Zauważmy, że \(A_{i-1}\subset B_i
	\subset A_{i+1}\) (do \(A_{i-1}\) nie należą oba elementy, które obróciliśmy,
	tworząc \(B_i\), a do \(A_{i+1}\) należą).

	Będziemy definiować funkcję monotoniczną \(f\), zaczynając od najkrótszych
	łańcuchów w \(\mathcal C\). Na tych długości co najwyżej \(2\) (istnieją, bo
	środkowe poziomy są większe od nieśrodkowych, więc łańcuchy zawierające coś z
	nieśrodkowych poziomów nie pokryją środkowych) mamy maksymalnie \(3\) opcje
	(oba elementy dostają tą samą lub różne wartości). Rozważmy łańcuch
	\(\set{A_0,\ldots, A_k}\) i zbiór \(\set{B_1,\ldots,B_{k-1}}\) zbiorów
	otrzymanych z elementów łańcucha przez opisane wyżej przekształcenie. Funkcja
	\(f\) jest już na nich zdefiniowana. Jeśli \(f(B_1)=1\), to \(f(A_2)=1\) z
	monotoniczności i pozostaje nam wybór wartości na dwóch zbiorach z
	rozważanego łańcucha. Jeśli \(f(B_{k-1})=0\), to \(f(A_{k-2})=0\) z
	monotoniczności i ponownie pozostaje nam wybór wartości na dwóch zbiorach.
	Jeśli \(f(B_1)=0\) i \(f(B_{k-1})=1\), to istnieje takie \(i\in [k-2]\), że
	\(f(B_i)=0\) i \(f(B_{i+1})=1\). Wtedy z monotoniczności \(f(A_{i-1})=0\) i
	\(f(A_{i+1})=1\), więc również zostały nam do wybrania dwie wartości. Zatem dla
	każdego z \(\binom{n}{\floor{\frac n2}}\) łańcuchów mamy możliwość dokonania co
	najwyżej\footnote{Podczas konstrukcji może okazać się, że próbując ustalić dany łańcuch
		sprzeczne ze sobą ograniczenia -- może się tak zdażyć jeżeli istnieje kilka miejsc gdzie \(B_i = 0, B_{i+1} = 1\).
		W takiej sytuacji wiemy, że taka funkcja nie istnieje, ale nie psuje to ograniczenia górnego.} \(3\) wyborów, czyli razem mamy \(3^{\binom{n}{\floor{\frac n2}}}\) możliwości.
\end{proof}



\section{Cienie i twierdzenie Erdősa-Ko-Rado}
\begin{definition}
	Dla zbioru \(\mathcal B \subset \displaystyle\binom{[n]}{k}\) jego cieniem
	dolnym nazywamy zbiór \[\Delta\mathcal B = \set{A : \exists_{B\in\mathcal B,
				x\in B} \ A=B\setminus\set{x}},\] a cieniem górnym nazywamy zbiór
	\[\nabla\mathcal B = \set{A : \exists_{B\in\mathcal B, x\in [n]\setminus B} \
			A=B\cup\set{x}}.\] Elementy cienia odpowiednio tracą lub zyskują jeden element
	-- cień jest obcięciem stożka do najbliższego poziomu.
\end{definition}

\begin{theorem}[O rozmiarze cienia]
	Dla \(\mathcal B \subset \displaystyle\binom{[n]}{k}\) zachodzą następujące nierówności:
	\begin{align*}
		\card{\Delta\mathcal B} & \ge \frac k{n-k+1}\card{\mathcal B},   \\
		\card{\nabla\mathcal B} & \ge \frac {n-k}{k+1}\card{\mathcal B}. \\
	\end{align*}
	Z powyższego twierdzenia wynika, że \(\card{\Delta\mathcal B} \ge \card{\mathcal B}\) dla \(k \ge
	\frac {n+1}2\) oraz \(\card{\nabla\mathcal B} \ge \card{\mathcal B}\) dla \(k \le \frac {n-1}2\).
\end{theorem}
\begin{proof}
	Zliczamy moc zbioru
	\(W = \set{(A,B) : B\in\mathcal B, A\in\Delta\mathcal B, A\subset B}\).
	Jest ona równa \(k\card{\mathcal B}\), bo każdy element \(\mathcal B\) ma dokładnie \(k\)
	swoich elementów cienia. Jednocześnie każdy element cienia może mieć co najwyżej
	\(n-(k-1)\) swoich nadzbiorów w \(\mathcal B\), więc \(\card W \le \card{\Delta\mathcal B}(n-k+1)\),
	co dowodzi pierwszej nierówności.
	Analogiczne zliczenie dla górnego cienia (każdy element \(\mathcal B\) ma \(n-k\)
	swoich elementów cienia, element cienia ma co najwyżej \(k+1\) elementów
	\(\mathcal B\)) daje drugą nierówność.
\end{proof}

\begin{theorem}[Twierdzenie Spernera, bis]
	Największy antyłańcuch w \(\mathbb B_n\) ma rozmiar \(\binom{n}{\lfloor \frac n2
		\rfloor}\).
\end{theorem}
\begin{proof}[Dowód przez cienie]
	Niech \(\mathcal A\) będzie antyłańcuchem w \(\mathbb B_n\) i niech \(\mathcal A_j
	= \mathcal A \cap \binom{[n]}{j}\). Jeśli \(i = \min\set{j : \mathcal
		A_j\ne\emptyset}\), to dla \(i\le \frac{n-1}2\) zbiór \(\mathcal A' = (\mathcal
	A\setminus \mathcal A_i)\cup \ \nabla\mathcal A_i\) ma większą moc od
	\(\mathcal{A}\) oraz dalej jest antyłańcuchem -- jeśli coś jest nad cieniem
	górnym \(\mathcal A_i\), to jest też nad \(\mathcal A_i\), więc \(\mathcal A\) nie
	byłby antyłańcuchem. Podobnie, jeśli weźmiemy \(k = \max\set{j : \mathcal
		A_j\ne\emptyset}\) i będzie \(k\ge \frac{n+1}2\). Możemy więc po kolei przesuwać
	kolejne poziomy bliżej środka kraty. Jeśli \(2\nmid n\), to możemy wybrać
	dowolny ze środkowych poziomów, bo nierówności z cieniami na to pozwalają.
\end{proof}

\begin{definition}
	Rodzina zbiorów \(\mathcal F\) jest przecinająca się, jeśli \(\forall_{X, Y\in \mathcal F} \ X \cap Y \neq \emptyset\).
\end{definition}

\begin{theorem}
	Największa rodzina przecinająca się w \(\mathbb B_n\) ma rozmiar \(2^{n-1}\).
\end{theorem}
\begin{proof}
	Zauważmy, że dla rodziny przecinającej \(\mathcal F\) nie może jednocześnie
	zachodzić \(X\in\mathcal F\) i \(\overline X\in\mathcal F\). Zatem jest
	\(\card{\mathcal F}\le 2^{n-1}\). Przykładem takiej rodziny mogą być wszystkie
	podzbiory \(\mathbb B_n\) zawierające \(1\).
\end{proof}

\begin{theorem}[Erd\H{o}s-Ko-Rado]
	Niech \(\mathcal F\subseteq \binom{[n]}{k}\) będzie przecinająca się i niech
	\(2k\le n\). Maksymalny rozmiar \(\mathcal F\) to \(\binom{n-1}{k-1}\).
\end{theorem}
\begin{proof}
	Najpierw zauważmy, że dla \(2k>n\) można wziąć \(\mathcal F = \binom{[n]}{k}\), bo
	wszystkie takie zbiory muszą się przecinać.

	Faktyczny dowód zaczniemy, rozważając cykl \(\sigma\) elementów
	\([n]\) (tj. permutację o jednym cyklu). Przedziałem \(k\)-elementowym w \(\sigma\)
	nazwiemy ciąg \(k\) elementów występujących kolejno w \(\sigma\), być może zapętlając
	się modulo \(n\). Pokażemy, że do \(\mathcal F\) może należeć co najwyżej \(k\)
	przedziałów dla każdego takiego cyklu \(\sigma\).
	Załóżmy, że \(X=\set{x_1,\ldots,x_k}\in\mathcal
	F\) jest przedziałem w \(\sigma.\) Zauważmy, że pary przedziałów, z których
	jeden ma prawy koniec w \(x_i\), a drugi ma lewy koniec w \(x_{i+1}\) dla \(i\in
	[k]\) są jedynymi przedziałami, które mogą należeć do \(\mathcal F\) i co
	najwyżej jeden z każdej pary należy do \(\mathcal F\) (bo muszą się wzajemnie
	przecinać i przecinać \(X\), a warunek \(2k\le n\) zapewnia, że
	nie przetną się ,,z drugiej strony''). Zatem zbiór
	\(W = \set{(X,\sigma) : X\in\mathcal F, \sigma \text{ cyklem w $[n]$ }, X \text{ przedziałem w
			$\sigma$}}\) ma co najwyżej \(k(n-1)!\) elementów (po \(k\) na każdy cykl, a cyklów jest \((n-1)!\)).
	Jednocześnie każdy zbiór z \(\mathcal F\) można dopełnić do cyklu, stawiając go
	na początku cyklu i permutując jego elementy i pozostałe elementy, co daje nam
	\(\card W = \card{\mathcal F}k!(n-k)!\), zatem \(\card{\mathcal F} \le
	\frac{k(n-1)!}{k!(n-k)!} = \binom{n-1}{k-1}\).

	Aby znaleźć rodzinę spełniającą to ograniczenie, można wziąć wszystkie elementy z
	\(\displaystyle\binom{[n-1]}{k-1}\) z dorzuconym elementem \(n\).
\end{proof}



\section{Twierdzenie Kruskala-Katony/Lovása}
\begin{theorem}[\(k\)-kaskadowa reprezentacja liczb naturalnych]
	Niech \(m,k\in\natural_1\). Istnieją takie liczby \(a_k > a_{k-1} > \ldots > a_s \ge s
	\ge 1\), że
	\[m = \binom{a_k}{k} + \binom{a_{k-1}}{k-1} + \ldots + \binom{a_s}{s},\]
	a ponadto taka reprezentacja jest jedyna.
\end{theorem}
\begin{proof}
	Istnienie dowodzimy indukując się po \((k,m)\), dla \(k=1\) mamy \(m =
	\binom{m}{1}\), dla \(m=1\) mamy \(m = \binom{k}{k}\). W kroku indukcyjnym niech
	\(a_k = \max\set{a : \binom{a}{k}\le m}\), mamy \(m = \binom{a_k}{k} + m'\), a
	\(m'\) z indukcji ma \((k-1)\)-kaskadową reprezentację (lub jest równe \(0\), co
	kończy konstrukcję), w której jest \(a_{k-1} < a_k\), bo inaczej \(m \ge
	\binom{a_k}{k} + \binom{a_k}{k-1} = \binom{a_k+1}{k}\) wbrew definicji \(a_k\).

	Załózmy nie wprost, że taka reprezentacja nie jest jedyna, a \(m\) jest
	minimalnym przykładem tego. Wtedy \(m = \binom{a_k}{k} + \ldots +
	\binom{a_s}{s} = \binom{a'_k}{k} + \ldots + \binom{a'_{s'}}{s'}\) i \(a_k\ne
	a'_k\) (inaczej można odjąć te same czynniki i otrzymać mniejszy
	kontrprzykład). Bez straty ogólności \(a_k > a'_k\). Wtedy jednak
	\(\binom{a'_k}{k} + \ldots + \binom{a'_{s'}}{s'} \le \binom{a_k-1}{k} +
	\binom{a_k-2}{k-1} + \ldots + \binom{a_k-k}{1} < \binom{a_k}{k} \le m\), co
	daje sprzeczność (druga nierówność wynika z tożsamości dwumianów
	\(\sum_{i=0}^{k}\binom{n-1+i}{i} = \binom{n+k}{k}\)).
\end{proof}

\begin{definition}[Porządek ,,colex'']
	Na zbiorze \(\binom{\natural}{k}\) definiujemy porządek koleksykograficzny: dla
	\(A,B \in \binom{\natural}{k}\) jest \(A <_{\text{col}} B\) wtedy i tylko wtedy, gdy
	\(\max(A\div B)\in B\). Oznacza to, że o porządku ,,colex'' decyduje ostatni (największy)
	różniący się element (a nie jak w porządku leksykograficznym najmniejszy, stąd nazwa).
\end{definition}

\begin{theorem}[Twierdzenie Kruskala-Katony]
	Niech \(\mathcal F\subset\binom{\natural}{k}\) i \(\card{\mathcal F} = m =
	\binom{a_k}{k} + \binom{a_{k-1}}{k-1} + \ldots + \binom{a_s}{s}\). Wtedy
	\[\card{\Delta\mathcal F} \ge \binom{a_k}{k-1} + \binom{a_{k-1}}{k-2} +
		\ldots + \binom{a_s}{s-1}.\]
	Co więcej, takie ograniczenie jest najlepsze możliwe.
\end{theorem}
\begin{proof}
	Najpierw pokażemy, że istnieje rodzina spełniająca to ograniczenie. Weźmy
	rodzinę \(\mathcal C(m,k)\) pierwszych \(m\) elementów z \(\binom{\natural}{k}\) w
	porządku koleksykograficznym. Mając zadaną \(k\)-kaskadową reprezentację \(m\)
	widzimy, że \(\mathcal C(m,k)\) składa się z \(\binom{[a_k]}{k}\), zbiorów
	powstałych przez dodanie \(\set{a_k+1}\) do \(\binom{[a_{k-1}]}{k-1}\), dodanie
	\(\set{a_k+1, a_{k-1}+1}\) do \(\binom{a_{k-2}}{k-2}\) i tak dalej, aż do zbiorów
	powstałych przez dodanie \(\set{a_k+1,\ldots,a_{s+1}+1}\) do \(\binom{[a_s]}{s}\)
	-- bierzemy tyle ile się da na najmniejszym możliwym zbiorze, potem zostają
	nam zbiory, w których jest liczba o jeden większa i rekurencyjnie bierzemy
	mniejsze zbiory. Cień takiej rodziny składa się z \(\binom{[a_k]}{k-1}\),
	zbiorów powstałych przez dodanie \(\set{a_k+1}\) do \(\binom{[a_{k-1}]}{k-2}\),
	dodanie \(\set{a_k+1, a_{k-1}+1}\) do \(\binom{a_{k-2}}{k-3}\) i tak dalej, aż do
	zbiorów powstałych przez dodanie \(\set{a_k+1,\ldots,a_{s+1}+1}\) do
	\(\binom{[a_s]}{s-1}\) -- biorąc cień kolejnych z tych zbiorów usunięcie
	któregoś z wyróżnionych elementów da nam jeden z otrzymanych wcześniej
	zbiorów, wszystkie inne dadzą coś nowego. To daje nam poszukiwaną wielkość
	cienia.

	Pokazanie, że osiągnięta wartość jest faktycznie najmniejsza, przebiega
	identycznie jak dowód twierdzenia Lov\'asza (który zaraz pokażemy),
	z tym, że trzeba wielokrotnie stosować rekurencyjny wzór na współczynniki dwumianowe.
\end{proof}

\begin{definition} Niech \(\mathcal F \subset \binom{\natural}{k}\) dla pewnego \(k \geq 1\) oraz ustalmy \(i \geq 2\). Operator przesunięcia \(S_i\) tworzy nową rodzinę \(S_i(\mathcal F) = \{S_i(F) : F \in \mathcal F\}\), gdzie
	\[ S_i(F) =
		\begin{cases}
			F \setminus \{i\} \cup \{1\} & \text{jeśli } i \in F, 1 \notin F \text{ oraz } F \setminus \{i\} \cup \{1\} \notin\mathcal F, \\
			F                            & \text{w przeciwnym przypadku}.
		\end{cases}
		.\]
	Jeśli \(S_i(F)=F\) z powodu istnienia już przesuniętego zbioru w rodzinie, to mówimy, że \(F\) został zablokowany.
\end{definition}

\begin{lemma}
	\label{kk_rozmiar}
	Dla każdego skończonego \(\mathcal F\subset \binom{\natural}{k}\) i \(i\ge 2\) jest
	\(|S_i(\mathcal F)| = |\mathcal F|\).
\end{lemma}
\begin{proof}
	Różne zbiory są przesuwane w różne zbiory, a zbiór nie zostanie przesunięty,
	jeśli jego przesunięcie już jest w rodzinie.
\end{proof}

\begin{lemma}
	\label{kk_zamiana}
	Dla dowolnego skończonego \(\mathcal F \subset \binom{\natural}{k}\) i dowolnego \(i
	\geq 2\) jest \(\Delta S_i(\mathcal F) \subseteq S_i(\Delta\mathcal F)\).
\end{lemma}
\begin{proof}
	Dowód wymaga rozważenia czterech przypadków. Przypuśćmy, że \(E \in \Delta
	S_i(\mathcal F)\), więc \(E = S_i(F) \setminus \{x\}\) dla pewnego \(F \in
	\mathcal F\) i \(x \in S_i(F)\).

	Najpierw załóżmy, że \(1, i \notin S_i(F)\). Ponieważ \(1 \notin S_i(F)\), musimy
	mieć \(S_i(F) = F\), a zatem \(E \subset F\). Zatem \(E \in \Delta\mathcal  F\), a
	ponieważ \(i \notin E\), to \(S_i(E) = E\). W związku z tym \(E \in
	S_i(\Delta\mathcal F)\).

	Teraz przypuśćmy, że \(1, i \in S_i(F)\). Ponieważ \(i \in S_i(F)\), mamy \(S_i(F)
	= F\), a zatem \(E \in \Delta\mathcal F\), jak wcześniej. Jeśli \(x \neq 1\), to
	\(1 \in E\), i zatem \(E = S_i(E) \in S_i(\Delta\mathcal F)\). Jeśli \(x = 1\), to
	\(E' = E \setminus \{i\} \cup \{1\} \subset F\), a zatem \(E' \in \Delta\mathcal
	F\). To oznacza, że \(E\) jest zablokowane i \(S_i(E) = E\), co implikuje \(E \in
	S_i(\Delta\mathcal F)\).

	W trzecim przypadku przypuśćmy, że \(S_i(F) \cap \{1, i\} = \{i\}\). Ponieważ \(i
	\in S_i(F)\), musimy mieć \(S_i(F) = F\). Jednakże, jako że \(i \in F\) i \(1
	\notin F\), \(F\) musiało być zablokowane przez \(F' = F \setminus \{i\} \cup
	\{1\} \in\mathcal F\). Ponieważ \(E \subset S_i(F) = F\), \(E \in \Delta\mathcal
	F\). Jeśli \(x = i\), to \(i \notin E\), i zatem \(E = S_i(E) \in
	S_i(\Delta\mathcal F)\). Jeśli \(x \neq i\), to \(E\) byłoby zablokowane przez \(E'
	= F' \setminus \{x\} \in \Delta\mathcal F\), i zatem \(E = S_i(E) \in
	S_i(\Delta\mathcal F)\) również w tym przypadku.

	Ostatni przypadek to gdy \(S_i(F) \cap \{1, i\} = \{1\}\). Zauważmy, że \(i
	\notin E\) i zatem \(S_i(E) = E\). W związku z tym, jeśli \(E \in \Delta\mathcal
	F\), to \(E = S_i(E) \in S_i(\Delta\mathcal F)\). Jeśli \(F\) nie przesunął się,
	to \(F = S_i(F)\) i \(E \in \Delta\mathcal F\). Jeśli \(F\) przesunął się, to
	\(S_i(F) = F \setminus \{i\} \cup \{1\}\). Jeśli \(x = 1\), to \(E \subset F\) i
	zatem jak wcześniej \(E \in \Delta\mathcal F\). Jeśli \(x \neq 1\), niech \(E' = E
	\setminus \{1\} \cup \{i\}\) i zauważmy, że \(E' \subset F\), i zatem \(E' \in
	\Delta\mathcal F\). Wtedy albo \(E \in \Delta\mathcal F\), albo \(E'\) nie jest
	zablokowane przed przesunięciem, i \(E = S_i(E') \in S_i(\Delta\mathcal F)\).
	To kończy analizę przypadków.
\end{proof}

\begin{definition}
	Rodzinę \(\mathcal F \subset \binom{\natural}{k}\) nazywamy stabilną, jeśli
	\(S_i(\mathcal F) = \mathcal F\) dla każdego \(i\ge 2\).
\end{definition}

\begin{lemma}
	\label{kk_stabilna}
	Dla każdej skończonej rodziny \(\mathcal F\subset \binom{\natural}{k}\) istnieje
	rodzina stabilna \(\mathcal G\subset\binom{\natural}{k}\) taka, że \(\card{\mathcal G}
	= \card{\mathcal F}\) i \(\card{\Delta \mathcal G} \le \card{\Delta \mathcal F}\).
\end{lemma}
\begin{proof}
	Dla stabilnej \(\mathcal F\) można wziąć \(\mathcal G = \mathcal F\), a inaczej
	można wziąć \(\mathcal F' = S_i(\mathcal F) \ne \mathcal F\) dla pewnego \(i\ge
	2\) -- Lematy \ref{kk_rozmiar} i \ref{kk_zamiana} dają pożądane wielkości
	odpowiednich zbiorów. Możemy w ten sposób przesuwać rodzinę, póki się da. Ten
	proces się zakończy, bo każde przesunięcie zwiększa liczbę zbiorów
	zawierających \(1\).
\end{proof}

\begin{lemma}
	\label{kk_zawieranie}
	Dla każdej stabilnej rodziny \(\mathcal F\subset \binom{\natural}{k}\) zachodzi
	\(\Delta\mathcal F_0 \subseteq \mathcal F'_1\), gdzie \(\mathcal F = \mathcal
	F_0 \sqcup \mathcal F_1\) i \(\mathcal F_0 = \{F \in \mathcal F : 1 \notin F\}\)
	oraz \(\mathcal F_1 = \{F \in \mathcal F : 1 \in F\}\) i \(\mathcal F_1' = \{F
	\setminus \{1\} : F \in \mathcal F_1\}\).
\end{lemma}
\begin{proof}
	Przypuśćmy, że \(E \in \Delta\mathcal F_0\). Wówczas musimy mieć \(E = F
	\setminus \{x\}\) dla pewnego \(F \in\mathcal F_0\) oraz \(x \in F\). Ponieważ \(F
	\in\mathcal F_0\), \(x \geq 2\). Ponieważ \(\mathcal F\) jest stabilna, to
	\(S_x(\mathcal F) =\mathcal F\), a zatem \(S_x(F) = F\). To oznacza, że \(F\) był
	zablokowany, więc \(F' = F \setminus \{x\} \cup \{1\} \in \mathcal F\) i w
	szczególności jest w \(\mathcal F_1\). Zatem \(E = (F \setminus \{x\} \cup
	\{1\}) \setminus \{1\} \in \mathcal F_1'\).
\end{proof}

\begin{lemma}
	\label{kk_suma}
	Dla każdej stabilnej rodziny \(\mathcal F\subset \binom{\natural}{k}\) zachodzi
	\(\card{\Delta\mathcal F} = \card{\mathcal F'_1} + \card{\Delta\mathcal
		F'_1}\), gdzie \(\mathcal F = \mathcal F_0 \sqcup \mathcal F_1\) i \(\mathcal F_0
	= \{F \in \mathcal F : 1 \notin F\}\) oraz \(\mathcal F_1 = \{F \in \mathcal F
	: 1 \in F\}\) i \(\mathcal F_1' = \{F \setminus \{1\} : F \in \mathcal F_1\}\).
\end{lemma}
\begin{proof}
	Oczywiście mamy \(\Delta\mathcal F = \Delta\mathcal F_0 \cup \Delta\mathcal
	F_1\). W Lemacie \ref{kk_zawieranie} pokazaliśmy, że \(\Delta\mathcal F_0
	\subseteq \mathcal F_1'\). Niech \(\mathcal F'' = \set{F\cup\set{1} :
		F\in\Delta \mathcal F'_1}\) Pokażemy, że \(\Delta\mathcal F_1 = \mathcal
	F'_1\cup \mathcal F''\). Te dwa zbiory są rozłączne (elementy tylko jednego
	zawierają \(1\)), a w pierwszym z nich zawiera się \(\mathcal F_0\), więc da nam
	to żądaną równość.

	To, że \(\mathcal F_1' \subseteq \Delta\mathcal F_1\), wynika z jego definicji,
	ponieważ dla każdego \(F' \in \mathcal F_1'\) mamy \(F' = F \setminus \{1\}\) dla
	pewnego \(F \in \mathcal F_1\). Usunięcięcie elementu i dodanie \(1\) do elementu
	\(\mathcal F'_1\) (przy definiowaniu \(\mathcal F''\)) można zrobić w odwrotnej
	kolejności, więc \(\mathcal F''\subseteq \Delta\mathcal F_1\). Jednocześnie w
	tych dwóch zbiorach znajdują się wszystkie elementy cienia \(\mathcal F_1\) --
	jedne z nich powstają przez usunięcie \(1\), a drugie przez usunięcie
	czegokolwiek innego. To dowodzi zawierania w drugą stronę i kończy dowód.
\end{proof}

\begin{theorem}[Lov\'{a}sz]
	Niech \(\mathcal F\subset\binom{\natural}{k}\) i \(\card{\mathcal F} = m =
	\binom{x}{k}\), gdzie \(x\in\real\).
	(dla przypomnienia, definiujemy \(\binom{x}{k} = \frac{x^{\underline{k}}}{k!}\)).
	Wtedy \[\card{\Delta\mathcal F} \ge \binom{x}{k-1}.\]
\end{theorem}
\begin{proof}
	Przeprowadzimy indukcję po \((k,m)\). Dla \(k=1\) cień zawiera zbiór pusty i
	wymagamy od niego rozmiaru \(1\). Dla \(m=1 = \binom{k}{k}\) cień składa się z \(k
	= \binom{k}{k-1}\) elementów. Dalej zakładamy, że \(k,m\ge 2\). Z Lematu
	\ref{kk_stabilna} możemy założyć, że \(\mathcal F\) jest stabilna. Niech
	\(\mathcal F = \mathcal F_0 \sqcup \mathcal F_1\) i \(\mathcal F_0 = \{F \in
	\mathcal F : 1 \notin F\}\) oraz \(\mathcal F_1 = \{F \in \mathcal F : 1 \in
	F\}\) i \(\mathcal F_1' = \{F \setminus \{1\} : F \in \mathcal F_1\}\).
	Pokażemy, że \(\card{\mathcal F'_1} \ge \binom{x-1}{k-1}\).

	Załóżmy, że tak nie jest. Mamy \(m=\card{\mathcal F} = \card{\mathcal F_0} +
	\card{\mathcal F_1}\) oraz \(\card{\mathcal F'_1} = \card{\mathcal F_1}\), zatem
	\(\card{\mathcal F_0} > \binom{x}{k} - \binom{x-1}{k-1} = \binom{x-1}{k}\). Dla
	stabilnej rodziny \(\mathcal F\) rodzina \(\mathcal F_1\) jest niepusta i
	\(\card{\mathcal F_0} < m\), więc z indukcji i Lematu \ref{kk_zawieranie} jest
	\(\card{\mathcal F'_1} \ge \card{\Delta\mathcal F_0} \ge \binom{x-1}{k-1}\), co
	daje sprzeczność z założeniem nie wprost.

	Z indukcji mamy teraz \(\card{\Delta\mathcal F'_1} \ge \binom{x-1}{k-2}\). Z
	Lematu \ref{kk_suma} mamy więc \(\card{\Delta\mathcal F} = \card{\mathcal
		F'_1} + \card{\Delta\mathcal F'_1} \ge \binom{x-1}{k-1} + \binom{x-1}{k-2} =
	\binom{x}{k-1}\), co kończy dowód.
\end{proof}






