Dla przypomnienia z MFI, posetem nazywamy parę \((X, \preceq)\), gdzie \(\preceq\  \subset X \times X\)
jest relacją zwrotną, przechodnią i antysymetryczną.
Przez \(\mathbb B_n\) oznaczamy poset \((\mathcal P([n]), \subseteq)\).

\section{Twierdzenie Dilwortha}
\begin{theorem}[Twierdzenie Dilwortha]
	Jeśli długość maksymalnego antyłańcucha w posecie wynosi k, poset ten da się pokryć w całości z użyciem k łańcuchów.
\end{theorem}

\begin{proof}
	Robimy indukcję po liczbie elementów posetu; gdy poset $P$ składa się z jednego elementu twierdzenie zachodzi w trywialny sposób. Załóżmy teraz, że mamy poset składający się z $n$ elementów i o najdłuższym antyłańcuchu długości $k$; antyłańcuch ten nazwiemy $A$. Zdefiniujmy teraz zbiory $U$ i $D$, które będziemy określać jako \textit{upset} i \textit{downset}. Do zbioru $U$ należą wszystkie elementy $P$, takie że są większe od jakiegokolwiek elementu z $A$. Do zbioru $D$ należą zaś wszystkie elementy $P$, takie że są mniejsze od jakiegokolwiek elementu z $A$. Bardziej formalnie:
	\begin{equation*}
		U = \{\,x \in P \mid \exists_{y \in A}  y \leq x \} \setminus A
	\end{equation*}
	\begin{equation*}
		D = \{\,x \in P \mid \exists_{y \in A} y \geq x \} \setminus A
	\end{equation*}

	Pierwsza obserwacja którą należy wykonać, to taka że $U \cup D \cup A = P$. Jest to oczywiste; jeśli istniałby jakiś element z $L$ który nie należałby ani do downsetu ani do upsetu ani do antyłańcucha maksymalnego, to z faktu że nie należy ani do downsetu ani do upsetu wynikałoby, że musiałby należeć do antyłańcucha maksymalnego (bo nie jest porównywalny z żadnym elementem z antyłańcucha maksymalnego).

	Druga obserwacja: nie istnieje element, który należy jednocześnie do upsetu i downsetu. Załóżmy nie wprost, że tak jest: mamy jakieś $x, y, z$ takie, że $x \geq y$,  $x \leq z$, i $y,z \in A$. Wówczas otrzymujemy że $y \leq x \leq z$, a więc z tranzytywności $P$ mamy że $y \leq z$, ale $y, z$ są nieporównywalne bo są w jednym łańcuchu. Otrzymana sprzeczność dowodzi obserwację.

	Teraz musimy rozpatrzyć trzy przypadki:
	\begin{enumerate}
		\item $U = D = \emptyset$ \\
		      Bardzo fajny przypadek, głównie dlatego że trywialny do udowodnienia; każdy element z antyłańcucha tworzy jednoelementowy łańcuch zawierający tylko siebie samego, mamy podział $P$ na $k$ łańcuchów.
		\item $U = \emptyset \hspace{5pt} \mathbf{ALBO} \hspace{5pt} D = \emptyset$ \\
		      Wyrzucam z $P$ jakiś łańcuch maksymalny (jakikolwiek). Teraz są różne przypadki: \begin{enumerate}
			      \item Nie ma już antyłańcucha maksymalnego o długości $k$, wtedy z założenia indukcyjnego $P$ da się podzielić, dorzucamy nasz łańcuch maksymalny z powrotem i mamy pokrycie takie jakie byśmy chcieli mieć,
			      \item Istnieje jakiś antyłańcuch maksymalny (inny niż nasz startowy, mogło tak być) długości $k$ i mający downset i upset niepuste; wtedy patrz przypadek trzeci,
			      \item Każdy antyłańcuch maksymalny nie ma albo downsetu albo upsetu, ale wtedy coś nie gra bo jego jeden z elementów powinien był się znaleźć w łańcuchu maksymalnym który wyleciał (bo element maksymalny łańcucha maksymalnego musiał być mniejszy niż jakiś element z antyłańcucha maksymalnego skoro był w downsecie, a więc mogłem ten element ,,dokleić'' i analogicznie w odwrotnym przypadku).
		      \end{enumerate}
		\item $U \not = \emptyset \wedge D \not = \emptyset$ \\
		      Rozpatruję sobie zbiory $B = A \cup U, C = A \cup D$. Oba z nich z założenia indukcyjnego da się podzielić na $k$ łańcuchów. Ponadto każdy element $A$ ma łańcuch różny od wszystkich innych elementów $A$ w swoim pokryciu łańcuchowym dla zbiorów $B$ i $C$ (2 elementy z jednego antyłańcucha nie mogą być w jednym łańcuchu). W takim razie po prostu ,,sklejam'' łańcuchy z $B$ i $C$ w danym elemencie $A$ i mam poprawne pokrycie łańcuchowe całego posetu $P$.

	\end{enumerate}

\end{proof}

\section{Twierdzenie dualne do Dilwortha}
\begin{theorem}[Twierdzenie dualne do twierdzenia Dilwortha]
	Długość maksymalnego łańcucha w posecie wynosi
	Długość maksymalnego łańcucha w posecie jest równa ilości antyłańcuchów w najmniejszym pokryciu antyłańcuchowym
	(tj. rozkładzie posetu na podzbiory będące antyłańcuchami).
	Długość tę dla danego posetu nazywamy szerokością i oznaczamy \(\posetheight(P)\).
\end{theorem}

\begin{proof}
	W jedną stronę nierówność jest trywialna -- mając łańcuch o długości \(k\), każdy z jego
	elementów musi trafić do innego antyłańcucha. Dowodzimy więc nierówność w drugą stronę.
	Niech \((P, \leq)\) będzie posetem -- zdefiniujmy sobie funkcję \(\varphi\) idącą z elementów
	\(P\) w liczby naturalne, taką że \(\varphi(x)\) jest to moc najdłuższego łańcucha w \(P\),
	którego maksimum wynosi \(x\). Zauważmy, że \(\varphi\) może przyjmować jedynie wartości
	w zakresie \(1 \dots k\), bo \(k\) to długość najdłuższego łanćucha w ogóle. Zauważamy,
	że wszystkie elementy \(P\) które przechodzą na jakąś liczbę \(m\) muszą być ze sobą
	nieporównywalne, a więc formować antyłańcuch. Gdyby tak nie było i istniałyby jakieś
	elementy \(x, y\), takie że, bez straty ogólności, \(x \leq y\) i
	\(\varphi(x) = \varphi(y) = z\), to łańcuch w którym \(x\) jest elementem maksymalnym
	i który ma długość \(z\) możemy ,,rozszerzyć'' dodając do niego \(y\), które stałoby się
	nowym elementem maksymalnym; tym samym maksymalna długość łańcucha w którym \(y\)
	byłoby elementem maksymalnym wynosiłaby nie \(z\), a \(z+1\), co prowadziłoby do sprzeczności.
	W takim razie dla każdej liczby naturalnej w zakresie \(1 \dots k\) mamy jakiś antyłańcuch
	i wiemy, że te antyłańcuchy w sumie muszą pokrywać cały poset \(P\), co kończy dowód.
\end{proof}


\section[Lemat Erdősa-Szekeresa o podciągach monotonicznych]{Lemat Erdősa-Szekeresa o podciągach\\monotonicznych}
\begin{theorem}[Lemat Erdősa-Szekeresa o podciągach monotonicznych]
	W ciągu składającym się z \(n \cdot m + 1\) liczb naturalnych (\(n,m \leq 1\)) znajduje
	się podciąg niemalejący długości co najmniej \(n + 1\) lub nierosnący długości co najmniej
	\(m + 1\).
\end{theorem}

\begin{proof}
	Zdefiniujmy sobie porządek częściowy na elementach ciągu. Mówimy, że \(a \preceq b\),
	gdy \(b\) występuje później niż \(a\) w ciągu oraz \(a \geq b\). Zauważmy, że łańcuch w
	tak zdefiniowanym posecie jest podciągiem nierosnącym naszego ciągu, zaś antyłańcuch
	musi być podciągiem niemalejącym.
	Z twierdzenia dualnego do twierdzenia Dilwortha wnioskujemy, że w dowolnym posecie
	zachodzi \(\posetwidth(P) \cdot \posetheight(P) \geq \card{P}\).
	Prowadzi to nas już zasadniczo do tezy, którą możemy teraz dowieść nie wprost:
	załóżmy, że istnieje taki ciąg długości \(n \cdot m + 1\), w którym każdy podciąg
	niemalejący ma długość maksymalnie \(n\), a nierosnący ma długość maksymalnie \(m\).
	Oznacza to, że najdłuższy łańcuch w naszym wcześniej zdefiniowanym
	posecie ma długość \(m\), a antyłańcuch długość \(n\), skąd otrzymujemy że
	\(n \cdot m \geq n \cdot m + 1\), co prowadzi nas do sprzeczności.
\end{proof}



\section{Nierówność LYM}
\begin{theorem}[Nierówność LYM (Lubella, Yamamoto, Meshalkina)]
	Niech \(\mathcal{D}\) będzie antyłańcuchem kraty \(\mathbb B_n\).
	Wtedy zachodzi:
	\begin{equation}
		\sum_{X \in \mathcal{D}} \frac{1}{\binom{n}{\card{X}}} \leq 1.
	\end{equation}
	Alternatywnie, jeżeli \(f_n\) to liczba zbiorów o mocy \(n\) w \(\mathcal{D}\),
	to twierdzenie można zapisać jako:
	\begin{equation}
		\sum_{i=0}^{n} \frac{f_n}{\binom{n}{k}} \leq 1.
	\end{equation}
\end{theorem}

\begin{proof}
	Dowód opierać się będzie na pewnej ,,dziwnej'' funkcji \(\nu: \mathbb B_n \to \mathcal{P}(S_n)\),
	gdzie \(S_n\) to zbiór wszystkich permutacji \(n\)-elementowych.\footnote{Dla fanów algebry jest to zbiór podkładowy grupy symetrycznej na zbiorze \([n]\).}
	Naszą funkcję definiujemy w następujący sposób:
	\[\nu(X) = \set{\pi \in S_n: \set{\pi(1), \pi(2), \pi(\card{X})} = X},\]
	czyli innymi słowy jest to zbiór wszystkich permutacji, których pierwsze \(\card{X}\)
	elementów należy do \(X\). Zauważmy teraz dwa ciekawe fakty:
	\begin{enumerate}
		\item \(\card{\nu(X)} = \card{X} \cdot (n - \card{X})\) \\
		      Fakt ten wynika z prostego zliczania -- łatwo zauważyć, że wszystkie permutacje
		      w \(\nu(X)\) są postaci \[\pi = \left(\sigma(1), \sigma(2), \ldots, \sigma_{\card{X}},
			      \rho(1), \rho(2), \ldots, \rho(n - \card{X})\right),\]
		      gdzie \(\sigma\) jest ,,permutacją'' \(X\) (tj. bijekcją z \([\card{X}]\) na \(X\)),
		      a \(\rho\) jest analogicznym ustawieniem elementów z \([n] \setminus X\). Z reguły mnożenia
		      otrzymujemy więc moc zbioru \(\nu(X)\).
		\item \(X \neq Y,\, \nu(X) \cap \nu(Y) \neq \emptyset \implies X, Y - \text{porównywalne}\) \\
		      Załóżmy BSO, że \(\card{X} \leq \card{Y}\), i niech \(\pi\) będzie elementem świadczącym
		      niepustości przecięcia, tj. \(\pi \in \nu(X)\) oraz \(\pi \in \nu(Y)\). Z definicji \(\nu\)
		      wiemy, że \(X = \set{\pi(1), \pi(2), \ldots, \pi(\card{X})}\) oraz
		      \(Y = \set{\pi(1), \pi(2), \ldots, \pi(\card{Y})}\) -- ale z tego oczywiście
		      wynika, że \(X \subset Y\). Z tej obserwacji możemy wywnioskować, że dla dowolnego
		      antyłańcucha \(\mathcal{D} \subset \mathbb B_n\) i \(X, Y \in \mathcal{D},\, X \neq Y\)
		      zachodzi \(\nu(X) \cap \nu(Y) = \emptyset\).
	\end{enumerate}
	Czyli z drugiej obserwacji wynika, że \(\bigsqcup_{X \in \mathcal{D}} \nu(X) \subset S_n\), bo każdej permutacji
	odpowiada co najwyżej jeden zbiór z antyłańcucha \(\mathcal{D}\) -- wykonamy więc kilka transformacji:
	\begin{align*}
		\bigsqcup_{X \in \mathcal{D}} \nu(X)                              & \subset S_n     & \implies \\
		\sum_{X \in \mathcal{D}} \card{\nu(X)}                            & \leq \card{S_n} & \iff     \\
		\sum_{X \in \mathcal{D}} \card{X} \cdot (n - \card{X})            & \leq n!         & \iff     \\
		\sum_{X \in \mathcal{D}} \frac{\card{X} \cdot (n - \card{X})}{n!} & \leq 1          & \iff     \\
		\sum_{X \in \mathcal{D}} \frac{1}{\binom{n}{\card{X}}}            & \leq 1
	\end{align*}
\end{proof}


\section{Twierdzenie Spernera}
\begin{theorem}[Twierdzenie Spernera]
	Najdłuższy antyłańcuch \(\mathcal D\) w kracie zbiorów \(\mathbb B_n\) ma moc \(\binom{n}{\lfloor\frac{n}{2}\rfloor} = \binom{n}{\lceil\frac{n}{2}\rceil}\).
\end{theorem}

\begin{proof}[Dowód przez nierówność LYM]
	Przedstawimy dowody dla \(\binom{n}{\lfloor\frac n2\rfloor}\) -- dla sufitu są one analogiczne.
	Jesteśmy w stanie wskazać antyłańcuch takiej długości -- \(\binom{[n]}{\lfloor\frac{n}{2}\rfloor}\).
	Wystaczy pokazać więc, że nie istnieje dłuższy antyłańcuch.
	Niech \(\mathcal{D}\) będzie antyłańcuchem, wtedy:
	\begin{align*}
		\forall_{0 \leq k \leq n}: \binom{n}{k}                       & \leq \binom{n}{\lfloor\frac{n}{2}\rfloor}                            & \implies \text{(tr. obserwacja)} \\
		1 \geq \sum_{X \in \mathcal{D}} \frac{1}{\binom{n}{\card{X}}} & \geq \frac{\card{\mathcal{D}}}{\binom{n}{\lfloor\frac{n}{2}\rfloor}} & \implies \text{(nier. LYM)}      \\
		\binom{n}{\lfloor\frac{n}{2}\rfloor}                          & \geq \card{\mathcal{D}}                                              & \text{(mnożenie stronami)}
	\end{align*}
\end{proof}

\begin{definition}[Łańcuchy symetryczne]
	Łańcuch \(C\) w \(\mathbb{B}_n\) nazywamy symetrycznym, jeśli \(C=\{X_k, X_{k+1},
	\ldots, X_{n-k}\}\), gdzie \(X_k \subset X_{k+1} \subset \ldots \subset
	X_{n-k}\) oraz \(|X_i|=i\) dla pewnego \(k\). Taki łańcuch narysowany na kracie
	jest symetryczny względem środkowego poziomu.
\end{definition}

\begin{proof}[Dowód tw. Spernera przez łańcuchy symetryczne]
	Rozważmy podział \(\mathbb B_n\) na łańcuchy symetryczne. Każdy taki łańcuch
	zawiera dokładnie jeden element ze środkowego poziomu (rozmiaru  \(\lfloor \frac
	n2 \rfloor\)), a więc podział ma \(\binom{n}{\lfloor \frac n2 \rfloor}\)
	elementów. Z twierdzenia Dilwortha antyłańcuch nie może mieć więcej niż tyle
	elementów.
\end{proof}
Niestety powyższy dowód nie jest jeszcze kompletny, bo nie wiemy jeszcze czy taki
podział na łańcuchy symetryczne wogóle istnieje. Na szczęście właśnie to pokażemy

\begin{theorem}[Podział na łańcuchy symetryczne, rekurencyjnie]
	Dla każdej kraty boolowskiej \(\mathbb B_n\) istnieje jej podział na łańcuchy symetryczne.
\end{theorem}
\begin{proof}
	\(\mathbb B_0\) ma jeden element, on sam jest symetrycznym łańcuchem. Mając
	podział \(\mathbb B_n\) na symetryczne łańcuchy \(\mathcal C\), konstruujemy
	podział \(\mathbb B_{n+1}\): dla \(C = \{X_k,\ldots, X_{n-k}\}\in\mathcal C\)
	łańcuchami w \(\mathbb B_{n+1}\) są \(C' = \{X_k,X_{k+1},\ldots,
	X_{n-k}\cup\{n+1\}\}\) oraz \(C'' = \{X_k\cup\{n+1\},
	x_{k+1}\cup\{n+1\},\ldots, X_{n-k-1}\cup\{n+1\}\}\). Te łańcuchy są symetryczne
	w \(\mathbb B_{n+1}\) (pierwszy to zbiory ze środkowych poziomów mające od \(k\)
	do \(n+1-k\) elementów, a drugi od \(k+1\) do \(n+1-k-1\)), a stworzenie takich
	łańcuchów dla wszystkich \(C\in\mathcal C\) daje nam podział.
\end{proof}



\section{Nawiasowania i liczby Dedekinda}
Powrócimy na chwilę do wprowadzonych w poprzednim dziale łańcuchów symetrycznych:
\begin{theorem}[Podział na łańcuchy symetryczne, przez nawiasowania]
	Dla każdej kraty boolowskiej \(\mathbb B_n\) istnieje jej podział na łańcuchy symetryczne.
\end{theorem}
\begin{proof}[Dowód]
	Niech \(A \subseteq [n]\) -- będziemy utożsamiać \(A\) z ciągiem \(n\) nawiasów, gdzie
	\(i\)-ty nawias jest zamykający wtedy i tylko wtedy, gdy gdy \(i \in A\). Innymi słowy
	bierzemy funkcję charakterystyczną \(\chi_A: [n] \to \set{0, 1}\) i traktujemy
	ją jako ciąg gdzie \(0\) to \texttt{(}, a \(1\) to \texttt{)}\footnote{Jeżeli nie jesteście pewni,
		skąd wynika taka interpretacja, to warto sobie przypomnieć, że w XX wieku, gdy
		powstawała kombinatoryka o wiele łatwiej było o środki psychoaktywne}. Weźmy teraz owy ciąg i ,,sparujmy''
	wszystkie możliwe nawiasy -- tj. wybieramy sobie wszystkie spójne podciągi,
	gdzie każdy nawias otwierający ma odpowiadający mu nawias zamykający i wice-wersa.
	Niech \(M_A\) będzie zbiorem wszystkich sparowanych nawiasów, \(I_A = M_A \cap A\)
	(tj. zbiór sparowanych nawiasów zamykających), a \(F_A = [n] \setminus M_A\) (tj. zbiór niesparowanych elementów).
	Niech \(F_A = \set{x_1, x_2, \ldots, x_k}\), gdzie \(x_1 < x_2 < \ldots < x_k\).
	Definiujemy teraz łańcuch \(\mathcal C\): jego kolejne elementy to \(C_0 =
	I_A, C_1 = I_A\cup\set{x_1}, \ldots, C_k = I_A\cup\set{x_1,\ldots,x_k} = C_{k-1} \cup \set{x_k}\).
	Mamy \(\card{C_0} = \frac{\card{M_A}}2\) oraz \(\card{C_k} = \frac{\card{M_A}}2 +
	\card{F_A}\), więc \(\card{C_0}+\card{C_k} = \card{M_A}+\card{F_A} = n\) i \(\mathcal C\) jest symetryczny.

	Zauważmy również, że dla dowolnego zbioru z \(\mathcal C\), patrząc na elementy \((x_1,\ldots,x_k)\)
	w nawiasowaniu, dostaniemy ciąg najpierw zamkniętych nawiasów, a potem otwartych -- wynika to z prostej analizy definicji.
	Zauważmy, że zbiór \(A\) jednoznacznie wyznacza \(M_A\), a co za tym idzie funkcja \(M = \lambda A \to M_A\) definiuje
	rozkład \(\mathbb B_n\) na klasy abstrakcji\footnote{Własności relacji równoważności przenoszą się z równości zbiorów}.
	Ciekawszym jest fakt, że klasa, do której należy \(A\) to właśnie skonstruowany przez nas łańcuch \(\mathcal C\)
	-- mając ustalone \(M_A\) mamy dowolność tylko na elementach \(F_A\),
	w \(C_0\) one wszystkie są otwarte, po kolei domykamy kolejne, dodając kolejne elementy \(F_A\)
	do zbiorów tworzących łańcuch -- łańcuch nie zepsuje się z racji wcześniejszej obserwacji
	o postaci \(F_A\) jako ciąg postaci \texttt{))...))((...((}.
	Udowodniliśmy więc, że podział na klasy wyznaczony przez operator \(M\) dzieli \(\mathbb B_n\) na symetryczne łańcuchy.
\end{proof}
Powyższy dowód jest alternatywną konstrukcją podziału na łańcuchy symetryczne, który możemy wykorzystać
w dowodzie twierdzenia Spernera.

\begin{definition}
	Liczba Dedekinda \(D_n\) to liczba antyłańcuchów w \(\mathbb B_n\).
\end{definition}

\begin{theorem}[Ograniczenie na liczby Dedekinda]
	Liczba Dedekinda \(D_n\) ograniczona jest nierównościami
	\[2^{\binom{n}{\floor{\frac n2}}} \le D_n \le 3^{\binom{n}{\floor{\frac n2}}}.\]
\end{theorem}
\begin{proof}
	Dolne ograniczenie wynika z tego, że największy antyłańcuch ma
	\(\binom{n}{\floor{\frac n2}}\) elementów, a dowolny jego podzbiór jest antyłańcuchem.

	Pozostało wykazać ograniczenie górne.
	Wykorzystamy obserwację, że antyłańcuchy można utożsamiać z ich stożkami dolnymi
	(zbiorami elementów, które są mniejsze lub równe elementom antyłańcucha) --
	antyłańcuch zadaje swój stożek dolny i można go odzyskać biorąc
	elementy maksymalne. Natomiast stożki dolne można utożsamiać z monotonicznymi
	funkcjami \(\mathbb B_n \to \set{0,1}\), które są dopełnieniami funkcji
	charakterystycznych tych zbiorów. Będziemy zliczać funkcje monotoniczne.

	Rozważmy podział na łańcuchy symetryczne \(\mathcal C\) zadany przez konstrukcję
	z nawiasowaniem. Każdy łańcuch \(C\in\mathcal C\) ma ustalone elementy
	sparowane, a zmieniają się elementy niesparowane. Dla \(\set{A_0,\ldots,
		A_k}\in\mathcal C\), gdzie \(A_0\subset\ldots\subset A_k\) i \(k\ge 2\), w zbiorze
	\(A_i\) \(i\)-ty niesparowany nawias jest ostatnim domkniętym. Dla \(0<i<k\)
	istnieje za nim nawias otwarty. Możemy obrócić te nawiasy i sparować je,
	tworząc zbiór \(B_i\), który należy do pewnego krótszego łańcucha w \(\mathcal
	C\) (są w nim dwa nowe sparowane nawiasy). Zauważmy, że \(A_{i-1}\subset B_i
	\subset A_{i+1}\) (do \(A_{i-1}\) nie należą oba elementy, które obróciliśmy,
	tworząc \(B_i\), a do \(A_{i+1}\) należą).

	Będziemy definiować funkcję monotoniczną \(f\), zaczynając od najkrótszych
	łańcuchów w \(\mathcal C\). Na tych długości co najwyżej \(2\) (istnieją, bo
	środkowe poziomy są większe od nieśrodkowych, więc łańcuchy zawierające coś z
	nieśrodkowych poziomów nie pokryją środkowych) mamy maksymalnie \(3\) opcje
	(oba elementy dostają tą samą lub różne wartości). Rozważmy łańcuch
	\(\set{A_0,\ldots, A_k}\) i zbiór \(\set{B_1,\ldots,B_{k-1}}\) zbiorów
	otrzymanych z elementów łańcucha przez opisane wyżej przekształcenie. Funkcja
	\(f\) jest już na nich zdefiniowana. Jeśli \(f(B_1)=1\), to \(f(A_2)=1\) z
	monotoniczności i pozostaje nam wybór wartości na dwóch zbiorach z
	rozważanego łańcucha. Jeśli \(f(B_{k-1})=0\), to \(f(A_{k-2})=0\) z
	monotoniczności i ponownie pozostaje nam wybór wartości na dwóch zbiorach.
	Jeśli \(f(B_1)=0\) i \(f(B_{k-1})=1\), to istnieje takie \(i\in [k-2]\), że
	\(f(B_i)=0\) i \(f(B_{i+1})=1\). Wtedy z monotoniczności \(f(A_{i-1})=0\) i
	\(f(A_{i+1})=1\), więc również zostały nam do wybrania dwie wartości. Zatem dla
	każdego z \(\binom{n}{\floor{\frac n2}}\) łańcuchów mamy możliwość dokonania co
	najwyżej\footnote{Podczas konstrukcji może okazać się, że próbując ustalić dany łańcuch
		sprzeczne ze sobą ograniczenia -- może się tak zdażyć jeżeli istnieje kilka miejsc gdzie \(B_i = 0, B_{i+1} = 1\).
		W takiej sytuacji wiemy, że taka funkcja nie istnieje, ale nie psuje to ograniczenia górnego.} \(3\) wyborów, czyli razem mamy \(3^{\binom{n}{\floor{\frac n2}}}\) możliwości.
\end{proof}



\section{Cienie i twierdzenie Erdősa-Ko-Rado}
\begin{definition}
	Dla zbioru \(\mathcal B \subset \displaystyle\binom{[n]}{k}\) jego cieniem
	dolnym nazywamy zbiór \[\Delta\mathcal B = \set{A : \exists_{B\in\mathcal B,
				x\in B} \ A=B\setminus\set{x}},\] a cieniem górnym nazywamy zbiór
	\[\nabla\mathcal B = \set{A : \exists_{B\in\mathcal B, x\in [n]\setminus B} \
			A=B\cup\set{x}}.\] Elementy cienia odpowiednio tracą lub zyskują jeden element
	-- cień jest obcięciem stożka do najbliższego poziomu.
\end{definition}

\begin{theorem}[O rozmiarze cienia]
	Dla \(\mathcal B \subset \displaystyle\binom{[n]}{k}\) zachodzą następujące nierówności:
	\begin{align*}
		\card{\Delta\mathcal B} & \ge \frac k{n-k+1}\card{\mathcal B},   \\
		\card{\nabla\mathcal B} & \ge \frac {n-k}{k+1}\card{\mathcal B}. \\
	\end{align*}
	Z powyższego twierdzenia wynika, że \(\card{\Delta\mathcal B} \ge \card{\mathcal B}\) dla \(k \ge
	\frac {n+1}2\) oraz \(\card{\nabla\mathcal B} \ge \card{\mathcal B}\) dla \(k \le \frac {n-1}2\).
\end{theorem}
\begin{proof}
	Zliczamy moc zbioru
	\(W = \set{(A,B) : B\in\mathcal B, A\in\Delta\mathcal B, A\subset B}\).
	Jest ona równa \(k\card{\mathcal B}\), bo każdy element \(\mathcal B\) ma dokładnie \(k\)
	swoich elementów cienia. Jednocześnie każdy element cienia może mieć co najwyżej
	\(n-(k-1)\) swoich nadzbiorów w \(\mathcal B\), więc \(\card W \le \card{\Delta\mathcal B}(n-k+1)\),
	co dowodzi pierwszej nierówności.
	Analogiczne zliczenie dla górnego cienia (każdy element \(\mathcal B\) ma \(n-k\)
	swoich elementów cienia, element cienia ma co najwyżej \(k+1\) elementów
	\(\mathcal B\)) daje drugą nierówność.
\end{proof}

\begin{theorem}[Twierdzenie Spernera, bis]
	Największy antyłańcuch w \(\mathbb B_n\) ma rozmiar \(\binom{n}{\lfloor \frac n2
		\rfloor}\).
\end{theorem}
\begin{proof}[Dowód przez cienie]
	Niech \(\mathcal A\) będzie antyłańcuchem w \(\mathbb B_n\) i niech \(\mathcal A_j
	= \mathcal A \cap \binom{[n]}{j}\). Jeśli \(i = \min\set{j : \mathcal
		A_j\ne\emptyset}\), to dla \(i\le \frac{n-1}2\) zbiór \(\mathcal A' = (\mathcal
	A\setminus \mathcal A_i)\cup \ \nabla\mathcal A_i\) ma większą moc od
	\(\mathcal{A}\) oraz dalej jest antyłańcuchem -- jeśli coś jest nad cieniem
	górnym \(\mathcal A_i\), to jest też nad \(\mathcal A_i\), więc \(\mathcal A\) nie
	byłby antyłańcuchem. Podobnie, jeśli weźmiemy \(k = \max\set{j : \mathcal
		A_j\ne\emptyset}\) i będzie \(k\ge \frac{n+1}2\). Możemy więc po kolei przesuwać
	kolejne poziomy bliżej środka kraty. Jeśli \(2\nmid n\), to możemy wybrać
	dowolny ze środkowych poziomów, bo nierówności z cieniami na to pozwalają.
\end{proof}

\begin{definition}
	Rodzina zbiorów \(\mathcal F\) jest przecinająca się, jeśli \(\forall_{X, Y\in \mathcal F} \ X \cap Y \neq \emptyset\).
\end{definition}

\begin{theorem}
	Największa rodzina przecinająca się w \(\mathbb B_n\) ma rozmiar \(2^{n-1}\).
\end{theorem}
\begin{proof}
	Zauważmy, że dla rodziny przecinającej \(\mathcal F\) nie może jednocześnie
	zachodzić \(X\in\mathcal F\) i \(\overline X\in\mathcal F\). Zatem jest
	\(\card{\mathcal F}\le 2^{n-1}\). Przykładem takiej rodziny mogą być wszystkie
	podzbiory \(\mathbb B_n\) zawierające \(1\).
\end{proof}

\begin{theorem}[Erd\H{o}s-Ko-Rado]
	Niech \(\mathcal F\subseteq \binom{[n]}{k}\) będzie przecinająca się i niech
	\(2k\le n\). Maksymalny rozmiar \(\mathcal F\) to \(\binom{n-1}{k-1}\).
\end{theorem}
\begin{proof}
	Najpierw zauważmy, że dla \(2k>n\) można wziąć \(\mathcal F = \binom{[n]}{k}\), bo
	wszystkie takie zbiory muszą się przecinać.

	Faktyczny dowód zaczniemy, rozważając cykl \(\sigma\) elementów
	\([n]\) (tj. permutację o jednym cyklu). Przedziałem \(k\)-elementowym w \(\sigma\)
	nazwiemy ciąg \(k\) elementów występujących kolejno w \(\sigma\), być może zapętlając
	się modulo \(n\). Pokażemy, że do \(\mathcal F\) może należeć co najwyżej \(k\)
	przedziałów dla każdego takiego cyklu \(\sigma\).
	Załóżmy, że \(X=\set{x_1,\ldots,x_k}\in\mathcal
	F\) jest przedziałem w \(\sigma.\) Zauważmy, że pary przedziałów, z których
	jeden ma prawy koniec w \(x_i\), a drugi ma lewy koniec w \(x_{i+1}\) dla \(i\in
	[k]\) są jedynymi przedziałami, które mogą należeć do \(\mathcal F\) i co
	najwyżej jeden z każdej pary należy do \(\mathcal F\) (bo muszą się wzajemnie
	przecinać i przecinać \(X\), a warunek \(2k\le n\) zapewnia, że
	nie przetną się ,,z drugiej strony''). Zatem zbiór
  \(W = \set{(X,\sigma) : X\in\mathcal F, \sigma \text{ cyklem w $[n]$ }, X \text{ przedziałem w
      $\sigma$}}\) ma co najwyżej \(k(n-1)!\) elementów (po \(k\) na każdy cykl, a cyklów jest \((n-1)!\)).
	Jednocześnie każdy zbiór z \(\mathcal F\) można dopełnić do cyklu, stawiając go
	na początku cyklu i permutując jego elementy i pozostałe elementy, co daje nam
	\(\card W = \card{\mathcal F}k!(n-k)!\), zatem \(\card{\mathcal F} \le
	\frac{k(n-1)!}{k!(n-k)!} = \binom{n-1}{k-1}\).

	Aby znaleźć rodzinę spełniającą to ograniczenie, można wziąć wszystkie elementy z
	\(\displaystyle\binom{[n-1]}{k-1}\) z dorzuconym elementem \(n\).
\end{proof}



\section{Twierdzenie Kruskala-Katony/Lovása}
\begin{theorem}[\(k\)-kaskadowa reprezentacja liczb naturalnych]
	Niech \(m,k\in\natural_1\). Istnieją takie liczby \(a_k > a_{k-1} > \ldots > a_s \ge s
	\ge 1\), że
	\[m = \binom{a_k}{k} + \binom{a_{k-1}}{k-1} + \ldots + \binom{a_s}{s},\]
	a ponadto taka reprezentacja jest jedyna.
\end{theorem}
\begin{proof}
	Istnienie dowodzimy indukując się po \((k,m)\), dla \(k=1\) mamy \(m =
	\binom{m}{1}\), dla \(m=1\) mamy \(m = \binom{k}{k}\). W kroku indukcyjnym niech
	\(a_k = \max\set{a : \binom{a}{k}\le m}\), mamy \(m = \binom{a_k}{k} + m'\), a
	\(m'\) z indukcji ma \((k-1)\)-kaskadową reprezentację (lub jest równe \(0\), co
	kończy konstrukcję), w której jest \(a_{k-1} < a_k\), bo inaczej \(m \ge
	\binom{a_k}{k} + \binom{a_k}{k-1} = \binom{a_k+1}{k}\) wbrew definicji \(a_k\).

	Załózmy nie wprost, że taka reprezentacja nie jest jedyna, a \(m\) jest
	minimalnym przykładem tego. Wtedy \(m = \binom{a_k}{k} + \ldots +
	\binom{a_s}{s} = \binom{a'_k}{k} + \ldots + \binom{a'_{s'}}{s'}\) i \(a_k\ne
	a'_k\) (inaczej można odjąć te same czynniki i otrzymać mniejszy
	kontrprzykład). Bez straty ogólności \(a_k > a'_k\). Wtedy jednak
	\(\binom{a'_k}{k} + \ldots + \binom{a'_{s'}}{s'} \le \binom{a_k-1}{k} +
	\binom{a_k-2}{k-1} + \ldots + \binom{a_k-k}{1} < \binom{a_k}{k} \le m\), co
	daje sprzeczność (druga nierówność wynika z tożsamości dwumianów
	\(\sum_{i=0}^{k}\binom{n-1+i}{i} = \binom{n+k}{k}\)).
\end{proof}

\begin{definition}[Porządek ,,colex'']
	Na zbiorze \(\binom{\natural}{k}\) definiujemy porządek koleksykograficzny: dla
	\(A,B \in \binom{\natural}{k}\) jest \(A <_{\text{col}} B\) wtedy i tylko wtedy, gdy
	\(\max(A\div B)\in B\). Oznacza to, że o porządku ,,colex'' decyduje ostatni (największy)
	różniący się element (a nie jak w porządku leksykograficznym najmniejszy, stąd nazwa).
\end{definition}

\begin{theorem}[Twierdzenie Kruskala-Katony]
	Niech \(\mathcal F\subset\binom{\natural}{k}\) i \(\card{\mathcal F} = m =
	\binom{a_k}{k} + \binom{a_{k-1}}{k-1} + \ldots + \binom{a_s}{s}\). Wtedy
	\[\card{\Delta\mathcal F} \ge \binom{a_k}{k-1} + \binom{a_{k-1}}{k-2} +
		\ldots + \binom{a_s}{s-1}.\]
	Co więcej, takie ograniczenie jest najlepsze możliwe.
\end{theorem}
\begin{proof}
	Najpierw pokażemy, że istnieje rodzina spełniająca to ograniczenie. Weźmy
	rodzinę \(\mathcal C(m,k)\) pierwszych \(m\) elementów z \(\binom{\natural}{k}\) w
	porządku koleksykograficznym. Mając zadaną \(k\)-kaskadową reprezentację \(m\)
	widzimy, że \(\mathcal C(m,k)\) składa się z \(\binom{[a_k]}{k}\), zbiorów
	powstałych przez dodanie \(\set{a_k+1}\) do \(\binom{[a_{k-1}]}{k-1}\), dodanie
	\(\set{a_k+1, a_{k-1}+1}\) do \(\binom{a_{k-2}}{k-2}\) i tak dalej, aż do zbiorów
	powstałych przez dodanie \(\set{a_k+1,\ldots,a_{s+1}+1}\) do \(\binom{[a_s]}{s}\)
	-- bierzemy tyle ile się da na najmniejszym możliwym zbiorze, potem zostają
	nam zbiory, w których jest liczba o jeden większa i rekurencyjnie bierzemy
	mniejsze zbiory. Cień takiej rodziny składa się z \(\binom{[a_k]}{k-1}\),
	zbiorów powstałych przez dodanie \(\set{a_k+1}\) do \(\binom{[a_{k-1}]}{k-2}\),
	dodanie \(\set{a_k+1, a_{k-1}+1}\) do \(\binom{a_{k-2}}{k-3}\) i tak dalej, aż do
	zbiorów powstałych przez dodanie \(\set{a_k+1,\ldots,a_{s+1}+1}\) do
	\(\binom{[a_s]}{s-1}\) -- biorąc cień kolejnych z tych zbiorów usunięcie
	któregoś z wyróżnionych elementów da nam jeden z otrzymanych wcześniej
	zbiorów, wszystkie inne dadzą coś nowego. To daje nam poszukiwaną wielkość
	cienia.

	Pokazanie, że osiągnięta wartość jest faktycznie najmniejsza, przebiega
	identycznie jak dowód twierdzenia Lov\'asza (który zaraz pokażemy),
	z tym, że trzeba wielokrotnie stosować rekurencyjny wzór na współczynniki dwumianowe.
\end{proof}

\begin{definition} Niech \(\mathcal F \subset \binom{\natural}{k}\) dla pewnego \(k \geq 1\) oraz ustalmy \(i \geq 2\). Operator przesunięcia \(S_i\) tworzy nową rodzinę \(S_i(\mathcal F) = \{S_i(F) : F \in \mathcal F\}\), gdzie
	\[ S_i(F) =
		\begin{cases}
			F \setminus \{i\} \cup \{1\} & \text{jeśli } i \in F, 1 \notin F \text{ oraz } F \setminus \{i\} \cup \{1\} \notin\mathcal F, \\
			F                            & \text{w przeciwnym przypadku}.
		\end{cases}
		.\]
	Jeśli \(S_i(F)=F\) z powodu istnienia już przesuniętego zbioru w rodzinie, to mówimy, że \(F\) został zablokowany.
\end{definition}

\begin{lemma}
	\label{kk_rozmiar}
	Dla każdego skończonego \(\mathcal F\subset \binom{\natural}{k}\) i \(i\ge 2\) jest
	\(|S_i(\mathcal F)| = |\mathcal F|\).
\end{lemma}
\begin{proof}
	Różne zbiory są przesuwane w różne zbiory, a zbiór nie zostanie przesunięty,
	jeśli jego przesunięcie już jest w rodzinie.
\end{proof}

\begin{lemma}
	\label{kk_zamiana}
	Dla dowolnego skończonego \(\mathcal F \subset \binom{\natural}{k}\) i dowolnego \(i
	\geq 2\) jest \(\Delta S_i(\mathcal F) \subseteq S_i(\Delta\mathcal F)\).
\end{lemma}
\begin{proof}
	Dowód wymaga rozważenia czterech przypadków. Przypuśćmy, że \(E \in \Delta
	S_i(\mathcal F)\), więc \(E = S_i(F) \setminus \{x\}\) dla pewnego \(F \in
	\mathcal F\) i \(x \in S_i(F)\).

	Najpierw załóżmy, że \(1, i \notin S_i(F)\). Ponieważ \(1 \notin S_i(F)\), musimy
	mieć \(S_i(F) = F\), a zatem \(E \subset F\). Zatem \(E \in \Delta\mathcal  F\), a
	ponieważ \(i \notin E\), to \(S_i(E) = E\). W związku z tym \(E \in
	S_i(\Delta\mathcal F)\).

	Teraz przypuśćmy, że \(1, i \in S_i(F)\). Ponieważ \(i \in S_i(F)\), mamy \(S_i(F)
	= F\), a zatem \(E \in \Delta\mathcal F\), jak wcześniej. Jeśli \(x \neq 1\), to
	\(1 \in E\), i zatem \(E = S_i(E) \in S_i(\Delta\mathcal F)\). Jeśli \(x = 1\), to
	\(E' = E \setminus \{i\} \cup \{1\} \subset F\), a zatem \(E' \in \Delta\mathcal
	F\). To oznacza, że \(E\) jest zablokowane i \(S_i(E) = E\), co implikuje \(E \in
	S_i(\Delta\mathcal F)\).

	W trzecim przypadku przypuśćmy, że \(S_i(F) \cap \{1, i\} = \{i\}\). Ponieważ \(i
	\in S_i(F)\), musimy mieć \(S_i(F) = F\). Jednakże, jako że \(i \in F\) i \(1
	\notin F\), \(F\) musiało być zablokowane przez \(F' = F \setminus \{i\} \cup
	\{1\} \in\mathcal F\). Ponieważ \(E \subset S_i(F) = F\), \(E \in \Delta\mathcal
	F\). Jeśli \(x = i\), to \(i \notin E\), i zatem \(E = S_i(E) \in
	S_i(\Delta\mathcal F)\). Jeśli \(x \neq i\), to \(E\) byłoby zablokowane przez \(E'
	= F' \setminus \{x\} \in \Delta\mathcal F\), i zatem \(E = S_i(E) \in
	S_i(\Delta\mathcal F)\) również w tym przypadku.

	Ostatni przypadek to gdy \(S_i(F) \cap \{1, i\} = \{1\}\). Zauważmy, że \(i
	\notin E\) i zatem \(S_i(E) = E\). W związku z tym, jeśli \(E \in \Delta\mathcal
	F\), to \(E = S_i(E) \in S_i(\Delta\mathcal F)\). Jeśli \(F\) nie przesunął się,
	to \(F = S_i(F)\) i \(E \in \Delta\mathcal F\). Jeśli \(F\) przesunął się, to
	\(S_i(F) = F \setminus \{i\} \cup \{1\}\). Jeśli \(x = 1\), to \(E \subset F\) i
	zatem jak wcześniej \(E \in \Delta\mathcal F\). Jeśli \(x \neq 1\), niech \(E' = E
	\setminus \{1\} \cup \{i\}\) i zauważmy, że \(E' \subset F\), i zatem \(E' \in
	\Delta\mathcal F\). Wtedy albo \(E \in \Delta\mathcal F\), albo \(E'\) nie jest
	zablokowane przed przesunięciem, i \(E = S_i(E') \in S_i(\Delta\mathcal F)\).
	To kończy analizę przypadków.
\end{proof}

\begin{definition}
	Rodzinę \(\mathcal F \subset \binom{\natural}{k}\) nazywamy stabilną, jeśli
	\(S_i(\mathcal F) = \mathcal F\) dla każdego \(i\ge 2\).
\end{definition}

\begin{lemma}
	\label{kk_stabilna}
	Dla każdej skończonej rodziny \(\mathcal F\subset \binom{\natural}{k}\) istnieje
	rodzina stabilna \(\mathcal G\subset\binom{\natural}{k}\) taka, że \(\card{\mathcal G}
	= \card{\mathcal F}\) i \(\card{\Delta \mathcal G} \le \card{\Delta \mathcal F}\).
\end{lemma}
\begin{proof}
	Dla stabilnej \(\mathcal F\) można wziąć \(\mathcal G = \mathcal F\), a inaczej
	można wziąć \(\mathcal F' = S_i(\mathcal F) \ne \mathcal F\) dla pewnego \(i\ge
	2\) -- Lematy \ref{kk_rozmiar} i \ref{kk_zamiana} dają pożądane wielkości
	odpowiednich zbiorów. Możemy w ten sposób przesuwać rodzinę, póki się da. Ten
	proces się zakończy, bo każde przesunięcie zwiększa liczbę zbiorów
	zawierających \(1\).
\end{proof}

\begin{lemma}
	\label{kk_zawieranie}
	Dla każdej stabilnej rodziny \(\mathcal F\subset \binom{\natural}{k}\) zachodzi
	\(\Delta\mathcal F_0 \subseteq \mathcal F'_1\), gdzie \(\mathcal F = \mathcal
	F_0 \sqcup \mathcal F_1\) i \(\mathcal F_0 = \{F \in \mathcal F : 1 \notin F\}\)
	oraz \(\mathcal F_1 = \{F \in \mathcal F : 1 \in F\}\) i \(\mathcal F_1' = \{F
	\setminus \{1\} : F \in \mathcal F_1\}\).
\end{lemma}
\begin{proof}
	Przypuśćmy, że \(E \in \Delta\mathcal F_0\). Wówczas musimy mieć \(E = F
	\setminus \{x\}\) dla pewnego \(F \in\mathcal F_0\) oraz \(x \in F\). Ponieważ \(F
	\in\mathcal F_0\), \(x \geq 2\). Ponieważ \(\mathcal F\) jest stabilna, to
	\(S_x(\mathcal F) =\mathcal F\), a zatem \(S_x(F) = F\). To oznacza, że \(F\) był
	zablokowany, więc \(F' = F \setminus \{x\} \cup \{1\} \in \mathcal F\) i w
	szczególności jest w \(\mathcal F_1\). Zatem \(E = (F \setminus \{x\} \cup
	\{1\}) \setminus \{1\} \in \mathcal F_1'\).
\end{proof}

\begin{lemma}
	\label{kk_suma}
	Dla każdej stabilnej rodziny \(\mathcal F\subset \binom{\natural}{k}\) zachodzi
	\(\card{\Delta\mathcal F} = \card{\mathcal F'_1} + \card{\Delta\mathcal
		F'_1}\), gdzie \(\mathcal F = \mathcal F_0 \sqcup \mathcal F_1\) i \(\mathcal F_0
	= \{F \in \mathcal F : 1 \notin F\}\) oraz \(\mathcal F_1 = \{F \in \mathcal F
	: 1 \in F\}\) i \(\mathcal F_1' = \{F \setminus \{1\} : F \in \mathcal F_1\}\).
\end{lemma}
\begin{proof}
	Oczywiście mamy \(\Delta\mathcal F = \Delta\mathcal F_0 \cup \Delta\mathcal
	F_1\). W Lemacie \ref{kk_zawieranie} pokazaliśmy, że \(\Delta\mathcal F_0
	\subseteq \mathcal F_1'\). Niech \(\mathcal F'' = \set{F\cup\set{1} :
		F\in\Delta \mathcal F'_1}\) Pokażemy, że \(\Delta\mathcal F_1 = \mathcal
	F'_1\cup \mathcal F''\). Te dwa zbiory są rozłączne (elementy tylko jednego
	zawierają \(1\)), a w pierwszym z nich zawiera się \(\mathcal F_0\), więc da nam
	to żądaną równość.

	To, że \(\mathcal F_1' \subseteq \Delta\mathcal F_1\), wynika z jego definicji,
	ponieważ dla każdego \(F' \in \mathcal F_1'\) mamy \(F' = F \setminus \{1\}\) dla
	pewnego \(F \in \mathcal F_1\). Usunięcięcie elementu i dodanie \(1\) do elementu
	\(\mathcal F'_1\) (przy definiowaniu \(\mathcal F''\)) można zrobić w odwrotnej
	kolejności, więc \(\mathcal F''\subseteq \Delta\mathcal F_1\). Jednocześnie w
	tych dwóch zbiorach znajdują się wszystkie elementy cienia \(\mathcal F_1\) --
	jedne z nich powstają przez usunięcie \(1\), a drugie przez usunięcie
	czegokolwiek innego. To dowodzi zawierania w drugą stronę i kończy dowód.
\end{proof}

\begin{theorem}[Lov\'{a}sz]
	Niech \(\mathcal F\subset\binom{\natural}{k}\) i \(\card{\mathcal F} = m =
	\binom{x}{k}\), gdzie \(x\in\real\).
	(dla przypomnienia, definiujemy \(\binom{x}{k} = \frac{x^{\underline{k}}}{k!}\)).
	Wtedy \[\card{\Delta\mathcal F} \ge \binom{x}{k-1}.\]
\end{theorem}
\begin{proof}
	Przeprowadzimy indukcję po \((k,m)\). Dla \(k=1\) cień zawiera zbiór pusty i
	wymagamy od niego rozmiaru \(1\). Dla \(m=1 = \binom{k}{k}\) cień składa się z \(k
	= \binom{k}{k-1}\) elementów. Dalej zakładamy, że \(k,m\ge 2\). Z Lematu
	\ref{kk_stabilna} możemy założyć, że \(\mathcal F\) jest stabilna. Niech
	\(\mathcal F = \mathcal F_0 \sqcup \mathcal F_1\) i \(\mathcal F_0 = \{F \in
	\mathcal F : 1 \notin F\}\) oraz \(\mathcal F_1 = \{F \in \mathcal F : 1 \in
	F\}\) i \(\mathcal F_1' = \{F \setminus \{1\} : F \in \mathcal F_1\}\).
	Pokażemy, że \(\card{\mathcal F'_1} \ge \binom{x-1}{k-1}\).

	Załóżmy, że tak nie jest. Mamy \(m=\card{\mathcal F} = \card{\mathcal F_0} +
	\card{\mathcal F_1}\) oraz \(\card{\mathcal F'_1} = \card{\mathcal F_1}\), zatem
	\(\card{\mathcal F_0} > \binom{x}{k} - \binom{x-1}{k-1} = \binom{x-1}{k}\). Dla
	stabilnej rodziny \(\mathcal F\) rodzina \(\mathcal F_1\) jest niepusta i
	\(\card{\mathcal F_0} < m\), więc z indukcji i Lematu \ref{kk_zawieranie} jest
	\(\card{\mathcal F'_1} \ge \card{\Delta\mathcal F_0} \ge \binom{x-1}{k-1}\), co
	daje sprzeczność z założeniem nie wprost.

	Z indukcji mamy teraz \(\card{\Delta\mathcal F'_1} \ge \binom{x-1}{k-2}\). Z
	Lematu \ref{kk_suma} mamy więc \(\card{\Delta\mathcal F} = \card{\mathcal
		F'_1} + \card{\Delta\mathcal F'_1} \ge \binom{x-1}{k-1} + \binom{x-1}{k-2} =
	\binom{x}{k-1}\), co kończy dowód.
\end{proof}






