\begin{theorem}[Twierdzenie Dilwortha]
	Jeśli długość maksymalnego antyłańcucha w posecie wynosi k, poset ten da się pokryć w całości z użyciem k łańcuchów.
\end{theorem}

\begin{proof}
	Robimy indukcję po liczbie elementów posetu; gdy poset $P$ składa się z jednego elementu twierdzenie zachodzi w trywialny sposób. Załóżmy teraz, że mamy poset składający się z $n$ elementów i o najdłuższym antyłańcuchu długości $k$; antyłańcuch ten nazwiemy $A$. Zdefiniujmy teraz zbiory $U$ i $D$, które będziemy określać jako \textit{upset} i \textit{downset}. Do zbioru $U$ należą wszystkie elementy $P$, takie że są większe od jakiegokolwiek elementu z $A$. Do zbioru $D$ należą zaś wszystkie elementy $P$, takie że są mniejsze od jakiegokolwiek elementu z $A$. Bardziej formalnie:
	\begin{equation*}
		U = \{\,x \in P \mid \exists_{y \in A}  y \leq x \} \setminus A
	\end{equation*}
	\begin{equation*}
		D = \{\,x \in P \mid \exists_{y \in A} y \geq x \} \setminus A
	\end{equation*}

	Pierwsza obserwacja którą należy wykonać, to taka że $U \cup D \cup A = P$. Jest to oczywiste; jeśli istniałby jakiś element z $L$ który nie należałby ani do downsetu ani do upsetu ani do antyłańcucha maksymalnego, to z faktu że nie należy ani do downsetu ani do upsetu wynikałoby, że musiałby należeć do antyłańcucha maksymalnego (bo nie jest porównywalny z żadnym elementem z antyłańcucha maksymalnego).

	Druga obserwacja: nie istnieje element, który należy jednocześnie do upsetu i downsetu. Załóżmy nie wprost, że tak jest: mamy jakieś $x, y, z$ takie, że $x \geq y$,  $x \leq z$, i $y,z \in A$. Wówczas otrzymujemy że $y \leq x \leq z$, a więc z tranzytywności $P$ mamy że $y \leq z$, ale $y, z$ są nieporównywalne bo są w jednym łańcuchu. Otrzymana sprzeczność dowodzi obserwację.

	Teraz musimy rozpatrzyć trzy przypadki:
	\begin{enumerate}
		\item $U = D = \emptyset$ \\
		      Bardzo fajny przypadek, głównie dlatego że trywialny do udowodnienia; każdy element z antyłańcucha tworzy jednoelementowy łańcuch zawierający tylko siebie samego, mamy podział $P$ na $k$ łańcuchów.
		\item $U = \emptyset \hspace{5pt} \mathbf{ALBO} \hspace{5pt} D = \emptyset$ \\
		      Wyrzucam z $P$ jakiś łańcuch maksymalny (jakikolwiek). Teraz są różne przypadki: \begin{enumerate}
			      \item Nie ma już antyłańcucha maksymalnego o długości $k$, wtedy z założenia indukcyjnego $P$ da się podzielić, dorzucamy nasz łańcuch maksymalny z powrotem i mamy pokrycie takie jakie byśmy chcieli mieć,
			      \item Istnieje jakiś antyłańcuch maksymalny (inny niż nasz startowy, mogło tak być) długości $k$ i mający downset i upset niepuste; wtedy patrz przypadek trzeci,
			      \item Każdy antyłańcuch maksymalny nie ma albo downsetu albo upsetu, ale wtedy coś nie gra bo jego jeden z elementów powinien był się znaleźć w łańcuchu maksymalnym który wyleciał (bo element maksymalny łańcucha maksymalnego musiał być mniejszy niż jakiś element z antyłańcucha maksymalnego skoro był w downsecie, a więc mogłem ten element ,,dokleić'' i analogicznie w odwrotnym przypadku).
		      \end{enumerate}
		\item $U \not = \emptyset \wedge D \not = \emptyset$ \\
		      Rozpatruję sobie zbiory $B = A \cup U, C = A \cup D$. Oba z nich z założenia indukcyjnego da się podzielić na $k$ łańcuchów. Ponadto każdy element $A$ ma łańcuch różny od wszystkich innych elementów $A$ w swoim pokryciu łańcuchowym dla zbiorów $B$ i $C$ (2 elementy z jednego antyłańcucha nie mogą być w jednym łańcuchu). W takim razie po prostu ,,sklejam'' łańcuchy z $B$ i $C$ w danym elemencie $A$ i mam poprawne pokrycie łańcuchowe całego posetu $P$.

	\end{enumerate}

\end{proof}