\begin{theorem}
	Graf dwudzielny \(G\) można pokolorować krawędziowo przy użyciu \(\Delta(G)\) wierzchołków.
\end{theorem}
\begin{proof}
	Dowód przez indukcję po liczbie krawędzi; przypadek bazowy trywialny. Do poprawnie pokolorowanego grafu dwudzielnego \(G = (X,Y,E)\) dokładamy krawędź między jakimiś \(x\) i \(y\) i pokażemy, że taki graf również da się poprawnie pokolorować krawędziowo stosując \(\Delta(G)\) kolorów.

	Zauważam, że jako że krawędź między \(x \in X\) i \(y \in Y\) nie ma jeszcze koloru, to zarówno \(x\) jak i \(y\) muszą mieć jakiś kolor ,,wolny'', tzn. odpowiednio dla \(x\) jak i \(y\) muszą istnieć jakieś kolory \(\alpha, \beta\) takie, że krawędź o kolorze \(\alpha\) nie ,,wchodzi'' do \(x\), a krawędź o kolorze \(\beta\) nie ,,wchodzi'' do \(y\). Jeśli \(\alpha = \beta\) to kolorujemy krawędź między \(x\) i \(y\) na kolor \(\alpha\) no i mamy poprawne kolorowanie. Trochę nudne.

	Ciekawszy przypadek jest, gdy \(\alpha \not = \beta\): wtedy musimy jakoś sprytnie przekolorować \(G\), by móc pokolorować krawędź między \(x\) i \(y\) na jakiś kolor. Załóżmy że do \(x\) wchodzi krawędź o kolorze \(\beta\), a do \(y\) o kolorze \(\alpha\) (bo inaczej byśmy to trywialnie pokolorowali). Spójrzmy na wierzchołek z którym łączy się \(x\) krawędzią o kolorze \(\beta\); nazwijmy go \(y_1\). Zauważmy, że z \(y_1\) musi wychodzić krawędź o kolorze \(\alpha\), bo gdyby krawędź \(\alpha\) była wolna dla \(y_1\), to krawędź koloru \(\beta\) mógłbym przekolorować na kolor \(\alpha\), a więc kolor \(\beta\) w \(x\) by się ,,zwolnił'' i mógłbym przekolorować naszą oryginalną krawędź na \(\beta\), otrzymując poprawne kolorowanie spełniające tezę zadania.

	Wierzchołek z którym łączy się \(y_1\) za pomocą koloru \(\alpha\) nazwijmy zatem \(x_1\). Chyba już widzimy kierunek w którym to zmierza. \(x_1\) łączy się kolorem \(\beta\) z jakimś \(y_2\), bo jeśli nie to kolor \(\beta\) jest ,,wolny'' i możemy wcześniejszą krawędź przekolorować na niego, zwalniając jej kolor w wierzchołku \(y_1\) i umożliwiając dalsze przekolorowanie.

	Sekwencję takich wierzchołków, opisaną w ten sposób nazwiemy \(\beta\alpha-\)ścieżką. Zauważam że ścieżka ta nigdy nie może się ,,zacyklić'' w grafie dwudzielnym (w tym wrócić do \(x\) lub \(y\)) bo wygenerowałoby to sprzeczność z założeniami dotyczącymi kolorowania. Jednocześnie jako że graf jest skończony, musi ona gdzieś się skończyć. Znaczy to że dojdziemy w pewnym momencie do jakiegoś punktu, od którego będziemy mogli przekolorować całą tę ścieżkę i uzyskać poprawne kolorowanie całego grafu \(G\).
\end{proof}
