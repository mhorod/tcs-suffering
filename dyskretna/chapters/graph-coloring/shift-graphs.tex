    Shift grafy są kolejnym przykładem rodziny grafów bez trójkątów ale o~dowolnie dużej liczbie chromatycznej. Czemu są jako osobne wymaganie egzaminacyjne? Cytując Tadeusza Sznuka, odpowiedź brzmi: nie wiem, choć się domyślam. Otóż konstrukcja shift grafów (znana też jako konstrukcja Erdősa–Hajnala) jest jawna, natomiast te poprzednie są rekurencyjne. Konkretniej: będziemy konstruować ciąg takich grafów $G_n = \pars{V_n, E_n}$, że $\omega\pars{G_n} = 2$ oraz $\chi\pars{G_n} \geq \left\lceil\log_2n\right\rceil$. Zrobimy to następująco:
    \begin{gather*}
        V_n = \left\{[i, j] : 1 \leq i < j \leq n\right\}\\
        \{[i, j], [k, l]\} \in E_n \iff j = k
    \end{gather*}
    Innymi słowy, wierzchołkami naszego grafu będą przedziały o~końcach ze~zbioru $[n]$, bez przedziałów będących pojedynczym punktem, natomiast dwa przedziały są połączone krawędzią wtedy, gdy koniec jednego dotyka początku drugiego. Nie ma tutaj trójkątów. Dlaczego? Trójkąt musiałby być w postaci $[i, j], [j, k], [k, i]$, a~taka trójka przedziałów oczywiście nie może istnieć na prostej, bo wynikałoby z~tego coś w~stylu $i < j < k < i$.
    
    Przyszedł czas udowodnić fakt o~liczbie chromatycznej. Weźmy dowolne $k$-kolorowanie $G_n$ i~nazwijmy je $c$. Pokażemy, że $2^k \geq n$, czyli $k \geq \log_2n$, a~ponieważ pracujemy na liczbach całkowitych, to dostaniemy z~tego $k \geq \left\lceil\log_2n\right\rceil$. Zdefiniujmy zbiory $C_1, C_2, \ldots, C_n$ następująco:
    \begin{equation*}
        C_i = \left\{b \in [k] : b = c\pars{[j, i]} \text{ dla } j < i\right\}
    \end{equation*}
    Mówiąc najprościej, w~zbiorze $C_i$ są kolory przedziałów kończących się na punkcie $i$. Teraz pokażemy, że $C_i \neq C_j$ dla $i \neq j$. W~tym celu pokażemy, że dla każdej pary $j > i$ istnieje kolor należący do $C_j$ i~nienależący do $C_i$, czyli świadczący o~różności tych zbiorów. Jak tego dokonamy? Okazuje się, że bardzo prosto. Weźmy przedział $[i, j]$. Ma on jakiś kolor $b \in [k]$. Żaden przedział kończący się na $i$~nie może mieć koloru $b$, bo ma krawędź do $[i, j]$. Zatem $b \not\in C_i$. Ale oczywiście przedział $[i, j]$ kończy się na\dots $j$, a~skoro ma kolor $b$, to $b \in C_j$.
    
    Czyli, podsumowując, mamy zbiory $C_1, C_2, \ldots, C_n$ i~wszystkie różne. Oczywiście wszystkie są też podzbiorami $[k]$. Z~zasady szufladkowej wynika, że w~takim razie $n \leq 2^k$, co kończy dowód.
    \qed