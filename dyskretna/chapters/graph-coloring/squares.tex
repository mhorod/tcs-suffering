  \begin{theorem}
      Liczba kolorująca w grafach przecięć kwadratów na płaszczyźnie ograniczona jest przez $4 \omega - 3$.
  \end{theorem}
  \begin{proof}
      Ustalmy sobie porządek na kwadratach taki, że jeśli dany kwadrat $x$ ma krótszy bok niż kwadrat $y$, to $y$ jest w permutacji przed $x$. Jeżeli ktoś nie wie jak działają permutacje wierzchołków dla liczby kolorującej to sugeruję poczytać najpierw jak to wszystko działa. W sumie to nie sugeruję tylko nakazuję.

      Wracając, jako że im mniejszy jest kwadrat tym później jest w permutacji wierzchołków do cola, wszystkie kwadraty które są ,,na lewo'' od niego i mają do niego krawędź (tj. przecinają się z nim) muszą ,,w sobie'' mieć któryś z 4 wierzchołków tego kwadratu (przypominam że w grafie przecięć kwadratów na płaszczyźnie z definicji wszystkie kwadraty mają 2 krawędzie pionowo i 2 poziomo, nie dopuszczamy jakiegoś obracania). Fakt ten wynika z dowodu \textit{to widać}, ale serio -- jeśli ktoś znalazł sposób na władowanie kwadratu o większym boku w taki o mniejszym boku w grafie przecięć kwadratów \textbf{bez obracania} w taki sposób, że kwadrat o mniejszym boku nie ma żadnego wierzchołka w tym o większym boku, sugeruję kontakt z psychologiem.  

      No i fajnie, bo teraz ograniczam sobie liczbę kwadratów które mam ,,na lewo'' przez $4\omega - 4$. Dlaczego? Bo wszystkie kwadraty które przecinają wierzchołek $x$ mojego kwadratu (przypominam że kwadrat ma 4 wierzchołki, to zaraz będzie ważne) muszą formować jakąś klikę (no w sensie wszystkie mają punkt wspólny w postaci tego wierzchołka). Obecny kwadrat też wpada do tej kliki, więc wychodzi mi maksymalnie $\omega - 1$ sąsiadów na każdy wierzchołek. I w sumie to tyle, mamy ograniczenie liczby kolorującej więc mamy ograniczenie liczby chromatycznej. Fajnie. 
  \end{proof}