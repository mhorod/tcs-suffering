\section{Bezkontekstowość}

\begin{theorem}
    Niech \( L \) będzie językiem. Następujące stwierdzenia są równoważne:
    \begin{enumerate}
        \item \( L \) jest bezkontekstowy
        \item Istnieje PDA \( P \) taki że \( L = L(P) \)
        \item Istnieje PDA \( P \) taki że \( L = N(P) \)
    \end{enumerate}
\end{theorem}
\begin{proof}
\( (2) \iff (3) \) pokazaliśmy przed chwilą. Pokażmy teraz \( (3) \implies (1) \)

Nieterminale definiujemy jako
\[
    N = \set{S} \cup \set{(q, X, p) \mid q, p \in Q, X \in \Gamma }
\]

Intuicyjnie nieterminal \( (q, X, p) \) reprezentuje fragment słowa wyprodukowany między stanami \( q, p \) zjadając przy okazji szczyt stosu \( X \)

Dodajemy też produkcje
\[
    \forall_{p \in Q} : S \rightarrow (q_0, Z_0, p)
\]
\[
    \forall_{\dots} (q, X, r_k) \rightarrow a (r, Y_1, r_1) \dots ( r_{k-1}, Y_k, {r_k}) 
\]

Teraz musimy jeszcze pokazać, że ta gramatyka istotnie generuje to samo co akceptuje automat pustym stosem. tj.
\[
    (q, X, p) \rightarrow_G^* w \iff (q, w, X) \vdash_P^* (p, \eps, \eps) 
\]
\begin{description}
    \item \( \impliedby \)
        Indukcja po liczbie kroków wykonanych przez PDA.
    
    
    \item \( \implies \)
\end{description}

\end{proof}