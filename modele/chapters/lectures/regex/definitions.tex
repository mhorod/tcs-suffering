\section{Definicja}
\begin{definition}
	\textbf{Wyrażenie regularne} nad alfabetem \( \Sigma \) to najmniejszy język \( \re{\Sigma} \) nad alfabetem \( \Sigma' = \Sigma \cup \set{+, \cdot, ^*, 0, 1, (, )} \) taki, że:
	\begin{itemize}
		\item \( 0, 1 \in \re{\Sigma} \)
		\item \( \forall_{a \in \Sigma}: a \in \re{\Sigma} \)
		\item \( \forall_{d, d_1, d_2 \in \Sigma}: (d_1 + d_2), (d_1 \cdot d_2), d^* \in \re{\Sigma} \)
	\end{itemize}
\end{definition}

Warto zaznaczyć, że operujemy jedynie na napisach bez żadnej intepretacji -- symbole są jedynie symbolami, które będziemy zaraz interpretować odpowiednio.

Możemy teraz sobie dobrać do tego interpretację \( L: \re{\Sigma} \rightarrow \powerset\pars{\Sigma^*} \)
\begin{itemize}
	\item \( L(0) = \varnothing \)
	\item \( L(1) = \set{\eps} \)
	\item \( L(a) = \set{a} \)
	\item \( L(\alpha + \beta) = L(\alpha) \cup L(\beta) \)
	\item \( L(\alpha \cdot \beta) = L(\alpha)L(\beta) \)
	\item \( L(\alpha^*) = L(\alpha)^* \)
\end{itemize}
