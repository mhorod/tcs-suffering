\section{Problemy dla języków kontekstowych}

\subsection{Problem stopu}

\begin{lemma}
    Problem stopu (w wersji tupli automat i słowo) dla automatów ograniczonych liniowo jest rozwiązywalny.
\end{lemma}

\begin{proof}
    Pamięć zużywana przez LBA jest zależna silnie liniowo od wejścia - nawet jeżeli automat użyje wiele taśm, to jest to liczba dla konkretnego automatu ograniczona przez stałą.
    Alfabet znaków automatu również jest stały. Liczba możliwych konfiguracji jest więc ograniczona wykładniczo. Możemy więc zliczać liczbę kroków takiego automatu - po
    przekroczeniu pewnej liczby zależnej wyłącznie od automatu i długości wejścia, automat stan musi wystąpić dwa razy, czyli automat musi się zapętlić, czyli wiadomo, że nie zaakceptuje.
    Pozostaje jeszcze zauważyć, że niedeterminizm LBA nic tu nie psuje, bo możemy symulować różne ścieżki obliczeń z odpowiednimi licznikami.
\end{proof}

Dzięki rozwiązywalności problemu stopu, problemy dla LBA generalnie zachowują się jak problemy z R.