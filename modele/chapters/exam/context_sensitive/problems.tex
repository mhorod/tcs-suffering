\section{Problemy dla języków kontekstowych}

\subsection{Problem stopu}

\begin{lemma}
    Problem stopu (w wersji tupli automat + słowo) dla automatów ograniczonych liniowo jest rozwiązywalny.
\end{lemma}

\begin{proof}
    Pamięć zużywana przez LBA jest zależna silnie liniowo od wejścia - nawet jeżeli automat użyje wiele taśm, to jest to liczba dla konkretnego automatu ograniczona przez stałą.
    Alfabet znaków automatu również jest stały. Liczba możliwych konfiguracji jest więc ograniczona wykładniczo. Możemy więc zliczać liczbę kroków takiego automatu - po
    przekroczeniu pewnej liczby zależnej wyłącznie od automatu i długości wejścia, automat stan musi wystąpić dwa razy, czyli automat musi się zapętlić, czyli wiadomo, że nie zaakceptuje.
    Pozostaje jeszcze zauważyć, że niedeterminizm LBA nic tu nie psuje, bo możemy symulować różne ścieżki obliczeń z odpowiednimi licznikami.
\end{proof}

\subsection{Klasa złożoności LBA}

Aby zapisać licznik ilości kroków w problemie stopu, dzięki systemom liczbowym, wystarczy nam liniowo miejsca. Pociąga to natychmiast CSL \(\subset\) NPSPACE, zatem z tw. Savitch'a

\begin{corollary}
    CSL \(\subset\) NPSPACE
\end{corollary}

Podamy również bez dowodu następujące fakty:

\begin{lemma}
    CSL \( \not \subset \) NP
\end{lemma}

Jest to intuicyjne - nie ma powodu, by nie miał istnieć problem o liniowej pamięci ale wykładniczym czasie.

\begin{lemma}
    NP \( \not \subset \) CSL
\end{lemma}

Również jest to intuicyjne - NP dopuszcza wielomianową pamięć, LBA jedynie liniową.

Tutaj można jednak zauważyć, że przecież LBA może rozwiązać 3-SATa, który jest NP-zupełny. Jest to prawda, jednak 3-SAT jest NP-zupełny względem redukcji wielomianowej, nie liniowej.
Prowadzi nas to jednak do ciekawej obserwacji: możemy na maszynie Turinga zredukować dowolny problem z NP do 3-SAT, który możemy już rozwiązać LBA. Ostatecznie uzyskujemy ciekawą obserwację

\begin{corollary}
    Jeżeli istniała by wyrocznia LBA w P, to P = NP.
\end{corollary}

Nie ma to wielkiego praktycznego znaczenia, jednak warto to rozumieć. Podobne zadania pojawiały się również na egzaminie.