\section{Własności języków kontekstowych}

\subsection{Przecięcie}

\subsection{Konkatenacja}
\(G_1 = (N_1, \Sigma, P_1, S_1)\) \\
\(G_2 = (N_2, \Sigma, P_2, S_2)\)
\[ G = (S \cup N_1 \cup N_2, \Sigma, \{S \rightarrow S_1S_2\} \cup P_1 \cup P_2, S) 
\]
\[S \notin N_1 \cup N_2 \]

\subsection{Dopełnienie}
Idzie to z twierdzenia 6.6 (Immerman-Szelepcsény)


\subsection{Suma}
\(G_1 = (N_1, \Sigma_1, P_1, S_1)\) \\
\(G_2 = (N_2, \Sigma_2, P_2, S_2)\)
\[ G = (S \cup N_1 \cup N_2, \Sigma_1 \cup \Sigma_2, \{S \rightarrow S_1, S \rightarrow S_2\} \cup P_1 \cup P_2, S) 
\]
\[S \notin N_1 \cup N_2 \]

\subsection{Gwiazdka Kleenego}

