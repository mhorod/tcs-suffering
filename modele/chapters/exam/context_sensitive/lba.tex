\section{Automat ograniczony liniowo (LBA)}


\begin{definition}

 \textbf{Linear-Bounded Automaton} to ograniczona MT będąca największą tuplą na tym przedmiocie 
    \[
        LBA = (Q, \Sigma, \Gamma, \delta, q_0, \blank, \vdash, \dashv, F)
    \]
    gdzie
    \begin{itemize}
        \item \( Q \) -- skończony zbiór stanów
        \item \( \Sigma \subseteq \Gamma \) -- skończony alfabet wejściowy
        \item \( \Gamma \) -- skończony alfabet taśmowy
        \item \( \delta : Q \times \Gamma \rightarrow 
        Q \times \Gamma \times \set{-1, +1}
        \) -- stan, litera \( \rightarrow \) nowy stan, zmiana litery, ruch głowicą (lewo, prawo)
    \end{itemize}
    Przy czym jeśli \( ((q, \vdash), (q', X, d)) \in \delta \) to \( X = \vdash, d = 1 \) analogicznie po drugiej stronie taśmy
    
    \begin{itemize}
        \item \( q_0 \) -- stan startowy
        \item \( \blank \in \Gamma \setminus \Sigma \) -- pusty symbol / zero (domyślny symbol taśmy)
        \item \( F \) -- zbiór stanów akceptujących
    \end{itemize}
    
    Ponadto zakładamy, że \(Q \cap \Gamma = \varnothing\). 
\end{definition}

\begin{theorem}
    Dla każdej CSG G istnieje LBA A takie że \(L(G) = L(A) \)
\end{theorem}

\begin{proof}

\end{proof}