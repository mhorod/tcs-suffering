\section{Problem stopu}

Innym przydatnym przykładem nierozstrzygalnego problemu jest dość znany problem stopu czyli pytanie czy Maszyna Turinga \( M \) zatrzymuje się na słowie \( w \).

\begin{definition}
    Definiujemy język \( L_{HALT} \) następująco:
    \[
        L_{HALT} = \set{(M, w) : M \text{ zatrzymuje się na } w}
    \]
\end{definition}

\begin{theorem}
    \( L_{HALT} \in \re \setminus \r \)
\end{theorem}
\begin{proof}
    Oczywiście \( L_{HALT} \in \re \) bo możemy po prostu zasymulować \( M \) na \( w \) i jeśli się zatrzymamy to odpowiedź jest ,,TAK'' i wszystko super. Jeśli zaś odpowiedź to ,,NIE'' to nie musimy się w ogóle zatrzymywać.
    
    Aby pokazać że \( L_{HALT} \notin \r \) zakładamy nie wprost, że istnieje \( M \) z własnością stopu, taka że \( L(M) = L_{HALT} \).
    
    Konstruujemy teraz \( M' \) która:
    \begin{enumerate}
        \item wczytuje wejście \( x \)
        \item symuluje \( M \) na \( (x, x) \)
        \item jeśli \( M \) odpowiedziała ,,TAK'' to wpadamy w nieskończoną pętlę
        \item w przeciwnym razie zatrzymujemy się w dowolnym stanie
    \end{enumerate}
    
    Wrzucamy teraz do \( M \) wejście \( (M', M') \)
    
    Mamy dwie sytuacje:
    \begin{itemize}
        \item \( M \) zaakceptowała \( (M', M') \)
        
        Skoro \( (M', M') \in L(M) = L_{HALT} \) to oznacza, że
        wykonaliśmy krok (4) co ma miejsce jedynie gdy \( (x, x) \notin L(M) \). Tak się składa że u nas \( x = M' \) czyli \( (M', M') \notin L(M) \).
        
        \item \( M \) odrzuciła \( (M', M') \)
        
        Tutaj sytuacja jest podobna -- skoro \( (M', M') \notin L(M) = L_{HALT} \) to
        wykonaliśmy krok (3) co ma miejsce jedynie gdy \( (x, x) \in L(M) \), a ponieważ \( x = M' \) to \( (M', M') \in L(M) \).
    \end{itemize}
    
    Łącząc oba powyższe dostajemy \( (M', M') \in L(M) \iff (M', M') \notin L(M) \) co jest oczywiście sprzeczne. 
    
    W takim razie nie istnieje \( M \) z własnością stopu rozpoznająca \( L_{HALT} \), zatem \( L_{HALT} \notin \r \).
\end{proof}