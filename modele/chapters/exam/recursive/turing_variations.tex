\section{Niedeterministyczna Maszyna Turinga}

\subsection{Definicja}

\begin{definition}
    Niedeterministyczną Maszynę Turinga definiujemy tak samo jak deterministyczną Maszynę Turinga, przy czym funkcja \( \delta \) zostaje relacją; ciągnie to za sobą dosyć oczywiste konsekwencje w definiowaniu relacji \( \vdash\).
    
    Co ciekawe, definicja języka akceptowanego się nie zmienia; nadal zwyczajnie chcemy mieć stan akceptujący w domknięciu przechodnim.
\end{definition}

\subsection{Równoważność z deterministyczną Maszyną Turinga}

\begin{theorem}
    Następujące stwierdzenia są równoważne dla dowolnego języka \( L \subset \Sigma^*\):
    \begin{enumerate}
        \item Istnieje deterministyczna Maszyna Turinga DM taka, że \(L(DM) = L\).
        \item Istnieje niedeterministyczna Maszyna Turinga NM taka, że \(L(NM) = L \). 
    \end{enumerate}
\end{theorem}

\begin{proof}
    Dowód faktu, że mając deterministyczną Maszynę Turinga możemy utworzyć niedeterministyczną Maszynę Turinga jest trywialny (każda funkcja jest relacją). 
    
    Skupmy się zatem na dowodzie w drugą stronę; mamy daną niedeterministyczną Maszynę Turinga \(N\) i chcemy stworzyć równoważną deterministyczną Maszynę Turinga \(M\). Aby to uczynić, będziemy (intuicyjnie) chodzić BFS-em po nieskończonym grafie konfiguracji maszyny \(N\).
    
    Formalizacja tego jest bardzo ciężka, mimo faktu że idea jest prosta (zatem podamy ideę). W alfabecie chcemy mieć jakiś separator, który jedyne do czego będzie używany, to do rozpatrywania kolejnych konfiguracji ,,w kolejce''. 
    
    Na początku ,,w kolejce'' konfiguracji do analizy mamy jedynie konfigurację startową. Mając głowicę w jakimś położeniu, znając jej stan i to co jest ,,na prawo'' od niej, dopisujemy do kolejki symbol separacji konfiguracji, a następnie całą naszą konfigurację po wykonanym którymś przejściu z maszyny \(N\). Procedurę tę wykonujemy dla każdego możliwego przejścia z \(N\). 
    
    Gdy ,,przerobimy'' wszystkie możliwe przejścia z \(N\) (dla każdej sytuacji jest ich skończenie wiele, więc zakodowanie tego jak \(M\) ma się zachowywać by wygenerować wszystkie te konfiguracji jest osiągalne), przechodzimy do następnej konfiguracji z~kolejki. No BFS, co tu jeszcze można powiedzieć?
    
    Za każdym razem, gdy analizujemy nową konfigurację, sprawdzamy czy zapisany w niej stan jest akceptujący; jeśli tak, to wtedy akceptujemy słowo. Jeśli nie ma zdefiniowanych przejść dla określonej sytuacji, to nie robimy nic i przechodzimy do kolejnej konfiguracji w kolejce.
    
    Again, formalizowanie tego wymagałoby wprowadzenia dodatkowych stanów które trzymałyby informacje o tym, co ,,widzi'' maszyna \(N\); ponadto wszystkie stany maszyny \(N\) muszą wejść do alfabetu taśmowego maszyny \(M\) by móc je również trzymać taśmie. To wszystko jest wykonywalne, ale przykre w implementacji. 
\end{proof}

\section{k-taśmowa maszyna Turinga}

Jest równoważna zwykłej maszynie. Czemu ? Bo możemy trzymać wszystkie taśmy szeregowo na jednej taśmie. Aby zrobić przejście przechodzimy od lewej do prawej po całej taśmie i zbieramy informacje o położeniu i stanach głowic -- następnie wykonujemy odpowiednie przejścia rozszerzając przy tym reprezetancje taśm jeśli trzeba.


\section{PDA z k stosami}

Okazuje się że jeśli dodamy PDA drugi stos to jest on równoważny Maszynie Turinga -- pierwszego stosu będziemy używali jako taśmy, drugiego będziemy używali jako bufor aby móc dowolnie przeglądać i modyfikować pierwszy stos.


\section{Uniwersalna Maszyna Turinga}

Maszyny Turinga są super, ale czy taka maszyna potrafi zasymulować jakąś inną, być może dowolną maszynę? Otóż okazuje się że tak -- umiemy skonstruować taką Maszynę Turinga, która jak dostanie na wejściu zakodowaną Maszynę Turinga \( M \) oraz jej wejście \( w \) to będzie w stanie zasymulować zachowanie \( M \) na \( w \) i zwrócić (zakodowany w tym samym formacie co wejście) wynik symulacji.

Niech \( M = (Q, \Sigma, \Gamma, \delta, q_0, \blank, F) \) będzie (deterministyczną) maszyną którą chcemy symulować. Pokażemy jak zakodować \( M \) nad alfabetem \( \Sigma_U = \set{0, 1} \).

\begin{itemize}
    \item Niech \( Q = \set{q_1, \dots, q_n} \)
    
    Kodujemy \( q_i \) jako ciąg \( i \) jedynek.
    
    \item \( \blank \) będziemy kodować jako pojedynczą jedynkę.
    
    
    \item Podobnie niech \( \Sigma = \set{a_1, \dots, a_n} \)
    
    Kodujemy \( a_i \) jako ciąg \( i + 1 \) jedynek.
    
    \item Dla \( \Gamma = \set{z_1, \dots z_n} \)
    
    Kodujemy \( z_i \) jako ciąg \( i + \card{\Sigma + 1} \) jedynek -- dla odróżnienia od \( \Sigma \).
    
    \item \( \delta \) jest funkcją, czyli zbiorem par \( ((q_1, z_1), (q_2, z_2, \pm 1) \)
    
    Parę \( ((q_i, z_j), (q_k, z_l), 1) \) zakodujemy jako
    
    \[
        \underbrace{1 \dots 1}_i
        0
        \underbrace{1 \dots 1}_j
        0
        \underbrace{1 \dots 1}_k
        0
        \underbrace{1 \dots 1}_l
        0
        1
    \]
    
    Natomiast parę \( ((q_i, z_j), (q_k, z_l), -1) \) zakodujemy jako
    
    \[
        \underbrace{1 \dots 1}_i
        0
        \underbrace{1 \dots 1}_j
        0
        \underbrace{1 \dots 1}_k
        0
        \underbrace{1 \dots 1}_l
        0
        11
    \]
    
    Całą deltę kodujemy kodując wszystkie jej pary (w dowolnej kolejności) i oddzielając je dwoma zerami.
    
    \item \( F = \set{q_{i_1}, \dots q_{i_n}} \) kodujemy jako
    \[
        \underbrace{1 \dots 1}_{i_1} 0 \dots 0 \underbrace{1 \dots 1}_{i_n}
    \]
\end{itemize}

Całą maszynę \( M \) zapisujemy wpisując \( \delta, q_0, F \) oddzielając je ciągiem \( 000 \).

Słowo \( w = a_{i_1} \dots a_{i_n} \) kodujemy jako
\[
    \underbrace{1 \dots 1}_{i_1} 0 \dots 0 \underbrace{1 \dots 1}_{i_n}
\]

Teraz musimy jeszcze powiedzieć jak kodujemy konfigurację maszyny \( M \) a robimy to dość prosto --
Konfigurację \( X_1 \dots X_k q_i X_{k+1} \dots X_n \) kodujemy jako
\[
    0\underbrace{1 \dots 1}_{\text{kod } X_1} 0 \underbrace{1 \dots 1}_{\text{kod } X_k} 00 \underbrace{1 \dots 1}_{\text{kod } q_i} 00 \underbrace{1 \dots 1}_{\text{kod } X_{k + 1}} 0 \underbrace{1 \dots 1}_{\text{kod } X_n} 0
\]

Możemy założyć, że kodowanie \( M \) jest zapisane na taśmie wejściowej po której nigdy nie piszemy, zaś konfigurację \( M \) będziemy trzymać na osobnej taśmie.

Krok symulacji przebiega teraz następująco:
\begin{enumerate}
    \item Zlokalizuj głowicę \( M \) (szukamy \( 00 \))
    \item Znajdź w kodowaniu \( \delta \) wpis odpowiadający przejściu na \( q_i X_k \)
    \item Wykonaj zadane przejście -- możemy wpisać nową konfigurację na osobną, tymczasową taśmę i przepisać ją z powrotem na taśmę na której operujemy.
    
\end{enumerate}


