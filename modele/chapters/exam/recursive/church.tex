\section{Teza Churcha}

\textbf{Teza Churcha} jest nieformalną tezą głoszącą, że jakaś funkcja matematyczna z liczb naturalnych może być policzona za pomocą \textit{efektywnej metody} (to znaczy takiej, którą można w jakiś sposób w pełni zautomatyzować) wtedy i tylko wtedy gdy może być obliczona przez Maszynę Turinga. 

W pierwszej połowie XX wieku próbowano sformalizować pojęcie \textbf{funkcji obliczalnych}, w związku z czym powstały 3 sformalizowane modele obliczeń:

\begin{enumerate}
    \item Rachunek \(\lambda\)
    \item Funkcje pierwotnie rekurencyjne
    \item Maszyny Turinga
\end{enumerate}

Okazuje się, że funkcja jest \(\lambda\)-obliczalna wtedy i tylko wtedy gdy jest pierwotnie rekurencyjna, oraz że jest pierwotnie rekurencyjna wtedy i tylko wtedy gdy jest obliczalna przez Maszynę Turinga.

Z racji równoważności tych trzech modeli obliczeń ludzie doszli do wniosku, że być może nie ma lepszego sposobu na automatyczne obliczanie funkcji. 