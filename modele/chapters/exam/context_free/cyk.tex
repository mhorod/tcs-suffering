\section{Algorytm CYK}

Na wejściu dostajemy gramatykę w postaci Chomsky'ego i słowo, odpowiadamy czy gramatyka może wygenerować słowo. Algorytm jest dynamiczny, zdefiniujmy zatem \(dp[i,j]\) - lista wszystkich nieterminali, które wygenerują nam podsłowo (słowa na wejściu) długości \(i\) oraz o początku w \(j\). Algorytm zaczynamy od wypełnienia list dla \(i=1\) - z racji na postać Chomsky'ego, będzie tam po jednym (lub ew. zero) nieterminali generujących daną literkę. Następnie, iterując się po rosnących \(i\) obliczamy dynamicznie kolejne listy - dla każdej komórki tabeli sprawdzamy wszystkie możliwe podziały danego podsłowa na dwa krótsze podsłowa i dodajemy wszystkie nieterminale, które wygenerują nam te podsłowa (czyli wygenerują dwa nieterminale takie, że pierwszy wygeneruje pierwsze podsłowo a drugi drugie); wystarczy sprawdzić podziały na dwa podsłowa, gdyż gramatyka jest w postaci Chomsky'ego.

Na końcu sprawdzamy czy w podproblemie odpowiadającemu całemu słowu (i początku w jego początku) jest symbol startowy, i zwracamy że się da, jeśli jest. Poprawność jest prosta do udowodnienia - wystarczy zauważyć, że skoro dla każdego podsłowa sprawdzamy wszystkie jego podziały, to nic nie przeoczymy.