\section{Konstrukcja CFG i PDA dla prostych języków}
\label{example-cfg-pda}

Skonstruujemy CFG i PDA jednego z prostszych i bardziej znanych języków bezkontekstowych:
\[
	\{ a^nb^ia^n | n, i > 0 \}
\]

\subsection{CFG}

Konstruując gramatyki bezkontekstowe, najczęściej chcemy zauważyć symetrię i wygenerować odpowiednie części od środka.\
W tym wypadku oczywiście symetryczna jest liczba \(a\) po obu stronach. Konstruujemy więc przejścia, które pozwolą nam wygenerować :

\(S \rightarrow aS_aa\)

\(S_a \rightarrow aS_aa\)

Mamy więc formę zdaniową postaci \(a^nS_aa^n\). Do szczęścia brakuje nam zastąpienia \(S_a\) przez  \(b^i\). Dodajmy więc:

\(S_a \rightarrow S_b\)

\(S_b \rightarrow bS_b | \eps\).

\subsection{PDA}

Konstruując PDA, często chcemy aby używał stosu do dopasowywania rzeczy które już widział do obecnych. W tym wypadku, wystarczy nam policzyć \(a\) przed\
pierwszym wystąpieniem \(b\) i sprawdzić, czy po tym wystąpieniu będzie ich tyle samo. Robimy to w dość prosty sposób:

\begin{enumerate}
	\item W pierwszym stanie, wrzucamy żeton (inny niż symbol startowy) na stos jeśli trafimy na \(a\). Przechodzimy do drugiego stanu jeśli trafimy na \(b\). Uważni czytelnicy zauważą,\
	      że musimy jeszcze zagwarantować przynajmniej jedno \(a\) - w przeciwieństwie do DFA nie dodajemy stanu, wystarczy że jeśli trafimy na \(b\) a na stosie mamy symbol startowy to odrzucamy wejście.
	\item W drugim stanie, nie ruszamy stosu jeśli trafimy na \(b\), i przechodzimy do trzeciego, jeśli trafimy na \(a\).
	\item W trzecim stanie, zdejmujemy żeton ze stosu jeśli trafimy na \(a\) i odrzucamy słowo, jeśli trafimy na \(b\).
	\item W tym miejscu musimy się zdecydować, co z symbolem startowym na stosie. Możemy go zdjąć wrzucając pierwszy żeton - wówczas nasz automat będzie akceptował pustym stosem.\
	      Jeśli jednak wolimy akceptację stanem akceptującym, to musimy taki stan dorobić - tworzymy do niego \(\eps\)-przejście z stanu 3 oraz symbolu startowego stosu. Wczytanie\
	      dowolnej literki w tym stanie powoduje odrzucenie słowa\footnote{taki "wrażliwy" stan końcowy jest również motywem wartym zapamiętania}.
\end{enumerate}
