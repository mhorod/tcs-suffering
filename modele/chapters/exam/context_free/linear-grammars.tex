\section{Gramatyki liniowe}

\begin{definition}
	Mówimy, że gramatyka jest \textbf{prawoliniowa} wtedy i tylko wtedy, gdy każda produkcja ma postać:

	\[
		A \rightarrow aB
	\]

	lub

	\[
		A \rightarrow a
	\]

	gdzie \( a \in \Sigma^* \) oraz \( A, B \in N \).

\end{definition}

\begin{definition}
	Mówimy, że gramatyka jest \textbf{silnie prawoliniowa} wtedy i tylko wtedy, gdy każda produkcja ma postać:

	\[
		A \rightarrow aB
	\]

	lub

	\[
		A \rightarrow \eps
	\]

	gdzie \(a \in \Sigma\) i \(A, B \in N\).
\end{definition}

\begin{theorem}
	Następujące stwierdzenia są sobie równoważne dla dowolnego języka \( L \subset \Sigma^* \):

	\begin{enumerate}
		\item \(L\) jest generowany przez pewno wyrażenie regularne.
		\item \(L\) jest generowany przez pewną gramatykę prawoliniową.
		\item \(L\) jest generowany przez pewną gramatykę silnie prawoliniową.
	\end{enumerate}
\end{theorem}

\begin{proof}
	Równoważność \((2)\) i \((3)\) jest dosyć widoczna: każda gramatyka silnie prawoliniowa jest w szczególności prawoliniowa, natomiast w drugą stronę możemy wykonać konwersję bardzo podobną do tej, którą stosowaliśmy przy otrzymywaniu CNF. Tym samym pozostawiamy ją jako ćwiczenie dla czytelnika.

	Ciekawszą rzeczą jest konwersja DFA na gramatyki prawoliniowe i z powrotem.

	Jeśli mamy DFA \(A\), to konwertujemy je do postaci gramatyki \(G\) następująco:

	\begin{enumerate}
		\item Ustalamy zbiór nieterminali gramatyki \(N\) jako zbiór stanów \(Q\) naszego DFA.
		\item Dla każdej litery \(a \in \Sigma\) i każdego stanu \(q\in Q\) dodajemy produkcję taką, że:
		      \[
			      q \rightarrow a \cdot \delta(q,a)
		      \]
		\item Nieterminal startowy naszej gramatyki to stan \(q_s\).
		\item Dla każdego stanu \(q \in F\) dodajemy produkcję:
		      \[
			      q \rightarrow \eps
		      \]
	\end{enumerate}

	Otrzymana gramatyka jest w oczywisty sposób prawoliniowa (i to silnie!). Wypadałoby jakoś dowieść równoważność tej gramatyki i DFA, ale to nawet widać: jeśli DFA akceptuje słowo \(w\), to oznacza że mam w nim ścieżkę od stanu startowego do końcowego, która ,,zjada'' po kolei literki z \(w\); jestem w stanie ją odtworzyć chodząc tymi samymi przejściami z gramatyki. Formalnie pewnie poleciałaby jakoaś indukcja.

	Dowód w drugą stronę przebiega bardzo podobnie, bo jeśli mam jakiś wywód, to jestem w stanie go trywialne przekształcić na ścieżkę w DFA, korzystając z konstrukcji.

	Teraz konstruujemy automat z gramatyki silnie prawoliniowej; robimy tym razem \(\eps\)-NFA, bo możemy i będzie prościej. Konstrukcja leci bardzo podobnie, tylko że na odwrót: każdy nieterminal \(A\) staje się stanem i ma przejście po literze \(a\) do stanu \(B\) jeśli w gramatyce znajduje się produkcja postaci:

	\[
		A \rightarrow aB
	\]

	Zauważmy, że to musi być teraz NFA, bo może być wiele produkcji przechodzących z jednego nieterminala dodając tę samą literkę.

	Stanem startowym staje się nieterminal startowy, a akceptującymi stają się te nieterminale, które mogą przejść na \(\eps\).

	Jeśli mamy wywód akceptujący dla słowa \(w\) w gramatyce, to nasze NFA z pewnością je zaakceptuje (idąc krawędziami ,,pochodzącymi'' od produkcji). Z kolei jeśli nasze NFA akceptuje jakieś słowo, to jestem w stanie ,,odtworzyć'' to przejście w gramatyce (to serio widać, formalnie pewnie poleci jakaś indukcja po długości słowa).

\end{proof}