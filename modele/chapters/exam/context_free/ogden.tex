\section{Lemat Ogdena}

W istocie bardzo podobny do lematu o pompowaniu, pozwala nam zaznaczać litery słowa i liczyć tylko te zaznaczone.

\begin{theorem}[Lemat Ogdena]
	Jeżeli język \(L\) nad \(\Sigma^*\) jest bezkontekstowy, to:

	\( \exists_{n_0 \in \natural} \) \\
	\( \forall_{w \in L} \) \\
	\( \exists_{a, b, c, d, e \in \Sigma^*} \hspace{5pt} w = abcde \land \card{w}_T \geq n_0 \land \card{bcd}_T \leq n_0 \land \card{bd}_T \geq 1 \) \\
	\( \forall_{i \in \natural} \hspace{5pt} ab^{i}cd^{i}e \in L\)
\end{theorem}
\begin{proof}
	Dowód jest analogiczny jak dowód Lematu o pompowaniu. Główna zmiana to zaznaczenie w drzewie wywodu wszystkich ścieżek z zaznaczonych liter do symbolu startowego. Dalej w dowodzie rozważamy tylko te węzły, których oba dzieci są zaznaczone, i postępujemy analogicznie.
\end{proof}