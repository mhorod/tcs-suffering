\section{Redukcje}

\begin{lemma}
    Jeśli \( L_1 \) jest C-trudny i \( L_1 \leq_p L_2 \) to \( L_2 \) też jest C-trudny. 
\end{lemma}

\begin{proof}
    Skoro \( L_1 \) jest C-trudny to znaczy, że dla każdego \( L' \in C \) mamy redukcję \( f \) z \( L' \) do \( L_1 \).
    Mamy też redukcję \( g \) z \( L_1 \) do \( L_2 \).
    
    Składając \( g \circ f \) dostajemy redukcję z \( L' \)  do \( L_2 \) co dowodzi, że \( L_2 \) jest trudniejszy niż dowolny \( L' \n C \)
\end{proof}

\begin{lemma}
    \( L \) jest C-trudny wtedy i tylko wtedy, gdy \( \complement{L} \) jest  coC-trudny. 
\end{lemma}

\begin{proof}

Dowodzimy równoważność w obie strony:

\begin{itemize}
    \item     \( \implies \)
    
    Weźmy sobie \( L' \in C\). Jako, że \(L'\) jest trudne, to \(L' <_{p} L\).  Wiemy wobec tego, że istnieje taka funkcja\footnote{obliczalna w czasie wielomianowym bla bla bla} \(f\), obliczalna w czasie wielomianowym, że \( x \in L' \iff f(x) \in L\).
    
    Równoważnie, \( x \not\in L' \iff f(x) \not \in L\). No ale to z kolei oznacza, że \( x \in \complement{L'} \iff f(x) \in \complement{L}\). Wobec tego, dla każdego problemu należącego do \(coC\) jest tak, że redukuje się on do \( \complement{L} \), czyli \(\complement{L}\) jest \(coC\)-trudny.

    \item \( \impliedby \) 
    Centralnie piszemy to samo, ale w drugą stronę.
    
    
    Bierzemy sobie \( L' \in coC\). Z racji faktu, że \(L'\) jest trudne mamy, że \(L' <_{p} \complement{L}\). To oznacza, że istnieje funkcja \(f\), taka że \( x \in L' \iff f(x) \in \complement{L} \). 
    
    Równoważnie, \( x \not \in L' \iff f(x) \not \in \complement{L} \). Z tego wiemy, że \( x \in \complement{L'} \iff f(x) \in L \). Wobec tego, każdy problem z \(C\) redukuje się do \(L\), czyli \(L\) jest \(C\)-trudne. 
    

\end{itemize}

\end{proof}
